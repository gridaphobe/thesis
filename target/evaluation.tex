\section{Evaluation} \label{sec:evaluation}

We have built a prototype implementation of \toolname\footnote{\url{http://hackage.haskell.org/package/target-0.1.1.0}} and next, 
describe an evaluation on a series of benchmarks ranging from 
textbook examples of algorithms and data structures to widely 
used Haskell libraries like \textsc{containers} and \textsc{xmonad}.
%
Our goal in this evaluation is two-fold. 
%
First, we describe micro-benchmarks (\ie functions)
that \emph{quantitatively compare} \toolname with 
the existing state-of-the-art, property-based testing
tools for Haskell -- namely \smallcheck and \quickcheck\ -- 
to determine whether \toolname is indeed able to generate
highly constrained inputs more effectively.
%
Second, we describe macro-benchmarks (\ie modules) that 
evaluate the amount of \emph{code coverage} that we 
get from type-targeted testing.
%
%% Third, using our results as a base, we present a 
%% qualitative discussion of \toolname as a \emph{gradual} 
%% approach that bridges informal and formal verification.

%% ES: This doesn't actually make any sense..
% An important optimization in our implementation is to 
% perform the post-condition checking in Haskell instead of by querying
% the SMT solver. We accomplish this by adding an additional @toReft@
% method to @Targetable@ that translates the concrete inputs that we
% @decode@ \emph{back} into logical expressions,
% %
% \begin{code}
%   toReft :: a -> Var
% \end{code}
% %
% which we then substitute into the output-type before checking the
% concrete output. @toReft@ can be simply implemented as:
% %
% \begin{code}
%   toReft v = app c (map toReft vs)
%     where (c, vs) = splitCtor v
%           -- app :: Ctor -> [Var] -> Var
% \end{code}
% %
% where @app@ is a pure version of @apply@, \ie it constructs a
% logical expression like @apply@ but does nothing else.

% \newcommand\XX{\multicolumn{1}{c}{X}}
% \newcommand{\mysec}[1]{\SI{#1}{\second}}
\begin{figure}[ht!]
  \centering
  % \includegraphics[width=0.49\linewidth]{figs/List-insert}
  % \includegraphics[width=0.49\linewidth]{figs/RBTree-add}
  % \includegraphics[width=0.49\linewidth]{figs/Map-delete}
  % \includegraphics[width=0.49\linewidth]{figs/Map-difference}
  % \includegraphics[width=0.49\linewidth]{figs/XMonad-focus-left}
  % \begin{tikzpicture}
  %    \begin{customlegend}[legend columns=4,legend style={align=center,draw=none},legend entries={\toolname,\smallcheck,\lazysmallcheck,\lazysmallcheck (slow)}]
  %    \addlegendimage{color=blue,mark=square*}
  %    \addlegendimage{color=red,mark=*}   
  %    \addlegendimage{color=orange,mark=diamond*}
  %    \addlegendimage{color=black,mark=x}
  %    \end{customlegend}
  % \end{tikzpicture}

  \begin{tikzpicture}
    \begin{groupplot}[
      group style = {group size = 3 by 1, horizontal sep=15pt,},
      groupplot ylabel={Time (sec)},
      groupplot xlabel={Depth},
      group/only outer labels,
      ymode=log,
      ymax=10000,
      ymin=0.0001
    ]
    % \begin{semilogyaxis}[
    \nextgroupplot[
      title=\textsc{List.insert}
    ]
    \addplot table[smooth,col sep=comma,x index=0,y index=1] {target/csv/List.insert.csv};
    \addplot table[smooth,col sep=comma,x index=0,y index=2] {target/csv/List.insert.csv};
    \addplot table[smooth,col sep=comma,x index=0,y index=3] {target/csv/List.insert.csv};
    % \end{semilogyaxis}
  % \end{tikzpicture}
  % \begin{tikzpicture}
    % \begin{semilogyaxis}[
    \nextgroupplot[
      title=\textsc{RBTree.add},
      legend columns=4,
      legend entries={\toolname,\smallcheck,\lazysmallcheck,\lazysmallcheck (slow)},
      legend to name=legend,
    ]
    \addplot table[smooth,col sep=comma,x index=0,y index=1] {target/csv/RBTree.add.csv};
    \addplot table[smooth,col sep=comma,x index=0,y index=2] {target/csv/RBTree.add.csv};
    \addplot table[smooth,col sep=comma,x index=0,y index=3] {target/csv/RBTree.add.csv};
    \addplot table[smooth,col sep=comma,x index=0,y index=4] {target/csv/RBTree.add.csv};
    % \end{semilogyaxis}
  % \end{tikzpicture}
  % \begin{tikzpicture}
    % \begin{semilogyaxis}[
    \nextgroupplot[
      title=\textsc{XMonad.focus\_left}
    ]
    \addplot table[smooth,col sep=comma,x index=0,y index=1] {target/csv/XMonad.focus_left.csv};
    \addplot table[smooth,col sep=comma,x index=0,y index=2] {target/csv/XMonad.focus_left.csv};
    \addplot table[smooth,col sep=comma,x index=0,y index=3] {target/csv/XMonad.focus_left.csv};
    % \end{semilogyaxis}
    \end{groupplot}
  \end{tikzpicture}
  \begin{tikzpicture}
    \begin{groupplot}[
      group style = {group size = 2 by 1, horizontal sep=15pt,},
      groupplot ylabel={Time (sec)},
      groupplot xlabel={Depth},
      group/only outer labels,
      ymode=log,
      ymax=10000,
      ymin=0.0001
    ]
    % \begin{semilogyaxis}[
    \nextgroupplot[
      title=\textsc{Map.delete}
    ]
    \addplot table[smooth,col sep=comma,x index=0,y index=1] {target/csv/Map.delete.csv};
    \addplot table[smooth,col sep=comma,x index=0,y index=2] {target/csv/Map.delete.csv};
    \addplot table[smooth,col sep=comma,x index=0,y index=3] {target/csv/Map.delete.csv};
    \addplot table[smooth,col sep=comma,x index=0,y index=4] {target/csv/Map.delete.csv};
    % \end{semilogyaxis}
  % \end{tikzpicture}
  % \begin{tikzpicture}
    % \begin{semilogyaxis}[
    \nextgroupplot[
      title=\textsc{Map.difference}
    ]
    \addplot table[smooth,col sep=comma,x index=0,y index=1] {target/csv/Map.difference.csv};
    \addplot table[smooth,col sep=comma,x index=0,y index=2] {target/csv/Map.difference.csv};
    \addplot table[smooth,col sep=comma,x index=0,y index=3] {target/csv/Map.difference.csv};
    \addplot table[smooth,col sep=comma,x index=0,y index=4] {target/csv/Map.difference.csv};
    % \end{semilogyaxis}
    \end{groupplot}
  \end{tikzpicture}\\
  \ref{legend}

  \caption{Results of comparing \toolname with \quickcheck, \smallcheck, and Lazy
    \smallcheck on a series of functions. \toolname, \smallcheck, and Lazy
    \smallcheck were both configured to check the first 1000 inputs that
    satisfied the precondition at increasing depth parameters, with a 60 minute
    timeout per depth; \quickcheck was run with the default settings, \ie it had
    to produce 100 test cases. \toolname, \smallcheck, and \lazysmallcheck were
    configured to use the same notion of depth, in order to ensure they would
    generate the same number of valid inputs at each depth level. \quickcheck was
    unable to successfully complete any run due to the low probability of
    generating valid inputs at random.}\label{fig:comparisonresults}
\end{figure}



\subsection{Comparison with \quickcheck and \smallcheck}\label{sec:comparison}

We compare \toolname with \quickcheck and \smallcheck by using 
a set of benchmarks with highly constrained inputs. 
%
For each benchmark we compared \toolname with \smallcheck and
\quickcheck, with the latter two using the generate-and-filter 
approach, wherein a value is generated and subsequently discarded if
it does not meet the desired constraint.
%
While one could possibly write custom ``operational'' generators 
for each property, the point of this evaluation is compare the 
different approaches ability to enable ``declarative'' specification 
driven testing.
%
Next, we describe the benchmarks and then summarize the results of the comparison
(Figure~\ref{fig:comparisonresults}).



\mypara{Inserting into a sorted \List}
%
Our first benchmark is the \Insert function from the homonymous 
sorting routine. We use the specification that given an element 
and a sorted list, @insert x xs@ should evaluate to a sorted list.
We express this with the type
%
\begin{code}
  type Sorted a = List <{\hd v -> hd < v}> a
  insert :: a -> Sorted a -> Sorted a
\end{code}
%
where the ordering constraint is captured by an abstract 
refinement~\cite{Vazou13} which states that \emph{each} 
list head @hd@ is less than every element @v@ in its tail.

\mypara{Inserting into a Red-Black Tree}
%
Next, we consider insertion into a Red-Black tree.
%
\begin{code}
  data RBT a = Leaf  | Node Col a (RBT a) (RBT a)
  data Col   = Black | Red
\end{code}
%
Red-black trees must satisfy three invariants:
%
(1)~red nodes always have black children,
(2)~the black height of all paths from the root to a leaf is the same, and
(3)~the elements in the tree should be ordered.
%
We capture (1) via a measure that recursively checks each @Red@ node has @Black@ children.
%
\begin{code}
  measure isRB :: RBT a -> Prop
  isRB Leaf           = true
  isRB (Node c x l r) = isRB l && isRB r &&
                        (c == Red => isBlack l && isBlack r)
\end{code}
%
We specify (2) by defining the @Black@ height as:
%
\begin{code}
  measure bh :: RBT a -> Int
  bh Leaf           = 0
  bh (Node c x l r) = bh l + (if c == Red then 0 else 1)
\end{code}
%
and then checking that the @Black@ height of both subtrees is the same:
%
\begin{code}
  measure isBH :: RBT a -> Prop
  isBH Leaf           = true
  isBH (Node c x l r) = isBH l && isBH r && bh l == bh r
\end{code}
%
Finally, we specify the (3), the ordering invariant as:
%
\begin{code}
  type OrdRBT a = RBT <{\r v -> v < r}, {\r v -> r < v}> a
\end{code}
%
\ie with two abstract refinements for the left and right subtrees
respectively, which state that the root @r@ is greater than (resp. less than)
each element @v@ in the subtrees. Finally, a valid Red-Black tree is:
%
\begin{code}
  type OkRBT a = {v:OrdRBT a | isRB v && isBH v}
\end{code}
%
Note that while the specification for the \emph{internal} invariants for Red-Black
trees is tricky, the specification for the public API -- \eg the @add@ function -- 
is straightforward:
%
\begin{code}
  add :: a -> OkRBT a -> OkRBT a
\end{code}

\mypara{Deleting from a Data.Map}\label{sec:delete-from-map}
%
Our third benchmark is the @delete@ function from the \hbox{@Data.Map@} module in 
the Haskell standard libraries. The @Map@ structure is a balanced binary
search tree that implements purely functional key-value dictionaries:
%
\begin{code}
  data Map k a = Tip | Bin Int k a (Map k a) (Map k a)
\end{code}
%
A valid @Data.Map@ must satisfy two properties:
%
(1)~the size of the left and right sub-trees must be 
    within a factor of three of each other, and
(2)~the keys must obey a binary search ordering.
%
We specify the balancedness invariant~(1) with a measure
%
\begin{code}
  measure isBal :: Map k a -> Prop
  isBal (Tip)           = true
  isBal (Bin s k v l r) = isBal l && isBal r &&
                          (sz l + sz r <= 1 ||
                           sz l <= 3 * sz r <= 3 * sz l)
\end{code}
%
and combine it with an ordering invariant (like @OrdRBT@) to specify valid trees.
%
\begin{code}
  type OkMap k a = {v : OrdMap k a | isBal v}
\end{code}
%
We can check that @delete@ preserves the invariants by 
checking that its output is an @OkMap k a@.
However, we can also go one step further and check 
the functional correctness property that @delete@ 
removes the given key, with a type:
%
\begin{code}
  delete :: Ord k => k:k -> m:OkMap k a 
         -> {v:OkMap k a | MinusKey v m k}
\end{code}
%
where the predicate @MinusKey@ is defined as:
%
\begin{code}
  predicate MinusKey M1 M2 K 
    = keys M1 = difference (keys M2) (singleton K)
\end{code}
%
using the measure @keys@ describing the contents of the @Map@:
%
\begin{code}
  measure keys :: Map k a -> Set k
  keys (Tip)           = empty () 
  keys (Bin s k v l r) = union (singleton k) 
                               (union (keys l) (keys r))
\end{code}

\mypara{Refocusing XMonad StackSets} \label{sec:refocus-stackset}
%
Our last benchmark comes from the tiling window manager XMonad. 
%
The key invariant of XMonad's internal @StackSet@ data structure 
is that the elements (windows) must all be \emph{unique}, \ie contain
no duplicates.
%
XMonad comes with a test-suite of over 100 \quickcheck properties;
we select one which states that moving the focus between windows 
in a @StackSet@ should not affect the \emph{order} of the windows.
%
\begin{code}
  prop_focus_left_master n s =
    index (foldr (const focusUp) s [1..n]) == index s
\end{code}
%
With \quickcheck, the user writes a custom generator for valid @StackSet@s
and then runs the above function on test inputs created by the generator, 
to check if in each case, the result of the above is @True@.

With \toolname, it is possible to test such properties \emph{without} 
requiring custom generators. Instead the user writes a declarative 
specification:
%
\begin{code}
  type OkStackSet = {v:StackSet | NoDuplicates v}
\end{code}
%
(We refer the reader to~\cite{VazouRealWorld14} for a full 
discussion of how to specify @NoDuplicates@).
%
Next, we define a refinement type:
%
\begin{code}
  type TTrue = {v:Bool | Prop v}
\end{code}
%
that is only inhabited by @True@, and use it to type the \quickcheck 
property as:
%
\begin{code}
  prop_focus_left_master :: Nat -> OkStackSet -> TTrue 
\end{code}
%
This property is particularly difficult to \emph{verify}; however,
\toolname is able to automatically
generate valid inputs to \emph{test} that @prop_focus_left_master@
always returns @True@.

%%% The high level of abstraction inherent in the @StackSet@ definition
%%% works in our favor here, as we can instantiate the relevant type parameter (the
%%% window) to \Char and leave the others as @()@ to drastically reduce
%%% the search space.


\mypara{Results}
%
Figure~\ref{fig:comparisonresults} summarizes the results of the comparison.
%
\quickcheck was unable to successfully complete \emph{any} 
benchmark to the low probability of generating properly 
constrained values at random.

\begin{description}
\item[List Insert] \toolname is able to test @insert@ all the way to 
   depth 20, whereas \lazysmallcheck times out at depth 19.

\item[Red-Black Tree Insert] \toolname is able to test @add@ up to depth 12,
  while \lazysmallcheck times out at depth 6.
  
\item[Map Delete] \toolname is able to check @delete@ up to depth 10, whereas
   \lazysmallcheck times out at depth 7 if it checks ordering first,
    or depth 6 if it checks balancedness first.

\item[StackSet Refocus] \toolname and is able to check this property 
    up to depth 8, while \lazysmallcheck times out at depth 7.
\end{description}

\toolname sees a performance hit with properties 
that require reasoning with the theory of Sets \eg 
the no-duplicates invariant of @StackSet@. 
%
While \lazysmallcheck times out at a higher depths, when it completes
\eg at depth 6, it does so in 0.7s versus \toolname's 9 minutes.
%
We suspect this is because the theory of sets are a relatively recent
addition to SMT solvers \cite{arrayZ3}, and with further improvements 
in SMT technology, these numbers will get significantly better.


Overall, we found that for \emph{small inputs} \lazysmallcheck 
is substantially faster as exhaustive enumeration is tractable,
and does not incur the overhead of communicating with an external 
general-purpose solver.
%
Additionally, \lazysmallcheck benefits from pruning predicates 
that exploit laziness and only force a small portion of the 
structure (\eg ordering). 
%
However, we found that constraints that force the entire 
structure (\eg balancedness), or composing predicates in the 
wrong \emph{order}, can force \lazysmallcheck to enumerate 
the entire exponentially growing search space.

\toolname, on the other hand, scales nicely to larger input sizes,
allowing systematic and exhaustive testing of larger, more complex
inputs. This is because \toolname eschews \emph{explicit} 
enumeration-and-filtering (which results in searching for 
fewer needles in larger haystacks as the sizes increas), 
in favor of \emph{symbolically} searching for valid models 
via SMT, making \toolname robust to the strictness or ordering 
of constraints.



\subsection{Measuring Code Coverage}\label{sec:code-coverage}

The second question we seek to answer is whether \toolname is suitable for testing entire
libraries, \ie how much of the program can be automatically exercised using our
system? Keeping in mind the well-known issues with treating code coverage as an
indication of test-suite quality~\cite{marick1999misuse}, we
consider this experiment a negative filter.

To this end, we ran \toolname against the entire user-facing API of 
\hbox{@Data.Map@,} our @RBTree@ library, and @XMonad.StackSet@ -- using 
the constrained refined types (\eg @OkMap@, @OkRBT@, @OkStackSet@) as 
the specification for the exposed types -- and measured the expression 
and branch coverage, as reported by @hpc@~\cite{gill2007haskell}.
%
We used an increasing timeout ranging from one to thirty minutes
per exported function.

\mypara{Results}
%
The results of our experiments are shown in Figure~\ref{fig:coverage}. 
Across all three libraries, \toolname achieved at least 70\% expression 
and 64\% alternative coverage at the shortest timeout of one minute per function. 
Interestingly, the coverage metrics for @RBTree@ and @Data.Map@ remain relatively constant as we increase
the timeouts, with a small jump in expression coverage between 10 and 20 minutes.
@XMonad@ on the other hand, jumps from 70\% expression and 64\% alternative
coverage with a one minute timeout, to 96\% expression and 94\% alternative
with a ten minute timeout.

% @Data.Map@ and @RBTree@ show no change in coverage metrics 
% beyond a 5 minute timeout, while @XMonad@ has another bump in coverage 
% between 10 and 15 minutes.

There are three things to consider when examining these results. 
%
First is that some expressions are not evaluated due to Haskell's 
laziness (\eg the values contained in a @Map@). 
%
Second is that some expressions \emph{should not} be evaluated 
and some branches \emph{should not} be taken, as these only happen
when an unexpected error condition is triggered (\ie these expressions
should be dead code).
%
\toolname considers any inputs that trigger an uncaught exception a 
valid counterexample; the pre-conditions should rule out these inputs, 
and so we expect not to cover those expressions with \toolname.

The last remark is not intrinsically related to \toolname, 
but rather our means of collecting the coverage data. @hpc@ includes 
@otherwise@ guards in the ``always-true'' category, even though they 
cannot evaluate to anything else. 
%
@Data.Map@ contained 56 guards, of which 24 were marked ``always-true''. We
manually counted 21 \hbox{@otherwise@} guards, the remaining 3 ``always-true''
guards compared the size of subtrees when rebalancing to determine whether a
single or double rotation was needed; we were unable to trigger the double
rotation in these cases.
%
\hbox{@XMonad@} contained 9 guards, of which 4 were ``always-true''. 3 of these
were @otherwise@ guards; the remaining ``always-true'' guard dynamically checked
a function's pre-condition. If the pre-condition check had failed an error would
have been thrown by the next case, we consider it a success of \toolname that
the error branch was not triggered.


\begin{figure}[t!]
\centering
% \includegraphics[width=0.49\linewidth]{figs/MapCoverage}
% \includegraphics[width=0.49\linewidth]{figs/XMonad-StackSetCoverage}
% \includegraphics[width=0.49\linewidth]{figs/RBTreeCoverage}
  \begin{tikzpicture}
    \begin{groupplot}[
      group style = {group size = 3 by 1, horizontal sep=15pt,},
      groupplot ylabel={\% Coverage},
      groupplot xlabel={Timeout (min)},
      group/only outer labels,
      ymin=0,
      ymax=1
    ]
    % \begin{axis}[
    \nextgroupplot[
      title=\textsc{Data.Map},
      legend columns=3,
      legend entries={expressions,booleans,always-true,always-false,alternatives,local-functions},
      legend to name=legend,
    ]
    \addplot table[smooth,col sep=comma,x index=0,y index=1] {target/csv/MapCoverage.csv};
    \addplot table[smooth,col sep=comma,x index=0,y index=2] {target/csv/MapCoverage.csv};
    \addplot table[smooth,col sep=comma,x index=0,y index=3] {target/csv/MapCoverage.csv};
    \addplot table[smooth,col sep=comma,x index=0,y index=4] {target/csv/MapCoverage.csv};
    \addplot table[smooth,col sep=comma,x index=0,y index=5] {target/csv/MapCoverage.csv};
    \addplot table[smooth,col sep=comma,x index=0,y index=6] {target/csv/MapCoverage.csv};
    % \end{axis}
  % \end{tikzpicture}
  % \begin{tikzpicture}
    % \begin{axis}[
    \nextgroupplot[
      title=\textsc{XMonad.StackSet},
    ]
    \addplot table[smooth,col sep=comma,x index=0,y index=1] {target/csv/StackSetCoverage.csv};
    \addplot table[smooth,col sep=comma,x index=0,y index=2] {target/csv/StackSetCoverage.csv};
    \addplot table[smooth,col sep=comma,x index=0,y index=3] {target/csv/StackSetCoverage.csv};
    \addplot table[smooth,col sep=comma,x index=0,y index=4] {target/csv/StackSetCoverage.csv};
    \addplot table[smooth,col sep=comma,x index=0,y index=5] {target/csv/StackSetCoverage.csv};
    \addplot table[smooth,col sep=comma,x index=0,y index=6] {target/csv/StackSetCoverage.csv};
    % \end{axis}
  % \end{tikzpicture}
  % \begin{tikzpicture}
    % \begin{axis}[
    \nextgroupplot[
      title=\textsc{RBTree}
    ]
    \addplot table[smooth,col sep=comma,x index=0,y index=1] {target/csv/RBTreeCoverage.csv};
    \addplot table[smooth,col sep=comma,x index=0,y index=5] {target/csv/RBTreeCoverage.csv};
    \addplot table[smooth,col sep=comma,x index=0,y index=6] {target/csv/RBTreeCoverage.csv};
    % \end{axis}
    \end{groupplot}
  \end{tikzpicture}\\
  \ref{legend}
% \begin{verbatim}
% 81% expressions used (2202/2712)
% 42% boolean coverage (24/57)
%      41% guards (23/56), 26 always True,
%          3 always False, 4 unevaluated
%     100% 'if' conditions (1/1)
%     100% qualifiers (0/0)
% 95% alternatives used (370/388)
% 98% local declarations used (49/50)
% 92% top-level declarations used (134/145)
% \end{verbatim}
\caption{Coverage-testing of \texttt{Data.Map.Base}, \texttt{RBTree}, and
  \texttt{XMonad.StackSet} using \toolname. Each exported function was tested
  with increasing depth limits until a single run hit a timeout ranging from one
  to thirty minutes. Lower is better for ``always-true'' and ``always-false'',
  higher is better for everything else.}\label{fig:coverage}
\end{figure}

%%% NUKE \ES{can we use an hpc overlay to make it ignore the "always true" otherwise
%%% NUKE   guards? seems the party line is that one should focus on expression and
%%% NUKE   alternative coverage, not boolean... so perhaps we can report expression, alternative, and alwaysFalse}
%%% NUKE \RJ{Dont know what you mean, is this note LIVE? or can we DELETE?}

% Although
% @hpc@ reports only 42\% boolean coverage for @Data.Map@, manual inspection
% revealed that 22 of the guards marked by @hpc@ as ``always True'' are
% @otherwise@ guards and can never be false. In that light, it would be more
% accurate to consider 46/57 booleans as covered, \ie 82\% coverage. The remaining
% ``always True'' branches compared the size of subtrees when rebalancing to
% determine whether a single or double rotation was needed, in some cases we were
% unable to generate sufficiently large trees in one minute to trigger a double
% rotation. The two guards that were always false were due to the simplistic
% generator we currently use for higher-order functions always returning false.

\subsection{Discussion}\label{sec:discussion}

To sum up, our experiments demonstrate that \toolname generates valid inputs:
%
(1) where \quickcheck fails outright, due to the low probability of
    generating random values satisfying a property;
%
(2) more efficiently than \lazysmallcheck, which relies on lazy
    pruning predicates; and
%
(3) providing high code coverage for real-world libraries with no
    hand-written test cases.

% \subsection{Limitations of \toolname}\label{sec:limitations}

Of course our approach is not without drawbacks; we highlight five classes
of pitfalls the user may encounter.

\mypara{Laziness} in the function or in the output refinement can cause exceptions
  to go un-thrown if the output value is not fully demanded. For example,
  \toolname would decide that the result @[1, undefined]@ inhabits @[Int]@ but not
  @[Score]@, as the latter would have to evaluate @0 <= undefined < 100@. This
  limitation is not specific to our system, rather it is fundamental to any tool
  that exercises lazy programs. Furthermore, \toolname only generates
  inductively-defined values, it cannot generate infinite or cyclic structures,
  nor will the generated values ever contain $\bot$.

\mypara{Polymorphism} Like any other tool that actually runs the function under scrutiny,
  \toolname can only test monomorphic instantiations of polymorphic
  functions. For example, when testing @XMonad@ we instantiated the ``window''
  parameter to @Char@ and all other type parameters to @()@, as the properties
  we were testing only examined the window. This helped drastically reduce the
  search space, both for \toolname and \smallcheck.

  % Our monomorphism restriction simplifies \toolname's implementation as we do
  % not have to consider type-class or equality constraints when generating test
  % values, but it also reduces the generalizability of \toolname's
  % result. 
  % Parametricity helps by telling us that the choice of
  % concrete instantiation will not affect the behavior of the function, but
  % in the presence of type-classes the benefit is reduced as we only know that
  % the specific instance we tested is correct.

\mypara{Advanced type-system features} such as GADTs and Existential types
  may prevent GHC from deriving a @Generic@ instance, which would force the
  programmer to write her own @Targetable@ instance. Though tedious, the single
  hand-written instance allows \toolname to automatically generate values
  satisfying disparate constraints, which is still an improvement over the
  generate-and-filter approach.
  
\mypara{Refinement types} are less expressive than properties written in the
  host language. If the pre-conditions are not expressible in \toolname's logic,
  the user will have to use the generate-and-filter approach, losing the benefits
  of symbolic enumeration.
  
\mypara{Input explosion} \toolname excels when the space of valid inputs is
  a sparse subset of the space of all inputs. If the input space is not
  sufficiently constrained, \toolname may spend lose its competitive advantage
  over other tools due to the overhead of using a general-purpose solver.

%% 1. laziness
%%    - potential for untriggered exceptions
%%    - our generated values never include bot
%% 2. Advanced type-system features
%%    - we can only provide default instances for Generic types
%%      - no GADTs or existentials
%% 3. Polymorphic functions
%%    - can only test monomorphic instantiation
%%    - types must be defaulted either by user or GHC
%%    - limitation shared by any testing tool

%%% Local Variables:
%%% mode: latex
%%% TeX-master: "main"
%%% End:
