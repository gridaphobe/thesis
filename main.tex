\documentclass[12pt]{ucsddissertation}

\usepackage{amsthm}

\usepackage[tt=false, type1=true]{libertine}
\usepackage[libertine]{newtxmath}
\usepackage[varqu]{zi4}
\usepackage[T1]{fontenc} % spell out all text encodings used

% mathptmx is a Times Roman look-alike (don't use the times package)
% It isn't clear if Times is required. The OGS manual lists several
% "standard fonts" but never says they need to be used.
%\usepackage{mathptmx}

\usepackage[NoDate]{currvita}
\usepackage{array}
\usepackage{tabularx}
\usepackage{booktabs}
\usepackage{subcaption}
\usepackage{ragged2e}
\usepackage{microtype}
\usepackage[breaklinks=true,pdfborder={0 0 0}]{hyperref}
\usepackage{graphicx}
\AtBeginDocument{%
	\settowidth\cvlabelwidth{\cvlabelfont 0000--0000}%
}

\usepackage{framed}

% OGS recommends increasing the margins slightly.
\increasemargins{.1in}

\usepackage[numbers,sort&compress,square]{natbib}
\usepackage{bibentry}
\nobibliography*
% \usepackage[natbib=true,style=numeric,firstinits=true,isbn=false,doi=false,url=false]{biblatex}
% \addbibresource{main.bib}



%
% the following standard packages may be helpful, but are not required
%
\usepackage{amsmath}
\usepackage{amssymb}
% \usepackage{SIunits}            % typset units correctly
% \usepackage{courier}            % standard fixed width font
% \usepackage[scaled]{helvet} % see www.ctan.org/get/macros/latex/required/psnfss/psnfss2e.pdf
\usepackage{url}                  % format URLs
\usepackage{enumitem}      % adjust spacing in enums
%\usepackage[colorlinks=false,allcolors=blue,breaklinks,draft=false]{hyperref}   % hyperlinks, including DOIs and URLs in bibliography
% known bug: http://tex.stackexchange.com/questions/1522/pdfendlink-ended-up-in-different-nesting-level-than-pdfstartlink
\newcommand{\doi}[1]{doi:~\href{http://dx.doi.org/#1}{\Hurl{#1}}}   % print a hyperlinked DOI
\usepackage{xspace}
\usepackage{commands}
\def\sectionautorefname{\S}
\def\subsectionautorefname{\S}

\usepackage{graphicx}
%\usepackage{pdfpages}
%\usepackage[doi]{natbib}

\renewcommand{\qedsymbol}{\ensuremath{\blacksquare}}
% \newtheorem{lemma}{Lemma}
% \newtheorem{corollary}{Corollary}
% \newtheorem{theorem}{Theorem}
% \newtheorem{definition}{Definition}
\theoremstyle{plain}% default
\newtheorem{thm}{Theorem} % [section]
\newtheorem{lem}[thm]{Lemma}
\newtheorem{prop}[thm]{Proposition}
\newtheorem*{cor}{Corollary}
\theoremstyle{definition}
\newtheorem{defn}{Definition} % [section]

\usepackage{listings}          % format code
\lstset{
  language=Caml,
  basicstyle=\ttfamily,
  keywordstyle=\ttfamily,
}
\lstnewenvironment{code}{
\lstset{
  language=Caml,
  basicstyle=\ttfamily,
  keywordstyle=\ttfamily,
}}
{}
\lstnewenvironment{ecode}{
\lstset{
  language=Caml,
  basicstyle=\ttfamily,
  keywordstyle=\ttfamily,
  numbers=left,
  frame=leftline,
  xleftmargin=7.5mm,
  moredelim=[is][\bfseries]{==}{==},
  moredelim=[is][\underbar]{__}{__},
  moredelim=[is][\bfseries\underbar]{_=}{=_},
  escapeinside={(*@}{@*)},
}}
{}
\lstnewenvironment{mcode}{
\lstset{
  language=Caml,
  basicstyle=\ttfamily,
  keywordstyle=\ttfamily,
  % numbers=left,
  % frame=leftline,
  % xleftmargin=7.5mm,
  escapeinside={(*}{*)},
  mathescape=true,
}}
{}


\usepackage[inference]{semantic}
\usepackage{mdframed}

\usepackage{fancybox}
\makeatletter
\newenvironment{CenteredBox}{%
\begin{Sbox}}{% Save the content in a box
\end{Sbox}\centerline{\parbox{\wd\@Sbox}{\TheSbox}}}% And output it centered
\makeatother

\usepackage{tikz}
\usetikzlibrary{positioning}
\usepackage{pgfplots}
\usepackage{pgfkeys}
\pgfplotsset{width=0.38\linewidth,compat=1.11}
%\pgfplotsset{compat=1.12}
\usepgfplotslibrary{groupplots}
\usepackage{pgfplotstable}

% magic from http://tex.stackexchange.com/questions/117759/create-x-and-y-label-which-overlaps-for-multiple-plots
\makeatletter
\pgfplotsset{
    groupplot xlabel/.initial={},
    every groupplot x label/.style={
        at={($({group c1r\pgfplots@group@rows.west}|-{group c1r\pgfplots@group@rows.outer south})!0.5!({group c\pgfplots@group@columns r\pgfplots@group@rows.east}|-{group c\pgfplots@group@columns r\pgfplots@group@rows.outer south})$)},
        anchor=north,
    },
    groupplot ylabel/.initial={},
    every groupplot y label/.style={
            rotate=90,
        at={($({group c1r1.north}-|{group c1r1.outer
west})!0.5!({group c1r\pgfplots@group@rows.south}-|{group c1r\pgfplots@group@rows.outer west})$)},
        anchor=south
    },
    execute at end groupplot/.code={%
      \node [/pgfplots/every groupplot x label]
{\pgfkeysvalueof{/pgfplots/groupplot xlabel}};
      \node [/pgfplots/every groupplot y label]
{\pgfkeysvalueof{/pgfplots/groupplot ylabel}};
    },
    group/only outer labels/.style =
{
group/every plot/.code = {%
    \ifnum\pgfplots@group@current@row=\pgfplots@group@rows\else%
        \pgfkeys{xticklabels = {}, xlabel = {}}\fi%
    \ifnum\pgfplots@group@current@column=1\else%
        \pgfkeys{yticklabels = {}, ylabel = {}}\fi%
}
}
}

\def\endpgfplots@environment@groupplot{%
    \endpgfplots@environment@opt%
    \pgfkeys{/pgfplots/execute at end groupplot}%
    \endgroup%
}
\makeatother


\usepackage{ifthen}
\newcommand{\isTechReport}{true} % true or false
\newcommand\includeTechReport[1]{%
  \ifthenelse{\equal{\isTechReport}{true}}
    {{#1}}
    {\ignorespaces}
\xspace}


\usepackage{tcolorbox}

% colors from http://colorbrewer2.org/?type=qualitative&scheme=Set2&n=3
% \definecolor{tree}{HTML}{66C2A5}
% \definecolor{sherrloc}{HTML}{FC8D62}
% \definecolor{fix}{HTML}{8DA0CB}
% \definecolor{sherrloc}{HTML}{FC8D62}
\definecolor{tree}{HTML}{8dd3c7}
\definecolor{sherrloc}{HTML}{FFFFB3}
\definecolor{fix}{HTML}{bebada}

\tcbset{every box/.style={on line,colframe=black,size=fbox,boxrule=1pt}}
\newtcbox{\hlTree}{colback=tree,toprule=0pt,bottomrule=0pt}
\newtcbox{\hlOcaml}{colback=tree,toprule=0pt,bottomrule=0pt}
\newtcbox{\hlSherrloc}{colback=sherrloc,leftrule=0pt,rightrule=0pt}
\newtcbox{\hlFix}{colback=fix!50}


% Required information
\title{Dynamic Witnesses for Static Type Errors}
\author{Eric Lee Seidel}
\degree{Computer Science \& Engineering}{Doctor of Philosophy}
% Each member of the committee should be listed as Professor Foo Bar.
% If Professor is not the correct title for one, then titles should be
% omitted entirely.
\chair{Professor Ranjit Jhala}
% \cochair{Professor Gamma Delta} % Optional
% Your committee members (other than the chairs) must be in alphabetical order
\committee{Professor William Griswold}
\committee{Professor Philip Guo}
\committee{Professor James Hollan}
\committee{Professor Sorin Lerner}
\degreeyear{2017}

% Start the document
\begin{document}
% Begin with frontmatter and so forth
\frontmatter
\maketitle
\makecopyright
\makesignature

% Optional
% \begin{dedication}
% \setsinglespacing
% \raggedright % It would be better to use \RaggedRight from ragged2e
% \parindent0pt\parskip\baselineskip
% In recognition of reading this manual before beginning to format the
% doctoral dissertation or master's thesis; for following the
% instructions written herein; for consulting with OGS Academic Affairs
% Advisers; and for not relying on other completed manuscripts, this
% manual is dedicated to all graduate students about to complete the
% doctoral dissertation or master's thesis.

% In recognition that this is my one chance to use whichever
% justification, spacing, writing style, text size, and/or textfont that
% I want to while still keeping my headings and margins consistent.
% \end{dedication}

% Optional
% \begin{epigraph}
% \vskip0pt plus.5fil
% \setsinglespacing
% {\flushright
% Beware of bugs in the above code; I have only proved it correct, not tried it.

% \vskip\baselineskip
% \textit{Donald Knuth}\par}
% \vfil
% % \begin{center}
% % You write with ease to show your breeding,\\
% % But easy writing's curst hard reading.

% % \vskip\baselineskip
% % \textit{Richard Brinsley Sheridan}
% % \end{center}
% % \vfil
% % \noindent Writing, at its best, is a lonely life. Organizations for
% % writers palliate the writer's loneliness, but I doubt if they improve
% % his writing. He grows in public stature as he sheds his loneliness and
% % often his work deteriorates. For he does his work alone and if he is a
% % good enough writer he must face eternity, or the lack of it, each day.

% % \vskip\baselineskip
% % \hskip0pt plus1fil\textit{Ernest Hemingway}\hskip0pt plus4fil\null

% % \vfil
% \end{epigraph}

% Next comes the table of contents, list of figures, list of tables,
% etc. If you have code listings, you can use \listoflistings (or
% \lstlistoflistings) to have it be produced here as well. Same with
% \listofalgorithms.
\tableofcontents
\listoffigures
\listoftables

% Preface
% \begin{preface}
% Almost nothing is said in the manual about the preface. There is no
% indication about how it is to be typeset. Given that, one is forced to
% simply typeset it and hope it is accepted. It is, however, optional
% and may be omitted.
% \end{preface}

% Your fancy acks here. Keep in mind you need to ack each paper you
% use. See the examples here. In addition, each chapter ack needs to
% be repeated at the end of the relevant chapter.
\begin{acknowledgements}
PEOPLE TO ACKNOWLEDGE (not in order)
\begin{itemize}
\item Jhala (advisor)
\item Griswold (quasi-advisor)
\item Ingolf (recruiter)
\item Weimer (great collaborator)
\item Troeger (inspiring instructor)
\item Gab (early mentor)
\item PL\@UCSD (in particular Niki)
\item Megan \& Family (support)
\end{itemize}

\end{acknowledgements}

% Stupid vita goes next
\begin{vita}
\noindent
\begin{cv}{}
\begin{cvlist}{}
\item[2012] Bachelor of Science, City College of New York
\item[2016] Master of Science, University of California, San Diego
\item[2017] Doctor of Philosophy, University of California, San Diego
\end{cvlist}
\end{cv}

% This puts in the PUBLICATIONS header. Note that it appears inside
% the vita environment. It is optional.
\publications
\noindent\bibentry{Seidel2016-ul}
\vskip\baselineskip
\noindent\bibentry{Elliott2015-xu}
\vskip\baselineskip
\noindent\bibentry{Seidel2015-pe}
\vskip\baselineskip
\noindent\bibentry{Vazou2014-gx}
\vskip\baselineskip
\noindent\bibentry{Vazou2014-xk}
\vskip\baselineskip
\noindent\bibentry{Seidel2012-fw}
\vskip\baselineskip
\noindent\bibentry{Khoo2012-tk}
\vskip\baselineskip
\noindent\bibentry{Allen2010-jr}
\vskip\baselineskip
\noindent\bibentry{Seidel2010-rx}


% \noindent``Distributions of Control Points in a System for Analysis of Stress
% Distribution'' IRE Transactions of the I.R.E\@. Professional Group on
% Automatic Control, vol. AC-7, pp 272--289, September 2005

% This puts in the FIELDS OF STUDY. Also inside vita and also
% optional.
% \fieldsofstudy
% \noindent Major Field: Engineering (Specialization or Focused Studies)
% \vskip\baselineskip
% Studies in Applied Mathematics\par
% Professors Alpha Beta and Gamma Delta
% \vskip\baselineskip
% Studies in Mechanices\par
% Professors Epsilon Zeta and Eta Theta
% \vskip\baselineskip
% Studies in Electromagnetism\par
% Professors Iota Kappa and Lambda Mu
\end{vita}

% Put your maximum 350 word abstract here.
\begin{dissertationabstract}
Static type systems are a powerful tool for reasoning about the safety
of programs.
%
Global type inference eliminates one of the prime complaints against
static types, that the annotation burden is too high.
%
However, this introduces its own problems as the type checker must now
make assumptions about what the programmer intended to do.
%
A single incorrect assumption can lead the type checker to erroneously
blame an expression far from the true error the programmer made, which
can be particularly confusing for newcomers who have not yet constructed
a mental model for how the type checker works.

In this dissertation we present a pair of complementary techniques to
\emph{localize} and \emph{explain} type errors, with an emphasis on the
errors encountered by novice users.
%
We tackle the localization problem by using machine learning to learn a
model of the errors made by students in an introductory course. Then, we
use the model to produce a ranked list of likely error locations in new
programs. Our models can be trained on a modest amount of data, \eg a
single instance of a course, and we envision a future where each
introductory course is accompanied by a model of its students' errors.
%
To better explain the error to novice users, we present a runtime error
that the type system would have prevented. We interleave type-checking
and execution to search for a set of program inputs that would lead
execution to a bad state, and present the execution trace to the user in
an interactive debugger. This allows the user to explore why their
program was rejected and connects the dynamic (runtime) semantics to the
static (typing) semantics.

We have evaluated our techniques empirically using a new dataset of
ill-typed student programs collected from two instances of an
undergraduate programming languages course at UC San Diego.
%
We have also performed user studies with novice users, comparing the
output of our techniques with the state of the art in type error
diagnosis.
%
Our results show that these are practical, lightweight techniques for
improving the error messages produced by type checkers.
\end{dissertationabstract}

% This is where the main body of your dissertation goes!
\mainmatter

% Optional Introduction
% \begin{dissertationintroduction}
% This optional section is barely described in the OGS manual other than
% saying it is optional and that it appears in the table of contents
% between the Abstract and the first chapter.

% No formatting guidelines appear so presumably, it should be formatted
% like an ordinary chapter. It should appear after the
% \verb!\mainmatter! macro because it should start on page~1.
% \end{dissertationintroduction}

\lstMakeShortInline{|}

%\part{Introduction}
\chapter{Introduction}
\label{chp:intro}
Static type systems are a marvelous invention.
%
They rule out, at compile-time, an entire class of run-time failures,
\eg dereferencing a |null|-pointer or passing an |int| where a |bool|
was expected.
%
Languages like \ocaml and \haskell make
the value-proposition for types even more
appealing by using constraints to automatically
synthesize the types for all program terms,
without troubling the programmer for any
annotations.
%
Unfortunately, this automation comes at a price.
%
Type annotations signify the programmer's intent, and help to correctly
blame the erroneous sub-term when the code is ill-typed.
%
In the absence of such signifiers, automatic type inference algorithms
are prone to report type errors far from their source
\citep{Wand1986-nw}.
%
While this can seem like a minor annoyance to veteran programmers,
\citet{Joosten1993-yx} have found that novices often focus their
attention on the \emph{location} reported and disregard the
\emph{message}.

In this dissertation we present two new techniques designed to help
\emph{localize} and \emph{explain} type errors.
%
First, in this chapter, we begin with a review of the existing
literature on type errors.
%
Second, we present a new dataset of novice interactions with the \ocaml
type-checker, which will form the backbone of our evaluation.
%
Finally, we present our techniques and evaluate them in the context of
localizing and explaining novice type errors.

\section{A Running Example}
\label{sec:intro:sumList}

\begin{figure}[t!]
\small
\begin{minipage}{0.45\linewidth}
\begin{ecode}
let rec sumList xs =
  match xs with
  | []   -> (*@\hlSherrloc{[]}@*)
  | h::t -> h + (*@\hlTree{sumList t}@*)
\end{ecode}
\end{minipage}
\begin{minipage}{0.49\linewidth}
\begin{verbatim}
File "sumList.ml", line 4, characters 16-25:
  This expression has type 'a list
  but an expression was expected of type int

\end{verbatim}
\end{minipage}
\caption{(left) An ill-typed \ocaml program that should sum the elements of a
  list, with highlights indicating two possible blame assignments based on:
  %
  (1) the \hlTree{\ocaml} compiler; and
  %
  (2) the student's \hlSherrloc{fix}.
  %
  % The \underline{underlined} expressions are equally valid
  % locations to blame. The expression blamed by the \ocaml compiler
  % is in \textbf{bold}.
  %
  % FIXME: This bolding is ambiguous, because ``let rec'', ``match'' and
  % ``with'' are also bolded (by \\ecode)! You need to find another way to
  % highlight what ocaml is yelling about.
  %
  (right) The error reported by \ocaml.}
\label{fig:intro:sumList}
\end{figure}

Consider the \ocaml program in \autoref{fig:intro:sumList}, which is
supposed to sum the integers in a list.
%
This program was written by an undergraduate student at UC San Diego,
and works as follows.
%
In functional languages like \ocaml, lists are recursively defined as
either the empty list (written |[]| and pronounced ``nil''), or a single
element |h| followed by the rest of the list |t| (written |h::t| and
pronounced ``h cons t'')\footnote{These variable names are conventional,
  \texttt{h} stands for ``head'' and \texttt{t} for ``tail''.}.
%
Given an input |xs|, the student |match|es it against the
two forms a list can take.
%
In the |[]| case she returns another empty list |[]|, and in the
|h::t| case she adds |h| to the recursive sum of |t|.

The observant reader will notice that this program is incorrect, given
\emph{any} non-empty list of integers, the addition on line 4 will
attempt to add an integer to |[]|, which is an invalid operation.
%
In fact, the program is \emph{ill-typed} and the \ocaml compiler rejects
it with the error message in \autoref{fig:intro:sumList}.
%
Unfortunately, \ocaml's error \emph{blames} the recursive call to
|sumList|, explaining that |sumList| returns a |list|, while the |+|
operator requires an |int|.
%
The real error is on line 3, where the student returns |[]| rather than
|0| as the sum of an empty list.

As we will see throughout the rest of this chapter, this rather simple
program is sufficient to illustrate many of the difficulties of
automatically locating the source of a type error, and explaining it to
the programmer.

\begin{figure}[t]
\centering
\[
\boxed{
\begin{array}{rcl}
e & ::=    & x \spmid \efun{x}{e} \spmid \eapp{e}{e} \spmid \elet{x}{e}{e} \\
  & \spmid & n \spmid \eplus{e}{e}\\
  % & \spmid & b \spmid \eif{e}{e}{e} \\
  % & \spmid & \epair{e}{e} \spmid \epcase{e}{x}{x}{e} \\
  & \spmid & \enil \spmid \econs{e}{e} \spmid \elcase{e}{e}{x}{x}{e} \\[0.05in]

n & ::= &  0, 1, -1, \ldots \\[0.05in]

% b & ::= &  \etrue \spmid \efalse \\[0.05in]

t & ::= & \alpha \spmid % \tbool \spmid
          \tint \spmid \tarr{t}{t} \spmid % \tprod{t}{t} \spmid
          \tlist{t} \\[0.05in]
\end{array}
}
\]
\caption{A simple $\lambda$-calculus with integers and lists.}
\label{fig:intro:simple-lambda}
\end{figure}



\chapter{A Dataset of Novice Type Errors}
\label{chp:data-collection}
In this chapter we describe a collection of novice interactions with the
\ocaml compiler --- including, importantly, type errors --- that we
gathered at UC San Diego over two quarters of the undergraduate CSE 130
course (IRB \#140608).
%
We have made the anonymized data publicly
available~\citep{Seidel2017-ko}, and hope that other researchers will
find it as valuable as we have.

The CSE 130 course is an upper-level (\ie generally consisting of third-
and fourth-year students) course that introduces students to typed
functional languages, specifically \ocaml.
%
For most students, this course will be their first exposure to both
functional programming and type systems with global inference.
%
We generally spend the first five weeks covering basic functional
programming in \ocaml, and then spend the last five weeks in
\textsc{Scala} covering more advanced concepts like traits and
monads (in the guise of |for-yield| comprehensions).
%
In the \ocaml portion of the course we cover standard functional idioms
like (tail-) recursion, higher-order functions, and user-defined
algebraic datatypes.

We recruited students from two instances of the course, Spring 2014
(\SPRING) and Fall 2015 (\FALL), to use an instrumented version of the
\ocamltop\footnote{\url{https://www.typerex.org/ocaml-top.html}} editor, which logged each of their
interactions with the \ocaml top-level system.
%
46 students from the \SPRING quarter and 56 students from the \FALL
quarter participated, for a total of 102 participants.
%
The participants used our instrumented editor to complete the first
three programming assignments, which involved writing 23 \ocaml
programs.
%

\begin{figure}
\centering
\includegraphics[width=\linewidth]{ocaml-top.png}
\caption{The \ocamltop editor.}
\label{fig:intro:ocaml-top}
\end{figure}

\autoref{fig:intro:ocaml-top} shows the main interface of \ocamltop,
which contains an editor pane on the left and an instance of the \ocaml
top-level interpreter on the right.
%
The students interact with the top-level system by selecting text in the
editor and pressing the ``play'' button in the toolbar, which sends the
selected text to the interpreter for evaluation.
%
The editor also maintains a cursor into the open file to track how much
of the file has been evaluated; this allows it to intelligently send the
next top-level definition to the interpreter if no text is selected.
%
The toolbar also has a ``stop'' button to abort evaluation (\eg to abort
infinite loops), a ``rewind'' button to restart the interpreter, and a
``fast-forward'' button to load the entire file into the interpreter.
%
\ocamltop always sends each definition to the interpreter individually,
even when using the ``fast-forward'' button, this will become important
when we extract ill-typed programs from the interaction traces.

We instrumented \ocamltop to record each of the student's interactions
with the top-level interpreter.
%
Specifically, we modified it to log an event each time a student pressed
one of the four interaction buttons on the toolbar (or used the
equivalent keyboard shortcuts).
%
This gives rise to three kinds of interaction events:
%
\begin{description}
\item[\textsc{Eval}] The student sent one or more definitions to be
  evaluated by the interpreter, by pressing either ``play'' or
  ``fast-forward''. In addition to logging the event, we logged both the
  offsets into the file that mark the region of text that was evaluated,
  and the list of top-level definitions that were evaluated.
\item[\textsc{Abort}] The student aborted evaluation by pressing
  ``stop''.
\item[\textsc{Stop}] The student restarted the interpreter by pressing
  ``rewind''.
\end{description}
%
For each event we also recorded the filename to identify the homework
the student was working on, the current UNIX timestamp, the entire body
of the file, and the offset of the cursor into the file.
%
Thus, for each participating student we can see and replay, with fine
granularity, the steps they took to solve the programming assignments.

We then post-processed the interaction traces to add the \ocaml
interpreter's responses to the students' submissions.
%
Recall that \ocamltop always sends single top-level definitions to the
interpreter, which maintains an environment of defined types and
functions.
%
This is inconvenient for extracting ill-typed programs, as the vast
majority of submitted definitions will depend on other definitions
that were submitted previously.
%
Thus, we modified the \ocaml interpreter to track dependencies between
top-level function and type definitions, so that for each definition a
student submitted, we could produce a self-contained, \emph{minimal}
program that would have the same behavior.

For each submitted definition, we collected the interpreter's response
and the minimal self-contained program, and classified the response as
either a syntax error, a scoping error (\eg an unbound variable), a type
error, or no error.
%
As we are primarily concerned with type errors in this work, we did not
capture the actual result of evaluating the definition, \ie the
resulting value, but it would be easy to extend the replay procedure to
do so.
%
We then stored the post-processed interaction traces as sequences of
JSON objects, in the format described by
\autoref{fig:intro:data-format}.
%
\begin{figure}
\centering
\begin{lstlisting}
{
    "file": "hw1.ml" | "hw2.ml" | "hw3.ml",
    "time": number,
    "body": string,
    "cursor": number,
    "event": {
        "type": "abort" | "eval" | "stop",
        "region": {
            "start": number,
            "stop": number
        }
    },
    "ocaml": [{
        "in": string,
        "out": string,
        "type": "scope" | "syntax" | "type" | "",
        "min": string
    }]
}
\end{lstlisting}
\caption{Format of the post-processed interaction events as JSON
  objects.}
\label{fig:intro:data-format}
\end{figure}
%
From this format it is quite convenient to run various analyses, \eg
what are the hardest assignments (measured by time spent or by errors
encountered), what is the relative frequency of various errors, \etc,
though for this work we will only use the dataset as a source of type
errors and fixes.

%% TODO: weird, this doesnt seem to be a problem anymore??
% In retrospect, it is not clear that the evaluation model used by
% \ocamltop, \ie loading individual definitions rather than the entire
% file, is the best model for students learning the language.
% %
% This model can lead to very subtle issues where the state of the program
% as represented by the file (and, presumably, the state according to the
% student's mind) diverges from the state of the interpreter.
% %
% Consider the following, suppose the student defines her own type for
% lists of integers and then defines a simple function to sum the integers
% of a list.
% %
% \begin{code}
%   type intlist = Nil | Cons of int * intlist

%   let rec sum xs = match xs with
%     | Nil         -> 0
%     | Cons (h, t) -> h + sum t
% \end{code}
% %
% The \ocaml interpreter will accept her definitions, having inferred that
% |sum| must have type |intlist -> int|.
% %
% But suppose now she decides to make |intlist| polymorphic in the
% element, changing the |intlist| definition to
% %
% \begin{code}
%   type 'a intlist = Nil | Cons of 'a * 'a intlist
% \end{code}
% %
% Then she tries to call the |sum| function.
% %


%\part{Witnessing Type Errors}
\chapter{Dynamic Witnesses for Static Type Errors}
\label{chp:nanomaly}
\lstMakeShortInline{@}
\newcommand\toolname{\tool{NanoMaLy}}
\section{Introduction}\label{sec:intro}

Should the programmer spend her time writing \emph{better types}
or \emph{thorough tests}?  
%
Types have long been the most pervasive means of describing the 
intended behavior of code. However, a type signature is often a 
very coarse description; the actual inputs and outputs
may be a subset of the values described by the types. 
%
For example, the set of ordered integer lists is a very 
sparse subset of the set of all integer lists. 
%
Thus, to validate functions that produce or consume such values, 
the programmer must painstakingly enumerate these values by hand 
or via ad-hoc generators for unit tests.

We present a new technique called \emph{type targeted testing}, 
abbreviated to \toolname, that enables the generation of unit
tests from precise \emph{refinement types}.
%
Over the last decade, various groups have shown how refinement 
types -- which compose the usual types with logical refinement predicates
that characterize the subset of actual type inhabitants -- 
can be used to specify and formally verify a wide variety 
of correctness properties of programs~\cite{pfenningxi98,Dunfield07,fstar,VazouICFP14}.
%
Our insight is that through the lens of SMT
solvers, refinement types can be viewed as a high-level, 
declarative, test generation technique.

\toolname tests an implementation function against a refinement 
type specification using a \emph{query-decode-check} loop.
%
% NV: input implies the types TARGET gets as input
First, \toolname translates the argument types into a logical
\emph{query} for which we obtain a satisfying assignment 
(or model) from the SMT solver.
%
Next, \toolname \emph{decodes} the SMT solver's model to obtain
concrete input values for the function.
%
Finally, \toolname executes the function on the inputs 
to get the corresponding output, which we \emph{check} 
belongs to the specified result type. 
%
If the check fails, the inputs are returned as a counterexample, 
otherwise
%
\toolname refutes the given model to force the SMT solver to 
return a different set of inputs. 
%
This process is repeated for a given number of 
iterations, or until \emph{all} inputs up to a certain size 
have been tested.

% Vs. Testing
\toolname offers several benefits over other testing techniques.
%
Refinement types provide a succinct description of the 
input and output requirements, eliminating the need to 
enumerate individual test cases by hand or to write 
custom generators.
%
Furthermore, \toolname generates \emph{all} 
values (up to a given size) that inhabit a type, and thus
does not skip any corner cases that a hand-written generator 
might miss.
%
Finally, while the above advantages can be recovered by a brute-force
generate-and-filter approach that discards inputs that do not meet
some predicate, we show that our SMT-based method can be significantly
more efficient for enumerating valid inputs in a highly-constrained space.
% and hence, sparse space.

% Vs. Verification 
\toolname paves a \emph{gradual path} from testing to verification, 
that affords several advantages over verification.
%
First, the programmer has an \emph{incentive} to write formal 
specifications using refinement types. \toolname provides the 
immediate gratification of an automatically generated, 
exhaustive suite of unit tests that can expose errors.
Thus, the programmer is rewarded without paying, up front, 
the extra price of annotations, hints, strengthened 
inductive invariants, or tactics needed for formally 
verifying the specification.
%
Second, our approach makes it possible to use refinement 
types to formally verify \emph{some} parts of the program, 
while using tests to validate other parts that may
be too difficult to verify
%
\toolname integrates the two modes by using refinement
types as the uniform specification mechanism. 
Functions in the verified half can be formally checked 
\emph{assuming} the functions in the tested half adhere 
to their specifications. 
We could even use refinements to generate dynamic 
contracts~\cite{Findler01} around the tested half 
if so desired.
%
Third, even when formally verifying the type specifications, 
the generated tests can act as valuable \emph{counterexamples} 
to help \emph{debug} the specification or implementation in 
the event that the program is rejected by the verifier.

% Vs. SymEx 
Finally, \toolname offers several concrete advantages over previous
property-based testing techniques, which also have the potential for 
gradual verification.
%
First, instead of specifying properties with arbitrary code 
\cite{claessen_quickcheck:_2000,runciman_smallcheck_2008} 
which complicates the task of subsequent formal verification, 
with \toolname the properties are specified via refinement 
types, for which there are already several existing formal 
verification algorithms~\cite{VazouICFP14}.
%
Second, while symbolic execution tools~\cite{DART,CUTE,Veanes08} 
can generate tests from arbitrary code contracts (\eg assertions) 
we find that highly constrained inputs trigger path explosion 
which precludes the use of such tools for gradual verification.

% In the rest of the paper...
In the rest of this paper, we start with an overview of 
how \toolname can be used and how its query-decode-check 
loop is implemented (\S~\ref{sec:overview}).
%
Next, we formalize a general framework for type-targeted 
testing (\S~\ref{sec:framework}) and show how it can be 
instantiated to generating tests for lists (\S~\ref{sec:list}), 
and then automatically generalized to other 
types (\S~\ref{sec:generic}).
%
All the benefits of \toolname come at a price; 
we are limited to properties that can be specified with 
refinement types. 
%
We present an empirical evaluation that shows
\toolname is efficient and expressive enough to capture 
a variety of sophisticated properties,
%
demonstrating that type-targeted 
testing is
a sweet spot between automatic testing 
and verification (\S~\ref{sec:evaluation}).

%%% Second, \toolname's symbolic, SMT-based approach makes it possible
%%% to systematically generate values that satisfy highly constrained 
%%% pre-conditions which otherwise thwart the generate-and-filter 
%%% approach of traditional property-based tools.
%%% %
%%% While this work shows a great deal of promise for making 
%%% formal verification practical, it does not obviate the 
%%% need for testing.
%%% %
%%% First, it can take a great deal of effort and expertise 
%%% to write down the annotations (beyond the end-to-end 
%%% specification) needed to verify a program 
%%% %
%%% Second, there maybe requirements (\eg performance) that 
%%% are not easily specified via refinements.
%%% ES INTRO We present \toolname, an automatic test-generator that uses SMT-solvers 
%%% ES INTRO to generate test-cases, taking advantange of their efficient decision 
%%% ES INTRO procedures to quickly prune the search space of all inputs. 
%%% ES INTRO 
%%% ES INTRO We observe based on prior work that many interesting properties of
%%% ES INTRO functions can be specified in an SMT-decidable logic, in our case the
%%% ES INTRO logic of linear arithetic, equality, and uninterpreted functions
%%% ES INTRO (\smtlogic).
%%% ES INTRO 
%%% ES INTRO Building on top of the \liquidhaskell program verification
%%% ES INTRO tool~\cite{VazouRealWorld14}, we show how a single data generator can
%%% ES INTRO be used to produce values of a type that satisfy disparate
%%% ES INTRO constraints, thereby easing the tedium of testing
%%% ES INTRO (\S~\ref{sec:liquidcheck}). 
%%% ES INTRO 
%%% ES INTRO We further show how to use a single generator to generically produce
%%% ES INTRO values of \emph{any} algebraic datatype (\S~\ref{sec:generic-generation}). 
%%% ES INTRO 
%%% ES INTRO Finally, we note that due to our shared specification language with
%%% ES INTRO \liquidhaskell, our properties are immediately ammenable to formal
%%% ES INTRO verification, should the developer wish to invest the extra time
%%% ES INTRO (\S~\ref{sec:discussion}).

\section{Overview}\label{sec:overview}

We start with a series of examples pertaining to a small grading
library called @Scores@. The examples provide a bird's eye view of 
how a user interacts with \toolname, how \toolname is implemented,
and the advantages of type-based testing.

\mypara{Refinement Types}
A refinement type is one where the basic types are decorated 
with logical predicates drawn from an efficiently decidable 
theory. For example,
%
\begin{code}
  type Nat   = {v:Int | 0 <= v}
  type Pos   = {v:Int | 0 <  v}
  type Rng N = {v:Int | 0 <= v && v <  N}
\end{code}
%
are refinement types describing the set of integers that are 
non-negative, strictly positive, and in the interval @[0, N)@ 
respectively. We will also build up function and collection 
types over base refinement types like the above. 
%
In this paper, we will not address the issue of \emph{checking}
refinement type signatures~\cite{VazouICFP14}.
%
We assume the code is typechecked, \eg by GHC, against the 
standard type signatures obtained by erasing the refinements.
Instead, we focus on using the refinements to 
synthesize tests to \emph{execute} the function, and to find 
\emph{counterexamples} that violate %demonstrate the function does not meet
the given specification.

\subsection{Testing with Types}

\mypara{Base Types}
Let us write a function @rescale@ that takes a source range @[0,r1)@, 
a target range @[0,r2)@, and a score @n@ from the source range,
and returns the linearly scaled score in the target range.
%
For example, @rescale 5 100 2@ should return @40@. 
Here is a first attempt at @rescale@ 
%
\begin{code}
  rescale :: r1:Nat -> r2:Nat -> s:Rng r1 -> Rng r2 
  rescale r1 r2 s = s * (r2 `div` r1)   
\end{code}
%
When we run \toolname, it immediately reports 
%
\begin{code}
  Found counter-example: (1, 0, 0) 
\end{code}
%
Indeed, @rescale 1 0 0@ results in @0@ which is not in the target 
@Rng 0@, as the latter is empty! We could fix this in various ways, 
\eg by requiring the ranges are non-empty:
%
\begin{code}
  rescale :: r1:Pos -> r2:Pos -> s:Rng r1 -> Rng r2 
\end{code}
%
Now, \toolname accepts the function and reports
%
\begin{code}
  OK. Passed all tests.
\end{code}
%
Thus, using the refinement type \emph{specification} for @rescale@, 
\toolname systematically tests the \emph{implementation} by generating 
all valid inputs (up to a given size bound) that respect the 
pre-conditions, running the function, and checking that the 
output satisfies the post-condition.
%
Testing against random, unconstrained inputs would be of limited value 
as the function is not designed to work on all @Int@ values. While in 
this case we could filter invalid inputs, we shall show
that \toolname can be more effective.

\mypara{Containers}
Let us suppose we have normalized all scores to be out of @100@
%
\begin{code}
  type Score = Rng 100
\end{code}
%
Next, let us write a function to compute a \emph{weighted} average 
of a list of scores.
%
\begin{code}
  average     :: [(Int, Score)] -> Score
  average []  = 0
  average wxs = total `div` n
    where
      total   = sum [w * x | (w, x) <- wxs ]
      n       = sum [w     | (w, _) <- wxs ]
\end{code}
%
It can be tricky to \emph{verify} this function as it requires non-linear reasoning
about an unbounded collection. However, we can gain a great degree of confidence by
systematically testing it using the type specification; indeed, \toolname responds:
%
\begin{code}
  Found counter-example: [(0,0)]
\end{code}
%
Clearly, an unfortunate choice of weights can trigger a divide-by-zero; we can fix 
this by requiring the weights be non-zero:
%
\begin{code}
  average :: [({v:Int | v /= 0}, Score)] -> Score
\end{code}
%
but now \toolname responds with
%
\begin{code}
  Found counter-example: [(-3,3),(3,0)]
\end{code}
% 
which also triggers the divide-by-zero! We will play it safe and require positive weights,
%
\begin{code}
  average :: [(Pos, Score)] -> Score
\end{code}
%
at which point \toolname reports that all tests pass.

\mypara{Ordered Containers}
The very nature of our business requires that at the end of the day,
we order students by their scores. We can represent ordered lists by 
requiring the elements of the tail @t@ to be greater than the head @h@:
%
\begin{code}
data OrdList a = [] | (:) {h :: a, t :: OrdList {v:a | h <= v}}
\end{code}
%
Note that erasing the refinement predicates gives us plain old Haskell lists.
We can now write a function to insert a score into an ordered list:
%
\begin{code}
  insert :: (Ord a) => a -> OrdList a -> OrdList a 
\end{code}
%
\toolname automatically generates all ordered lists (up to a given size)
and executes @insert@ to check for any errors. Unlike randomized testers, 
\toolname is not thwarted by the ordering constraint, and does not require a
custom generator from the user.

\mypara{Structured Containers} 
Everyone has a few bad days. Let us write a function that takes the 
@best k@ scores for a particular student. That is, the output
must satisfy a \emph{structural} constraint -- that its size 
equals @k@. We can encode the size of a list with a logical 
measure function~\cite{VazouICFP14}:
%
\begin{code}
  measure len :: [a] -> Nat
  len []      = 0
  len (x:xs)  = 1 + len xs
\end{code}
%
Now, we can stipulate that the output indeed has @k@ scores:
%
\begin{code}
  best      :: k:Nat -> [Score] -> {v:[Score] | k = len v}
  best k xs = take k $ reverse $ sort xs
\end{code}
%
Now, \toolname quickly finds a counterexample:
%
\begin{code}
  Found counter-example: (2,[])
\end{code}
%
Of course -- we need to have at least @k@ scores to start with! 
%
\begin{code}
best :: k:Nat -> {v:[Score]|k <= len v} -> {v:[Score]|k = len v}
\end{code}
%
and now, \toolname is assuaged and reports no counterexamples.
%
While randomized testing would suffice for @best@, we will see 
more sophisticated structural properties such as height balancedness, 
which stymie random testers, but are easily handled by \toolname.

\mypara{Higher-order Functions} 
Perhaps instead of taking the $k$ best grades, we would like
to pad each individual grade, and, furthermore, we want to
be able to experiment with different padding functions. Let
us rewrite @average@ to take a functional argument, and
stipulate that it can only increase a @Score@.
%
\begin{code}
  padAverage       :: (s:Score -> {v:Score | s <= v}) 
                   -> [(Pos, Score)] -> Score
  padAverage f []  = f 0
  padAverage f wxs = total `div` n
    where
      total   = sum [w * f x | (w, x) <- wxs ]
      n       = sum [w       | (w, _) <- wxs ]
\end{code}
%
\toolname automatically checks that @padAverage@ is 
a safe generalization of @average@. Randomized 
testing tools can also generate functions, but those 
functions are unlikely to satisfy non-trivial constraints, 
thereby burdening the user with custom generators.


\subsection{Synthesizing Tests} 
\label{sec:synthesizing-tests}
Next, let us look under the hood to get an idea of how \toolname 
synthesizes tests from types. 
% INTRO
At a high-level, our strategy is to:
%
(1)~\emph{query}   an SMT solver for satisfying assigments to a set of logical 
                   constraints derived from the refinement type,
(2)~\emph{decode}  the model into Haskell values that are suitable inputs,
(3)~\emph{execute} the function on the decoded values to obtain the output, 
(4)~\emph{check}   that the output satisfies the output type,
(5)~\emph{refute}  the model to generate a different test, and 
repeat the above steps until all tests up to a certain size are executed.
%
We focus here on steps 1, 2, and 4 -- query, decode, and check -- the others are 
standard and require little explanation.

\mypara{Base Types}
Recall the initial (buggy) specification
%
\begin{code}
  rescale :: r1:Nat -> r2:Nat -> s:Rng r1 -> Rng r2 
\end{code}
%
\toolname \emph{encodes} input requirements for base types directly 
from their corresponding refinements. The constraints for multiple, 
related inputs are just the \emph{conjunction} of the constraints 
for each input. Hence, the constraint for @rescale@ is:
%
$$
\cstr{C_0} \defeq 0 \leq \cvar{r1} \wedge 0 \leq \cvar{r2} \wedge 0 \leq s < \cvar{r1} 
$$
%
In practice, $\cstr{C_0}$ will also contain conjuncts of the form $-N \leq x \leq N$ that
restrict @Int@-valued variables $x$ to be within the size bound $N$ supplied by
the user, but we will omit these throughout the paper for clarity.
%% %
%% For clarity, we omit the conjuncts of the form $-N \leq x \leq N$
%% that restrict @Int@-valued variables $x$ to be within the size
%% bound $N$ supplied by the user.

Note how easy it is to capture dependencies between inputs, 
\eg that the score @s@ be in the range defined by @r1@.
%
On querying the SMT solver with the above, we get a model
%
$[\cvar{r1} \mapsto 1, \cvar{r2} \mapsto 1, \cvar{s}  \mapsto 0]$.
%
\toolname decodes this model and executes \hbox{@rescale 1 1 0@} to obtain the value @v = 0@.
%
Then, \toolname validates @v@ against the post-condition by checking 
% that it inhabits the output type, \ie by checking 
the validity of the output type's constraint: 
%
$$\cvar{r2} = 1 \wedge \cvar{v} = 0 \wedge 0 \leq \cvar{v} \wedge \cvar{v} < \cvar{r2}$$
%
As the above is valid, \toolname moves on to generate another 
test by conjoining $\cstr{C_0}$ with a constraint that refutes 
the previous model:
%
$$
\cstr{C_1} \defeq \cstr{C_0} \wedge (\cvar{r1} \not = 1 \vee \cvar{r2} \not = 1 \vee \cvar{s} \not = 0)
$$
This time, the SMT solver returns a model: 
%
$[\cvar{r1} \mapsto 1, \cvar{r2} \mapsto 0, \cvar{s} \mapsto 0]$
%
which, when decoded and executed, yields the result $0$ that does \emph{not} 
inhabit the output type, and so is reported as a counterexample. 
%
When we fix the specification to only allow @Pos@ ranges, each test produces
a valid output, so \toolname reports that all tests pass.

\mypara{Containers}
Next, we use \toolname to test the implementation of @average@.
To do so, \toolname needs to generate Haskell lists with the appropriate constraints.
%
Since each list is recursively 
either ``nil'' 
or ``cons'', 
\toolname generates constraints that symbolically 
represent \emph{all} possible lists up to a given depth, 
using propositional \emph{choice variables} to 
symbolically pick between these two alternatives.
%
Every (satisfying) assignment of choices returned by 
the SMT solver gives \toolname the concrete data and 
constructors used at each level, allowing it to decode 
the assignment into a Haskell value.

For example, \toolname represents valid @[(Pos, Score)]@ 
inputs (of depth up to 3), required to test @average@, 
as the conjunction of $\cstr{C_{list}}$ and $\cstr{C_{data}}$:
%
\begin{eqnarray*}
\cstr{C_{list}} & \defeq & (\cvar{c}_{00} \Rightarrow \cvar{xs}_0 = \lnil) \wedge 
                          (\cvar{c}_{01} \Rightarrow \cvar{xs}_0 = \lcons{\cvar{x}_1}{\cvar{xs}_1}) \wedge 
                          (\cvar{c}_{00} \oplus \cvar{c}_{01}) \\
               & \wedge & (\cvar{c}_{10} \Rightarrow \cvar{xs}_1 = \lnil) \wedge
                          (\cvar{c}_{11} \Rightarrow \cvar{xs}_1 = \lcons{\cvar{x}_2}{\cvar{xs}_2}) \wedge 
                          (\cvar{c}_{01} \Rightarrow \cvar{c}_{10} \oplus \cvar{c}_{11}) \\
               & \wedge & (\cvar{c}_{20} \Rightarrow \cvar{xs}_2 = \lnil) \wedge 
                          (\cvar{c}_{21} \Rightarrow \cvar{xs}_2 = \lcons{\cvar{x}_3}{\cvar{xs}_3}) \wedge 
                          (\cvar{c}_{11} \Rightarrow \cvar{c}_{20} \oplus \cvar{c}_{21}) \\
               & \wedge & (\cvar{c}_{30} \Rightarrow \cvar{xs}_3 = \lnil) \wedge 
                          (\cvar{c}_{21} \Rightarrow \cvar{c}_{30}) \\[0.1in]
\cstr{C_{data}} & \defeq & (\cvar{c}_{01} \Rightarrow \cvar{x}_1 = \ltup{\cvar{w}_1}{\cvar{s}_1} \ \wedge\ 0 < \cvar{w}_1 \ \wedge\ 0 \leq \cvar{s}_1 < 100) \\
               & \wedge & (\cvar{c}_{11} \Rightarrow \cvar{x}_2 = \ltup{\cvar{w}_2}{\cvar{s}_2} \ \wedge\ 0 < \cvar{w}_2 \ \wedge\ 0 \leq \cvar{s}_2 < 100) \\
               & \wedge & (\cvar{c}_{21} \Rightarrow \cvar{x}_3 = \ltup{\cvar{w}_3}{\cvar{s}_3} \ \wedge\ 0 < \cvar{w}_3 \ \wedge\ 0 \leq \cvar{s}_3 < 100)
\end{eqnarray*}
%
The first set of constraints $\cstr{C_{list}}$ describes all lists up to 
size 3. At each level $i$, the \emph{choice} variables $\cvar{c}_{i0}$ 
and $\cvar{c}_{i1}$ determine whether at that level the constructed 
list $\cvar{xs}_i$ is a ``nil'' or a ``cons''. 
%
In the constraints $\lnil$ and $(\lcons{}{})$ are \emph{uninterpreted} 
functions that represent ``nil'' and ``cons'' respectively. 
These functions only obey the congruence axiom and hence, can be 
efficiently analyzed by SMT solvers~\cite{Nelson81}.
%
The data at each level $\cvar{x}_i$ is constrained to be a pair of a 
positive weight $\cvar{w}_i$ and a valid score $\cvar{s}_i$.

The choice variables at each level are used to \emph{guard} the 
constraints on the next levels. 
%
First, if we are generating a ``cons'' at a given level, then
exactly one of the choice variables for the next level must be 
selected;
\eg  $\cvar{c}_{11} \Rightarrow \cvar{c}_{20} \oplus \cvar{c}_{21}$.
%
Second, the constraints on the data at a given level only hold 
if we are generating values for that level; \eg $\cvar{c}_{21}$ 
is used to guard the constraints on $\cvar{x}_3$, $\cvar{w}_3$ 
and $\cvar{s}_3$.
%
This is essential to avoid over-constraining the system 
which would cause \toolname to miss certain tests.

To \emph{decode} a model of the above into a Haskell value of type @[(Int, Int)]@,
we traverse constraints and use the valuations of the choice variables to 
build up the list appropriately.
%
At each level, if $\cvar{c}_{i0} \mapsto \ttrue$, then the list at that 
level is @[]@, otherwise $\cvar{c}_{i1} \mapsto \ttrue$ and we decode 
$\cvar{x}_{i+1}$ and $\cvar{xs}_{i+1}$ and ``cons'' the results.

We can iteratively generate \emph{multiple} inputs by adding a constraint that
refutes each prior model. As an important optimization, we only refute the
relevant parts of the model, \ie those needed to construct the list
(\S~\ref{sec:refute}).

\mypara{Ordered Containers}
%
Next, let us see how \toolname enables automatic testing with 
highly constrained inputs, such as the \emph{increasingly ordered} 
@OrdList@ values required by @insert@.
%
From the type definition, it is apparent that ordered
lists are the same as the usual lists described by
$\cstr{C_{list}}$, except that each unfolded \emph{tail} 
must only contain values that are greater than the 
corresponding \emph{head}.
%
That is, as we unfold @x1:x2:xs :: OrdList@ 
%
\begin{itemize}
\item At level @0@, we have @OrdList {v:Score| true}@
\item At level @1@, we have @OrdList {v:Score| x1 <= v}@
\item At level @2@, we have @OrdList {v:Score| x2 <= v && x1 <= v}@
\end{itemize}
%
and so on. Thus, we encode @OrdList Score@ (of depth up to 3) by
conjoining $\cstr{C_{list}}$ with  $\cstr{C_{score}}$ and $\cstr{C_{ord}}$,
which capture the valid score and ordering requirements respectively:
%
\begin{eqnarray*}
\cstr{C_{ord}}   & \defeq & (\cvar{c}_{11} \Rightarrow \cvar{x}_1 \leq \cvar{x}_2)
                \ \wedge\  (\cvar{c}_{21} \Rightarrow \cvar{x}_2 \leq \cvar{x}_3\ \wedge\ \cvar{x}_1 \leq \cvar{x}_3) \\[0.01in]
\cstr{C_{score}} & \defeq & (\cvar{c}_{01} \Rightarrow 0 \leq \cvar{x}_1 < 100)
                \ \wedge\  (\cvar{c}_{11} \Rightarrow 0 \leq \cvar{x}_2 < 100)
                \ \wedge\  (\cvar{c}_{21} \Rightarrow 0 \leq \cvar{x}_3 < 100)
\end{eqnarray*}

\mypara{Structured Containers}
Recall that @best k@ requires inputs whose \emph{structure} is constrained -- the 
size of the list should be no less than @k@. We specify size using special measure 
functions~\cite{VazouICFP14}, which let us relate the size of a list with that of
its unfolding, and hence, let us encode the notion of size inside the constraints:
%
\begin{eqnarray*}
\cstr{C_{size}} & \defeq & (\cvar{c}_{00} \Rightarrow \clen{\cvar{xs}_{0}} = 0) \wedge 
                          (\cvar{c}_{01} \Rightarrow \clen{\cvar{xs}_{0}} = 1 + \clen{\cvar{xs}_1}) \\
               & \wedge & (\cvar{c}_{10} \Rightarrow \clen{\cvar{xs}_{1}} = 0) \wedge 
                          (\cvar{c}_{11} \Rightarrow \clen{\cvar{xs}_{1}} = 1 + \clen{\cvar{xs}_2}) \\
               & \wedge & (\cvar{c}_{20} \Rightarrow \clen{\cvar{xs}_{2}} = 0) \wedge 
                          (\cvar{c}_{21} \Rightarrow \clen{\cvar{xs}_{2}} = 1 + \clen{\cvar{xs}_3}) \\
               & \wedge & (\cvar{c}_{30} \Rightarrow \clen{\cvar{xs}_{3}} = 0)
\end{eqnarray*}
%
At each unfolding, we instantiate the definition of the measure 
for each alternative of the datatype. 
%
In the constraints, $\clen{\cdot}$ is an uninterpreted function derived
from the measure definition. All of the relevant properties of the function
are spelled out by the unfolded constraints in $\cstr{C_{size}}$ and hence,
we can use SMT to search for models for the above constraint.
%
Hence, \toolname constrains the input type for @best@ as:
%
$$     0 \leq k 
\wedge \cstr{C_{list}} 
\wedge \cstr{C_{score}} 
\wedge \cstr{C_{size}} 
\wedge k \leq \clen{\cvar{xs}_0} $$
%
where the final conjunct comes from the top-level refinement that 
stipulates the input have at least @k@ scores.
%
Thus, \toolname only generates lists that are large enough. 
For example, in any model where $k = 2$, it will \emph{not} 
generate the empty or singleton list, as in those cases, 
$\clen{\cvar{xs}_0}$ would be $0$ (resp. $1$), violating the 
final conjunct above.

\mypara{Higher-order Functions}
Finally, \toolname's type-directed testing scales up to higher-order
functions using the same insight as in QuickCheck~\cite{claessen_quickcheck:_2000}, namely, 
to generate a function it suffices to be able to 
generate the \emph{output} of the function.
When tasked with the generation of a functional argument @f@, \toolname 
returns a Haskell function that when executed checks
whether its inputs satisfy @f@'s pre-conditions.
If they do, then @f@ uses \toolname to dynamically
query the SMT solver for an output that satisfies the 
constraints imposed by the concrete inputs.
Otherwise, @f@'s specifications are violated
and TARGET reports a counterexample.

This concludes our high-level tour of the benefits and 
implementation of \toolname. 
%
Notice that the property specification mechanism -- 
refinement types -- allowed us to get immediate feedback
that helped debug not just the code, but also the specification 
itself. 
%
Additionally, the specifications gave us machine-readable 
documentation about the behavior of functions, and a large 
unit test suite with which to automatically validate the 
implementation.
%
Finally, though we do not focus on it here, the specifications 
are amenable to formal verification should the programmer 
so desire.


%%% Local Variables:
%%% mode: latex
%%% TeX-master: "main"
%%% End:

\input{nanomaly/semantics}
\input{nanomaly/interactive}
\section{Evaluation} \label{sec:evaluation}

We have built a prototype implementation of \toolname\footnote{\url{http://hackage.haskell.org/package/target-0.1.1.0}} and next, 
describe an evaluation on a series of benchmarks ranging from 
textbook examples of algorithms and data structures to widely 
used Haskell libraries like \textsc{containers} and \textsc{xmonad}.
%
Our goal in this evaluation is two-fold. 
%
First, we describe micro-benchmarks (\ie functions)
that \emph{quantitatively compare} \toolname with 
the existing state-of-the-art, property-based testing
tools for Haskell -- namely \smallcheck and \quickcheck\ -- 
to determine whether \toolname is indeed able to generate
highly constrained inputs more effectively.
%
Second, we describe macro-benchmarks (\ie modules) that 
evaluate the amount of \emph{code coverage} that we 
get from type-targeted testing.
%
%% Third, using our results as a base, we present a 
%% qualitative discussion of \toolname as a \emph{gradual} 
%% approach that bridges informal and formal verification.

%% ES: This doesn't actually make any sense..
% An important optimization in our implementation is to 
% perform the post-condition checking in Haskell instead of by querying
% the SMT solver. We accomplish this by adding an additional @toReft@
% method to @Targetable@ that translates the concrete inputs that we
% @decode@ \emph{back} into logical expressions,
% %
% \begin{code}
%   toReft :: a -> Var
% \end{code}
% %
% which we then substitute into the output-type before checking the
% concrete output. @toReft@ can be simply implemented as:
% %
% \begin{code}
%   toReft v = app c (map toReft vs)
%     where (c, vs) = splitCtor v
%           -- app :: Ctor -> [Var] -> Var
% \end{code}
% %
% where @app@ is a pure version of @apply@, \ie it constructs a
% logical expression like @apply@ but does nothing else.

% \newcommand\XX{\multicolumn{1}{c}{X}}
% \newcommand{\mysec}[1]{\SI{#1}{\second}}
\begin{figure}[ht!]
  \centering
  % \includegraphics[width=0.49\linewidth]{figs/List-insert}
  % \includegraphics[width=0.49\linewidth]{figs/RBTree-add}
  % \includegraphics[width=0.49\linewidth]{figs/Map-delete}
  % \includegraphics[width=0.49\linewidth]{figs/Map-difference}
  % \includegraphics[width=0.49\linewidth]{figs/XMonad-focus-left}
  % \begin{tikzpicture}
  %    \begin{customlegend}[legend columns=4,legend style={align=center,draw=none},legend entries={\toolname,\smallcheck,\lazysmallcheck,\lazysmallcheck (slow)}]
  %    \addlegendimage{color=blue,mark=square*}
  %    \addlegendimage{color=red,mark=*}   
  %    \addlegendimage{color=orange,mark=diamond*}
  %    \addlegendimage{color=black,mark=x}
  %    \end{customlegend}
  % \end{tikzpicture}

  \begin{tikzpicture}
    \begin{groupplot}[
      group style = {group size = 3 by 1, horizontal sep=15pt,},
      groupplot ylabel={Time (sec)},
      groupplot xlabel={Depth},
      group/only outer labels,
      ymode=log,
      ymax=10000,
      ymin=0.0001
    ]
    % \begin{semilogyaxis}[
    \nextgroupplot[
      title=\textsc{List.insert}
    ]
    \addplot table[smooth,col sep=comma,x index=0,y index=1] {target/csv/List.insert.csv};
    \addplot table[smooth,col sep=comma,x index=0,y index=2] {target/csv/List.insert.csv};
    \addplot table[smooth,col sep=comma,x index=0,y index=3] {target/csv/List.insert.csv};
    % \end{semilogyaxis}
  % \end{tikzpicture}
  % \begin{tikzpicture}
    % \begin{semilogyaxis}[
    \nextgroupplot[
      title=\textsc{RBTree.add},
      legend columns=4,
      legend entries={\toolname,\smallcheck,\lazysmallcheck,\lazysmallcheck (slow)},
      legend to name=legend,
    ]
    \addplot table[smooth,col sep=comma,x index=0,y index=1] {target/csv/RBTree.add.csv};
    \addplot table[smooth,col sep=comma,x index=0,y index=2] {target/csv/RBTree.add.csv};
    \addplot table[smooth,col sep=comma,x index=0,y index=3] {target/csv/RBTree.add.csv};
    \addplot table[smooth,col sep=comma,x index=0,y index=4] {target/csv/RBTree.add.csv};
    % \end{semilogyaxis}
  % \end{tikzpicture}
  % \begin{tikzpicture}
    % \begin{semilogyaxis}[
    \nextgroupplot[
      title=\textsc{XMonad.focus\_left}
    ]
    \addplot table[smooth,col sep=comma,x index=0,y index=1] {target/csv/XMonad.focus_left.csv};
    \addplot table[smooth,col sep=comma,x index=0,y index=2] {target/csv/XMonad.focus_left.csv};
    \addplot table[smooth,col sep=comma,x index=0,y index=3] {target/csv/XMonad.focus_left.csv};
    % \end{semilogyaxis}
    \end{groupplot}
  \end{tikzpicture}
  \begin{tikzpicture}
    \begin{groupplot}[
      group style = {group size = 2 by 1, horizontal sep=15pt,},
      groupplot ylabel={Time (sec)},
      groupplot xlabel={Depth},
      group/only outer labels,
      ymode=log,
      ymax=10000,
      ymin=0.0001
    ]
    % \begin{semilogyaxis}[
    \nextgroupplot[
      title=\textsc{Map.delete}
    ]
    \addplot table[smooth,col sep=comma,x index=0,y index=1] {target/csv/Map.delete.csv};
    \addplot table[smooth,col sep=comma,x index=0,y index=2] {target/csv/Map.delete.csv};
    \addplot table[smooth,col sep=comma,x index=0,y index=3] {target/csv/Map.delete.csv};
    \addplot table[smooth,col sep=comma,x index=0,y index=4] {target/csv/Map.delete.csv};
    % \end{semilogyaxis}
  % \end{tikzpicture}
  % \begin{tikzpicture}
    % \begin{semilogyaxis}[
    \nextgroupplot[
      title=\textsc{Map.difference}
    ]
    \addplot table[smooth,col sep=comma,x index=0,y index=1] {target/csv/Map.difference.csv};
    \addplot table[smooth,col sep=comma,x index=0,y index=2] {target/csv/Map.difference.csv};
    \addplot table[smooth,col sep=comma,x index=0,y index=3] {target/csv/Map.difference.csv};
    \addplot table[smooth,col sep=comma,x index=0,y index=4] {target/csv/Map.difference.csv};
    % \end{semilogyaxis}
    \end{groupplot}
  \end{tikzpicture}\\
  \ref{legend}

  \caption{Results of comparing \toolname with \quickcheck, \smallcheck, and Lazy
    \smallcheck on a series of functions. \toolname, \smallcheck, and Lazy
    \smallcheck were both configured to check the first 1000 inputs that
    satisfied the precondition at increasing depth parameters, with a 60 minute
    timeout per depth; \quickcheck was run with the default settings, \ie it had
    to produce 100 test cases. \toolname, \smallcheck, and \lazysmallcheck were
    configured to use the same notion of depth, in order to ensure they would
    generate the same number of valid inputs at each depth level. \quickcheck was
    unable to successfully complete any run due to the low probability of
    generating valid inputs at random.}\label{fig:comparisonresults}
\end{figure}



\subsection{Comparison with \quickcheck and \smallcheck}\label{sec:comparison}

We compare \toolname with \quickcheck and \smallcheck by using 
a set of benchmarks with highly constrained inputs. 
%
For each benchmark we compared \toolname with \smallcheck and
\quickcheck, with the latter two using the generate-and-filter 
approach, wherein a value is generated and subsequently discarded if
it does not meet the desired constraint.
%
While one could possibly write custom ``operational'' generators 
for each property, the point of this evaluation is compare the 
different approaches ability to enable ``declarative'' specification 
driven testing.
%
Next, we describe the benchmarks and then summarize the results of the comparison
(Figure~\ref{fig:comparisonresults}).



\mypara{Inserting into a sorted \List}
%
Our first benchmark is the \Insert function from the homonymous 
sorting routine. We use the specification that given an element 
and a sorted list, @insert x xs@ should evaluate to a sorted list.
We express this with the type
%
\begin{code}
  type Sorted a = List <{\hd v -> hd < v}> a
  insert :: a -> Sorted a -> Sorted a
\end{code}
%
where the ordering constraint is captured by an abstract 
refinement~\cite{Vazou13} which states that \emph{each} 
list head @hd@ is less than every element @v@ in its tail.

\mypara{Inserting into a Red-Black Tree}
%
Next, we consider insertion into a Red-Black tree.
%
\begin{code}
  data RBT a = Leaf  | Node Col a (RBT a) (RBT a)
  data Col   = Black | Red
\end{code}
%
Red-black trees must satisfy three invariants:
%
(1)~red nodes always have black children,
(2)~the black height of all paths from the root to a leaf is the same, and
(3)~the elements in the tree should be ordered.
%
We capture (1) via a measure that recursively checks each @Red@ node has @Black@ children.
%
\begin{code}
  measure isRB :: RBT a -> Prop
  isRB Leaf           = true
  isRB (Node c x l r) = isRB l && isRB r &&
                        (c == Red => isBlack l && isBlack r)
\end{code}
%
We specify (2) by defining the @Black@ height as:
%
\begin{code}
  measure bh :: RBT a -> Int
  bh Leaf           = 0
  bh (Node c x l r) = bh l + (if c == Red then 0 else 1)
\end{code}
%
and then checking that the @Black@ height of both subtrees is the same:
%
\begin{code}
  measure isBH :: RBT a -> Prop
  isBH Leaf           = true
  isBH (Node c x l r) = isBH l && isBH r && bh l == bh r
\end{code}
%
Finally, we specify the (3), the ordering invariant as:
%
\begin{code}
  type OrdRBT a = RBT <{\r v -> v < r}, {\r v -> r < v}> a
\end{code}
%
\ie with two abstract refinements for the left and right subtrees
respectively, which state that the root @r@ is greater than (resp. less than)
each element @v@ in the subtrees. Finally, a valid Red-Black tree is:
%
\begin{code}
  type OkRBT a = {v:OrdRBT a | isRB v && isBH v}
\end{code}
%
Note that while the specification for the \emph{internal} invariants for Red-Black
trees is tricky, the specification for the public API -- \eg the @add@ function -- 
is straightforward:
%
\begin{code}
  add :: a -> OkRBT a -> OkRBT a
\end{code}

\mypara{Deleting from a Data.Map}\label{sec:delete-from-map}
%
Our third benchmark is the @delete@ function from the \hbox{@Data.Map@} module in 
the Haskell standard libraries. The @Map@ structure is a balanced binary
search tree that implements purely functional key-value dictionaries:
%
\begin{code}
  data Map k a = Tip | Bin Int k a (Map k a) (Map k a)
\end{code}
%
A valid @Data.Map@ must satisfy two properties:
%
(1)~the size of the left and right sub-trees must be 
    within a factor of three of each other, and
(2)~the keys must obey a binary search ordering.
%
We specify the balancedness invariant~(1) with a measure
%
\begin{code}
  measure isBal :: Map k a -> Prop
  isBal (Tip)           = true
  isBal (Bin s k v l r) = isBal l && isBal r &&
                          (sz l + sz r <= 1 ||
                           sz l <= 3 * sz r <= 3 * sz l)
\end{code}
%
and combine it with an ordering invariant (like @OrdRBT@) to specify valid trees.
%
\begin{code}
  type OkMap k a = {v : OrdMap k a | isBal v}
\end{code}
%
We can check that @delete@ preserves the invariants by 
checking that its output is an @OkMap k a@.
However, we can also go one step further and check 
the functional correctness property that @delete@ 
removes the given key, with a type:
%
\begin{code}
  delete :: Ord k => k:k -> m:OkMap k a 
         -> {v:OkMap k a | MinusKey v m k}
\end{code}
%
where the predicate @MinusKey@ is defined as:
%
\begin{code}
  predicate MinusKey M1 M2 K 
    = keys M1 = difference (keys M2) (singleton K)
\end{code}
%
using the measure @keys@ describing the contents of the @Map@:
%
\begin{code}
  measure keys :: Map k a -> Set k
  keys (Tip)           = empty () 
  keys (Bin s k v l r) = union (singleton k) 
                               (union (keys l) (keys r))
\end{code}

\mypara{Refocusing XMonad StackSets} \label{sec:refocus-stackset}
%
Our last benchmark comes from the tiling window manager XMonad. 
%
The key invariant of XMonad's internal @StackSet@ data structure 
is that the elements (windows) must all be \emph{unique}, \ie contain
no duplicates.
%
XMonad comes with a test-suite of over 100 \quickcheck properties;
we select one which states that moving the focus between windows 
in a @StackSet@ should not affect the \emph{order} of the windows.
%
\begin{code}
  prop_focus_left_master n s =
    index (foldr (const focusUp) s [1..n]) == index s
\end{code}
%
With \quickcheck, the user writes a custom generator for valid @StackSet@s
and then runs the above function on test inputs created by the generator, 
to check if in each case, the result of the above is @True@.

With \toolname, it is possible to test such properties \emph{without} 
requiring custom generators. Instead the user writes a declarative 
specification:
%
\begin{code}
  type OkStackSet = {v:StackSet | NoDuplicates v}
\end{code}
%
(We refer the reader to~\cite{VazouRealWorld14} for a full 
discussion of how to specify @NoDuplicates@).
%
Next, we define a refinement type:
%
\begin{code}
  type TTrue = {v:Bool | Prop v}
\end{code}
%
that is only inhabited by @True@, and use it to type the \quickcheck 
property as:
%
\begin{code}
  prop_focus_left_master :: Nat -> OkStackSet -> TTrue 
\end{code}
%
This property is particularly difficult to \emph{verify}; however,
\toolname is able to automatically
generate valid inputs to \emph{test} that @prop_focus_left_master@
always returns @True@.

%%% The high level of abstraction inherent in the @StackSet@ definition
%%% works in our favor here, as we can instantiate the relevant type parameter (the
%%% window) to \Char and leave the others as @()@ to drastically reduce
%%% the search space.


\mypara{Results}
%
Figure~\ref{fig:comparisonresults} summarizes the results of the comparison.
%
\quickcheck was unable to successfully complete \emph{any} 
benchmark to the low probability of generating properly 
constrained values at random.

\begin{description}
\item[List Insert] \toolname is able to test @insert@ all the way to 
   depth 20, whereas \lazysmallcheck times out at depth 19.

\item[Red-Black Tree Insert] \toolname is able to test @add@ up to depth 12,
  while \lazysmallcheck times out at depth 6.
  
\item[Map Delete] \toolname is able to check @delete@ up to depth 10, whereas
   \lazysmallcheck times out at depth 7 if it checks ordering first,
    or depth 6 if it checks balancedness first.

\item[StackSet Refocus] \toolname and is able to check this property 
    up to depth 8, while \lazysmallcheck times out at depth 7.
\end{description}

\toolname sees a performance hit with properties 
that require reasoning with the theory of Sets \eg 
the no-duplicates invariant of @StackSet@. 
%
While \lazysmallcheck times out at a higher depths, when it completes
\eg at depth 6, it does so in 0.7s versus \toolname's 9 minutes.
%
We suspect this is because the theory of sets are a relatively recent
addition to SMT solvers \cite{arrayZ3}, and with further improvements 
in SMT technology, these numbers will get significantly better.


Overall, we found that for \emph{small inputs} \lazysmallcheck 
is substantially faster as exhaustive enumeration is tractable,
and does not incur the overhead of communicating with an external 
general-purpose solver.
%
Additionally, \lazysmallcheck benefits from pruning predicates 
that exploit laziness and only force a small portion of the 
structure (\eg ordering). 
%
However, we found that constraints that force the entire 
structure (\eg balancedness), or composing predicates in the 
wrong \emph{order}, can force \lazysmallcheck to enumerate 
the entire exponentially growing search space.

\toolname, on the other hand, scales nicely to larger input sizes,
allowing systematic and exhaustive testing of larger, more complex
inputs. This is because \toolname eschews \emph{explicit} 
enumeration-and-filtering (which results in searching for 
fewer needles in larger haystacks as the sizes increas), 
in favor of \emph{symbolically} searching for valid models 
via SMT, making \toolname robust to the strictness or ordering 
of constraints.



\subsection{Measuring Code Coverage}\label{sec:code-coverage}

The second question we seek to answer is whether \toolname is suitable for testing entire
libraries, \ie how much of the program can be automatically exercised using our
system? Keeping in mind the well-known issues with treating code coverage as an
indication of test-suite quality~\cite{marick1999misuse}, we
consider this experiment a negative filter.

To this end, we ran \toolname against the entire user-facing API of 
\hbox{@Data.Map@,} our @RBTree@ library, and @XMonad.StackSet@ -- using 
the constrained refined types (\eg @OkMap@, @OkRBT@, @OkStackSet@) as 
the specification for the exposed types -- and measured the expression 
and branch coverage, as reported by @hpc@~\cite{gill2007haskell}.
%
We used an increasing timeout ranging from one to thirty minutes
per exported function.

\mypara{Results}
%
The results of our experiments are shown in Figure~\ref{fig:coverage}. 
Across all three libraries, \toolname achieved at least 70\% expression 
and 64\% alternative coverage at the shortest timeout of one minute per function. 
Interestingly, the coverage metrics for @RBTree@ and @Data.Map@ remain relatively constant as we increase
the timeouts, with a small jump in expression coverage between 10 and 20 minutes.
@XMonad@ on the other hand, jumps from 70\% expression and 64\% alternative
coverage with a one minute timeout, to 96\% expression and 94\% alternative
with a ten minute timeout.

% @Data.Map@ and @RBTree@ show no change in coverage metrics 
% beyond a 5 minute timeout, while @XMonad@ has another bump in coverage 
% between 10 and 15 minutes.

There are three things to consider when examining these results. 
%
First is that some expressions are not evaluated due to Haskell's 
laziness (\eg the values contained in a @Map@). 
%
Second is that some expressions \emph{should not} be evaluated 
and some branches \emph{should not} be taken, as these only happen
when an unexpected error condition is triggered (\ie these expressions
should be dead code).
%
\toolname considers any inputs that trigger an uncaught exception a 
valid counterexample; the pre-conditions should rule out these inputs, 
and so we expect not to cover those expressions with \toolname.

The last remark is not intrinsically related to \toolname, 
but rather our means of collecting the coverage data. @hpc@ includes 
@otherwise@ guards in the ``always-true'' category, even though they 
cannot evaluate to anything else. 
%
@Data.Map@ contained 56 guards, of which 24 were marked ``always-true''. We
manually counted 21 \hbox{@otherwise@} guards, the remaining 3 ``always-true''
guards compared the size of subtrees when rebalancing to determine whether a
single or double rotation was needed; we were unable to trigger the double
rotation in these cases.
%
\hbox{@XMonad@} contained 9 guards, of which 4 were ``always-true''. 3 of these
were @otherwise@ guards; the remaining ``always-true'' guard dynamically checked
a function's pre-condition. If the pre-condition check had failed an error would
have been thrown by the next case, we consider it a success of \toolname that
the error branch was not triggered.


\begin{figure}[t!]
\centering
% \includegraphics[width=0.49\linewidth]{figs/MapCoverage}
% \includegraphics[width=0.49\linewidth]{figs/XMonad-StackSetCoverage}
% \includegraphics[width=0.49\linewidth]{figs/RBTreeCoverage}
  \begin{tikzpicture}
    \begin{groupplot}[
      group style = {group size = 3 by 1, horizontal sep=15pt,},
      groupplot ylabel={\% Coverage},
      groupplot xlabel={Timeout (min)},
      group/only outer labels,
      ymin=0,
      ymax=1
    ]
    % \begin{axis}[
    \nextgroupplot[
      title=\textsc{Data.Map},
      legend columns=3,
      legend entries={expressions,booleans,always-true,always-false,alternatives,local-functions},
      legend to name=legend,
    ]
    \addplot table[smooth,col sep=comma,x index=0,y index=1] {target/csv/MapCoverage.csv};
    \addplot table[smooth,col sep=comma,x index=0,y index=2] {target/csv/MapCoverage.csv};
    \addplot table[smooth,col sep=comma,x index=0,y index=3] {target/csv/MapCoverage.csv};
    \addplot table[smooth,col sep=comma,x index=0,y index=4] {target/csv/MapCoverage.csv};
    \addplot table[smooth,col sep=comma,x index=0,y index=5] {target/csv/MapCoverage.csv};
    \addplot table[smooth,col sep=comma,x index=0,y index=6] {target/csv/MapCoverage.csv};
    % \end{axis}
  % \end{tikzpicture}
  % \begin{tikzpicture}
    % \begin{axis}[
    \nextgroupplot[
      title=\textsc{XMonad.StackSet},
    ]
    \addplot table[smooth,col sep=comma,x index=0,y index=1] {target/csv/StackSetCoverage.csv};
    \addplot table[smooth,col sep=comma,x index=0,y index=2] {target/csv/StackSetCoverage.csv};
    \addplot table[smooth,col sep=comma,x index=0,y index=3] {target/csv/StackSetCoverage.csv};
    \addplot table[smooth,col sep=comma,x index=0,y index=4] {target/csv/StackSetCoverage.csv};
    \addplot table[smooth,col sep=comma,x index=0,y index=5] {target/csv/StackSetCoverage.csv};
    \addplot table[smooth,col sep=comma,x index=0,y index=6] {target/csv/StackSetCoverage.csv};
    % \end{axis}
  % \end{tikzpicture}
  % \begin{tikzpicture}
    % \begin{axis}[
    \nextgroupplot[
      title=\textsc{RBTree}
    ]
    \addplot table[smooth,col sep=comma,x index=0,y index=1] {target/csv/RBTreeCoverage.csv};
    \addplot table[smooth,col sep=comma,x index=0,y index=5] {target/csv/RBTreeCoverage.csv};
    \addplot table[smooth,col sep=comma,x index=0,y index=6] {target/csv/RBTreeCoverage.csv};
    % \end{axis}
    \end{groupplot}
  \end{tikzpicture}\\
  \ref{legend}
% \begin{verbatim}
% 81% expressions used (2202/2712)
% 42% boolean coverage (24/57)
%      41% guards (23/56), 26 always True,
%          3 always False, 4 unevaluated
%     100% 'if' conditions (1/1)
%     100% qualifiers (0/0)
% 95% alternatives used (370/388)
% 98% local declarations used (49/50)
% 92% top-level declarations used (134/145)
% \end{verbatim}
\caption{Coverage-testing of \texttt{Data.Map.Base}, \texttt{RBTree}, and
  \texttt{XMonad.StackSet} using \toolname. Each exported function was tested
  with increasing depth limits until a single run hit a timeout ranging from one
  to thirty minutes. Lower is better for ``always-true'' and ``always-false'',
  higher is better for everything else.}\label{fig:coverage}
\end{figure}

%%% NUKE \ES{can we use an hpc overlay to make it ignore the "always true" otherwise
%%% NUKE   guards? seems the party line is that one should focus on expression and
%%% NUKE   alternative coverage, not boolean... so perhaps we can report expression, alternative, and alwaysFalse}
%%% NUKE \RJ{Dont know what you mean, is this note LIVE? or can we DELETE?}

% Although
% @hpc@ reports only 42\% boolean coverage for @Data.Map@, manual inspection
% revealed that 22 of the guards marked by @hpc@ as ``always True'' are
% @otherwise@ guards and can never be false. In that light, it would be more
% accurate to consider 46/57 booleans as covered, \ie 82\% coverage. The remaining
% ``always True'' branches compared the size of subtrees when rebalancing to
% determine whether a single or double rotation was needed, in some cases we were
% unable to generate sufficiently large trees in one minute to trigger a double
% rotation. The two guards that were always false were due to the simplistic
% generator we currently use for higher-order functions always returning false.

\subsection{Discussion}\label{sec:discussion}

To sum up, our experiments demonstrate that \toolname generates valid inputs:
%
(1) where \quickcheck fails outright, due to the low probability of
    generating random values satisfying a property;
%
(2) more efficiently than \lazysmallcheck, which relies on lazy
    pruning predicates; and
%
(3) providing high code coverage for real-world libraries with no
    hand-written test cases.

% \subsection{Limitations of \toolname}\label{sec:limitations}

Of course our approach is not without drawbacks; we highlight five classes
of pitfalls the user may encounter.

\mypara{Laziness} in the function or in the output refinement can cause exceptions
  to go un-thrown if the output value is not fully demanded. For example,
  \toolname would decide that the result @[1, undefined]@ inhabits @[Int]@ but not
  @[Score]@, as the latter would have to evaluate @0 <= undefined < 100@. This
  limitation is not specific to our system, rather it is fundamental to any tool
  that exercises lazy programs. Furthermore, \toolname only generates
  inductively-defined values, it cannot generate infinite or cyclic structures,
  nor will the generated values ever contain $\bot$.

\mypara{Polymorphism} Like any other tool that actually runs the function under scrutiny,
  \toolname can only test monomorphic instantiations of polymorphic
  functions. For example, when testing @XMonad@ we instantiated the ``window''
  parameter to @Char@ and all other type parameters to @()@, as the properties
  we were testing only examined the window. This helped drastically reduce the
  search space, both for \toolname and \smallcheck.

  % Our monomorphism restriction simplifies \toolname's implementation as we do
  % not have to consider type-class or equality constraints when generating test
  % values, but it also reduces the generalizability of \toolname's
  % result. 
  % Parametricity helps by telling us that the choice of
  % concrete instantiation will not affect the behavior of the function, but
  % in the presence of type-classes the benefit is reduced as we only know that
  % the specific instance we tested is correct.

\mypara{Advanced type-system features} such as GADTs and Existential types
  may prevent GHC from deriving a @Generic@ instance, which would force the
  programmer to write her own @Targetable@ instance. Though tedious, the single
  hand-written instance allows \toolname to automatically generate values
  satisfying disparate constraints, which is still an improvement over the
  generate-and-filter approach.
  
\mypara{Refinement types} are less expressive than properties written in the
  host language. If the pre-conditions are not expressible in \toolname's logic,
  the user will have to use the generate-and-filter approach, losing the benefits
  of symbolic enumeration.
  
\mypara{Input explosion} \toolname excels when the space of valid inputs is
  a sparse subset of the space of all inputs. If the input space is not
  sufficiently constrained, \toolname may spend lose its competitive advantage
  over other tools due to the overhead of using a general-purpose solver.

%% 1. laziness
%%    - potential for untriggered exceptions
%%    - our generated values never include bot
%% 2. Advanced type-system features
%%    - we can only provide default instances for Generic types
%%      - no GADTs or existentials
%% 3. Polymorphic functions
%%    - can only test monomorphic instantiation
%%    - types must be defaulted either by user or GHC
%%    - limitation shared by any testing tool

%%% Local Variables:
%%% mode: latex
%%% TeX-master: "main"
%%% End:

\input{nanomaly/related2}
\lstDeleteShortInline{@}

%\chapter{Type Targeted Testing}
%\lstMakeShortInline{@}
%\renewcommand\toolname{\tool{Target}}
%\section{Introduction}\label{sec:intro}

Should the programmer spend her time writing \emph{better types}
or \emph{thorough tests}?  
%
Types have long been the most pervasive means of describing the 
intended behavior of code. However, a type signature is often a 
very coarse description; the actual inputs and outputs
may be a subset of the values described by the types. 
%
For example, the set of ordered integer lists is a very 
sparse subset of the set of all integer lists. 
%
Thus, to validate functions that produce or consume such values, 
the programmer must painstakingly enumerate these values by hand 
or via ad-hoc generators for unit tests.

We present a new technique called \emph{type targeted testing}, 
abbreviated to \toolname, that enables the generation of unit
tests from precise \emph{refinement types}.
%
Over the last decade, various groups have shown how refinement 
types -- which compose the usual types with logical refinement predicates
that characterize the subset of actual type inhabitants -- 
can be used to specify and formally verify a wide variety 
of correctness properties of programs~\cite{pfenningxi98,Dunfield07,fstar,VazouICFP14}.
%
Our insight is that through the lens of SMT
solvers, refinement types can be viewed as a high-level, 
declarative, test generation technique.

\toolname tests an implementation function against a refinement 
type specification using a \emph{query-decode-check} loop.
%
% NV: input implies the types TARGET gets as input
First, \toolname translates the argument types into a logical
\emph{query} for which we obtain a satisfying assignment 
(or model) from the SMT solver.
%
Next, \toolname \emph{decodes} the SMT solver's model to obtain
concrete input values for the function.
%
Finally, \toolname executes the function on the inputs 
to get the corresponding output, which we \emph{check} 
belongs to the specified result type. 
%
If the check fails, the inputs are returned as a counterexample, 
otherwise
%
\toolname refutes the given model to force the SMT solver to 
return a different set of inputs. 
%
This process is repeated for a given number of 
iterations, or until \emph{all} inputs up to a certain size 
have been tested.

% Vs. Testing
\toolname offers several benefits over other testing techniques.
%
Refinement types provide a succinct description of the 
input and output requirements, eliminating the need to 
enumerate individual test cases by hand or to write 
custom generators.
%
Furthermore, \toolname generates \emph{all} 
values (up to a given size) that inhabit a type, and thus
does not skip any corner cases that a hand-written generator 
might miss.
%
Finally, while the above advantages can be recovered by a brute-force
generate-and-filter approach that discards inputs that do not meet
some predicate, we show that our SMT-based method can be significantly
more efficient for enumerating valid inputs in a highly-constrained space.
% and hence, sparse space.

% Vs. Verification 
\toolname paves a \emph{gradual path} from testing to verification, 
that affords several advantages over verification.
%
First, the programmer has an \emph{incentive} to write formal 
specifications using refinement types. \toolname provides the 
immediate gratification of an automatically generated, 
exhaustive suite of unit tests that can expose errors.
Thus, the programmer is rewarded without paying, up front, 
the extra price of annotations, hints, strengthened 
inductive invariants, or tactics needed for formally 
verifying the specification.
%
Second, our approach makes it possible to use refinement 
types to formally verify \emph{some} parts of the program, 
while using tests to validate other parts that may
be too difficult to verify
%
\toolname integrates the two modes by using refinement
types as the uniform specification mechanism. 
Functions in the verified half can be formally checked 
\emph{assuming} the functions in the tested half adhere 
to their specifications. 
We could even use refinements to generate dynamic 
contracts~\cite{Findler01} around the tested half 
if so desired.
%
Third, even when formally verifying the type specifications, 
the generated tests can act as valuable \emph{counterexamples} 
to help \emph{debug} the specification or implementation in 
the event that the program is rejected by the verifier.

% Vs. SymEx 
Finally, \toolname offers several concrete advantages over previous
property-based testing techniques, which also have the potential for 
gradual verification.
%
First, instead of specifying properties with arbitrary code 
\cite{claessen_quickcheck:_2000,runciman_smallcheck_2008} 
which complicates the task of subsequent formal verification, 
with \toolname the properties are specified via refinement 
types, for which there are already several existing formal 
verification algorithms~\cite{VazouICFP14}.
%
Second, while symbolic execution tools~\cite{DART,CUTE,Veanes08} 
can generate tests from arbitrary code contracts (\eg assertions) 
we find that highly constrained inputs trigger path explosion 
which precludes the use of such tools for gradual verification.

% In the rest of the paper...
In the rest of this paper, we start with an overview of 
how \toolname can be used and how its query-decode-check 
loop is implemented (\S~\ref{sec:overview}).
%
Next, we formalize a general framework for type-targeted 
testing (\S~\ref{sec:framework}) and show how it can be 
instantiated to generating tests for lists (\S~\ref{sec:list}), 
and then automatically generalized to other 
types (\S~\ref{sec:generic}).
%
All the benefits of \toolname come at a price; 
we are limited to properties that can be specified with 
refinement types. 
%
We present an empirical evaluation that shows
\toolname is efficient and expressive enough to capture 
a variety of sophisticated properties,
%
demonstrating that type-targeted 
testing is
a sweet spot between automatic testing 
and verification (\S~\ref{sec:evaluation}).

%%% Second, \toolname's symbolic, SMT-based approach makes it possible
%%% to systematically generate values that satisfy highly constrained 
%%% pre-conditions which otherwise thwart the generate-and-filter 
%%% approach of traditional property-based tools.
%%% %
%%% While this work shows a great deal of promise for making 
%%% formal verification practical, it does not obviate the 
%%% need for testing.
%%% %
%%% First, it can take a great deal of effort and expertise 
%%% to write down the annotations (beyond the end-to-end 
%%% specification) needed to verify a program 
%%% %
%%% Second, there maybe requirements (\eg performance) that 
%%% are not easily specified via refinements.
%%% ES INTRO We present \toolname, an automatic test-generator that uses SMT-solvers 
%%% ES INTRO to generate test-cases, taking advantange of their efficient decision 
%%% ES INTRO procedures to quickly prune the search space of all inputs. 
%%% ES INTRO 
%%% ES INTRO We observe based on prior work that many interesting properties of
%%% ES INTRO functions can be specified in an SMT-decidable logic, in our case the
%%% ES INTRO logic of linear arithetic, equality, and uninterpreted functions
%%% ES INTRO (\smtlogic).
%%% ES INTRO 
%%% ES INTRO Building on top of the \liquidhaskell program verification
%%% ES INTRO tool~\cite{VazouRealWorld14}, we show how a single data generator can
%%% ES INTRO be used to produce values of a type that satisfy disparate
%%% ES INTRO constraints, thereby easing the tedium of testing
%%% ES INTRO (\S~\ref{sec:liquidcheck}). 
%%% ES INTRO 
%%% ES INTRO We further show how to use a single generator to generically produce
%%% ES INTRO values of \emph{any} algebraic datatype (\S~\ref{sec:generic-generation}). 
%%% ES INTRO 
%%% ES INTRO Finally, we note that due to our shared specification language with
%%% ES INTRO \liquidhaskell, our properties are immediately ammenable to formal
%%% ES INTRO verification, should the developer wish to invest the extra time
%%% ES INTRO (\S~\ref{sec:discussion}).

%\section{Overview}\label{sec:overview}

We start with a series of examples pertaining to a small grading
library called @Scores@. The examples provide a bird's eye view of 
how a user interacts with \toolname, how \toolname is implemented,
and the advantages of type-based testing.

\mypara{Refinement Types}
A refinement type is one where the basic types are decorated 
with logical predicates drawn from an efficiently decidable 
theory. For example,
%
\begin{code}
  type Nat   = {v:Int | 0 <= v}
  type Pos   = {v:Int | 0 <  v}
  type Rng N = {v:Int | 0 <= v && v <  N}
\end{code}
%
are refinement types describing the set of integers that are 
non-negative, strictly positive, and in the interval @[0, N)@ 
respectively. We will also build up function and collection 
types over base refinement types like the above. 
%
In this paper, we will not address the issue of \emph{checking}
refinement type signatures~\cite{VazouICFP14}.
%
We assume the code is typechecked, \eg by GHC, against the 
standard type signatures obtained by erasing the refinements.
Instead, we focus on using the refinements to 
synthesize tests to \emph{execute} the function, and to find 
\emph{counterexamples} that violate %demonstrate the function does not meet
the given specification.

\subsection{Testing with Types}

\mypara{Base Types}
Let us write a function @rescale@ that takes a source range @[0,r1)@, 
a target range @[0,r2)@, and a score @n@ from the source range,
and returns the linearly scaled score in the target range.
%
For example, @rescale 5 100 2@ should return @40@. 
Here is a first attempt at @rescale@ 
%
\begin{code}
  rescale :: r1:Nat -> r2:Nat -> s:Rng r1 -> Rng r2 
  rescale r1 r2 s = s * (r2 `div` r1)   
\end{code}
%
When we run \toolname, it immediately reports 
%
\begin{code}
  Found counter-example: (1, 0, 0) 
\end{code}
%
Indeed, @rescale 1 0 0@ results in @0@ which is not in the target 
@Rng 0@, as the latter is empty! We could fix this in various ways, 
\eg by requiring the ranges are non-empty:
%
\begin{code}
  rescale :: r1:Pos -> r2:Pos -> s:Rng r1 -> Rng r2 
\end{code}
%
Now, \toolname accepts the function and reports
%
\begin{code}
  OK. Passed all tests.
\end{code}
%
Thus, using the refinement type \emph{specification} for @rescale@, 
\toolname systematically tests the \emph{implementation} by generating 
all valid inputs (up to a given size bound) that respect the 
pre-conditions, running the function, and checking that the 
output satisfies the post-condition.
%
Testing against random, unconstrained inputs would be of limited value 
as the function is not designed to work on all @Int@ values. While in 
this case we could filter invalid inputs, we shall show
that \toolname can be more effective.

\mypara{Containers}
Let us suppose we have normalized all scores to be out of @100@
%
\begin{code}
  type Score = Rng 100
\end{code}
%
Next, let us write a function to compute a \emph{weighted} average 
of a list of scores.
%
\begin{code}
  average     :: [(Int, Score)] -> Score
  average []  = 0
  average wxs = total `div` n
    where
      total   = sum [w * x | (w, x) <- wxs ]
      n       = sum [w     | (w, _) <- wxs ]
\end{code}
%
It can be tricky to \emph{verify} this function as it requires non-linear reasoning
about an unbounded collection. However, we can gain a great degree of confidence by
systematically testing it using the type specification; indeed, \toolname responds:
%
\begin{code}
  Found counter-example: [(0,0)]
\end{code}
%
Clearly, an unfortunate choice of weights can trigger a divide-by-zero; we can fix 
this by requiring the weights be non-zero:
%
\begin{code}
  average :: [({v:Int | v /= 0}, Score)] -> Score
\end{code}
%
but now \toolname responds with
%
\begin{code}
  Found counter-example: [(-3,3),(3,0)]
\end{code}
% 
which also triggers the divide-by-zero! We will play it safe and require positive weights,
%
\begin{code}
  average :: [(Pos, Score)] -> Score
\end{code}
%
at which point \toolname reports that all tests pass.

\mypara{Ordered Containers}
The very nature of our business requires that at the end of the day,
we order students by their scores. We can represent ordered lists by 
requiring the elements of the tail @t@ to be greater than the head @h@:
%
\begin{code}
data OrdList a = [] | (:) {h :: a, t :: OrdList {v:a | h <= v}}
\end{code}
%
Note that erasing the refinement predicates gives us plain old Haskell lists.
We can now write a function to insert a score into an ordered list:
%
\begin{code}
  insert :: (Ord a) => a -> OrdList a -> OrdList a 
\end{code}
%
\toolname automatically generates all ordered lists (up to a given size)
and executes @insert@ to check for any errors. Unlike randomized testers, 
\toolname is not thwarted by the ordering constraint, and does not require a
custom generator from the user.

\mypara{Structured Containers} 
Everyone has a few bad days. Let us write a function that takes the 
@best k@ scores for a particular student. That is, the output
must satisfy a \emph{structural} constraint -- that its size 
equals @k@. We can encode the size of a list with a logical 
measure function~\cite{VazouICFP14}:
%
\begin{code}
  measure len :: [a] -> Nat
  len []      = 0
  len (x:xs)  = 1 + len xs
\end{code}
%
Now, we can stipulate that the output indeed has @k@ scores:
%
\begin{code}
  best      :: k:Nat -> [Score] -> {v:[Score] | k = len v}
  best k xs = take k $ reverse $ sort xs
\end{code}
%
Now, \toolname quickly finds a counterexample:
%
\begin{code}
  Found counter-example: (2,[])
\end{code}
%
Of course -- we need to have at least @k@ scores to start with! 
%
\begin{code}
best :: k:Nat -> {v:[Score]|k <= len v} -> {v:[Score]|k = len v}
\end{code}
%
and now, \toolname is assuaged and reports no counterexamples.
%
While randomized testing would suffice for @best@, we will see 
more sophisticated structural properties such as height balancedness, 
which stymie random testers, but are easily handled by \toolname.

\mypara{Higher-order Functions} 
Perhaps instead of taking the $k$ best grades, we would like
to pad each individual grade, and, furthermore, we want to
be able to experiment with different padding functions. Let
us rewrite @average@ to take a functional argument, and
stipulate that it can only increase a @Score@.
%
\begin{code}
  padAverage       :: (s:Score -> {v:Score | s <= v}) 
                   -> [(Pos, Score)] -> Score
  padAverage f []  = f 0
  padAverage f wxs = total `div` n
    where
      total   = sum [w * f x | (w, x) <- wxs ]
      n       = sum [w       | (w, _) <- wxs ]
\end{code}
%
\toolname automatically checks that @padAverage@ is 
a safe generalization of @average@. Randomized 
testing tools can also generate functions, but those 
functions are unlikely to satisfy non-trivial constraints, 
thereby burdening the user with custom generators.


\subsection{Synthesizing Tests} 
\label{sec:synthesizing-tests}
Next, let us look under the hood to get an idea of how \toolname 
synthesizes tests from types. 
% INTRO
At a high-level, our strategy is to:
%
(1)~\emph{query}   an SMT solver for satisfying assigments to a set of logical 
                   constraints derived from the refinement type,
(2)~\emph{decode}  the model into Haskell values that are suitable inputs,
(3)~\emph{execute} the function on the decoded values to obtain the output, 
(4)~\emph{check}   that the output satisfies the output type,
(5)~\emph{refute}  the model to generate a different test, and 
repeat the above steps until all tests up to a certain size are executed.
%
We focus here on steps 1, 2, and 4 -- query, decode, and check -- the others are 
standard and require little explanation.

\mypara{Base Types}
Recall the initial (buggy) specification
%
\begin{code}
  rescale :: r1:Nat -> r2:Nat -> s:Rng r1 -> Rng r2 
\end{code}
%
\toolname \emph{encodes} input requirements for base types directly 
from their corresponding refinements. The constraints for multiple, 
related inputs are just the \emph{conjunction} of the constraints 
for each input. Hence, the constraint for @rescale@ is:
%
$$
\cstr{C_0} \defeq 0 \leq \cvar{r1} \wedge 0 \leq \cvar{r2} \wedge 0 \leq s < \cvar{r1} 
$$
%
In practice, $\cstr{C_0}$ will also contain conjuncts of the form $-N \leq x \leq N$ that
restrict @Int@-valued variables $x$ to be within the size bound $N$ supplied by
the user, but we will omit these throughout the paper for clarity.
%% %
%% For clarity, we omit the conjuncts of the form $-N \leq x \leq N$
%% that restrict @Int@-valued variables $x$ to be within the size
%% bound $N$ supplied by the user.

Note how easy it is to capture dependencies between inputs, 
\eg that the score @s@ be in the range defined by @r1@.
%
On querying the SMT solver with the above, we get a model
%
$[\cvar{r1} \mapsto 1, \cvar{r2} \mapsto 1, \cvar{s}  \mapsto 0]$.
%
\toolname decodes this model and executes \hbox{@rescale 1 1 0@} to obtain the value @v = 0@.
%
Then, \toolname validates @v@ against the post-condition by checking 
% that it inhabits the output type, \ie by checking 
the validity of the output type's constraint: 
%
$$\cvar{r2} = 1 \wedge \cvar{v} = 0 \wedge 0 \leq \cvar{v} \wedge \cvar{v} < \cvar{r2}$$
%
As the above is valid, \toolname moves on to generate another 
test by conjoining $\cstr{C_0}$ with a constraint that refutes 
the previous model:
%
$$
\cstr{C_1} \defeq \cstr{C_0} \wedge (\cvar{r1} \not = 1 \vee \cvar{r2} \not = 1 \vee \cvar{s} \not = 0)
$$
This time, the SMT solver returns a model: 
%
$[\cvar{r1} \mapsto 1, \cvar{r2} \mapsto 0, \cvar{s} \mapsto 0]$
%
which, when decoded and executed, yields the result $0$ that does \emph{not} 
inhabit the output type, and so is reported as a counterexample. 
%
When we fix the specification to only allow @Pos@ ranges, each test produces
a valid output, so \toolname reports that all tests pass.

\mypara{Containers}
Next, we use \toolname to test the implementation of @average@.
To do so, \toolname needs to generate Haskell lists with the appropriate constraints.
%
Since each list is recursively 
either ``nil'' 
or ``cons'', 
\toolname generates constraints that symbolically 
represent \emph{all} possible lists up to a given depth, 
using propositional \emph{choice variables} to 
symbolically pick between these two alternatives.
%
Every (satisfying) assignment of choices returned by 
the SMT solver gives \toolname the concrete data and 
constructors used at each level, allowing it to decode 
the assignment into a Haskell value.

For example, \toolname represents valid @[(Pos, Score)]@ 
inputs (of depth up to 3), required to test @average@, 
as the conjunction of $\cstr{C_{list}}$ and $\cstr{C_{data}}$:
%
\begin{eqnarray*}
\cstr{C_{list}} & \defeq & (\cvar{c}_{00} \Rightarrow \cvar{xs}_0 = \lnil) \wedge 
                          (\cvar{c}_{01} \Rightarrow \cvar{xs}_0 = \lcons{\cvar{x}_1}{\cvar{xs}_1}) \wedge 
                          (\cvar{c}_{00} \oplus \cvar{c}_{01}) \\
               & \wedge & (\cvar{c}_{10} \Rightarrow \cvar{xs}_1 = \lnil) \wedge
                          (\cvar{c}_{11} \Rightarrow \cvar{xs}_1 = \lcons{\cvar{x}_2}{\cvar{xs}_2}) \wedge 
                          (\cvar{c}_{01} \Rightarrow \cvar{c}_{10} \oplus \cvar{c}_{11}) \\
               & \wedge & (\cvar{c}_{20} \Rightarrow \cvar{xs}_2 = \lnil) \wedge 
                          (\cvar{c}_{21} \Rightarrow \cvar{xs}_2 = \lcons{\cvar{x}_3}{\cvar{xs}_3}) \wedge 
                          (\cvar{c}_{11} \Rightarrow \cvar{c}_{20} \oplus \cvar{c}_{21}) \\
               & \wedge & (\cvar{c}_{30} \Rightarrow \cvar{xs}_3 = \lnil) \wedge 
                          (\cvar{c}_{21} \Rightarrow \cvar{c}_{30}) \\[0.1in]
\cstr{C_{data}} & \defeq & (\cvar{c}_{01} \Rightarrow \cvar{x}_1 = \ltup{\cvar{w}_1}{\cvar{s}_1} \ \wedge\ 0 < \cvar{w}_1 \ \wedge\ 0 \leq \cvar{s}_1 < 100) \\
               & \wedge & (\cvar{c}_{11} \Rightarrow \cvar{x}_2 = \ltup{\cvar{w}_2}{\cvar{s}_2} \ \wedge\ 0 < \cvar{w}_2 \ \wedge\ 0 \leq \cvar{s}_2 < 100) \\
               & \wedge & (\cvar{c}_{21} \Rightarrow \cvar{x}_3 = \ltup{\cvar{w}_3}{\cvar{s}_3} \ \wedge\ 0 < \cvar{w}_3 \ \wedge\ 0 \leq \cvar{s}_3 < 100)
\end{eqnarray*}
%
The first set of constraints $\cstr{C_{list}}$ describes all lists up to 
size 3. At each level $i$, the \emph{choice} variables $\cvar{c}_{i0}$ 
and $\cvar{c}_{i1}$ determine whether at that level the constructed 
list $\cvar{xs}_i$ is a ``nil'' or a ``cons''. 
%
In the constraints $\lnil$ and $(\lcons{}{})$ are \emph{uninterpreted} 
functions that represent ``nil'' and ``cons'' respectively. 
These functions only obey the congruence axiom and hence, can be 
efficiently analyzed by SMT solvers~\cite{Nelson81}.
%
The data at each level $\cvar{x}_i$ is constrained to be a pair of a 
positive weight $\cvar{w}_i$ and a valid score $\cvar{s}_i$.

The choice variables at each level are used to \emph{guard} the 
constraints on the next levels. 
%
First, if we are generating a ``cons'' at a given level, then
exactly one of the choice variables for the next level must be 
selected;
\eg  $\cvar{c}_{11} \Rightarrow \cvar{c}_{20} \oplus \cvar{c}_{21}$.
%
Second, the constraints on the data at a given level only hold 
if we are generating values for that level; \eg $\cvar{c}_{21}$ 
is used to guard the constraints on $\cvar{x}_3$, $\cvar{w}_3$ 
and $\cvar{s}_3$.
%
This is essential to avoid over-constraining the system 
which would cause \toolname to miss certain tests.

To \emph{decode} a model of the above into a Haskell value of type @[(Int, Int)]@,
we traverse constraints and use the valuations of the choice variables to 
build up the list appropriately.
%
At each level, if $\cvar{c}_{i0} \mapsto \ttrue$, then the list at that 
level is @[]@, otherwise $\cvar{c}_{i1} \mapsto \ttrue$ and we decode 
$\cvar{x}_{i+1}$ and $\cvar{xs}_{i+1}$ and ``cons'' the results.

We can iteratively generate \emph{multiple} inputs by adding a constraint that
refutes each prior model. As an important optimization, we only refute the
relevant parts of the model, \ie those needed to construct the list
(\S~\ref{sec:refute}).

\mypara{Ordered Containers}
%
Next, let us see how \toolname enables automatic testing with 
highly constrained inputs, such as the \emph{increasingly ordered} 
@OrdList@ values required by @insert@.
%
From the type definition, it is apparent that ordered
lists are the same as the usual lists described by
$\cstr{C_{list}}$, except that each unfolded \emph{tail} 
must only contain values that are greater than the 
corresponding \emph{head}.
%
That is, as we unfold @x1:x2:xs :: OrdList@ 
%
\begin{itemize}
\item At level @0@, we have @OrdList {v:Score| true}@
\item At level @1@, we have @OrdList {v:Score| x1 <= v}@
\item At level @2@, we have @OrdList {v:Score| x2 <= v && x1 <= v}@
\end{itemize}
%
and so on. Thus, we encode @OrdList Score@ (of depth up to 3) by
conjoining $\cstr{C_{list}}$ with  $\cstr{C_{score}}$ and $\cstr{C_{ord}}$,
which capture the valid score and ordering requirements respectively:
%
\begin{eqnarray*}
\cstr{C_{ord}}   & \defeq & (\cvar{c}_{11} \Rightarrow \cvar{x}_1 \leq \cvar{x}_2)
                \ \wedge\  (\cvar{c}_{21} \Rightarrow \cvar{x}_2 \leq \cvar{x}_3\ \wedge\ \cvar{x}_1 \leq \cvar{x}_3) \\[0.01in]
\cstr{C_{score}} & \defeq & (\cvar{c}_{01} \Rightarrow 0 \leq \cvar{x}_1 < 100)
                \ \wedge\  (\cvar{c}_{11} \Rightarrow 0 \leq \cvar{x}_2 < 100)
                \ \wedge\  (\cvar{c}_{21} \Rightarrow 0 \leq \cvar{x}_3 < 100)
\end{eqnarray*}

\mypara{Structured Containers}
Recall that @best k@ requires inputs whose \emph{structure} is constrained -- the 
size of the list should be no less than @k@. We specify size using special measure 
functions~\cite{VazouICFP14}, which let us relate the size of a list with that of
its unfolding, and hence, let us encode the notion of size inside the constraints:
%
\begin{eqnarray*}
\cstr{C_{size}} & \defeq & (\cvar{c}_{00} \Rightarrow \clen{\cvar{xs}_{0}} = 0) \wedge 
                          (\cvar{c}_{01} \Rightarrow \clen{\cvar{xs}_{0}} = 1 + \clen{\cvar{xs}_1}) \\
               & \wedge & (\cvar{c}_{10} \Rightarrow \clen{\cvar{xs}_{1}} = 0) \wedge 
                          (\cvar{c}_{11} \Rightarrow \clen{\cvar{xs}_{1}} = 1 + \clen{\cvar{xs}_2}) \\
               & \wedge & (\cvar{c}_{20} \Rightarrow \clen{\cvar{xs}_{2}} = 0) \wedge 
                          (\cvar{c}_{21} \Rightarrow \clen{\cvar{xs}_{2}} = 1 + \clen{\cvar{xs}_3}) \\
               & \wedge & (\cvar{c}_{30} \Rightarrow \clen{\cvar{xs}_{3}} = 0)
\end{eqnarray*}
%
At each unfolding, we instantiate the definition of the measure 
for each alternative of the datatype. 
%
In the constraints, $\clen{\cdot}$ is an uninterpreted function derived
from the measure definition. All of the relevant properties of the function
are spelled out by the unfolded constraints in $\cstr{C_{size}}$ and hence,
we can use SMT to search for models for the above constraint.
%
Hence, \toolname constrains the input type for @best@ as:
%
$$     0 \leq k 
\wedge \cstr{C_{list}} 
\wedge \cstr{C_{score}} 
\wedge \cstr{C_{size}} 
\wedge k \leq \clen{\cvar{xs}_0} $$
%
where the final conjunct comes from the top-level refinement that 
stipulates the input have at least @k@ scores.
%
Thus, \toolname only generates lists that are large enough. 
For example, in any model where $k = 2$, it will \emph{not} 
generate the empty or singleton list, as in those cases, 
$\clen{\cvar{xs}_0}$ would be $0$ (resp. $1$), violating the 
final conjunct above.

\mypara{Higher-order Functions}
Finally, \toolname's type-directed testing scales up to higher-order
functions using the same insight as in QuickCheck~\cite{claessen_quickcheck:_2000}, namely, 
to generate a function it suffices to be able to 
generate the \emph{output} of the function.
When tasked with the generation of a functional argument @f@, \toolname 
returns a Haskell function that when executed checks
whether its inputs satisfy @f@'s pre-conditions.
If they do, then @f@ uses \toolname to dynamically
query the SMT solver for an output that satisfies the 
constraints imposed by the concrete inputs.
Otherwise, @f@'s specifications are violated
and TARGET reports a counterexample.

This concludes our high-level tour of the benefits and 
implementation of \toolname. 
%
Notice that the property specification mechanism -- 
refinement types -- allowed us to get immediate feedback
that helped debug not just the code, but also the specification 
itself. 
%
Additionally, the specifications gave us machine-readable 
documentation about the behavior of functions, and a large 
unit test suite with which to automatically validate the 
implementation.
%
Finally, though we do not focus on it here, the specifications 
are amenable to formal verification should the programmer 
so desire.


%%% Local Variables:
%%% mode: latex
%%% TeX-master: "main"
%%% End:

%\section{A Framework for Type Targeted Testing}\label{sec:framework}

Next, we describe a framework for type targeted testing, by formalizing
an abstract representation of refinement types~(\S~\ref{sec:reftypes}), 
describing the operations needed to generate tests from types~(\S~\ref{sec:targetable}), 
and then using the above to implement \toolname via a query-decode-check 
loop~(\S~\ref{sec:loop}). 
%
Subsequently, we instantiate the framework to obtain tests
for refined primitive types, lists, algebraic datatypes and higher-order 
functions~(\S~\ref{sec:list}).

%% Next, we describe an (abstract) representation of refinement types,
%% and show how to use them to implement the @query@, @decode@ and @check@ 
%% steps from Figure~\ref{fig:arch}.
 
\subsection{Refinement Types}\label{sec:reftypes}

\begin{figure}[t!]
\begin{mdframed}
\begin{CenteredBox}
\begin{code}
-- Manipulating Refinements
refinement :: RefType -> Refinement
subst      :: RefType -> [(Var, Var)] -> RefType

-- Manipulating Types
unfold     :: Ctor  -> RefType -> [(Var, RefType)]
binder     :: RefType -> Var
proxy      :: RefType -> Proxy a 
\end{code}
\end{CenteredBox}
\end{mdframed}
\caption{Refinement Type API}\label{fig:rtype}
\end{figure}

A refinement type is a type, where each component is 
decorated with a predicate from a refinement logic. 
%
For clarity, we describe refinement types and refinements 
abstractly as @RefType@ and @Refinement@ respectively.
%
We write @Var@ as an alias for @Refinement@ that is 
typically used to represent logical variables appearing
within the refinement.

\mypara{Notation} 
In the sequel, we will use % backticks
double brackets $\meta{}$ to represent the 
various entities in the meta-language used to describe \toolname. 
%
For example, 
|$\meta{k}$|,
|$\meta{k \leq len\ v}$|, and
|$\meta{\reft{v}{[Score]}{k \leq len\ v}}$|
% \hbox{@`{v:[Score] | x0 <= len v}`@}
are the @Var@, @Refinement@, and @RefType@
representing the corresponding entities written in the brackets.


Next, we describe the various operations over them 
needed to implement \toolname.
These operations, summarized in Figure~\ref{fig:rtype}, 
fall into two categories: those which manipulate the 
\emph{refinements} and those which manipulate the 
\emph{types}.

\mypara{Operating on Refinements} 
To generate constraints and check inhabitation, we use 
the function @refinement@ which returns the (top-level) refinement
that decorates the given refinement type.
%
We will generate fresh @Var@s to name values of components, and will 
use @subst@ to replace the free occurrences of variables in a given \hbox{@RefType@.}
%
Suppose that @t@ is the @RefType@ represented by
\hbox{|$\meta{\reft{v}{[Score]}{k \leq len\ v}}$|.} Then,
%
\begin{itemize}
\item{@refinement t@} evaluates to |$\meta{k \leq len\ v}$| and
\item{|subst t [($\meta{k}$, $\meta{x_0}$)]|} evaluates to |$\meta{\reft{v}{[Score]}{x_0 \leq len\ v}}$|.
\end{itemize}

\mypara{Operating on Types} 
To build up compound values (\eg lists) from components 
(\eg an integer and a list), 
%
@unfold@ breaks a @RefType@ (\eg a list of integers) into its 
constituents (\eg an integer and a list of integers) at a given 
constructor (\eg ``cons'').
%
@binder@ simply extracts the @Var@ representing the
value being refined from the \hbox{@RefType@.}
%
To write generic functions over @RefType@s and use Haskell's
type class machinery to @query@ and @decode@ components of
types, we associate with each refinement type a \emph{proxy}
representing the corresponding Haskell type (in practice
this must be passed around as a separate argument).
%
For example, if @t@ is \hbox{|$\meta{\reft{v}{[Score]}{k \leq len\ v}}$|,} 
%
\begin{itemize}
\item{|unfold $\meta{:}$ t|} evaluates to |[($\meta{x}$, $\meta{Score}$), ($\meta{xs}$, $\meta{[Score]}$)]|,
\item{@binder t@} evaluates to |$\meta{v}$|, and
\item{@proxy t@} evaluates to a value of type @Proxy [Int]@.
\end{itemize}

\subsection{The \texttt{Targetable} Type Class}\label{sec:targetable}

% query  :: Haskell-Logic -> Gen SMT-Logic
% decode :: SMT-Value     -> Haskell-Value
% encode :: Haskell-Value -> SMT-Logic 

Following \quickcheck, we encapsulate the key operations needed
for type-targeted testing in a type class @Targetable@ 
(Figure~\ref{fig:targetable}). 
%
\begin{figure}
\begin{mdframed}
\begin{CenteredBox}
\begin{code}
class Targetable a where
  query  :: Proxy a -> Int -> RefType -> SMT Var
  decode :: Var -> SMT a
  check  :: a -> RefType -> SMT (Bool, Var)
  toReft :: a -> Refinement
\end{code}
\end{CenteredBox}
\end{mdframed}
\caption{The class of types that can be tested by \toolname}\label{fig:targetable}
\end{figure}
%
This class characterizes the set of types which can be tested 
by \toolname. All of the operations can interact with an external SMT 
solver, and so return values in an @SMT@ monad.

\begin{itemize}
\item{@query@} takes a \emph{proxy} for the Haskell type
   for which we are generating values, an integer 
   \emph{depth} bound, and a \emph{refinement type}
   describing the desired constraints, and generates a set of 
   logical constraints and a @Var@ that represents the 
   constrained value.

\item{@decode@} takes a @Var@, generated via a previous 
   @query@ and queries the model returned by the SMT solver
   to construct a Haskell value of type @a@.
 
\item{@check@} takes a value of type @a@, translates 
   it back into logical form, and verifies that it inhabits
   the output type @t@.
   
\item{@toReft@} takes a value of type @a@ and translates it
   back into logical form (a specialization of @check@).
\end{itemize}

\subsection{The Query-Decode-Check Loop}\label{sec:loop}



Figure~\ref{fig:arch} summarizes the overall implementation of 
\toolname, which takes as input a function @f@ and its refinement 
type specification @t@ and proceeds to test the function against 
the specification via a \emph{query-decode-check} loop:
%
(1) First, we translate the refined @inputTypes@ into a logical \emph{query}.
%
(2) Next, we \emph{decode} the model (\ie satisfying assignment) for the 
    query returned by the SMT solver to obtain concrete @inputs@.
%
(3) Finally, we @execute@ the function @f@ on the @inputs@ to get the 
    corresponding @output@, which we @check@ belongs to the specified 
    @outputType@. If the @check@ fails, we return the @inputs@ as a counterexample.
%
After each test, \toolname, refutes the given test to force the SMT 
solver to return a different set of inputs, and this process is repeated until 
a user specified number of iterations. The @checkSMT@ call may fail
to find a model meaning that we have exhaustively tested all inputs upto
a given @testDepth@ bound. If all iterations succeed, \ie no counterexamples
are found, then \toolname returns @Ok@, indicating that @f@ satisfies @t@ 
up to the given depth bound.

\begin{figure}[ht!]
\begin{mdframed}
\begin{CenteredBox}
\begin{code} 
target f t = do 
  let txs = inputTypes t
  vars  <- forM txs $ \tx -> 
             query (proxy tx) testDepth tx -- Query
  forM [1 .. testNum] $ \_ -> do
    hasModel <- checkSMT 
    when hasModel $ do
      inputs <- forM vars decode           -- Decode
      output <- execute f inputs                        
      let su = zip (map binder txs) (map toReft inputs)
      let to = outputType t `subst` su
      (ok,_) <- check output to            -- Check
      if ok then 
        refuteSMT 
      else 
        throw (CounterExample inputs)     
  return Ok
\end{code}
\end{CenteredBox}
\end{mdframed}
\caption{Implementing \toolname via a \emph{query-decode-check} loop}\label{fig:arch}
\end{figure}

%\section{Instantiating the \toolname Framework}\label{sec:list}

Next, we describe a concrete instantiation of \toolname for lists.
%
We start with a constraint generation API~(\S~\ref{sec:constraint}). 
%
Then we use the API to implement the key operations 
\hbox{@query@~(\S~\ref{sec:query}),} 
\hbox{@decode@~(\S~\ref{sec:decode}),} 
\hbox{@check@~(\S~\ref{sec:check}),} and
\hbox{@refuteSMT@~(\S~\ref{sec:refute}),} 
thereby enabling \toolname to automatically test functions over lists.
We omit the definition of @toReft@ as it follows directly from the
definition of @check@.
%
Finally, we show how the list instance can be generalized to algebraic 
datatypes and higher-order functions~(\S~\ref{sec:generic}).

\subsection{SMT Solver Interface}\label{sec:constraint}

Figure~\ref{fig:smt} describes the interface to the SMT 
solvers that \toolname uses for constraint generation and 
model decoding. The interface has functions to 
%
(a)~generate logical variables of type @Var@, 
%
(b)~constrain their values using @Refinement@ predicates, and
%
(c)~determine the values assigned to the variables in satisfying models.

\begin{figure}[ht!]
\begin{mdframed}
\begin{CenteredBox}
\begin{code} 
fresh     :: SMT Var
guard     :: Var -> SMT a      -> SMT a 
constrain :: Var -> Refinement -> SMT ()

apply     :: Ctor -> [Var] -> SMT Var 
unapply   :: Var  -> SMT (Ctor, [Var])

oneOf     :: Var -> [(Var, Var)] -> SMT ()
whichOf   :: Var -> SMT Var

eval      :: Refinement -> SMT Bool
\end{code}
\end{CenteredBox}
\end{mdframed}
\caption{SMT Solver API}\label{fig:smt}
\end{figure}

\begin{itemize}

\item{@fresh@} allocates a new logical variable.

\item{@guard b act@} ensures that all the constraints 
generated by @act@ are \emph{guarded by} the choice 
variable @b@. That is, if @act@ generates the constraint 
$p$ then @guard b act@ generates the (implication) 
constraint ${b \Rightarrow p}$.

\item{@constrain x r@} generates a constraint that @x@ 
satisfies the refinement predicate @r@.

\item{@apply c xs@} generates a new @Var@ for the folded up value 
obtained by applying the constructor @c@ to the fields @xs@,
while also generating constraints from the measures. For example, 
{|apply $\meta{:}$ [$\meta{x_1}$, $\meta{xs_1}$]|} returns |$\meta{\lcons{\cvar{x}_1}{\cvar{xs}_1}}$|
%%%% \ES{$\lcons{\cvar{x}_1}{\cvar{xs}_1}$ doesn't look like a Var.. 
%%%% I guess this works since we've said that Var = Refinement, and 
%%%% it \emph{should} end up generating the constraints from the 
%%%% overview, but still.. I think we could easily confuse people 
%%%% here because of preconceived notions of what a Var is.}
and generates the constraint
${\clen{(\lcons{\cvar{x}_1}{\cvar{xs}_1})} = 1 + \clen{\cvar{xs}_1}}$.

\item{@unapply x@} returns the @Ctor@ and @Var@s from which the input 
@x@ was constructed. 

\item{@oneOf x cxs@} generates a constraint that @x@ equals exactly
one of the elements of @cxs@. For example, 
{|oneOf $\meta{xs_0}$ [($\meta{c_{00}}$,$\meta{[]}$),($\meta{c_{01}}$,$\meta{x_1 : xs_1}$)]|} 
yields:
$$(\cvar{c}_{00} \Rightarrow \cvar{xs}_0 = \lnil) \wedge 
  (\cvar{c}_{01} \Rightarrow \cvar{xs}_0 = \lcons{\cvar{x}_1}{\cvar{xs}_1}) \wedge 
  (\cvar{c}_{00} \oplus \cvar{c}_{01})$$

\item{@whichOf x@} returns the particular alternative that was 
assigned to @x@ in the current model returned by the 
SMT solver. Continuing the previous example, if the model sets 
|$\meta{c_{00}}$| (resp. |$\meta{c_{01}}$|) to $\ttrue$, |whichOf $\meta{xs_0}$| returns 
|$\meta{[]}$| (resp \hbox{|$\meta{x_1 : xs_1}$|).}

\item{@eval r@} checks the validity of a refinement with no free variables. For
  example, |eval $\meta{len\ (1 : []) > 0}$| would return @True@.

\end{itemize}

\subsection{Query}\label{sec:query}

Figure~\ref{fig:query} shows the procedure for constructing a 
@query@ from a refined list type, \eg the one required as an input 
to the @best@ or @insert@ functions from \S~\ref{sec:overview}.

\begin{figure}[t!]
\begin{mdframed}
\begin{CenteredBox}
\begin{mcode}
query p d t = do
  let cs = ctors d
  bs <- forM cs (\_ -> fresh)
  xs <- zipWithM (queryCtor (d-1) t) bs cs
  x  <- fresh 
  oneOf x     (zip bs xs)
  constrain x (refinement t)
  return x

queryCtor d t b c = guard b (do
  let fts = unfold c t
  fs'    <- scanM (queryField d) [] fts
  x      <- apply c fs'
  return x)

queryField d su (f, t) = do
  f' <- query (proxy t) d (t `subst` su)
  return ((f, f') : su, f')                    
ctors d
  | d > 0     = [ $\meta{:}$, $\meta{[]}$ ]
  | otherwise = [ $\meta{[]}$ ]
\end{mcode}
\end{CenteredBox}
\end{mdframed}
\caption{Generating a Query}\label{fig:query}
\end{figure}
% queryField :: Int -> Subst -> (Var, RefType) -> Gen (Subst, Var) 
% queryCtor :: Int -> RefType -> Choice -> Ctor -> Gen Var





\mypara{Lists}
@query@ returns a @Var@ that represent \emph{all} lists up to 
depth @d@ that satisfy the logical constraints associated with 
the refined list type @t@.
%
To this end, it invokes @ctors@ to obtain all of the suitable
constructors for depth @d@. For lists, when
the depth is @0@ we should only use the |$\meta{[]}$| constructor,
otherwise we can use either |$\meta{:}$| or |$\meta{[]}$|. 
This ensures that @query@ terminates after encoding all possible
lists up to a given depth \hbox{@d@.}
%
Next, it uses @fresh@ to generate a distinct \emph{choice} 
variable for each constructor, and calls \hbox{@queryCtor@ to}
generate constraints and a corresponding symbolic @Var@ 
for each constructor. 
%
The choice variable for each constructor is supplied to 
@queryCtor@ to ensure that the constraints are \emph{guarded}, 
\ie only required to hold \emph{if} the corresponding choice 
variable is selected in the model and not otherwise.
%
Finally, a fresh @x@ represents the value at depth @d@ and 
is constrained to be @oneOf@ the alternatives represented 
by the constructors, and to satisfy the top-level refinement of @t@.
%
% Note that the refinements of the components of @t@ will have already 
% been used to constrain the individual alternatives @xs@.


\mypara{Constructors}
@queryCtor@ takes as input the refined list type @t@, 
a depth @d@, a particular constructor @c@ for the list 
type, and generates a query describing the \emph{unfolding}
of @t@ at the constructor @c@, guarded by the choice 
variable @b@ that determines whether this alternative 
is indeed part of the value.
%
These constraints are the conjunction of
those describing the values of the individual fields 
which can be combined via @c@ to obtain a @t@ value.
%
To do so, @queryCtor@ first @unfold@s the type @t@ at 
@c@, obtaining a list of constituent fields and their
respective refinement types @fts@. Next, it uses 
%
\begin{code}
  scanM :: Monad m => (a -> b -> m (a, c)) -> a -> [b] -> m [c]
\end{code}
%
to traverse the fields from left to right, building up 
representations of values for the fields from their 
unfolded refinement types.
%
Finally, we invoke @apply@ on @c@ and the fields @fs'@ to 
return a symbolic representation of the constructed value 
that is constrained to satisfy the measure properties of @c@.

\mypara{Fields}
@queryField@ generates the actual constraints for a
single field @f@ with refinement type @t@, by invoking
@query@ on @t@.  
%
The @proxy@ enables us to resolve the appropriate 
type-class instance for generating the query for 
the field's value.
%
Each field is described by a new symbolic name @f'@ which is 
@subst@ituted for the formal name of the field @f@ in the
refinements of subsequent fields, thereby tracking dependencies
between the fields.
%
For example, these substitutions ensure the values in 
the tail are greater than the head as needed by 
@OrdList@ from \S~\ref{sec:overview}.

\subsection{Decode}\label{sec:decode}
%
\begin{figure}[t!]
\begin{mdframed}
\begin{minipage}{0.45\textwidth}
\begin{CenteredBox}
\begin{code}
decode x = do 
  x'      <- whichOf x
  (c,fs') <- unapply x'
  decodeCtor c fs'
\end{code}
\end{CenteredBox}
\end{minipage}
%
\begin{minipage}{0.55\textwidth}
\begin{CenteredBox}
\begin{mcode}
decodeCtor $\meta{[]}$ []    = return []
decodeCtor $\meta{:}$ [x,xs] = do
  v  <- decode x
  vs <- decode xs
  return (v:vs)
\end{mcode}
\end{CenteredBox}
\end{minipage}
\end{mdframed}
\caption{Decoding Models into Haskell Values}\label{fig:decode}
\end{figure}
%
Once we have generated the constraints we query the SMT solver 
for a model, and if one is found we must \emph{decode} it into
a concrete Haskell value with which to test the given function.
Figure~\ref{fig:decode} shows how to decode an SMT model for lists. 

\mypara{Lists} @decode@ takes as input the top-level symbolic
representation @x@ and queries the model to determine which
alternative was assigned by the solver to @x@, \ie a nil or a cons.
Once the alternative is determined, we use @unapply@ to destruct 
it into its constructor @c@ and fields @fs'@, which are recursively
decoded by @decodeCtor@.

\mypara{Constructors} @decodeCtor@ takes the constructor @c@ and
a list of symbolic representations for fields, and decodes each 
field into a value and applies the constructor to obtain the 
Haskell value.
%
For example, in the case of the |$\meta{[]}$| constructor, there are no
fields, so we return the empty list. In the case of the |$\meta{:}$| 
constructor, we decode the head and the tail, and cons them to 
return the decoded value. 
%
@decodeCtor@ has the type
%
\begin{code}  
  Targetable a => Ctor -> [Var] -> SMT [a]
\end{code}
%
\ie if @a@ is a decodable type, then @decodeCtor@ suffices to decode lists of @a@.
%
Primitives like integers that are directly encoded 
in the refinement logic are the base case -- \ie the 
value in the model is directly translated into the 
corresponding Haskell value.


\subsection{Check}\label{sec:encode}\label{sec:check}
%
\begin{figure}[t!]
\begin{mdframed}
\begin{CenteredBox}
\begin{mcode}
check v t = do
  let (c,vs) = splitCtor v
  let fts    = unfold c t
  (bs, vs') <- fmap unzip (scanM checkField [] (zip vs fts))
  v'        <- apply c vs'
  let t'     = t `subst` [(binder t, v')]
  b'        <- eval (refinement t')
  return (and (b:bs), v')
  
checkField su (v, (f, t)) = do
  (b, v') <- check v (t `subst` su)
  return ((f, v') : su, (b, v'))

splitCtor []     = ($\meta{[]}$, [])
splitCtor (x:xs) = ($\meta{:}$, [x,xs])
\end{mcode}
\end{CenteredBox}
\end{mdframed}
\caption{Checking Outputs}\label{fig:check}
\end{figure}
%
The third step of the query-decode-check loop is to verify
that the output produced by the function under test indeed
satisfies the output refinement type of the function.
%
We accomplish this by \emph{encoding} the output value as a
logical expression, and evaluating the output refinement
applied to the logical representation of the output value.

@check@, shown in Figure~\ref{fig:check}, takes a Haskell
(output) value @v@ and the (output) refinement type @t@, and
recursively verifies each component of the output type. It
converts each component into a logical representation,
@subst@itutes the logical expression for the symbolic value,
and @eval@uates the resulting @Refinement@.

%% We accomplish this by \emph{encoding} the output value as a 
%% set of of logical constraints, and then querying the SMT 
%% solver to check consistency with the output type.

%% @check@, shown in Figure~\ref{fig:check}, takes a Haskell
%% (output) value @v@ and the (output) refinement type @t@,
%% converts @v@ into a logical representation while adding the
%% constraints imposed by @t@, and then checks the consistency
%% of the resulting constraint system.
%% %
%% @encode@ recursively translates the Haskell value into a set
%% of logical constraints, by unfolding the type @t@ to obtain
%% the types of the value's compenents, encoding the components
%% to obtain symbolic fields, invoking @apply@ on the fields to
%% assert the measure properties of the constructor, and
%% finally constraining the resulting symbolic value to satisfy
%% @t@'s refinement.

\subsection{Refuting Models} \label{sec:refute}

Finally, \toolname invokes @refuteSMT@ to \emph{refute} a 
given model in order to force the SMT solver to produce a 
different model that will yield a different test input.
%
A na\"{\i}ve implementation of refutation is as follows.
%
Let $X$ be the set of all variables appearing in the constraints.
%
Suppose that in the current model, each variable $x$ is assigned 
the value $\val{x}$.
%
Then, to refute the model, we add a \emph{refutation constraint} 
$
\vee_{x \in X} x \not = \val{x}
$.
That is, we stipulate that \emph{some} variable be assigned a 
different value.

The na\"{\i}ve  implementation is extremely inefficient.
The SMT solver is free to pick a different value for some 
\emph{irrelevant} variable which was not even used for decoding.
%
As a result, the next model can, after decoding, yield the 
\emph{same} Haskell value, thereby blowing up the number of 
iterations needed to generate all tests of a given size.

\toolname solves this problem by forcing the SMT solver to return
models that yield \emph{different decoded tests} in each iteration.
To this end \toolname restricts the refutation constraint to 
the set of variables that were actually used to @decode@ the 
Haskell value.
%
We track this set by instrumenting the @SMT@ monad to log the 
set of variables and choice-variables that are transitively 
queried via the recursive calls to @decode@.
%
That is, each call to @decode@ logs its argument, and each call 
to @whichOf@ logs the choice variable corresponding to the 
alternative that was returned.
%
Let $R$ be the resulting set of \emph{decode-relevant} variables.
\toolname refutes the model by using a \emph{relevant refutation constraint}
$
\vee_{x \in R} x \not = \val{x}
$
which ensures that the next model decodes to a different value.% than all preceding ones.


\subsection{Generalizing \toolname To Other Types}\label{sec:generic}

The implementation in \S~\ref{sec:list} is for % refined 
List types, but @ctors@, @decodeCtor@, and @splitCtor@ are the only 
functions that are List-specific. 
%
Thus, we can easily generalize the implementation to:
%
\begin{itemize}
%
\item{\emph{primitive datatypes}}, \eg integers, by returning an empty 
    list of constructors,
%    
\item{\emph{algebraic datatypes}}, by implementing @ctors@, @decodeCtor@, and @splitCtor@ for that type.
%
\item{\emph{higher-order functions}}, by lifting instances of @a@ to functions returning @a@.
\end{itemize}

\mypara{Algebraic Datatypes}
% 
Our List implementation has three pieces of type-specific logic:
%
\begin{itemize}
\item{@ctors@}, which returns a list of constructors to unfold;
%
\item{@decodeCtor@}, which decodes a specific @Ctor@; and
%
\item{@splitCtor@}, which splits a Haskell value into a pair of its @Ctor@ and fields. 
\end{itemize}

Thus, to instantiate \toolname on a new data type, all we need is to 
implement these three operations for the type. This implementation
essentially follows the concrete template for Lists.
In fact, we observe that the recipe is entirely mechanical boilerplate,  
and can be fully automated for \emph{all} algebraic data types by using 
a \emph{generics} library.

Any algebraic datatype (ADT) can be represented as a \emph{sum-of-products} 
of component types. A generics library, such as \GhcGenerics~\cite{magalhaes_generic_2010}, 
provides a \emph{univeral} sum-of-products type and functions to automatically 
convert any ADT to and from the universal representation.
Thus, to obtain @Targetable@ instances for \emph{any} ADT it suffices
to define a @Targetable@ instance for the \emph{universal} type.

Once the universal type is @Targetable@ we can automatically get an 
instance for any new user-defined ADT (that is an instance of @Generic@) as follows:
%
(1)~to generate a \emph{query} we simply create a query for 
    \GhcGenerics' universal representation of the refined type,
%
(2)~to \emph{decode} the results from the SMT solver, we 
    decode them into the universal representation and then use 
    \GhcGenerics to map them back into the user-defined type,
%
(3)~to \emph{check} that a given value inhabits a user-defined 
    refinement type, we check that the universal representation 
    of the value inhabits the type's universal counterpart.

The @Targetable@ instance for the universal representation is a 
generalized version of the List instance from \S~\ref{sec:list}, 
that relies on various technical details of \GhcGenerics.
% Thus, we defer it to
% \ifthenelse{\equal{\isTechReport}{true}}{Appendix~\ref{sec:genericapp}}{the Appendix}.

\mypara{Higher Order Functions} 
Our type-directed approach to specification makes it easy to extend
\toolname to higher-order functions. Concretely, it suffices to 
implement a type-class instance:
%
\begin{code}
  instance (Targetable input, Targetable output) 
    => Targetable (input -> output)
\end{code}
%
In essence, this instance uses the @Targetable@ 
instances for @input@ and @output@ to 
create an instance for functions from @input -> output@,
after which Haskell's type class machinery suffices to 
generate concrete function values.

To create such instances, we use the insight from 
\quickcheck, that to generate (constrained) functions,
we need only to generate \emph{output} values for the function. 
%
Following this route, we generate functions by creating 
new lambdas that take in the inputs from the calling context, 
and use their values to create queries for the output, after 
which we can call the SMT solver and decode the results 
to get concrete outputs that are returned by the lambda, 
completing the function definition. 
%
Note that we require @input@ to also be @Targetable@
so that we can encode the Haskell value in the refinement logic,
in order to constrain the output values suitably.
%
We additionally memoize the generated function to preserve the 
illusion of purity. 
%
It is also possible to, in the future, extend our 
implementation to refute functions by asserting 
that the output value for a given input be distinct 
from any previous outputs for that input.

%\section{Instantiating \toolname Generically}\label{sec:genericapp}

\GhcGenerics defines separate types for products, data constructors, sums, and
datatypes; and uses the @TypeFamilies@ extension~\cite{Chakravarty_ATS_2005} to define an 
associated generic representation @Rep a@ for any algebraic datatype. For 
example, the standard Haskell list would be represented by the generic type
%
\begin{code}
  Rep [a] = C1 U1 :+: C1 (Rec0 a :*: Rec0 [a])
\end{code}
%
where @C1@ denotes a data constructor, @U1@ an empty product (\eg for a nullary
constructor), @:+:@ a sum, and @:*:@ a product. Additionally, @Rec0@ indicates a
reference to a user-defined type, \ie values are translated to and from the
universal representation \emph{as-needed}. We omit some of the metadata to
highlight the structural similarity between the generic representation and the
original data definition.

\ES{TODO: explain that insight of ghc-generics is that you can treat all sums
  equally and all products equally, so general approach is to define two
  type-classes: one that handles sums and another that handles products.}

\ES{TODO: explain that generic rep is tree-structured, NOT list.}

% 3 type-specific pieces of overall approach:
%   1. obtaining list of @Con@s to unfold at a given depth
%   2. decoding a specific @Con@
%   3. encoding a @Con@ 

Recall that our implementation from \S~\ref{sec:list} contained three
pieces of type-specific logic, namely
%
(1) obtaining a list of @Ctor@s to unfold at a given depth (@ctors@),
%
(2) decoding a specific @Ctor@ (@decodeCtor@), and
%
(3) encoding a Haskell value as a logical expression (@encode@).
%
We now demonstrate how to generically implement these three steps for any
algebraic datatype, but first we will need two extensions to our refinement type
API, which we describe in Figure~\ref{fig:rtype-ext}.

\begin{figure}
\begin{mdframed}
\begin{CenteredBox}
\begin{code} 
ctorArity :: Ctor -> Int
mkCtor    :: Proxy (C1 c f) -> Ctor
\end{code}
\end{CenteredBox}
\end{mdframed}
\caption{Extensions to the refinement type API from Figure~\ref{fig:rtype}}\label{fig:rtype-ext}
\end{figure}

\begin{itemize}
\item{@ctorArity@} returns the number of fields that a @Ctor@ has.
\item{@mkCtor@} constructs a @Ctor@ from a \emph{proxy} for the constructor.
\end{itemize}

\subsection{Listing Constructors}\label{sec:generic-constructors}
Let us begin by writing a function @gCtors@ that will work just like @ctors@,
but for any datatype, \ie it will return a list of @Ctor@s that should be
unfolded at the given depth @d@. As is standard for \GhcGenerics we will define
a type-class for @gCtors@ and provide instances for sums, products, etc.
\GhcGenerics uses a number of different types for which we must provide class
instances, but only a few of the instances are interesting, which we show in
Figure~\ref{fig:generic-query}.
%
\begin{figure}[ht]
\begin{mdframed}
\begin{CenteredBox}
\begin{code}
  class GCtors f where
    gCtors :: Proxy f -> Int -> [Ctor]

  instance (GCtors f, GCtors g) => GCtors (f :+: g) where
    gCtors _ d = gCtors (Proxy :: Proxy f) d 
              ++ gCtors (Proxy :: Proxy g) d

  instance GCtors f => GCtors (C1 c f) where
    gCtors p 0
      | conArity c == 0 = [c]
      | otherwise       = []
      where
        c = mkCtor p
    gCtors p d
      = [mkCtor p]
\end{code}
\end{CenteredBox}
\end{mdframed}
\caption{Generic query generation}\label{fig:generic-query}
\end{figure}
%
For example, to obtain the list of @Ctor@s for a sum we simply concatenate the
lists obtained from the left- and right-hand sides of the sum.  When we reach a
specific constructor, we compare the constructor's arity with the depth; when we
reach depth 0 we only want to unfold \emph{nullary} constructors~\footnote{In
  practice one might want to do something smarter, like checking the minimum
  depth required to unfold the constructor.}.
\ES{The (C1 c f) might be confusing because the "c" was omitted in the "Rep [a]"
  example above...}
Now we can replace the call to @ctors@ in @queryList@ with
%
\begin{code}
  -- reproxyRep :: Proxy a -> Proxy (Rep a)
  let cs = gCtors (reproxyRep $ proxy t) d
\end{code}
%
\ES{i'm pretty sure (reproxyRep \$ proxy t) won't typecheck due to the existential..}
making our @query@ implementation fully datatype-generic.

% instance GCons f => GCons (D1 d f) where
%   gCons _ d = gCons (Proxy :: Proxy f) d

% class GHasDepth f where
%   gHasDepth :: Proxy f -> Int -> Bool

% instance (GHasDepth f, GHasDepth g) => GHasDepth (f :*: g) where
%   gHasDepth _ d = gHasDepth (Proxy :: Proxy f) d 
%                && gHasDepth (Proxy :: Proxy g) d

% instance GHasDepth U1 where
%   gHasDepth _ _ = True
  
% instance GHasDepth (Rec0 a) where
%   gHasDepth _ 0 = False
%   gHasDepth _ _ = True

\subsection{Decode}\label{sec:generic-decode}
Next we will tackle the process of \emph{decoding} a specific constructor from
the model. As above, we will define a type-class and show only the interesting
instances in Figure~\ref{fig:generic-decode}.
%
\begin{figure}[ht]
\begin{mdframed}
\begin{CenteredBox}
\begin{code}
  class GDecode f where
    gDecode :: Ctor -> [Var] -> Gen f

  instance (GDecode f, GDecode g) => GDecode (f :+: g) where
    gDecode c vs =  L1 <$> gDecode c vs
                <|> R1 <$> gDecode c vs

  instance GDecodeFields f => GDecode (C1 c f) where
    gDecode c vs 
      | c == mkCtor (Proxy :: Proxy (C1 c f))
      = C1 . snd <$> gDecodeFields vs
      | otherwise
      = empty

  class GDecodeFields f where
    gDecodeFields :: [Var] -> Gen ([Var], f)

  instance Targetable a => GDecodeFields (Rec0 a) where
    gDecodeFields (v:vs) = do
      x <- decode v
      return (vs, Rec0 x)
\end{code}
\end{CenteredBox}
\end{mdframed}
\caption{Generic decoding of Haskell values}\label{fig:generic-decode}
\end{figure}

Given an arbitrary sum, we do not know whether the constructor we are looking
for is in the left or right sub-sum, so we must try both.
%
Once we reach an individual constructor, we can check whether it is the correct
constructor using the forementioned @mkCtor@ function. If the check is
successful, we can go ahead and decode the constructor's fields using
@gDecodeFields@ and wrap them up, otherwise we signal that the next element of
the sum should be tried.

@gDecodeFields@ comes from an auxiliary type-class that we use to decode the
fields of a product. As \GhcGenerics represents sums and products as
\emph{trees} instead of lists, we have @gDecodeFields@ return the list of
@Var@s that still need to be decoded in addition to the decoded value.
%
Again, most of the instances are uninteresting and simply involve traversing the
product while decoding each field. The interesting instance arises when we want
to encode an individual field. Recall that products are represented with
recursive references to the user-defined type, \eg @Rec0 [a]@. So when we reach
an individual field, we will have to decode a value of the \emph{user-defined}
type.

We can now replace the @decodeCtor c fs'@ in our original implementation of @decode@
with
%
\begin{code}
  -- to :: Generic a => Rep a -> a
  to <$> gDecode c xs
\end{code}
%
% instance GDecode f => GDecode (D1 d f) where
%   gDecode c vs = D1 <$> gDecode c vs
%
% instance (GDecodeFields f, GDecodeFields g) => GDecodeFields (f :*: g) where
%   gDecodeFields vs = do
%     (vs', ls)  <- gDecodeFields vs
%     (vs'', rs) <- gDecodeFields vs'
%     return (vs'', ls :*: rs)
%     
% instance GDecodeFields U1 where
%   gDecodeFields vs = return (vs, U1)
%
\subsection{Check}\label{sec:generic-check}
Finally, let us examine how to generically \emph{check} that Haskell value 
inhabits a refinement type with the type-class in Figure~\ref{fig:generic-check}.
%
\begin{figure}[ht]
\begin{mdframed}
\begin{CenteredBox}
\begin{code}
  class GCheck f where
    gCheck :: f -> RefType -> Gen (Bool,Var)
    
  class GCheckFields f where
    gCheckFields :: f -> [(Var, RefType)]
                 -> SMT (Bool, [Var], [(Var, RefType)])
                 
  instance GCheckFields f => GCheck (C1 c f) where
    gCheck (C1 f) t = do
      let c       = mkCtor (Proxy :: Proxy (C1 c f))
      let fts     = unfold c t
      (b, vs, _) <- gCheckFields f fts
      v          <- apply c vs
      let t'      = t `subst` [(binder t, v)]
      b'         <- eval t'
      return (b', v)
      
  instance Targetable a => GCheckFields (Rec0 a) where
    gCheckFields (Rec0 a) ((f, t) : fts) = do
      (b, v)  <- check a t
      let fts' = fts `subst` [(f, v)]
      return (b, [v], fts')
\end{code}
\end{CenteredBox}
\end{mdframed}
\caption{Generic checking of Haskell values against refinement types.}\label{fig:generic-check}
\end{figure}

Checking a sum just involves stripping away levels of indirection until we reach
the actual constructor, at which point we need to unfold the constructor and
check its fields. We then @apply@ the constructor to the resulting symbolic
values, @subst@itute the resulting @Refinement@ for @t@'s @binder@ and
@eval@uate the result.

@gCheckFields@ checks the fields of a product, and is itself a type-class method.
%
As with @gDecodeFields@ the only interesting instance of @GCheckFields@ deals
with checking an individual field, where we have a value of the user-defined
type and must use the original @check@ method.

Now we can provide a default implementation for @check@
%
\begin{code}
  -- from :: Generic a => a -> Rep a
  check v t = gCheck (from v) t
\end{code}
%
thus replacing the last bit of type-specific logic in our @Targetable@
implementation.

% instance GEncode f => GEncode (D1 d f) where
%   gEncode (D1 f) t = gEncode f t

% instance (GEncode f, GEncode g) => GEncode (f :+: g) where
%   gEncode (L1 f) = gEncode f
%   gEncode (R1 g) = gEncode g

% instance (GEncodeFields f, GEncodeFields g) => GEncodeFields (f :*: g) where
%   gEncodeFields (f :*: g) ts = do
%     (fs,ts')  <- gEncodeFields f ts
%     (gs,ts'') <- gEncodeFields g ts'
%     return (fs ++ gs, ts'')

% instance GEncodeFields U1 where
%   gEncodeFields U1 = []


%%% Local Variables:
%%% mode: latex
%%% TeX-master: "main"
%%% End:

%\section{Evaluation} \label{sec:evaluation}

We have built a prototype implementation of \toolname\footnote{\url{http://hackage.haskell.org/package/target-0.1.1.0}} and next, 
describe an evaluation on a series of benchmarks ranging from 
textbook examples of algorithms and data structures to widely 
used Haskell libraries like \textsc{containers} and \textsc{xmonad}.
%
Our goal in this evaluation is two-fold. 
%
First, we describe micro-benchmarks (\ie functions)
that \emph{quantitatively compare} \toolname with 
the existing state-of-the-art, property-based testing
tools for Haskell -- namely \smallcheck and \quickcheck\ -- 
to determine whether \toolname is indeed able to generate
highly constrained inputs more effectively.
%
Second, we describe macro-benchmarks (\ie modules) that 
evaluate the amount of \emph{code coverage} that we 
get from type-targeted testing.
%
%% Third, using our results as a base, we present a 
%% qualitative discussion of \toolname as a \emph{gradual} 
%% approach that bridges informal and formal verification.

%% ES: This doesn't actually make any sense..
% An important optimization in our implementation is to 
% perform the post-condition checking in Haskell instead of by querying
% the SMT solver. We accomplish this by adding an additional @toReft@
% method to @Targetable@ that translates the concrete inputs that we
% @decode@ \emph{back} into logical expressions,
% %
% \begin{code}
%   toReft :: a -> Var
% \end{code}
% %
% which we then substitute into the output-type before checking the
% concrete output. @toReft@ can be simply implemented as:
% %
% \begin{code}
%   toReft v = app c (map toReft vs)
%     where (c, vs) = splitCtor v
%           -- app :: Ctor -> [Var] -> Var
% \end{code}
% %
% where @app@ is a pure version of @apply@, \ie it constructs a
% logical expression like @apply@ but does nothing else.

% \newcommand\XX{\multicolumn{1}{c}{X}}
% \newcommand{\mysec}[1]{\SI{#1}{\second}}
\begin{figure}[ht!]
  \centering
  % \includegraphics[width=0.49\linewidth]{figs/List-insert}
  % \includegraphics[width=0.49\linewidth]{figs/RBTree-add}
  % \includegraphics[width=0.49\linewidth]{figs/Map-delete}
  % \includegraphics[width=0.49\linewidth]{figs/Map-difference}
  % \includegraphics[width=0.49\linewidth]{figs/XMonad-focus-left}
  % \begin{tikzpicture}
  %    \begin{customlegend}[legend columns=4,legend style={align=center,draw=none},legend entries={\toolname,\smallcheck,\lazysmallcheck,\lazysmallcheck (slow)}]
  %    \addlegendimage{color=blue,mark=square*}
  %    \addlegendimage{color=red,mark=*}   
  %    \addlegendimage{color=orange,mark=diamond*}
  %    \addlegendimage{color=black,mark=x}
  %    \end{customlegend}
  % \end{tikzpicture}

  \begin{tikzpicture}
    \begin{groupplot}[
      group style = {group size = 3 by 1, horizontal sep=15pt,},
      groupplot ylabel={Time (sec)},
      groupplot xlabel={Depth},
      group/only outer labels,
      ymode=log,
      ymax=10000,
      ymin=0.0001
    ]
    % \begin{semilogyaxis}[
    \nextgroupplot[
      title=\textsc{List.insert}
    ]
    \addplot table[smooth,col sep=comma,x index=0,y index=1] {target/csv/List.insert.csv};
    \addplot table[smooth,col sep=comma,x index=0,y index=2] {target/csv/List.insert.csv};
    \addplot table[smooth,col sep=comma,x index=0,y index=3] {target/csv/List.insert.csv};
    % \end{semilogyaxis}
  % \end{tikzpicture}
  % \begin{tikzpicture}
    % \begin{semilogyaxis}[
    \nextgroupplot[
      title=\textsc{RBTree.add},
      legend columns=4,
      legend entries={\toolname,\smallcheck,\lazysmallcheck,\lazysmallcheck (slow)},
      legend to name=legend,
    ]
    \addplot table[smooth,col sep=comma,x index=0,y index=1] {target/csv/RBTree.add.csv};
    \addplot table[smooth,col sep=comma,x index=0,y index=2] {target/csv/RBTree.add.csv};
    \addplot table[smooth,col sep=comma,x index=0,y index=3] {target/csv/RBTree.add.csv};
    \addplot table[smooth,col sep=comma,x index=0,y index=4] {target/csv/RBTree.add.csv};
    % \end{semilogyaxis}
  % \end{tikzpicture}
  % \begin{tikzpicture}
    % \begin{semilogyaxis}[
    \nextgroupplot[
      title=\textsc{XMonad.focus\_left}
    ]
    \addplot table[smooth,col sep=comma,x index=0,y index=1] {target/csv/XMonad.focus_left.csv};
    \addplot table[smooth,col sep=comma,x index=0,y index=2] {target/csv/XMonad.focus_left.csv};
    \addplot table[smooth,col sep=comma,x index=0,y index=3] {target/csv/XMonad.focus_left.csv};
    % \end{semilogyaxis}
    \end{groupplot}
  \end{tikzpicture}
  \begin{tikzpicture}
    \begin{groupplot}[
      group style = {group size = 2 by 1, horizontal sep=15pt,},
      groupplot ylabel={Time (sec)},
      groupplot xlabel={Depth},
      group/only outer labels,
      ymode=log,
      ymax=10000,
      ymin=0.0001
    ]
    % \begin{semilogyaxis}[
    \nextgroupplot[
      title=\textsc{Map.delete}
    ]
    \addplot table[smooth,col sep=comma,x index=0,y index=1] {target/csv/Map.delete.csv};
    \addplot table[smooth,col sep=comma,x index=0,y index=2] {target/csv/Map.delete.csv};
    \addplot table[smooth,col sep=comma,x index=0,y index=3] {target/csv/Map.delete.csv};
    \addplot table[smooth,col sep=comma,x index=0,y index=4] {target/csv/Map.delete.csv};
    % \end{semilogyaxis}
  % \end{tikzpicture}
  % \begin{tikzpicture}
    % \begin{semilogyaxis}[
    \nextgroupplot[
      title=\textsc{Map.difference}
    ]
    \addplot table[smooth,col sep=comma,x index=0,y index=1] {target/csv/Map.difference.csv};
    \addplot table[smooth,col sep=comma,x index=0,y index=2] {target/csv/Map.difference.csv};
    \addplot table[smooth,col sep=comma,x index=0,y index=3] {target/csv/Map.difference.csv};
    \addplot table[smooth,col sep=comma,x index=0,y index=4] {target/csv/Map.difference.csv};
    % \end{semilogyaxis}
    \end{groupplot}
  \end{tikzpicture}\\
  \ref{legend}

  \caption{Results of comparing \toolname with \quickcheck, \smallcheck, and Lazy
    \smallcheck on a series of functions. \toolname, \smallcheck, and Lazy
    \smallcheck were both configured to check the first 1000 inputs that
    satisfied the precondition at increasing depth parameters, with a 60 minute
    timeout per depth; \quickcheck was run with the default settings, \ie it had
    to produce 100 test cases. \toolname, \smallcheck, and \lazysmallcheck were
    configured to use the same notion of depth, in order to ensure they would
    generate the same number of valid inputs at each depth level. \quickcheck was
    unable to successfully complete any run due to the low probability of
    generating valid inputs at random.}\label{fig:comparisonresults}
\end{figure}



\subsection{Comparison with \quickcheck and \smallcheck}\label{sec:comparison}

We compare \toolname with \quickcheck and \smallcheck by using 
a set of benchmarks with highly constrained inputs. 
%
For each benchmark we compared \toolname with \smallcheck and
\quickcheck, with the latter two using the generate-and-filter 
approach, wherein a value is generated and subsequently discarded if
it does not meet the desired constraint.
%
While one could possibly write custom ``operational'' generators 
for each property, the point of this evaluation is compare the 
different approaches ability to enable ``declarative'' specification 
driven testing.
%
Next, we describe the benchmarks and then summarize the results of the comparison
(Figure~\ref{fig:comparisonresults}).



\mypara{Inserting into a sorted \List}
%
Our first benchmark is the \Insert function from the homonymous 
sorting routine. We use the specification that given an element 
and a sorted list, @insert x xs@ should evaluate to a sorted list.
We express this with the type
%
\begin{code}
  type Sorted a = List <{\hd v -> hd < v}> a
  insert :: a -> Sorted a -> Sorted a
\end{code}
%
where the ordering constraint is captured by an abstract 
refinement~\cite{Vazou13} which states that \emph{each} 
list head @hd@ is less than every element @v@ in its tail.

\mypara{Inserting into a Red-Black Tree}
%
Next, we consider insertion into a Red-Black tree.
%
\begin{code}
  data RBT a = Leaf  | Node Col a (RBT a) (RBT a)
  data Col   = Black | Red
\end{code}
%
Red-black trees must satisfy three invariants:
%
(1)~red nodes always have black children,
(2)~the black height of all paths from the root to a leaf is the same, and
(3)~the elements in the tree should be ordered.
%
We capture (1) via a measure that recursively checks each @Red@ node has @Black@ children.
%
\begin{code}
  measure isRB :: RBT a -> Prop
  isRB Leaf           = true
  isRB (Node c x l r) = isRB l && isRB r &&
                        (c == Red => isBlack l && isBlack r)
\end{code}
%
We specify (2) by defining the @Black@ height as:
%
\begin{code}
  measure bh :: RBT a -> Int
  bh Leaf           = 0
  bh (Node c x l r) = bh l + (if c == Red then 0 else 1)
\end{code}
%
and then checking that the @Black@ height of both subtrees is the same:
%
\begin{code}
  measure isBH :: RBT a -> Prop
  isBH Leaf           = true
  isBH (Node c x l r) = isBH l && isBH r && bh l == bh r
\end{code}
%
Finally, we specify the (3), the ordering invariant as:
%
\begin{code}
  type OrdRBT a = RBT <{\r v -> v < r}, {\r v -> r < v}> a
\end{code}
%
\ie with two abstract refinements for the left and right subtrees
respectively, which state that the root @r@ is greater than (resp. less than)
each element @v@ in the subtrees. Finally, a valid Red-Black tree is:
%
\begin{code}
  type OkRBT a = {v:OrdRBT a | isRB v && isBH v}
\end{code}
%
Note that while the specification for the \emph{internal} invariants for Red-Black
trees is tricky, the specification for the public API -- \eg the @add@ function -- 
is straightforward:
%
\begin{code}
  add :: a -> OkRBT a -> OkRBT a
\end{code}

\mypara{Deleting from a Data.Map}\label{sec:delete-from-map}
%
Our third benchmark is the @delete@ function from the \hbox{@Data.Map@} module in 
the Haskell standard libraries. The @Map@ structure is a balanced binary
search tree that implements purely functional key-value dictionaries:
%
\begin{code}
  data Map k a = Tip | Bin Int k a (Map k a) (Map k a)
\end{code}
%
A valid @Data.Map@ must satisfy two properties:
%
(1)~the size of the left and right sub-trees must be 
    within a factor of three of each other, and
(2)~the keys must obey a binary search ordering.
%
We specify the balancedness invariant~(1) with a measure
%
\begin{code}
  measure isBal :: Map k a -> Prop
  isBal (Tip)           = true
  isBal (Bin s k v l r) = isBal l && isBal r &&
                          (sz l + sz r <= 1 ||
                           sz l <= 3 * sz r <= 3 * sz l)
\end{code}
%
and combine it with an ordering invariant (like @OrdRBT@) to specify valid trees.
%
\begin{code}
  type OkMap k a = {v : OrdMap k a | isBal v}
\end{code}
%
We can check that @delete@ preserves the invariants by 
checking that its output is an @OkMap k a@.
However, we can also go one step further and check 
the functional correctness property that @delete@ 
removes the given key, with a type:
%
\begin{code}
  delete :: Ord k => k:k -> m:OkMap k a 
         -> {v:OkMap k a | MinusKey v m k}
\end{code}
%
where the predicate @MinusKey@ is defined as:
%
\begin{code}
  predicate MinusKey M1 M2 K 
    = keys M1 = difference (keys M2) (singleton K)
\end{code}
%
using the measure @keys@ describing the contents of the @Map@:
%
\begin{code}
  measure keys :: Map k a -> Set k
  keys (Tip)           = empty () 
  keys (Bin s k v l r) = union (singleton k) 
                               (union (keys l) (keys r))
\end{code}

\mypara{Refocusing XMonad StackSets} \label{sec:refocus-stackset}
%
Our last benchmark comes from the tiling window manager XMonad. 
%
The key invariant of XMonad's internal @StackSet@ data structure 
is that the elements (windows) must all be \emph{unique}, \ie contain
no duplicates.
%
XMonad comes with a test-suite of over 100 \quickcheck properties;
we select one which states that moving the focus between windows 
in a @StackSet@ should not affect the \emph{order} of the windows.
%
\begin{code}
  prop_focus_left_master n s =
    index (foldr (const focusUp) s [1..n]) == index s
\end{code}
%
With \quickcheck, the user writes a custom generator for valid @StackSet@s
and then runs the above function on test inputs created by the generator, 
to check if in each case, the result of the above is @True@.

With \toolname, it is possible to test such properties \emph{without} 
requiring custom generators. Instead the user writes a declarative 
specification:
%
\begin{code}
  type OkStackSet = {v:StackSet | NoDuplicates v}
\end{code}
%
(We refer the reader to~\cite{VazouRealWorld14} for a full 
discussion of how to specify @NoDuplicates@).
%
Next, we define a refinement type:
%
\begin{code}
  type TTrue = {v:Bool | Prop v}
\end{code}
%
that is only inhabited by @True@, and use it to type the \quickcheck 
property as:
%
\begin{code}
  prop_focus_left_master :: Nat -> OkStackSet -> TTrue 
\end{code}
%
This property is particularly difficult to \emph{verify}; however,
\toolname is able to automatically
generate valid inputs to \emph{test} that @prop_focus_left_master@
always returns @True@.

%%% The high level of abstraction inherent in the @StackSet@ definition
%%% works in our favor here, as we can instantiate the relevant type parameter (the
%%% window) to \Char and leave the others as @()@ to drastically reduce
%%% the search space.


\mypara{Results}
%
Figure~\ref{fig:comparisonresults} summarizes the results of the comparison.
%
\quickcheck was unable to successfully complete \emph{any} 
benchmark to the low probability of generating properly 
constrained values at random.

\begin{description}
\item[List Insert] \toolname is able to test @insert@ all the way to 
   depth 20, whereas \lazysmallcheck times out at depth 19.

\item[Red-Black Tree Insert] \toolname is able to test @add@ up to depth 12,
  while \lazysmallcheck times out at depth 6.
  
\item[Map Delete] \toolname is able to check @delete@ up to depth 10, whereas
   \lazysmallcheck times out at depth 7 if it checks ordering first,
    or depth 6 if it checks balancedness first.

\item[StackSet Refocus] \toolname and is able to check this property 
    up to depth 8, while \lazysmallcheck times out at depth 7.
\end{description}

\toolname sees a performance hit with properties 
that require reasoning with the theory of Sets \eg 
the no-duplicates invariant of @StackSet@. 
%
While \lazysmallcheck times out at a higher depths, when it completes
\eg at depth 6, it does so in 0.7s versus \toolname's 9 minutes.
%
We suspect this is because the theory of sets are a relatively recent
addition to SMT solvers \cite{arrayZ3}, and with further improvements 
in SMT technology, these numbers will get significantly better.


Overall, we found that for \emph{small inputs} \lazysmallcheck 
is substantially faster as exhaustive enumeration is tractable,
and does not incur the overhead of communicating with an external 
general-purpose solver.
%
Additionally, \lazysmallcheck benefits from pruning predicates 
that exploit laziness and only force a small portion of the 
structure (\eg ordering). 
%
However, we found that constraints that force the entire 
structure (\eg balancedness), or composing predicates in the 
wrong \emph{order}, can force \lazysmallcheck to enumerate 
the entire exponentially growing search space.

\toolname, on the other hand, scales nicely to larger input sizes,
allowing systematic and exhaustive testing of larger, more complex
inputs. This is because \toolname eschews \emph{explicit} 
enumeration-and-filtering (which results in searching for 
fewer needles in larger haystacks as the sizes increas), 
in favor of \emph{symbolically} searching for valid models 
via SMT, making \toolname robust to the strictness or ordering 
of constraints.



\subsection{Measuring Code Coverage}\label{sec:code-coverage}

The second question we seek to answer is whether \toolname is suitable for testing entire
libraries, \ie how much of the program can be automatically exercised using our
system? Keeping in mind the well-known issues with treating code coverage as an
indication of test-suite quality~\cite{marick1999misuse}, we
consider this experiment a negative filter.

To this end, we ran \toolname against the entire user-facing API of 
\hbox{@Data.Map@,} our @RBTree@ library, and @XMonad.StackSet@ -- using 
the constrained refined types (\eg @OkMap@, @OkRBT@, @OkStackSet@) as 
the specification for the exposed types -- and measured the expression 
and branch coverage, as reported by @hpc@~\cite{gill2007haskell}.
%
We used an increasing timeout ranging from one to thirty minutes
per exported function.

\mypara{Results}
%
The results of our experiments are shown in Figure~\ref{fig:coverage}. 
Across all three libraries, \toolname achieved at least 70\% expression 
and 64\% alternative coverage at the shortest timeout of one minute per function. 
Interestingly, the coverage metrics for @RBTree@ and @Data.Map@ remain relatively constant as we increase
the timeouts, with a small jump in expression coverage between 10 and 20 minutes.
@XMonad@ on the other hand, jumps from 70\% expression and 64\% alternative
coverage with a one minute timeout, to 96\% expression and 94\% alternative
with a ten minute timeout.

% @Data.Map@ and @RBTree@ show no change in coverage metrics 
% beyond a 5 minute timeout, while @XMonad@ has another bump in coverage 
% between 10 and 15 minutes.

There are three things to consider when examining these results. 
%
First is that some expressions are not evaluated due to Haskell's 
laziness (\eg the values contained in a @Map@). 
%
Second is that some expressions \emph{should not} be evaluated 
and some branches \emph{should not} be taken, as these only happen
when an unexpected error condition is triggered (\ie these expressions
should be dead code).
%
\toolname considers any inputs that trigger an uncaught exception a 
valid counterexample; the pre-conditions should rule out these inputs, 
and so we expect not to cover those expressions with \toolname.

The last remark is not intrinsically related to \toolname, 
but rather our means of collecting the coverage data. @hpc@ includes 
@otherwise@ guards in the ``always-true'' category, even though they 
cannot evaluate to anything else. 
%
@Data.Map@ contained 56 guards, of which 24 were marked ``always-true''. We
manually counted 21 \hbox{@otherwise@} guards, the remaining 3 ``always-true''
guards compared the size of subtrees when rebalancing to determine whether a
single or double rotation was needed; we were unable to trigger the double
rotation in these cases.
%
\hbox{@XMonad@} contained 9 guards, of which 4 were ``always-true''. 3 of these
were @otherwise@ guards; the remaining ``always-true'' guard dynamically checked
a function's pre-condition. If the pre-condition check had failed an error would
have been thrown by the next case, we consider it a success of \toolname that
the error branch was not triggered.


\begin{figure}[t!]
\centering
% \includegraphics[width=0.49\linewidth]{figs/MapCoverage}
% \includegraphics[width=0.49\linewidth]{figs/XMonad-StackSetCoverage}
% \includegraphics[width=0.49\linewidth]{figs/RBTreeCoverage}
  \begin{tikzpicture}
    \begin{groupplot}[
      group style = {group size = 3 by 1, horizontal sep=15pt,},
      groupplot ylabel={\% Coverage},
      groupplot xlabel={Timeout (min)},
      group/only outer labels,
      ymin=0,
      ymax=1
    ]
    % \begin{axis}[
    \nextgroupplot[
      title=\textsc{Data.Map},
      legend columns=3,
      legend entries={expressions,booleans,always-true,always-false,alternatives,local-functions},
      legend to name=legend,
    ]
    \addplot table[smooth,col sep=comma,x index=0,y index=1] {target/csv/MapCoverage.csv};
    \addplot table[smooth,col sep=comma,x index=0,y index=2] {target/csv/MapCoverage.csv};
    \addplot table[smooth,col sep=comma,x index=0,y index=3] {target/csv/MapCoverage.csv};
    \addplot table[smooth,col sep=comma,x index=0,y index=4] {target/csv/MapCoverage.csv};
    \addplot table[smooth,col sep=comma,x index=0,y index=5] {target/csv/MapCoverage.csv};
    \addplot table[smooth,col sep=comma,x index=0,y index=6] {target/csv/MapCoverage.csv};
    % \end{axis}
  % \end{tikzpicture}
  % \begin{tikzpicture}
    % \begin{axis}[
    \nextgroupplot[
      title=\textsc{XMonad.StackSet},
    ]
    \addplot table[smooth,col sep=comma,x index=0,y index=1] {target/csv/StackSetCoverage.csv};
    \addplot table[smooth,col sep=comma,x index=0,y index=2] {target/csv/StackSetCoverage.csv};
    \addplot table[smooth,col sep=comma,x index=0,y index=3] {target/csv/StackSetCoverage.csv};
    \addplot table[smooth,col sep=comma,x index=0,y index=4] {target/csv/StackSetCoverage.csv};
    \addplot table[smooth,col sep=comma,x index=0,y index=5] {target/csv/StackSetCoverage.csv};
    \addplot table[smooth,col sep=comma,x index=0,y index=6] {target/csv/StackSetCoverage.csv};
    % \end{axis}
  % \end{tikzpicture}
  % \begin{tikzpicture}
    % \begin{axis}[
    \nextgroupplot[
      title=\textsc{RBTree}
    ]
    \addplot table[smooth,col sep=comma,x index=0,y index=1] {target/csv/RBTreeCoverage.csv};
    \addplot table[smooth,col sep=comma,x index=0,y index=5] {target/csv/RBTreeCoverage.csv};
    \addplot table[smooth,col sep=comma,x index=0,y index=6] {target/csv/RBTreeCoverage.csv};
    % \end{axis}
    \end{groupplot}
  \end{tikzpicture}\\
  \ref{legend}
% \begin{verbatim}
% 81% expressions used (2202/2712)
% 42% boolean coverage (24/57)
%      41% guards (23/56), 26 always True,
%          3 always False, 4 unevaluated
%     100% 'if' conditions (1/1)
%     100% qualifiers (0/0)
% 95% alternatives used (370/388)
% 98% local declarations used (49/50)
% 92% top-level declarations used (134/145)
% \end{verbatim}
\caption{Coverage-testing of \texttt{Data.Map.Base}, \texttt{RBTree}, and
  \texttt{XMonad.StackSet} using \toolname. Each exported function was tested
  with increasing depth limits until a single run hit a timeout ranging from one
  to thirty minutes. Lower is better for ``always-true'' and ``always-false'',
  higher is better for everything else.}\label{fig:coverage}
\end{figure}

%%% NUKE \ES{can we use an hpc overlay to make it ignore the "always true" otherwise
%%% NUKE   guards? seems the party line is that one should focus on expression and
%%% NUKE   alternative coverage, not boolean... so perhaps we can report expression, alternative, and alwaysFalse}
%%% NUKE \RJ{Dont know what you mean, is this note LIVE? or can we DELETE?}

% Although
% @hpc@ reports only 42\% boolean coverage for @Data.Map@, manual inspection
% revealed that 22 of the guards marked by @hpc@ as ``always True'' are
% @otherwise@ guards and can never be false. In that light, it would be more
% accurate to consider 46/57 booleans as covered, \ie 82\% coverage. The remaining
% ``always True'' branches compared the size of subtrees when rebalancing to
% determine whether a single or double rotation was needed, in some cases we were
% unable to generate sufficiently large trees in one minute to trigger a double
% rotation. The two guards that were always false were due to the simplistic
% generator we currently use for higher-order functions always returning false.

\subsection{Discussion}\label{sec:discussion}

To sum up, our experiments demonstrate that \toolname generates valid inputs:
%
(1) where \quickcheck fails outright, due to the low probability of
    generating random values satisfying a property;
%
(2) more efficiently than \lazysmallcheck, which relies on lazy
    pruning predicates; and
%
(3) providing high code coverage for real-world libraries with no
    hand-written test cases.

% \subsection{Limitations of \toolname}\label{sec:limitations}

Of course our approach is not without drawbacks; we highlight five classes
of pitfalls the user may encounter.

\mypara{Laziness} in the function or in the output refinement can cause exceptions
  to go un-thrown if the output value is not fully demanded. For example,
  \toolname would decide that the result @[1, undefined]@ inhabits @[Int]@ but not
  @[Score]@, as the latter would have to evaluate @0 <= undefined < 100@. This
  limitation is not specific to our system, rather it is fundamental to any tool
  that exercises lazy programs. Furthermore, \toolname only generates
  inductively-defined values, it cannot generate infinite or cyclic structures,
  nor will the generated values ever contain $\bot$.

\mypara{Polymorphism} Like any other tool that actually runs the function under scrutiny,
  \toolname can only test monomorphic instantiations of polymorphic
  functions. For example, when testing @XMonad@ we instantiated the ``window''
  parameter to @Char@ and all other type parameters to @()@, as the properties
  we were testing only examined the window. This helped drastically reduce the
  search space, both for \toolname and \smallcheck.

  % Our monomorphism restriction simplifies \toolname's implementation as we do
  % not have to consider type-class or equality constraints when generating test
  % values, but it also reduces the generalizability of \toolname's
  % result. 
  % Parametricity helps by telling us that the choice of
  % concrete instantiation will not affect the behavior of the function, but
  % in the presence of type-classes the benefit is reduced as we only know that
  % the specific instance we tested is correct.

\mypara{Advanced type-system features} such as GADTs and Existential types
  may prevent GHC from deriving a @Generic@ instance, which would force the
  programmer to write her own @Targetable@ instance. Though tedious, the single
  hand-written instance allows \toolname to automatically generate values
  satisfying disparate constraints, which is still an improvement over the
  generate-and-filter approach.
  
\mypara{Refinement types} are less expressive than properties written in the
  host language. If the pre-conditions are not expressible in \toolname's logic,
  the user will have to use the generate-and-filter approach, losing the benefits
  of symbolic enumeration.
  
\mypara{Input explosion} \toolname excels when the space of valid inputs is
  a sparse subset of the space of all inputs. If the input space is not
  sufficiently constrained, \toolname may spend lose its competitive advantage
  over other tools due to the overhead of using a general-purpose solver.

%% 1. laziness
%%    - potential for untriggered exceptions
%%    - our generated values never include bot
%% 2. Advanced type-system features
%%    - we can only provide default instances for Generic types
%%      - no GADTs or existentials
%% 3. Polymorphic functions
%%    - can only test monomorphic instantiation
%%    - types must be defaulted either by user or GHC
%%    - limitation shared by any testing tool

%%% Local Variables:
%%% mode: latex
%%% TeX-master: "main"
%%% End:

%\section{Related Work} \label{sec:related}

\toolname is closely related to a number of lines of work on connecting
formal specifications, execution, and automated constraint-based testing. 
Next, we describe the closest lines of work on test-generation and 
situate them with respect to our approach.

\subsection{Model-based Testing}
\label{sec:model-based-testing}
Model-based testing encompasses a broad range of black-box testing tools that
facilitate generating concrete test-cases from an abstract model of the system
under test. These systems generally (though not necessarily) model the system at
a holistic level using state machines to describe the desired
behavior~\cite{DiasNeto:2007:SMT:1353673.1353681}, and may or may not provide
fully automatic test-case generation. In addition to generating test-cases, many
model-based testing tools, \eg Spec Explorer~\cite{Veanes08} will produce extra artifacts
like visualizations to help the programmer understand the model. One could view
property-based testing, including our system, as a subset of model-based testing
focusing on lower-level properties of individual functions (unit-testing),
while using the type-structure of the functions under scrutiny to provide fully
automatic generation of test-cases.

\subsection{Property-based Testing}
\label{sec:property-based-testing}
Many property-based testing tools have been developed to automatically generate
test-suites. \quickcheck~\cite{claessen_quickcheck:_2000} randomly generates
inputs based on the property under scrutiny, but requires custom generators to
consistently generate constrained inputs. \cite{Claessen14Flops} extends
\quickcheck to randomly generate constrained values from a uniform distribution.
%
In contrast \smallcheck~\cite{runciman_smallcheck_2008} enumerates all possible
inputs up to some depth, which allows it to check existential properties in
addition to universal properties; however, it too has difficulty generating
inputs to properties with complex pre-conditions.
%
\lazysmallcheck~\cite{runciman_smallcheck_2008} addresses the issue of generating
constrained inputs by taking advantage of the inherent laziness of the
property, generating \emph{partially-defined} values (\ie values
containing $\bot$) and only filling in the holes if and when they are
demanded. 
%
Korat~\cite{Boyapati02} instruments a programmer-supplied
@repOk@ method, which checks class invariants and method pre-conditions, to
monitor which object fields are accessed. The authors observe that unaccessed fields
cannot have had an effect on the return value of @repOk@ and are thereby able to
exclude from the search space any objects that differ only in the values of the
unaccessed fields. 
%
While \lazysmallcheck and Korat's reliance on functions in the 
source language for specifying properties is convenient for the 
programmer (specification and implementation in the same language), 
it makes the method less amenable to formal verification, the
properties would need to be re-specified in another language 
that is restricted enough to facilitate verification.

\subsection{Symbolic Execution and Model-checking}
\label{sec:static-analysis}
Another popular technique for automatically generating test-cases is to analyze
the source code and attempt to construct inputs that will trigger different
paths through the program. DART~\cite{DART}, CUTE~\cite{CUTE}, 
and Pex~\cite{tillmann_pexwhite_2008} all use a
combination of symbolic and dynamic execution to explore different paths through
a program. 
%
While executing the program they collect \emph{path predicates},
conditions that characterize a path through a program, and at the end of a run
they negate the path predicates and query a constraint solver for another
assignment of values to program variables. This enables such tools to
efficiently explore many different paths through a program, but the technique
relies on the path predicates being expressible symbolically. When the
predicates are not expressible in the logic of the constraint solver, they fall
back to the values produced by the concrete execution, at a severe loss of
precision.
%
Instead of trying to trigger all paths through a program, one might 
simply try to trigger erroneous behavior. 
%
Check 'n' Crash~\cite{jcrasher} uses the ESC/Java analyzer~\cite{ESCJava} to discover
potential bugs and constructs concrete test-cases designed to trigger 
the bugs, if they exist. Similarly, \cite{ICSE04BLAST} uses the BLAST 
model-checker to construct test-cases that bring the program to 
a state satisfying some user-provided predicate.

In contrast to these approaches, \toolname (and more generally, property-based testing) 
treats the program as a \emph{black-box} and only requires that the pre- and 
post-conditions be expressible in the solver's logic. 
%
Of course, by expressing specifications in the source language, 
\eg as contracts, as in PEX~\cite{tillmann_pexwhite_2008}, one can use symbolic
execution to generate tests directly from specifications.
%
One concrete advantage of our approach over the symbolic execution based method
of PEX is that the latter generates tests by \emph{explicitly enumerating} paths
through the contract code, which suffers from a similar combinatorial 
problem as \smallcheck and \quickcheck. In contrast, \toolname performs the 
same search \emph{symbolically} within the SMT engine, which % we have demonstrated 
performs better for larger input sizes.


\subsection{Integrating Constraint-solving and Execution}
\label{sec:constraint-solving-execution}

\toolname is one of many tools that makes specifications 
executable via constraint solving. 
%
An early example of this approach is 
TestEra~\cite{Marinov:2001:TNF:872023.872551} 
that uses specifications written in the Alloy 
modeling language~\cite{jackson2002alloy} to 
generate all non-isomorphic Java objects that 
satisfy method pre-conditions and class invariants. 
%
As the specifications are written in Alloy, one can use 
Alloy's SAT-solver based model finding to symbolically 
enumerate candidate inputs.
%
Check 'n' Crash uses a similar idea, and SMT 
solvers to generate inputs that satisfy a given 
JML specification~\cite{jcrasher}.
%
Recent systems such as SBV~\cite{sbv} and 
Kaplan~\cite{Koksal:2012:CC:2103656.2103675} 
offer a monadic API for writing SMT constraints 
within the program, and use them to synthesize 
program values at \emph{run-time}. 
%
SBV provides a thin DSL over the logics understood 
by SMT solvers, whereas Kaplan integrates deeply 
with Scala, allowing the use of user-defined 
recursive types and functions. 
%
Test generation can be viewed as a special case 
of value-synthesis, and indeed Kaplan has been 
used to generate test-suites from preconditions 
in a similar manner to \toolname.

However, in all of the above (and also symbolic execution 
based methods like PEX or JCrasher), the specifications are 
\emph{assertions} in the Floyd-Hoare sense. 
%
Consequently, the techniques are limited to testing 
first-order functions over monomorphic data types.
%
In contrast, \toolname shows how to view \emph{types} as
executable specifications, which yields several advantages.
%
First, we can use types to compositionally lift specifications 
about flat values (\eg @Score@) over collections (\eg @[Score]@),
without requiring special recursive predicates to describe 
such collection invariants. 
% Similarly, we can use types to 
% smoothly lift specifications over arrow types, to test 
% higher-order functions.
%
Second, the compositional nature of types yields a 
compositional method for generating tests, allowing 
us to use type-class machinery to generate tests for
richer structures from tests for sub-structures.
%
Third, (refinement) types have proven to be effective 
for \emph{verifying} correctness properties in modern
modern languages that make ubiquitous use of parametric 
polymorphism and higher order 
functions~\cite{pfenningxi98,Dunfield07,SaraswatX10,fstar,VazouRealWorld14} 
and thus, we believe \toolname's approach of making refinement types
executable is a crucial step towards %
our goal of enabling 
\emph{gradual verification} for modern languages.

%%% BACKEND??? 
%%% That being said, Kaplan and SBV would be well-suited as a backend for \toolname.

%%%% NUKE SAID ELSEWHERE \subsection{Program Verification}
%%%% NUKE SAID ELSEWHERE \label{sec:program-verification}
%%%% NUKE SAID ELSEWHERE Many groups have worked on automatic verifiers that can prove 
%%%% NUKE SAID ELSEWHERE the absence of certain classes of bugs. Static contract 
%%%% NUKE SAID ELSEWHERE checkers~\cite{ESCJava,SpecSharp} use assertions and pre- and 
%%%% NUKE SAID ELSEWHERE post-conditions to verify program invariants. 
%%%% NUKE SAID ELSEWHERE %
%%%% NUKE SAID ELSEWHERE Our work builds on top of refinement type systems, which encode 
%%%% NUKE SAID ELSEWHERE complex invariants using predicates drawn from an SMT-decidable 
%%%% NUKE SAID ELSEWHERE logic~\cite{pfenningxi98}. 
%%%% NUKE SAID ELSEWHERE %
%%%% NUKE SAID ELSEWHERE The drawback to many of these systems is that the error messages 
%%%% NUKE SAID ELSEWHERE often become inscrutable, referring to temporary variables and 
%%%% NUKE SAID ELSEWHERE low-level constraints in the logic. 
%%%% NUKE SAID ELSEWHERE %
%%%% NUKE SAID ELSEWHERE In contrast a failing test-case provides the user with a concrete
%%%% NUKE SAID ELSEWHERE counter-example that he can examine to determine to source of 
%%%% NUKE SAID ELSEWHERE the error. 
%%%% NUKE SAID ELSEWHERE %
%%%% NUKE SAID ELSEWHERE We share a specification language with LiquidHaskell~\cite{VazouRealWorld14}, 
%%%% NUKE SAID ELSEWHERE providing the user with a seamless transition between testing 
%%%% NUKE SAID ELSEWHERE their program and proving it correct.

%%% REQUIRE auxiliary invariants -- e.g. RBTREE, balanced ---

%%% Local Variables:
%%% mode: latex
%%% TeX-master: "main"
%%% End:

%\lstDeleteShortInline{@}

%\part{Predicting Type Errors}
\chapter{Learning To Blame}
\label{chp:nate}
\renewcommand\toolname{\tool{Nate}}
\renewcommand\lang{\ensuremath{\lambda^{ML}}}
\input{nate/intro3}
\section{Overview}\label{sec:overview}

We start with a series of examples pertaining to a small grading
library called @Scores@. The examples provide a bird's eye view of 
how a user interacts with \toolname, how \toolname is implemented,
and the advantages of type-based testing.

\mypara{Refinement Types}
A refinement type is one where the basic types are decorated 
with logical predicates drawn from an efficiently decidable 
theory. For example,
%
\begin{code}
  type Nat   = {v:Int | 0 <= v}
  type Pos   = {v:Int | 0 <  v}
  type Rng N = {v:Int | 0 <= v && v <  N}
\end{code}
%
are refinement types describing the set of integers that are 
non-negative, strictly positive, and in the interval @[0, N)@ 
respectively. We will also build up function and collection 
types over base refinement types like the above. 
%
In this paper, we will not address the issue of \emph{checking}
refinement type signatures~\cite{VazouICFP14}.
%
We assume the code is typechecked, \eg by GHC, against the 
standard type signatures obtained by erasing the refinements.
Instead, we focus on using the refinements to 
synthesize tests to \emph{execute} the function, and to find 
\emph{counterexamples} that violate %demonstrate the function does not meet
the given specification.

\subsection{Testing with Types}

\mypara{Base Types}
Let us write a function @rescale@ that takes a source range @[0,r1)@, 
a target range @[0,r2)@, and a score @n@ from the source range,
and returns the linearly scaled score in the target range.
%
For example, @rescale 5 100 2@ should return @40@. 
Here is a first attempt at @rescale@ 
%
\begin{code}
  rescale :: r1:Nat -> r2:Nat -> s:Rng r1 -> Rng r2 
  rescale r1 r2 s = s * (r2 `div` r1)   
\end{code}
%
When we run \toolname, it immediately reports 
%
\begin{code}
  Found counter-example: (1, 0, 0) 
\end{code}
%
Indeed, @rescale 1 0 0@ results in @0@ which is not in the target 
@Rng 0@, as the latter is empty! We could fix this in various ways, 
\eg by requiring the ranges are non-empty:
%
\begin{code}
  rescale :: r1:Pos -> r2:Pos -> s:Rng r1 -> Rng r2 
\end{code}
%
Now, \toolname accepts the function and reports
%
\begin{code}
  OK. Passed all tests.
\end{code}
%
Thus, using the refinement type \emph{specification} for @rescale@, 
\toolname systematically tests the \emph{implementation} by generating 
all valid inputs (up to a given size bound) that respect the 
pre-conditions, running the function, and checking that the 
output satisfies the post-condition.
%
Testing against random, unconstrained inputs would be of limited value 
as the function is not designed to work on all @Int@ values. While in 
this case we could filter invalid inputs, we shall show
that \toolname can be more effective.

\mypara{Containers}
Let us suppose we have normalized all scores to be out of @100@
%
\begin{code}
  type Score = Rng 100
\end{code}
%
Next, let us write a function to compute a \emph{weighted} average 
of a list of scores.
%
\begin{code}
  average     :: [(Int, Score)] -> Score
  average []  = 0
  average wxs = total `div` n
    where
      total   = sum [w * x | (w, x) <- wxs ]
      n       = sum [w     | (w, _) <- wxs ]
\end{code}
%
It can be tricky to \emph{verify} this function as it requires non-linear reasoning
about an unbounded collection. However, we can gain a great degree of confidence by
systematically testing it using the type specification; indeed, \toolname responds:
%
\begin{code}
  Found counter-example: [(0,0)]
\end{code}
%
Clearly, an unfortunate choice of weights can trigger a divide-by-zero; we can fix 
this by requiring the weights be non-zero:
%
\begin{code}
  average :: [({v:Int | v /= 0}, Score)] -> Score
\end{code}
%
but now \toolname responds with
%
\begin{code}
  Found counter-example: [(-3,3),(3,0)]
\end{code}
% 
which also triggers the divide-by-zero! We will play it safe and require positive weights,
%
\begin{code}
  average :: [(Pos, Score)] -> Score
\end{code}
%
at which point \toolname reports that all tests pass.

\mypara{Ordered Containers}
The very nature of our business requires that at the end of the day,
we order students by their scores. We can represent ordered lists by 
requiring the elements of the tail @t@ to be greater than the head @h@:
%
\begin{code}
data OrdList a = [] | (:) {h :: a, t :: OrdList {v:a | h <= v}}
\end{code}
%
Note that erasing the refinement predicates gives us plain old Haskell lists.
We can now write a function to insert a score into an ordered list:
%
\begin{code}
  insert :: (Ord a) => a -> OrdList a -> OrdList a 
\end{code}
%
\toolname automatically generates all ordered lists (up to a given size)
and executes @insert@ to check for any errors. Unlike randomized testers, 
\toolname is not thwarted by the ordering constraint, and does not require a
custom generator from the user.

\mypara{Structured Containers} 
Everyone has a few bad days. Let us write a function that takes the 
@best k@ scores for a particular student. That is, the output
must satisfy a \emph{structural} constraint -- that its size 
equals @k@. We can encode the size of a list with a logical 
measure function~\cite{VazouICFP14}:
%
\begin{code}
  measure len :: [a] -> Nat
  len []      = 0
  len (x:xs)  = 1 + len xs
\end{code}
%
Now, we can stipulate that the output indeed has @k@ scores:
%
\begin{code}
  best      :: k:Nat -> [Score] -> {v:[Score] | k = len v}
  best k xs = take k $ reverse $ sort xs
\end{code}
%
Now, \toolname quickly finds a counterexample:
%
\begin{code}
  Found counter-example: (2,[])
\end{code}
%
Of course -- we need to have at least @k@ scores to start with! 
%
\begin{code}
best :: k:Nat -> {v:[Score]|k <= len v} -> {v:[Score]|k = len v}
\end{code}
%
and now, \toolname is assuaged and reports no counterexamples.
%
While randomized testing would suffice for @best@, we will see 
more sophisticated structural properties such as height balancedness, 
which stymie random testers, but are easily handled by \toolname.

\mypara{Higher-order Functions} 
Perhaps instead of taking the $k$ best grades, we would like
to pad each individual grade, and, furthermore, we want to
be able to experiment with different padding functions. Let
us rewrite @average@ to take a functional argument, and
stipulate that it can only increase a @Score@.
%
\begin{code}
  padAverage       :: (s:Score -> {v:Score | s <= v}) 
                   -> [(Pos, Score)] -> Score
  padAverage f []  = f 0
  padAverage f wxs = total `div` n
    where
      total   = sum [w * f x | (w, x) <- wxs ]
      n       = sum [w       | (w, _) <- wxs ]
\end{code}
%
\toolname automatically checks that @padAverage@ is 
a safe generalization of @average@. Randomized 
testing tools can also generate functions, but those 
functions are unlikely to satisfy non-trivial constraints, 
thereby burdening the user with custom generators.


\subsection{Synthesizing Tests} 
\label{sec:synthesizing-tests}
Next, let us look under the hood to get an idea of how \toolname 
synthesizes tests from types. 
% INTRO
At a high-level, our strategy is to:
%
(1)~\emph{query}   an SMT solver for satisfying assigments to a set of logical 
                   constraints derived from the refinement type,
(2)~\emph{decode}  the model into Haskell values that are suitable inputs,
(3)~\emph{execute} the function on the decoded values to obtain the output, 
(4)~\emph{check}   that the output satisfies the output type,
(5)~\emph{refute}  the model to generate a different test, and 
repeat the above steps until all tests up to a certain size are executed.
%
We focus here on steps 1, 2, and 4 -- query, decode, and check -- the others are 
standard and require little explanation.

\mypara{Base Types}
Recall the initial (buggy) specification
%
\begin{code}
  rescale :: r1:Nat -> r2:Nat -> s:Rng r1 -> Rng r2 
\end{code}
%
\toolname \emph{encodes} input requirements for base types directly 
from their corresponding refinements. The constraints for multiple, 
related inputs are just the \emph{conjunction} of the constraints 
for each input. Hence, the constraint for @rescale@ is:
%
$$
\cstr{C_0} \defeq 0 \leq \cvar{r1} \wedge 0 \leq \cvar{r2} \wedge 0 \leq s < \cvar{r1} 
$$
%
In practice, $\cstr{C_0}$ will also contain conjuncts of the form $-N \leq x \leq N$ that
restrict @Int@-valued variables $x$ to be within the size bound $N$ supplied by
the user, but we will omit these throughout the paper for clarity.
%% %
%% For clarity, we omit the conjuncts of the form $-N \leq x \leq N$
%% that restrict @Int@-valued variables $x$ to be within the size
%% bound $N$ supplied by the user.

Note how easy it is to capture dependencies between inputs, 
\eg that the score @s@ be in the range defined by @r1@.
%
On querying the SMT solver with the above, we get a model
%
$[\cvar{r1} \mapsto 1, \cvar{r2} \mapsto 1, \cvar{s}  \mapsto 0]$.
%
\toolname decodes this model and executes \hbox{@rescale 1 1 0@} to obtain the value @v = 0@.
%
Then, \toolname validates @v@ against the post-condition by checking 
% that it inhabits the output type, \ie by checking 
the validity of the output type's constraint: 
%
$$\cvar{r2} = 1 \wedge \cvar{v} = 0 \wedge 0 \leq \cvar{v} \wedge \cvar{v} < \cvar{r2}$$
%
As the above is valid, \toolname moves on to generate another 
test by conjoining $\cstr{C_0}$ with a constraint that refutes 
the previous model:
%
$$
\cstr{C_1} \defeq \cstr{C_0} \wedge (\cvar{r1} \not = 1 \vee \cvar{r2} \not = 1 \vee \cvar{s} \not = 0)
$$
This time, the SMT solver returns a model: 
%
$[\cvar{r1} \mapsto 1, \cvar{r2} \mapsto 0, \cvar{s} \mapsto 0]$
%
which, when decoded and executed, yields the result $0$ that does \emph{not} 
inhabit the output type, and so is reported as a counterexample. 
%
When we fix the specification to only allow @Pos@ ranges, each test produces
a valid output, so \toolname reports that all tests pass.

\mypara{Containers}
Next, we use \toolname to test the implementation of @average@.
To do so, \toolname needs to generate Haskell lists with the appropriate constraints.
%
Since each list is recursively 
either ``nil'' 
or ``cons'', 
\toolname generates constraints that symbolically 
represent \emph{all} possible lists up to a given depth, 
using propositional \emph{choice variables} to 
symbolically pick between these two alternatives.
%
Every (satisfying) assignment of choices returned by 
the SMT solver gives \toolname the concrete data and 
constructors used at each level, allowing it to decode 
the assignment into a Haskell value.

For example, \toolname represents valid @[(Pos, Score)]@ 
inputs (of depth up to 3), required to test @average@, 
as the conjunction of $\cstr{C_{list}}$ and $\cstr{C_{data}}$:
%
\begin{eqnarray*}
\cstr{C_{list}} & \defeq & (\cvar{c}_{00} \Rightarrow \cvar{xs}_0 = \lnil) \wedge 
                          (\cvar{c}_{01} \Rightarrow \cvar{xs}_0 = \lcons{\cvar{x}_1}{\cvar{xs}_1}) \wedge 
                          (\cvar{c}_{00} \oplus \cvar{c}_{01}) \\
               & \wedge & (\cvar{c}_{10} \Rightarrow \cvar{xs}_1 = \lnil) \wedge
                          (\cvar{c}_{11} \Rightarrow \cvar{xs}_1 = \lcons{\cvar{x}_2}{\cvar{xs}_2}) \wedge 
                          (\cvar{c}_{01} \Rightarrow \cvar{c}_{10} \oplus \cvar{c}_{11}) \\
               & \wedge & (\cvar{c}_{20} \Rightarrow \cvar{xs}_2 = \lnil) \wedge 
                          (\cvar{c}_{21} \Rightarrow \cvar{xs}_2 = \lcons{\cvar{x}_3}{\cvar{xs}_3}) \wedge 
                          (\cvar{c}_{11} \Rightarrow \cvar{c}_{20} \oplus \cvar{c}_{21}) \\
               & \wedge & (\cvar{c}_{30} \Rightarrow \cvar{xs}_3 = \lnil) \wedge 
                          (\cvar{c}_{21} \Rightarrow \cvar{c}_{30}) \\[0.1in]
\cstr{C_{data}} & \defeq & (\cvar{c}_{01} \Rightarrow \cvar{x}_1 = \ltup{\cvar{w}_1}{\cvar{s}_1} \ \wedge\ 0 < \cvar{w}_1 \ \wedge\ 0 \leq \cvar{s}_1 < 100) \\
               & \wedge & (\cvar{c}_{11} \Rightarrow \cvar{x}_2 = \ltup{\cvar{w}_2}{\cvar{s}_2} \ \wedge\ 0 < \cvar{w}_2 \ \wedge\ 0 \leq \cvar{s}_2 < 100) \\
               & \wedge & (\cvar{c}_{21} \Rightarrow \cvar{x}_3 = \ltup{\cvar{w}_3}{\cvar{s}_3} \ \wedge\ 0 < \cvar{w}_3 \ \wedge\ 0 \leq \cvar{s}_3 < 100)
\end{eqnarray*}
%
The first set of constraints $\cstr{C_{list}}$ describes all lists up to 
size 3. At each level $i$, the \emph{choice} variables $\cvar{c}_{i0}$ 
and $\cvar{c}_{i1}$ determine whether at that level the constructed 
list $\cvar{xs}_i$ is a ``nil'' or a ``cons''. 
%
In the constraints $\lnil$ and $(\lcons{}{})$ are \emph{uninterpreted} 
functions that represent ``nil'' and ``cons'' respectively. 
These functions only obey the congruence axiom and hence, can be 
efficiently analyzed by SMT solvers~\cite{Nelson81}.
%
The data at each level $\cvar{x}_i$ is constrained to be a pair of a 
positive weight $\cvar{w}_i$ and a valid score $\cvar{s}_i$.

The choice variables at each level are used to \emph{guard} the 
constraints on the next levels. 
%
First, if we are generating a ``cons'' at a given level, then
exactly one of the choice variables for the next level must be 
selected;
\eg  $\cvar{c}_{11} \Rightarrow \cvar{c}_{20} \oplus \cvar{c}_{21}$.
%
Second, the constraints on the data at a given level only hold 
if we are generating values for that level; \eg $\cvar{c}_{21}$ 
is used to guard the constraints on $\cvar{x}_3$, $\cvar{w}_3$ 
and $\cvar{s}_3$.
%
This is essential to avoid over-constraining the system 
which would cause \toolname to miss certain tests.

To \emph{decode} a model of the above into a Haskell value of type @[(Int, Int)]@,
we traverse constraints and use the valuations of the choice variables to 
build up the list appropriately.
%
At each level, if $\cvar{c}_{i0} \mapsto \ttrue$, then the list at that 
level is @[]@, otherwise $\cvar{c}_{i1} \mapsto \ttrue$ and we decode 
$\cvar{x}_{i+1}$ and $\cvar{xs}_{i+1}$ and ``cons'' the results.

We can iteratively generate \emph{multiple} inputs by adding a constraint that
refutes each prior model. As an important optimization, we only refute the
relevant parts of the model, \ie those needed to construct the list
(\S~\ref{sec:refute}).

\mypara{Ordered Containers}
%
Next, let us see how \toolname enables automatic testing with 
highly constrained inputs, such as the \emph{increasingly ordered} 
@OrdList@ values required by @insert@.
%
From the type definition, it is apparent that ordered
lists are the same as the usual lists described by
$\cstr{C_{list}}$, except that each unfolded \emph{tail} 
must only contain values that are greater than the 
corresponding \emph{head}.
%
That is, as we unfold @x1:x2:xs :: OrdList@ 
%
\begin{itemize}
\item At level @0@, we have @OrdList {v:Score| true}@
\item At level @1@, we have @OrdList {v:Score| x1 <= v}@
\item At level @2@, we have @OrdList {v:Score| x2 <= v && x1 <= v}@
\end{itemize}
%
and so on. Thus, we encode @OrdList Score@ (of depth up to 3) by
conjoining $\cstr{C_{list}}$ with  $\cstr{C_{score}}$ and $\cstr{C_{ord}}$,
which capture the valid score and ordering requirements respectively:
%
\begin{eqnarray*}
\cstr{C_{ord}}   & \defeq & (\cvar{c}_{11} \Rightarrow \cvar{x}_1 \leq \cvar{x}_2)
                \ \wedge\  (\cvar{c}_{21} \Rightarrow \cvar{x}_2 \leq \cvar{x}_3\ \wedge\ \cvar{x}_1 \leq \cvar{x}_3) \\[0.01in]
\cstr{C_{score}} & \defeq & (\cvar{c}_{01} \Rightarrow 0 \leq \cvar{x}_1 < 100)
                \ \wedge\  (\cvar{c}_{11} \Rightarrow 0 \leq \cvar{x}_2 < 100)
                \ \wedge\  (\cvar{c}_{21} \Rightarrow 0 \leq \cvar{x}_3 < 100)
\end{eqnarray*}

\mypara{Structured Containers}
Recall that @best k@ requires inputs whose \emph{structure} is constrained -- the 
size of the list should be no less than @k@. We specify size using special measure 
functions~\cite{VazouICFP14}, which let us relate the size of a list with that of
its unfolding, and hence, let us encode the notion of size inside the constraints:
%
\begin{eqnarray*}
\cstr{C_{size}} & \defeq & (\cvar{c}_{00} \Rightarrow \clen{\cvar{xs}_{0}} = 0) \wedge 
                          (\cvar{c}_{01} \Rightarrow \clen{\cvar{xs}_{0}} = 1 + \clen{\cvar{xs}_1}) \\
               & \wedge & (\cvar{c}_{10} \Rightarrow \clen{\cvar{xs}_{1}} = 0) \wedge 
                          (\cvar{c}_{11} \Rightarrow \clen{\cvar{xs}_{1}} = 1 + \clen{\cvar{xs}_2}) \\
               & \wedge & (\cvar{c}_{20} \Rightarrow \clen{\cvar{xs}_{2}} = 0) \wedge 
                          (\cvar{c}_{21} \Rightarrow \clen{\cvar{xs}_{2}} = 1 + \clen{\cvar{xs}_3}) \\
               & \wedge & (\cvar{c}_{30} \Rightarrow \clen{\cvar{xs}_{3}} = 0)
\end{eqnarray*}
%
At each unfolding, we instantiate the definition of the measure 
for each alternative of the datatype. 
%
In the constraints, $\clen{\cdot}$ is an uninterpreted function derived
from the measure definition. All of the relevant properties of the function
are spelled out by the unfolded constraints in $\cstr{C_{size}}$ and hence,
we can use SMT to search for models for the above constraint.
%
Hence, \toolname constrains the input type for @best@ as:
%
$$     0 \leq k 
\wedge \cstr{C_{list}} 
\wedge \cstr{C_{score}} 
\wedge \cstr{C_{size}} 
\wedge k \leq \clen{\cvar{xs}_0} $$
%
where the final conjunct comes from the top-level refinement that 
stipulates the input have at least @k@ scores.
%
Thus, \toolname only generates lists that are large enough. 
For example, in any model where $k = 2$, it will \emph{not} 
generate the empty or singleton list, as in those cases, 
$\clen{\cvar{xs}_0}$ would be $0$ (resp. $1$), violating the 
final conjunct above.

\mypara{Higher-order Functions}
Finally, \toolname's type-directed testing scales up to higher-order
functions using the same insight as in QuickCheck~\cite{claessen_quickcheck:_2000}, namely, 
to generate a function it suffices to be able to 
generate the \emph{output} of the function.
When tasked with the generation of a functional argument @f@, \toolname 
returns a Haskell function that when executed checks
whether its inputs satisfy @f@'s pre-conditions.
If they do, then @f@ uses \toolname to dynamically
query the SMT solver for an output that satisfies the 
constraints imposed by the concrete inputs.
Otherwise, @f@'s specifications are violated
and TARGET reports a counterexample.

This concludes our high-level tour of the benefits and 
implementation of \toolname. 
%
Notice that the property specification mechanism -- 
refinement types -- allowed us to get immediate feedback
that helped debug not just the code, but also the specification 
itself. 
%
Additionally, the specifications gave us machine-readable 
documentation about the behavior of functions, and a large 
unit test suite with which to automatically validate the 
implementation.
%
Finally, though we do not focus on it here, the specifications 
are amenable to formal verification should the programmer 
so desire.


%%% Local Variables:
%%% mode: latex
%%% TeX-master: "main"
%%% End:

\input{nate/learning}
\section{Evaluation} \label{sec:evaluation}

We have built a prototype implementation of \toolname\footnote{\url{http://hackage.haskell.org/package/target-0.1.1.0}} and next, 
describe an evaluation on a series of benchmarks ranging from 
textbook examples of algorithms and data structures to widely 
used Haskell libraries like \textsc{containers} and \textsc{xmonad}.
%
Our goal in this evaluation is two-fold. 
%
First, we describe micro-benchmarks (\ie functions)
that \emph{quantitatively compare} \toolname with 
the existing state-of-the-art, property-based testing
tools for Haskell -- namely \smallcheck and \quickcheck\ -- 
to determine whether \toolname is indeed able to generate
highly constrained inputs more effectively.
%
Second, we describe macro-benchmarks (\ie modules) that 
evaluate the amount of \emph{code coverage} that we 
get from type-targeted testing.
%
%% Third, using our results as a base, we present a 
%% qualitative discussion of \toolname as a \emph{gradual} 
%% approach that bridges informal and formal verification.

%% ES: This doesn't actually make any sense..
% An important optimization in our implementation is to 
% perform the post-condition checking in Haskell instead of by querying
% the SMT solver. We accomplish this by adding an additional @toReft@
% method to @Targetable@ that translates the concrete inputs that we
% @decode@ \emph{back} into logical expressions,
% %
% \begin{code}
%   toReft :: a -> Var
% \end{code}
% %
% which we then substitute into the output-type before checking the
% concrete output. @toReft@ can be simply implemented as:
% %
% \begin{code}
%   toReft v = app c (map toReft vs)
%     where (c, vs) = splitCtor v
%           -- app :: Ctor -> [Var] -> Var
% \end{code}
% %
% where @app@ is a pure version of @apply@, \ie it constructs a
% logical expression like @apply@ but does nothing else.

% \newcommand\XX{\multicolumn{1}{c}{X}}
% \newcommand{\mysec}[1]{\SI{#1}{\second}}
\begin{figure}[ht!]
  \centering
  % \includegraphics[width=0.49\linewidth]{figs/List-insert}
  % \includegraphics[width=0.49\linewidth]{figs/RBTree-add}
  % \includegraphics[width=0.49\linewidth]{figs/Map-delete}
  % \includegraphics[width=0.49\linewidth]{figs/Map-difference}
  % \includegraphics[width=0.49\linewidth]{figs/XMonad-focus-left}
  % \begin{tikzpicture}
  %    \begin{customlegend}[legend columns=4,legend style={align=center,draw=none},legend entries={\toolname,\smallcheck,\lazysmallcheck,\lazysmallcheck (slow)}]
  %    \addlegendimage{color=blue,mark=square*}
  %    \addlegendimage{color=red,mark=*}   
  %    \addlegendimage{color=orange,mark=diamond*}
  %    \addlegendimage{color=black,mark=x}
  %    \end{customlegend}
  % \end{tikzpicture}

  \begin{tikzpicture}
    \begin{groupplot}[
      group style = {group size = 3 by 1, horizontal sep=15pt,},
      groupplot ylabel={Time (sec)},
      groupplot xlabel={Depth},
      group/only outer labels,
      ymode=log,
      ymax=10000,
      ymin=0.0001
    ]
    % \begin{semilogyaxis}[
    \nextgroupplot[
      title=\textsc{List.insert}
    ]
    \addplot table[smooth,col sep=comma,x index=0,y index=1] {target/csv/List.insert.csv};
    \addplot table[smooth,col sep=comma,x index=0,y index=2] {target/csv/List.insert.csv};
    \addplot table[smooth,col sep=comma,x index=0,y index=3] {target/csv/List.insert.csv};
    % \end{semilogyaxis}
  % \end{tikzpicture}
  % \begin{tikzpicture}
    % \begin{semilogyaxis}[
    \nextgroupplot[
      title=\textsc{RBTree.add},
      legend columns=4,
      legend entries={\toolname,\smallcheck,\lazysmallcheck,\lazysmallcheck (slow)},
      legend to name=legend,
    ]
    \addplot table[smooth,col sep=comma,x index=0,y index=1] {target/csv/RBTree.add.csv};
    \addplot table[smooth,col sep=comma,x index=0,y index=2] {target/csv/RBTree.add.csv};
    \addplot table[smooth,col sep=comma,x index=0,y index=3] {target/csv/RBTree.add.csv};
    \addplot table[smooth,col sep=comma,x index=0,y index=4] {target/csv/RBTree.add.csv};
    % \end{semilogyaxis}
  % \end{tikzpicture}
  % \begin{tikzpicture}
    % \begin{semilogyaxis}[
    \nextgroupplot[
      title=\textsc{XMonad.focus\_left}
    ]
    \addplot table[smooth,col sep=comma,x index=0,y index=1] {target/csv/XMonad.focus_left.csv};
    \addplot table[smooth,col sep=comma,x index=0,y index=2] {target/csv/XMonad.focus_left.csv};
    \addplot table[smooth,col sep=comma,x index=0,y index=3] {target/csv/XMonad.focus_left.csv};
    % \end{semilogyaxis}
    \end{groupplot}
  \end{tikzpicture}
  \begin{tikzpicture}
    \begin{groupplot}[
      group style = {group size = 2 by 1, horizontal sep=15pt,},
      groupplot ylabel={Time (sec)},
      groupplot xlabel={Depth},
      group/only outer labels,
      ymode=log,
      ymax=10000,
      ymin=0.0001
    ]
    % \begin{semilogyaxis}[
    \nextgroupplot[
      title=\textsc{Map.delete}
    ]
    \addplot table[smooth,col sep=comma,x index=0,y index=1] {target/csv/Map.delete.csv};
    \addplot table[smooth,col sep=comma,x index=0,y index=2] {target/csv/Map.delete.csv};
    \addplot table[smooth,col sep=comma,x index=0,y index=3] {target/csv/Map.delete.csv};
    \addplot table[smooth,col sep=comma,x index=0,y index=4] {target/csv/Map.delete.csv};
    % \end{semilogyaxis}
  % \end{tikzpicture}
  % \begin{tikzpicture}
    % \begin{semilogyaxis}[
    \nextgroupplot[
      title=\textsc{Map.difference}
    ]
    \addplot table[smooth,col sep=comma,x index=0,y index=1] {target/csv/Map.difference.csv};
    \addplot table[smooth,col sep=comma,x index=0,y index=2] {target/csv/Map.difference.csv};
    \addplot table[smooth,col sep=comma,x index=0,y index=3] {target/csv/Map.difference.csv};
    \addplot table[smooth,col sep=comma,x index=0,y index=4] {target/csv/Map.difference.csv};
    % \end{semilogyaxis}
    \end{groupplot}
  \end{tikzpicture}\\
  \ref{legend}

  \caption{Results of comparing \toolname with \quickcheck, \smallcheck, and Lazy
    \smallcheck on a series of functions. \toolname, \smallcheck, and Lazy
    \smallcheck were both configured to check the first 1000 inputs that
    satisfied the precondition at increasing depth parameters, with a 60 minute
    timeout per depth; \quickcheck was run with the default settings, \ie it had
    to produce 100 test cases. \toolname, \smallcheck, and \lazysmallcheck were
    configured to use the same notion of depth, in order to ensure they would
    generate the same number of valid inputs at each depth level. \quickcheck was
    unable to successfully complete any run due to the low probability of
    generating valid inputs at random.}\label{fig:comparisonresults}
\end{figure}



\subsection{Comparison with \quickcheck and \smallcheck}\label{sec:comparison}

We compare \toolname with \quickcheck and \smallcheck by using 
a set of benchmarks with highly constrained inputs. 
%
For each benchmark we compared \toolname with \smallcheck and
\quickcheck, with the latter two using the generate-and-filter 
approach, wherein a value is generated and subsequently discarded if
it does not meet the desired constraint.
%
While one could possibly write custom ``operational'' generators 
for each property, the point of this evaluation is compare the 
different approaches ability to enable ``declarative'' specification 
driven testing.
%
Next, we describe the benchmarks and then summarize the results of the comparison
(Figure~\ref{fig:comparisonresults}).



\mypara{Inserting into a sorted \List}
%
Our first benchmark is the \Insert function from the homonymous 
sorting routine. We use the specification that given an element 
and a sorted list, @insert x xs@ should evaluate to a sorted list.
We express this with the type
%
\begin{code}
  type Sorted a = List <{\hd v -> hd < v}> a
  insert :: a -> Sorted a -> Sorted a
\end{code}
%
where the ordering constraint is captured by an abstract 
refinement~\cite{Vazou13} which states that \emph{each} 
list head @hd@ is less than every element @v@ in its tail.

\mypara{Inserting into a Red-Black Tree}
%
Next, we consider insertion into a Red-Black tree.
%
\begin{code}
  data RBT a = Leaf  | Node Col a (RBT a) (RBT a)
  data Col   = Black | Red
\end{code}
%
Red-black trees must satisfy three invariants:
%
(1)~red nodes always have black children,
(2)~the black height of all paths from the root to a leaf is the same, and
(3)~the elements in the tree should be ordered.
%
We capture (1) via a measure that recursively checks each @Red@ node has @Black@ children.
%
\begin{code}
  measure isRB :: RBT a -> Prop
  isRB Leaf           = true
  isRB (Node c x l r) = isRB l && isRB r &&
                        (c == Red => isBlack l && isBlack r)
\end{code}
%
We specify (2) by defining the @Black@ height as:
%
\begin{code}
  measure bh :: RBT a -> Int
  bh Leaf           = 0
  bh (Node c x l r) = bh l + (if c == Red then 0 else 1)
\end{code}
%
and then checking that the @Black@ height of both subtrees is the same:
%
\begin{code}
  measure isBH :: RBT a -> Prop
  isBH Leaf           = true
  isBH (Node c x l r) = isBH l && isBH r && bh l == bh r
\end{code}
%
Finally, we specify the (3), the ordering invariant as:
%
\begin{code}
  type OrdRBT a = RBT <{\r v -> v < r}, {\r v -> r < v}> a
\end{code}
%
\ie with two abstract refinements for the left and right subtrees
respectively, which state that the root @r@ is greater than (resp. less than)
each element @v@ in the subtrees. Finally, a valid Red-Black tree is:
%
\begin{code}
  type OkRBT a = {v:OrdRBT a | isRB v && isBH v}
\end{code}
%
Note that while the specification for the \emph{internal} invariants for Red-Black
trees is tricky, the specification for the public API -- \eg the @add@ function -- 
is straightforward:
%
\begin{code}
  add :: a -> OkRBT a -> OkRBT a
\end{code}

\mypara{Deleting from a Data.Map}\label{sec:delete-from-map}
%
Our third benchmark is the @delete@ function from the \hbox{@Data.Map@} module in 
the Haskell standard libraries. The @Map@ structure is a balanced binary
search tree that implements purely functional key-value dictionaries:
%
\begin{code}
  data Map k a = Tip | Bin Int k a (Map k a) (Map k a)
\end{code}
%
A valid @Data.Map@ must satisfy two properties:
%
(1)~the size of the left and right sub-trees must be 
    within a factor of three of each other, and
(2)~the keys must obey a binary search ordering.
%
We specify the balancedness invariant~(1) with a measure
%
\begin{code}
  measure isBal :: Map k a -> Prop
  isBal (Tip)           = true
  isBal (Bin s k v l r) = isBal l && isBal r &&
                          (sz l + sz r <= 1 ||
                           sz l <= 3 * sz r <= 3 * sz l)
\end{code}
%
and combine it with an ordering invariant (like @OrdRBT@) to specify valid trees.
%
\begin{code}
  type OkMap k a = {v : OrdMap k a | isBal v}
\end{code}
%
We can check that @delete@ preserves the invariants by 
checking that its output is an @OkMap k a@.
However, we can also go one step further and check 
the functional correctness property that @delete@ 
removes the given key, with a type:
%
\begin{code}
  delete :: Ord k => k:k -> m:OkMap k a 
         -> {v:OkMap k a | MinusKey v m k}
\end{code}
%
where the predicate @MinusKey@ is defined as:
%
\begin{code}
  predicate MinusKey M1 M2 K 
    = keys M1 = difference (keys M2) (singleton K)
\end{code}
%
using the measure @keys@ describing the contents of the @Map@:
%
\begin{code}
  measure keys :: Map k a -> Set k
  keys (Tip)           = empty () 
  keys (Bin s k v l r) = union (singleton k) 
                               (union (keys l) (keys r))
\end{code}

\mypara{Refocusing XMonad StackSets} \label{sec:refocus-stackset}
%
Our last benchmark comes from the tiling window manager XMonad. 
%
The key invariant of XMonad's internal @StackSet@ data structure 
is that the elements (windows) must all be \emph{unique}, \ie contain
no duplicates.
%
XMonad comes with a test-suite of over 100 \quickcheck properties;
we select one which states that moving the focus between windows 
in a @StackSet@ should not affect the \emph{order} of the windows.
%
\begin{code}
  prop_focus_left_master n s =
    index (foldr (const focusUp) s [1..n]) == index s
\end{code}
%
With \quickcheck, the user writes a custom generator for valid @StackSet@s
and then runs the above function on test inputs created by the generator, 
to check if in each case, the result of the above is @True@.

With \toolname, it is possible to test such properties \emph{without} 
requiring custom generators. Instead the user writes a declarative 
specification:
%
\begin{code}
  type OkStackSet = {v:StackSet | NoDuplicates v}
\end{code}
%
(We refer the reader to~\cite{VazouRealWorld14} for a full 
discussion of how to specify @NoDuplicates@).
%
Next, we define a refinement type:
%
\begin{code}
  type TTrue = {v:Bool | Prop v}
\end{code}
%
that is only inhabited by @True@, and use it to type the \quickcheck 
property as:
%
\begin{code}
  prop_focus_left_master :: Nat -> OkStackSet -> TTrue 
\end{code}
%
This property is particularly difficult to \emph{verify}; however,
\toolname is able to automatically
generate valid inputs to \emph{test} that @prop_focus_left_master@
always returns @True@.

%%% The high level of abstraction inherent in the @StackSet@ definition
%%% works in our favor here, as we can instantiate the relevant type parameter (the
%%% window) to \Char and leave the others as @()@ to drastically reduce
%%% the search space.


\mypara{Results}
%
Figure~\ref{fig:comparisonresults} summarizes the results of the comparison.
%
\quickcheck was unable to successfully complete \emph{any} 
benchmark to the low probability of generating properly 
constrained values at random.

\begin{description}
\item[List Insert] \toolname is able to test @insert@ all the way to 
   depth 20, whereas \lazysmallcheck times out at depth 19.

\item[Red-Black Tree Insert] \toolname is able to test @add@ up to depth 12,
  while \lazysmallcheck times out at depth 6.
  
\item[Map Delete] \toolname is able to check @delete@ up to depth 10, whereas
   \lazysmallcheck times out at depth 7 if it checks ordering first,
    or depth 6 if it checks balancedness first.

\item[StackSet Refocus] \toolname and is able to check this property 
    up to depth 8, while \lazysmallcheck times out at depth 7.
\end{description}

\toolname sees a performance hit with properties 
that require reasoning with the theory of Sets \eg 
the no-duplicates invariant of @StackSet@. 
%
While \lazysmallcheck times out at a higher depths, when it completes
\eg at depth 6, it does so in 0.7s versus \toolname's 9 minutes.
%
We suspect this is because the theory of sets are a relatively recent
addition to SMT solvers \cite{arrayZ3}, and with further improvements 
in SMT technology, these numbers will get significantly better.


Overall, we found that for \emph{small inputs} \lazysmallcheck 
is substantially faster as exhaustive enumeration is tractable,
and does not incur the overhead of communicating with an external 
general-purpose solver.
%
Additionally, \lazysmallcheck benefits from pruning predicates 
that exploit laziness and only force a small portion of the 
structure (\eg ordering). 
%
However, we found that constraints that force the entire 
structure (\eg balancedness), or composing predicates in the 
wrong \emph{order}, can force \lazysmallcheck to enumerate 
the entire exponentially growing search space.

\toolname, on the other hand, scales nicely to larger input sizes,
allowing systematic and exhaustive testing of larger, more complex
inputs. This is because \toolname eschews \emph{explicit} 
enumeration-and-filtering (which results in searching for 
fewer needles in larger haystacks as the sizes increas), 
in favor of \emph{symbolically} searching for valid models 
via SMT, making \toolname robust to the strictness or ordering 
of constraints.



\subsection{Measuring Code Coverage}\label{sec:code-coverage}

The second question we seek to answer is whether \toolname is suitable for testing entire
libraries, \ie how much of the program can be automatically exercised using our
system? Keeping in mind the well-known issues with treating code coverage as an
indication of test-suite quality~\cite{marick1999misuse}, we
consider this experiment a negative filter.

To this end, we ran \toolname against the entire user-facing API of 
\hbox{@Data.Map@,} our @RBTree@ library, and @XMonad.StackSet@ -- using 
the constrained refined types (\eg @OkMap@, @OkRBT@, @OkStackSet@) as 
the specification for the exposed types -- and measured the expression 
and branch coverage, as reported by @hpc@~\cite{gill2007haskell}.
%
We used an increasing timeout ranging from one to thirty minutes
per exported function.

\mypara{Results}
%
The results of our experiments are shown in Figure~\ref{fig:coverage}. 
Across all three libraries, \toolname achieved at least 70\% expression 
and 64\% alternative coverage at the shortest timeout of one minute per function. 
Interestingly, the coverage metrics for @RBTree@ and @Data.Map@ remain relatively constant as we increase
the timeouts, with a small jump in expression coverage between 10 and 20 minutes.
@XMonad@ on the other hand, jumps from 70\% expression and 64\% alternative
coverage with a one minute timeout, to 96\% expression and 94\% alternative
with a ten minute timeout.

% @Data.Map@ and @RBTree@ show no change in coverage metrics 
% beyond a 5 minute timeout, while @XMonad@ has another bump in coverage 
% between 10 and 15 minutes.

There are three things to consider when examining these results. 
%
First is that some expressions are not evaluated due to Haskell's 
laziness (\eg the values contained in a @Map@). 
%
Second is that some expressions \emph{should not} be evaluated 
and some branches \emph{should not} be taken, as these only happen
when an unexpected error condition is triggered (\ie these expressions
should be dead code).
%
\toolname considers any inputs that trigger an uncaught exception a 
valid counterexample; the pre-conditions should rule out these inputs, 
and so we expect not to cover those expressions with \toolname.

The last remark is not intrinsically related to \toolname, 
but rather our means of collecting the coverage data. @hpc@ includes 
@otherwise@ guards in the ``always-true'' category, even though they 
cannot evaluate to anything else. 
%
@Data.Map@ contained 56 guards, of which 24 were marked ``always-true''. We
manually counted 21 \hbox{@otherwise@} guards, the remaining 3 ``always-true''
guards compared the size of subtrees when rebalancing to determine whether a
single or double rotation was needed; we were unable to trigger the double
rotation in these cases.
%
\hbox{@XMonad@} contained 9 guards, of which 4 were ``always-true''. 3 of these
were @otherwise@ guards; the remaining ``always-true'' guard dynamically checked
a function's pre-condition. If the pre-condition check had failed an error would
have been thrown by the next case, we consider it a success of \toolname that
the error branch was not triggered.


\begin{figure}[t!]
\centering
% \includegraphics[width=0.49\linewidth]{figs/MapCoverage}
% \includegraphics[width=0.49\linewidth]{figs/XMonad-StackSetCoverage}
% \includegraphics[width=0.49\linewidth]{figs/RBTreeCoverage}
  \begin{tikzpicture}
    \begin{groupplot}[
      group style = {group size = 3 by 1, horizontal sep=15pt,},
      groupplot ylabel={\% Coverage},
      groupplot xlabel={Timeout (min)},
      group/only outer labels,
      ymin=0,
      ymax=1
    ]
    % \begin{axis}[
    \nextgroupplot[
      title=\textsc{Data.Map},
      legend columns=3,
      legend entries={expressions,booleans,always-true,always-false,alternatives,local-functions},
      legend to name=legend,
    ]
    \addplot table[smooth,col sep=comma,x index=0,y index=1] {target/csv/MapCoverage.csv};
    \addplot table[smooth,col sep=comma,x index=0,y index=2] {target/csv/MapCoverage.csv};
    \addplot table[smooth,col sep=comma,x index=0,y index=3] {target/csv/MapCoverage.csv};
    \addplot table[smooth,col sep=comma,x index=0,y index=4] {target/csv/MapCoverage.csv};
    \addplot table[smooth,col sep=comma,x index=0,y index=5] {target/csv/MapCoverage.csv};
    \addplot table[smooth,col sep=comma,x index=0,y index=6] {target/csv/MapCoverage.csv};
    % \end{axis}
  % \end{tikzpicture}
  % \begin{tikzpicture}
    % \begin{axis}[
    \nextgroupplot[
      title=\textsc{XMonad.StackSet},
    ]
    \addplot table[smooth,col sep=comma,x index=0,y index=1] {target/csv/StackSetCoverage.csv};
    \addplot table[smooth,col sep=comma,x index=0,y index=2] {target/csv/StackSetCoverage.csv};
    \addplot table[smooth,col sep=comma,x index=0,y index=3] {target/csv/StackSetCoverage.csv};
    \addplot table[smooth,col sep=comma,x index=0,y index=4] {target/csv/StackSetCoverage.csv};
    \addplot table[smooth,col sep=comma,x index=0,y index=5] {target/csv/StackSetCoverage.csv};
    \addplot table[smooth,col sep=comma,x index=0,y index=6] {target/csv/StackSetCoverage.csv};
    % \end{axis}
  % \end{tikzpicture}
  % \begin{tikzpicture}
    % \begin{axis}[
    \nextgroupplot[
      title=\textsc{RBTree}
    ]
    \addplot table[smooth,col sep=comma,x index=0,y index=1] {target/csv/RBTreeCoverage.csv};
    \addplot table[smooth,col sep=comma,x index=0,y index=5] {target/csv/RBTreeCoverage.csv};
    \addplot table[smooth,col sep=comma,x index=0,y index=6] {target/csv/RBTreeCoverage.csv};
    % \end{axis}
    \end{groupplot}
  \end{tikzpicture}\\
  \ref{legend}
% \begin{verbatim}
% 81% expressions used (2202/2712)
% 42% boolean coverage (24/57)
%      41% guards (23/56), 26 always True,
%          3 always False, 4 unevaluated
%     100% 'if' conditions (1/1)
%     100% qualifiers (0/0)
% 95% alternatives used (370/388)
% 98% local declarations used (49/50)
% 92% top-level declarations used (134/145)
% \end{verbatim}
\caption{Coverage-testing of \texttt{Data.Map.Base}, \texttt{RBTree}, and
  \texttt{XMonad.StackSet} using \toolname. Each exported function was tested
  with increasing depth limits until a single run hit a timeout ranging from one
  to thirty minutes. Lower is better for ``always-true'' and ``always-false'',
  higher is better for everything else.}\label{fig:coverage}
\end{figure}

%%% NUKE \ES{can we use an hpc overlay to make it ignore the "always true" otherwise
%%% NUKE   guards? seems the party line is that one should focus on expression and
%%% NUKE   alternative coverage, not boolean... so perhaps we can report expression, alternative, and alwaysFalse}
%%% NUKE \RJ{Dont know what you mean, is this note LIVE? or can we DELETE?}

% Although
% @hpc@ reports only 42\% boolean coverage for @Data.Map@, manual inspection
% revealed that 22 of the guards marked by @hpc@ as ``always True'' are
% @otherwise@ guards and can never be false. In that light, it would be more
% accurate to consider 46/57 booleans as covered, \ie 82\% coverage. The remaining
% ``always True'' branches compared the size of subtrees when rebalancing to
% determine whether a single or double rotation was needed, in some cases we were
% unable to generate sufficiently large trees in one minute to trigger a double
% rotation. The two guards that were always false were due to the simplistic
% generator we currently use for higher-order functions always returning false.

\subsection{Discussion}\label{sec:discussion}

To sum up, our experiments demonstrate that \toolname generates valid inputs:
%
(1) where \quickcheck fails outright, due to the low probability of
    generating random values satisfying a property;
%
(2) more efficiently than \lazysmallcheck, which relies on lazy
    pruning predicates; and
%
(3) providing high code coverage for real-world libraries with no
    hand-written test cases.

% \subsection{Limitations of \toolname}\label{sec:limitations}

Of course our approach is not without drawbacks; we highlight five classes
of pitfalls the user may encounter.

\mypara{Laziness} in the function or in the output refinement can cause exceptions
  to go un-thrown if the output value is not fully demanded. For example,
  \toolname would decide that the result @[1, undefined]@ inhabits @[Int]@ but not
  @[Score]@, as the latter would have to evaluate @0 <= undefined < 100@. This
  limitation is not specific to our system, rather it is fundamental to any tool
  that exercises lazy programs. Furthermore, \toolname only generates
  inductively-defined values, it cannot generate infinite or cyclic structures,
  nor will the generated values ever contain $\bot$.

\mypara{Polymorphism} Like any other tool that actually runs the function under scrutiny,
  \toolname can only test monomorphic instantiations of polymorphic
  functions. For example, when testing @XMonad@ we instantiated the ``window''
  parameter to @Char@ and all other type parameters to @()@, as the properties
  we were testing only examined the window. This helped drastically reduce the
  search space, both for \toolname and \smallcheck.

  % Our monomorphism restriction simplifies \toolname's implementation as we do
  % not have to consider type-class or equality constraints when generating test
  % values, but it also reduces the generalizability of \toolname's
  % result. 
  % Parametricity helps by telling us that the choice of
  % concrete instantiation will not affect the behavior of the function, but
  % in the presence of type-classes the benefit is reduced as we only know that
  % the specific instance we tested is correct.

\mypara{Advanced type-system features} such as GADTs and Existential types
  may prevent GHC from deriving a @Generic@ instance, which would force the
  programmer to write her own @Targetable@ instance. Though tedious, the single
  hand-written instance allows \toolname to automatically generate values
  satisfying disparate constraints, which is still an improvement over the
  generate-and-filter approach.
  
\mypara{Refinement types} are less expressive than properties written in the
  host language. If the pre-conditions are not expressible in \toolname's logic,
  the user will have to use the generate-and-filter approach, losing the benefits
  of symbolic enumeration.
  
\mypara{Input explosion} \toolname excels when the space of valid inputs is
  a sparse subset of the space of all inputs. If the input space is not
  sufficiently constrained, \toolname may spend lose its competitive advantage
  over other tools due to the overhead of using a general-purpose solver.

%% 1. laziness
%%    - potential for untriggered exceptions
%%    - our generated values never include bot
%% 2. Advanced type-system features
%%    - we can only provide default instances for Generic types
%%      - no GADTs or existentials
%% 3. Polymorphic functions
%%    - can only test monomorphic instantiation
%%    - types must be defaulted either by user or GHC
%%    - limitation shared by any testing tool

%%% Local Variables:
%%% mode: latex
%%% TeX-master: "main"
%%% End:

\input{nate/qualitative}
\input{nate/threats}
\input{nate/discussion}
\input{nate/related2}
The goal of this work has been to improve the diagnostic feedback that
compilers provide when a program with no type annotations fails to type-check.
%
To that end, we have made three key contributions that advance the state
of the art in type error diagnosis.

\paragraph{Contribution 1: A Dataset of Novice Type Errors}
Our first contribution was a new dataset of novice interactions with the
\ocaml top-level interpreter (in particular, type errors they
encountered and their fixes).
%
The dataset contains thousands of ill-typed programs written by over one
hundred undergraduate students at UC San Diego, as well as the
subsequent fixes.
%
This is the largest set of novice type errors that we are aware of, has
formed the backbone of our evaluation, and will hopefully be similarly
useful to other researchers in the future.

\paragraph{Contribution 2: Dynamic Witnesses for Static Type Errors}
Second, we presented a novel technique for explaining type errors in
terms of the underlying runtime error the type system prevented.
%
We interleave type inference and execution to search for a witness to
the type error, a set of inputs that would cause the program to crash
at runtime.
%
We borrow the notion of ``holes'' from the automatic testing literature
to avoid spurious witnesses by delaying the selection of a concrete
input until execution has reached a point where we can be sure of its
type.
%
Once our search procedure finds a witness, we compute a full execution
trace that demonstrates how the program would evolve and eventually
crash.
%
We present this trace to the user in an interactive debugger that allows
the user to explore the erroneous computation in a familiar setting.

We proved that our search procedure produces general witnesses, \ie if
we can find a witness the program must be untypeable.
%
We showed empirically that most novice type errors, around 85\%, admit
witnesses, and that the vast majority can be found in under one second
by our search procedure.
%
We also found that students who were given our witnesses were more
likely to correctly explain and fix a type error than students who were
just given \ocaml's error message.
%
Finally, we found that our witnesses can also serve as a localization
method for type errors by treating the stuck term as a sink for typing
constraints and the values contained within it as sources.
%
Our witness-based localizations are substantially more accurate than
\ocaml's errors and competitive with the state of the art.

\paragraph{Contribution 3: Data-Driven Diagnosis of Type Errors}
Finally, we presented a novel technique for localizing type errors based
on observations of past errors and their fixes.
%
We use machine learning to train a classifier that predicts, given a
term from an ill-typed program, whether the term is likely to be changed
in the eventual fix (\ie is that term to blame for the error).
%
Given a new ill-typed program, we run the classifier for all program
terms and use its confidence score to rank the terms by the likelihood
that they should be blamed, selecting only the top three to present to
the user.
%
The classifier makes predictions based solely on the syntax and types of
the term and its immediate parent and children, and, crucially, whether
the term is part of a minimal type error slice.

Our classifier's top-ranked prediction is at least \ToolnameWinSherrloc
percentage points more accurate than the state of the art in type error
localization, and given three predictions it exceeds 90\% accuracy.
%
Furthermore, the classifier can be trained on a modest amount of data;
we obtained our results by training on programs from a single instance
of our undergraduate programming languages course at UC San Diego.
%
This makes us confident that even if our model does not generalize to
programs from other courses (or more broadly, to arbitrary \ocaml
programs), it is quite reasonable for instructors to train models of the
specific errors made by students in their courses.


\section{Future Work}
\label{sec:conc:future-work}
We will conclude this dissertation with a brief discussion of some
exciting future directions for this line of work.

\paragraph{Other Type Systems and Analyses}
This dissertation has focused on improving type errors for typed
functional languages based on the Hindley-Milner type system, but
there are a great many other type systems in use.
%
Thus, one promising direction for future work would be adapting our
techniques to other systems.

Both \tool{NanoMaLy} and \tool{Nate} have been designed to be parametric
in the type system, so in principle it should be straightforward to
adapt them to other languages and type systems.
%
\tool{NanoMaLy}'s use of the type system is mostly confined to the
\forcesym procedure that performs type-checking, thus one would need to
adapt \forcesym to the target type system. If the dynamic semantics of
the target language differ significantly from \ocaml's, one would also
need to replace the evaluation rules, but this is not a significant
burden either. \tool{NanoMaLy}'s evaluation rules are just those of
\ocaml, with a call to \forcesym inserted before every primitive
reduction.
%
In contrast, \tool{Nate} only uses the type system as a source of
features, so one would only need to extract alternative features from
the target type system.

We will next briefly outline a few classes of type systems and how
supporting them might differ from \ocaml.

\subparagraph{Dependent Types}
%
Dependent types~\citep{Bertot2013-ao,Norell2007-cj,Brady2013-nl}
and their close cousins,
refinement types~\citep{Xi1998-we,Dunfield2007-ei,Rondon2008-ea,Swamy2011-rq},
%
allow the programmer to specify complex invariants on their programs and
data, and can statically prevent many runtime errors that \ocaml cannot.
%
These systems have been used to prove
%
the absence out-of-bounds accesses~\citep{Xi1998-we,Rondon2008-ea,Vazou2014-xk},
%
complex data-structure invariants (\eg red-black tree balancing)~\citep{Vazou2014-xk,Kawaguchi2009-cl},
%
security policies~\citep{Bengtson2011-ep,Swamy2011-rq},
%
and even compiler correctness~\citep{Leroy2009-zs}.

As one might expect, it is much harder to prove such properties about
your programs than traditional type-safety, and thus programmers using
such systems may spend much more time investigating type errors.
%
Another consequence of the increased expressiveness is that these
systems generally do not support global type inference as it becomes
undecidable.
%
Thus, the debugging type errors in these systems is more about
\emph{understanding} the error than \emph{localizing} it.
%
A crucial question the programmer must answer is whether the error is a
legitimate bug in her program, or if the type checker simply lacks
enough information to \emph{prove} the program correct.

\tool{NanoMaLy}'s approach to searching for witnesses to type errors
could thus be quite helpful in these systems.
%
The presence of a witness proves that there is an actual bug, while the
absence may suggest that the programmer needs to supply some additional
lemmas to convince the type checker.
%
We have done some preliminary work in this area, showing that dependent
and refinement type signatures can be thought of as generators and
oracles for comprehensive test suites~\citep{Seidel2015-pe}.
%
However, that work only checked that a function satisfies its top-level
signature, we did not search for witnesses to the misuse of other
functions internally.
%
\citet{Petiot2016-fe} present a technique for determining if a proof
failure is due to a legitimate bug or a lack of available information,
but they only evaluate it in the context of first-order, imperative C
programs.
%
It would be very interesting to apply these techniques to dependent
and refinement type systems for functional languages.

\subparagraph{Objects}
%
Objects are a common feature in popular languages like \tool{C++},
\tool{Java}, and \tool{C\#}, offering code reuse via inheritance and
behavioral abstraction via interfaces.
%
A core feature of type systems that support objects is \emph{subtyping},
allowing values of the sub-type to be seamlessly used anywhere values of
the super-type are expected.
%
While these languages are gradually adopting type inference for local
variables, they generally require type annotations for
functions\footnote{\ocaml's own object system is a notable exception
  here.} as the addition of subtyping would force the compiler to guess
the programmer's intent for the input types.
%
For instance, did she want the function to accept objects of a specific
type, or objects of any type that implement a particular interface?

Since the type checker is (generally) not responsible for guessing the
programmer's intent in these systems, we suspect that type error
localization is unlikely to be as serious problem as it is in \ocaml;
however, these languages may still benefit from \tool{NanoMaLy}'s
approach to explaining type errors in terms of runtime errors.
%
Novice users, in particular, may find it easier to understand the error
when presented as a concrete runtime trace.
%
\citet{Bayne2011-cn} demonstrate a tool similar to \tool{NanoMaLy} that
allows programmers to execute ill-typed \tool{Java} programs, though
their aim is user-driven testing of a program in spite of potentially
irrelevant type errors, while ours is automated explanations of
type errors.
%
Furthermore, subtyping adds a similar question of whether the type error
is legitimate or if a function was simply being too conservative in its
input and output types, thus testing may help guide the programmer to a
solution as suggested above.

% \subparagraph{Gradual Types}
% %
% \ES{TODO}

\subparagraph{Information Flow Control}
%
Several authors have proposed type systems for tracking who may access
or modify certain pieces of data, \eg medical records or paper reviews,
in order to ensure confidentiality~\citep{Denning1977-kk,Heintze1998-hu,Myers2000-zd,Pottier2003-et,Stefan2011-cd}.
%
These systems typically associate a security label --- ranging from
simple ``high'' or ``low'' security label, to a set of privileged actors
--- in addition to a type with each object in the system, and ensure
that only actors with sufficient privileges can access restricted
objects.
%
A major complication from traditional type-checking is that \emph{implicit}
information flows must be tracked in addition to explicit flows.
%
While an explicit information flow could be returning a row from a
database table, an implicit flow occurs when the program makes an
observable decision based on a piece of information, \eg by branching
on the value of an object.
%
Implicit flows must be restricted so that malicious users cannot infer
privileged data by carefully constructed inputs to a system;
%
a common tactic is to conservatively propagate security labels via
implicit flows.

As with objects, these type systems do not perform global inference for
the security labels, thus the type checker is not trying to infer the
programmer's intent; however, the implicit flows can create a similar
issue of errors being reported far from their source.
%
Thus, we suspect that these systems could benefit from both approaches
presented in this dissertation: \tool{Nate}'s data-driven localizations
could improve the accuracy of error reports, and \tool{NanoMaLy}'s
witnesses could help explain the errors in terms of the undesirable
leaking of privileged data.

\paragraph{Fixing Type Errors}
Throughout this entire dissertation we have focused on localizing and
explaining type errors, but % many type errors are rather mundane things.
% %
% For example, perhaps we used \ocaml's |+| operator for |int|s rather
% than the |+.| operator for |float|s.
% %
% For these errors in particular
it would be nice to have a tool that
could simply \emph{fix} the error without any user intervention.

In addition to the techniques we mentioned in
\autoref{sec:fixing-type-errors} there is a wealth of existing work in
the broader field of automatic program repair~\citep[see][\S~4, for a
survey]{Le_Goues2013-ag} that we could draw from.
%
A core challenge in program repair is fault localization, \ie
determining where the repair should take place.
%
Thus, \textsc{Nate}'s ability to accurately locate the source of type
errors could provide a strong foundation on top of which to build repair
systems.

Another exciting opportunity for future work would be using machine
learning to predict fixes \emph{in addition} to blame labels.
%
In fact, this could be a very natural extension of our work on
\textsc{Nate}, we can frame it as a classification problem as follows.
%
Given a term from an ill-typed program, that we have identified as a
candidate for blame, we want to predict which local syntactic feature
should be enabled for the corresponding term in the program's fix.
%
For example, in our |sumList| example, we would like the classifier to
predict that the \textsc{Is-Int} feature should be enabled, instead of
the \textsc{Is-[]} feature, in the fix.
%
The problem then becomes a \emph{multi-label} classification problem, as
there are many possible syntactic features to choose from, but these
problems are also well studied in the machine learning literature.

This approach could work nicely for relatively simple fixes like for
|sumList|, or a use of the integer |+| rather than the floating-point
|+.|, \etc, but many fixes will require multiple edits to the program.
%
For example, even adding an extra argument to a function call would be a
multi-edit fix in \ocaml: first we would need to insert a new
application node with the old node as its left child, then we would need
to synthesize a new term for the right child.
%
Clearly, such a fix cannot be generated by \textsc{Nate} as is, since
there are an infinite number of possible terms but a finite number of
labels that we can predict.
%
However, there has been much work in the machine learning community on
models that can generate structured data, most notably
images~\citep{Gregor2015-ra} and text~\citep{Bahdanau2014-gt}, but also
program terms~\citep{Raychev2016-xk,Raychev2014-jv}.
%
A sizable challenge to adopting these techniques is that they tend to
require vast amounts of data to learn a precise model, and we have a
comparatively small amount of type-error data.
%
However, all is not lost.
%
The type-error data allowed us to predict an accurate location for the
fix, but the fix itself should be a type-\emph{correct} program, and
there are an abundance of type-correct programs publicly available in
software repositories like \textsc{GitHub}.
%
So it may be possible to train a precise model of type-correct programs
using publicly available data, and then use that model to generate
structured fixes to ill-typed programs.

% If we cannot confidently predict a syntactic fix, perhaps we can predict
% a type-level fix, \eg the |+| term should be replaced by a term with type
% |float -> float -> float|.

% \ES{investigate using witnesses to drive fixes?}

%%% Local Variables:
%%% mode: latex
%%% TeX-master: "main"
%%% End:


%\part{Conclusion}
\chapter{Conclusion}

\appendix
\chapter{Proofs}
\input{nanomaly/proofs}

%\chapter{User Studies}

\chapter{\tool{NanoMaLy} User Study}
\label{sec:nanomaly:user-study-exams}
% \includepdf[pages={-},pagecommand={},scale=0.65,frame,fitpaper]{user-study.pdf}
\newpage
\section{Version A}
\noindent\fbox{\includegraphics[width=\textwidth,page=4]{nanomaly/user-study.pdf}}
\newpage
\noindent\fbox{\includegraphics[width=\linewidth,page=5]{nanomaly/user-study.pdf}}
\newpage
\noindent\fbox{\includegraphics[width=\linewidth,page=6]{nanomaly/user-study.pdf}}
\newpage
\section{Version B}
\noindent\fbox{\includegraphics[width=\linewidth,page=1]{nanomaly/user-study.pdf}}
\newpage
\noindent\fbox{\includegraphics[width=\linewidth,page=2]{nanomaly/user-study.pdf}}
\newpage
\noindent\fbox{\includegraphics[width=\linewidth,page=3]{nanomaly/user-study.pdf}}

% Stuff at the end of the dissertation goes in the back matter
\backmatter
\bibliographystyle{abbrvnat} % Or whatever style you want like plainnat
\bibliography{main}
%\bibliography{nate/main,target/sw}
%\printbibliography

\end{document}
