The goal of this work has been to improve the diagnostic feedback that
compilers provide when a program fails to type-check.
%
To that end, we have made three key contributions that advance the state
of the art in type error diagnosis.

\paragraph{Contribution 1: A Dataset of Novice Type Errors}
Our first contribution was a new dataset of novice interactions with the
\ocaml top-level interpreter (in particular, type errors they
encountered and their fixes).
%
The dataset contains thousands of ill-typed programs written by over one
hundred undergraduate students at UC San Diego, as well as the
subsequent fixes.
%
This is the largest set of novice type errors that we are aware of, has
formed the backbone of our evaluation, and will hopefully be similarly
useful to other researchers in the future.

\paragraph{Contribution 2: Dynamic Witnesses for Static Type Errors}
Second, we presented a novel technique for explaining type errors in
terms of the underlying runtime error the type system prevented.
%
We interleave type inference and execution to search for a witness to
the type error, a set of inputs that would cause the program to crash
at runtime.
%
We borrow the notion of ``holes'' from the automatic testing literature
to avoid spurious witnesses by delaying the selection of a concrete
input until execution has reached a point where we can be sure of its
type.
%
Once our search procedure finds a witness, we compute a full execution
trace that demonstrates how the program would evolve and eventually
crash.
%
We present this trace to the user in an interactive debugger that allows
the user to explore the erroneous computation in a familiar setting.

We proved that our search procedure produces general witnesses, \ie if
we can find a witness the program must be untypeable.
%
We showed empirically that most novice type errors, around 85\%, admit
witnesses, and that the vast majority can be found in under one second
by our search procedure.
%
We also found that students who were given our witnesses were more
likely to correctly explain and fix a type error than students who were
just given \ocaml's error message.
%
Finally, we found that our witnesses can also serve as a localization
method for type errors by treating the stuck term as a sink for typing
constraints and the values contained within it as sources.
%
Our witness-based localizations are substantially more accurate than
\ocaml's errors and competitive with the state of the art.

\paragraph{Contribution 3: Data-Driven Diagnosis of Type Errors}
Finally, we presented a novel technique for localizing type errors based
on observations of past errors and their fixes.
%
We use machine learning to train a classifier that predicts, given a
term from an ill-typed program, whether the term is likely to be changed
in the eventual fix (\ie is that term to blame for the error).
%
Given a new ill-typed program, we run the classifier for all program
terms and use its confidence score to rank the terms by the likelihood
that they should be blamed, selecting only the top three to present to
the user.
%
The classifier makes predictions based solely on the syntax and types of
the term and its immediate parent and children, and, crucially, whether
the term is part of a minimal type error slice.

Our classifier's top-ranked prediction is at least \ToolnameWinSherrloc
percentage points more accurate than the state of the art in type error
localization, and given three predictions it exceeds 90\% accuracy.
%
Furthermore, the classifier can be trained on a modest amount of data;
we obtained our results by training on programs from a single instance
of our undergraduate programming languages course at UC San Diego.
%
This makes us confident that even if our model does not generalize to
programs from other courses (or more broadly, to arbitrary \ocaml
programs), it is quite reasonable for instructors to train models of the
specific errors made by students in their courses.


\section{Future Work}
\label{sec:conc:future-work}
We will conclude this dissertation with a brief discussion of some
exciting future directions for this line of work.

\paragraph{Other Type Systems}


\paragraph{Fixing Type Errors}
Throughout this entire dissertation we have focused on localizing and
explaining type errors, but % many type errors are rather mundane things.
% %
% For example, perhaps we used \ocaml's |+| operator for |int|s rather
% than the |+.| operator for |float|s.
% %
% For these errors in particular
it would be nice to have a tool that
could simply \emph{fix} the error without any user intervention.

In addition to the techniques we mentioned in
\autoref{sec:fixing-type-errors} there is a wealth of existing work in
the broader field of automatic program repair~\citep[see][\S~4, for a
survey]{Le_Goues2013-ag} that we could draw from.
%
A core challenge in program repair is fault localization, \ie
determining where the repair should take place.
%
Thus, \textsc{Nate}'s ability to accurately locate the source of type
errors could provide a strong foundation on top of which to build repair
systems.

Another exciting opportunity for future work would be using machine
learning to predict fixes \emph{in addition} to blame labels.
%
In fact, this could be a very natural extension of our work on
\textsc{Nate}, we can frame it as a classification problem as follows.
%
Given a term from an ill-typed program, that we have identified as a
candidate for blame, we want to predict which local syntactic feature
should be enabled for the corresponding term in the program's fix.
%
For example, in our |sumList| example, we would like the classifier to
predict that the \textsc{Is-Int} feature should be enabled, instead of
the \textsc{Is-[]} feature, in the fix.
%
The problem then becomes a \emph{multi-label} classification problem, as
there are many possible syntactic features to choose from, but these
problems are also well studied in the machine learning literature.

This approach could work nicely for relatively simple fixes like for
|sumList|, or a use of the integer |+| rather than the floating-point
|+.|, \etc, but many fixes will require multiple edits to the program.
%
For example, even adding an extra argument to a function call would be a
multi-edit fix in \ocaml: first we would need to insert a new
application node with the old node as its left child, then we would need
to synthesize a new term for the right child.
%
Clearly, such a fix cannot be generated by \textsc{Nate} as is, since
there are an infinite number of possible terms but a finite number of
labels that we can predict.
%
However, there has been much work in the machine learning community on
models that can generate structured data, most notably
images~\citep{Gregor2015-ra} and text~\citep{Bahdanau2014-gt}, but also
program terms~\citep{Raychev2016-xk,Raychev2014-jv}.
%
A sizable challenge to adopting these techniques is that they tend to
require vast amounts of data to learn a precise model, and we have a
comparatively small amount of type-error data.
%
However, all is not lost.
%
The type-error data allowed us to predict an accurate location for the
fix, but the fix itself should be a type-\emph{correct} program, and
there are an abundance of type-correct programs publicly available in
software repositories like \textsc{GitHub}.
%
So it may be possible to train a precise model of type-correct programs
using publicly available data, and then use that model to generate
structured fixes to ill-typed programs.

% If we cannot confidently predict a syntactic fix, perhaps we can predict
% a type-level fix, \eg the |+| term should be replaced by a term with type
% |float -> float -> float|.

% \ES{investigate using witnesses to drive fixes?}

%%% Local Variables:
%%% mode: latex
%%% TeX-master: "main"
%%% End:
