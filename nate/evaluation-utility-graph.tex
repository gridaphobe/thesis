\begin{figure}[t]
\centering
\begin{subfigure}[t]{\linewidth}
\centering
\begin{tikzpicture}
\begin{axis}[
  name=slice,
  % scale only axis,
  %at=(feature.above north), anchor=below south east,
  ybar stacked,
  width=0.5\linewidth,
  height=5cm,
  %title={Impact of Type Error Slice},
  ylabel={Accuracy},
  bar width=0.5cm,
  ymin=0,
  ymax=1,
  ytick={0.0, 0.1, 0.2, 0.3, 0.4, 0.5, 0.6, 0.7, 0.8, 0.9, 1.0},
  yticklabel={\pgfmathparse{\tick*100}\pgfmathprintnumber{\pgfmathresult}\,\%},
  ytick style={draw=none},
  ymajorgrids = true,
  symbolic x coords={op+slice-no-slice, op+slice, op+slice-only-slice},
  enlarge x limits=0.25,
  xtick=data,
  xtick style={draw=none},
  xticklabels={\textsc{Syntax}, +\InSlice, \textsc{Filter}},
  x tick label style={font=\small},
  y tick label style={font=\small},
  reverse legend,
  transpose legend,
  legend style={
    %legend pos = outer north east,
    at={(1.75,0.5)},
    anchor=center,
    legend columns=5
  },
]

% ES: NOTE: ORDER OF PLOTS/LEGEND ENTRIES MATTERS

\addplot+[stack plots=false, draw=black, fill=none, thick, bar shift=.25cm] table[x=features, y=recall] {\SliceHiddenBench};
\addlegendentry{Recall}
\addplot[draw=black, fill=green1, bar shift=.25cm] table[x=features, y=top-1] {\SliceHiddenBench};
\addlegendentry{Top-1}
\addplot[draw=black, fill=green2, bar shift=.25cm] table[x=features, y expr=\thisrow{top-2} - \thisrow{top-1}] {\SliceHiddenBench};
\addlegendentry{Top-2}
\addplot[draw=black, fill=green3, bar shift=.25cm] table[x=features, y expr=\thisrow{top-3} - \thisrow{top-2}] {\SliceHiddenBench};
\addlegendentry{Top-3}
\addlegendimage{empty legend}
\addlegendentry{\hiddenFH}

\resetstackedplots

\addplot+[stack plots=false, draw=black, fill=none, thick, bar shift=-.25cm] table[x=features, y=recall] {\SliceLinearBench};
\addlegendentry{Recall}
\addplot[draw=black, fill=blue1, bar shift=-.25cm] table[x=features, y=top-1] {\SliceLinearBench};
\addlegendentry{Top-1}
\addplot[draw=black, fill=blue2, bar shift=-.25cm] table[x=features, y expr=\thisrow{top-2} - \thisrow{top-1}] {\SliceLinearBench};
\addlegendentry{Top-2}
\addplot[draw=black, fill=blue3, bar shift=-.25cm] table[x=features, y expr=\thisrow{top-3} - \thisrow{top-2}] {\SliceLinearBench};
\addlegendentry{Top-3}
\addlegendimage{empty legend}
\addlegendentry{\linear}
\end{axis}
\begin{axis}[
  ybar stacked,
  width=0.5\linewidth,
  height=5cm,
  ylabel={Recall},
  axis y line*=right,
  ymin=0,
  ymax=1,
  ytick={0.0, 0.1, 0.2, 0.3, 0.4, 0.5, 0.6, 0.7, 0.8, 0.9, 1.0},
  yticklabel={\pgfmathparse{\tick*100}\pgfmathprintnumber{\pgfmathresult}\,\%},
  ytick style={draw=none},
  ymajorgrids = false,
  xmin=0, xmax=1,
  hide x axis,
]
\end{axis}
\end{tikzpicture}
\caption{Impact of type error slice on blame accuracy.}\label{fig:slice-utility}
\end{subfigure}

%\hfill\mbox{}

\vspace{1\baselineskip}


\begin{subfigure}[t]{\linewidth}
\begin{tikzpicture}
\begin{axis}[
  % name=feature,
  % scale only axis,
  ybar stacked,
  width=0.9\linewidth,
  height=6cm,
  %title={Impact of Contextual Feature Classes},
  ylabel={Accuracy},
  bar width=0.5cm,
  ymin=0.5,
  ymax=1,
  ytick={0.5, 0.6, 0.7, 0.8, 0.9, 1.0},
  yticklabel={\pgfmathparse{\tick*100}\pgfmathprintnumber{\pgfmathresult}\,\%},
  ytick style={draw=none},
  ymajorgrids = true,
  symbolic x coords={op, op+size, op+context, op+type, op+context+size, op+type+size, op+context+type, op+context+type+size},
  % enlarge x limits=0.5,
  xtick=data,
  xtick style={draw=none},
  xticklabel style={align=center},
  xticklabels={
    \textsc{Syntax}\\(45),
    \textsc{+Size}\\(46), \textsc{+Ctxt}\\(225), \textsc{+Type}\\(100),
    +C+S\\(226), +T+S\\(101), +C+T\\(281),
    +C+T+S\\(282)
    % + Size, + Context, + Type,
    % + Context + Size, + Type + Size, + Context + Type,
    % + Context + Type + Size
  },
  x tick label style={font=\small},
  y tick label style={font=\small},
  %x tick label style={rotate=45, anchor=north east},
  % reverse legend,
  % transpose legend,
  % legend style={
  %   anchor = south east,
  %   at = {(1,1)},
  %   %legend pos = outer north east,
  %   legend columns=4},
]

% ES: NOTE: ORDER OF PLOTS/LEGEND ENTRIES MATTERS

\addplot+[stack plots=false, draw=black, fill=none, thick, bar shift=.25cm] table[x=features, y=recall] {\FeatureHiddenBench};
%\addlegendentry{Recall}
\addplot[draw=black, fill=green1, bar shift=.25cm] table[x=features, y=top-1] {\FeatureHiddenBench};
% \addlegendentry{Top-1}
\addplot[draw=black, fill=green2, bar shift=.25cm] table[x=features, y expr=\thisrow{top-2} - \thisrow{top-1}] {\FeatureHiddenBench};
% \addlegendentry{Top-2}
\addplot[draw=black, fill=green3, bar shift=.25cm] table[x=features, y expr=\thisrow{top-3} - \thisrow{top-2}] {\FeatureHiddenBench};
% \addlegendentry{Top-3}
% \addlegendimage{empty legend}
% \addlegendentry{\hiddenFH}

\resetstackedplots

\addplot+[stack plots=false, draw=black, fill=none, thick, bar shift=-.25cm] table[x=features, y=recall] {\FeatureLinearBench};
\addplot[draw=black, fill=blue1, bar shift=-.25cm] table[x=features, y=top-1] {\FeatureLinearBench};
% \addlegendentry{Top-1}
\addplot[draw=black, fill=blue2, bar shift=-.25cm] table[x=features, y expr=\thisrow{top-2} - \thisrow{top-1}] {\FeatureLinearBench};
% \addlegendentry{Top-2}
\addplot[draw=black, fill=blue3, bar shift=-.25cm] table[x=features, y expr=\thisrow{top-3} - \thisrow{top-2}] {\FeatureLinearBench};
% \addlegendentry{Top-3}
% \addlegendimage{empty legend}
% \addlegendentry{\linear}


%\legend{Top-1, Top-2, Top-3}
\end{axis}
\begin{axis}[
  ybar stacked,
  width=0.9\linewidth,
  height=6cm,
  ylabel={Recall},
  axis y line*=right,
  ymin=0.5,
  ymax=1,
  ytick={0.5, 0.6, 0.7, 0.8, 0.9, 1.0},
  yticklabel={\pgfmathparse{\tick*100}\pgfmathprintnumber{\pgfmathresult}\,\%},
  ytick style={draw=none},
  ymajorgrids = false,
  xmin=0, xmax=1,
  hide x axis,
]
\end{axis}

\end{tikzpicture}
\caption{
  %
  Impact of additional features on blame accuracy.
  %
  Starting from a baseline of local syntactic features,
  we add each combination of
  expression size, contextual syntactic, and typing features.
  %
  The total number of features is given in parentheses.
}
\label{fig:context-utility}
\end{subfigure}

\caption{
  %
  Results of our experiments on feature utility.
%
}
\label{fig:slice-utility-results}
\end{figure}
