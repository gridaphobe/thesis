\mysection{Learning to Blame}
\label{sec:nate:learning}
In this section, we describe our approach to localizing type errors, in the
context
of \lang (\autoref{fig:syntax}), a simple lambda calculus with integers,
booleans, pairs, and lists.
%
Our goal is to instantiate the $\blamesym$ function of
\autoref{fig:api}, which takes as input a $\Model$ of type errors and an
ill-typed program $e$, and returns an ordered list of subexpressions
from $e$ paired with the confidence score $\Runit$ that they should be
blamed.

A $\Model$ is produced by $\trainsym$, which performs
supervised learning on a training set of feature vectors $\V$ and
(boolean) labels $\B$.
%
Once trained, we can $\evalsym$uate a $\Model$ on a new input,
producing the confidence $\Runit$ that the blame label should be
applied.
%
We describe multiple $\Model$s and their instantiations of
$\trainsym$ and $\evalsym$
(\autoref{sec:nate:models}).

Of course, the $\Model$ expects feature vectors $\V$ and blame labels
$\B$, but we are given program pairs.
%
So our first step must be to define a suitable translation from program
pairs to feature vectors and labels, \ie we must define the
$\extractsym$ function in \autoref{fig:api}.
%
We model features as real-valued functions of
terms, and extract a feature vector for each \emph{subterm}
of the ill-typed program (\autoref{sec:nate:features}).
%
Then we define the blame labels for the training set to be the
subexpressions that changed between the ill-typed program and its
subsequent fix, and model $\blamesym$ as a function from a program pair
to the set of expressions that changed (\autoref{sec:nate:labels}).
%
The $\extractsym$ function, then, extracts $\featuresym$ from each
subexpression and computes the blamed expressions according to
$\labelsym$.

% First, we define the inputs to the model, a set of \emph{features}
% (\autoref{sec:nate:features}) that we will use to describe programs.
% %
% Second, we define the expected outputs of the model, a set of
% \emph{labels} (\autoref{sec:nate:labels}) that we will use to assign blame.
% %
% Third, we describe the actual \emph{models} (\autoref{sec:nate:models}) that we
% will evaluate.

\begin{figure}[t]
\centering
\[
\boxed{
\begin{array}{rcl}
e & ::=    & x \spmid \efun{x}{e} \spmid \eapp{e}{e} \spmid \elet{x}{e}{e} \\
  & \spmid & n \spmid \eplus{e}{e}\\
  & \spmid & b \spmid \eif{e}{e}{e} \\
  & \spmid & \epair{e}{e} \spmid \epcase{e}{x}{x}{e} \\
  & \spmid & \enil \spmid \econs{e}{e} \spmid \elcase{e}{e}{x}{x}{e} \\[0.05in]

n & ::= &  0, 1, -1, \ldots \\[0.05in]

b & ::= &  \etrue \spmid \efalse \\[0.05in]

t & ::= & \alpha \spmid \tbool \spmid \tint \spmid \tfun{t}{t} \spmid \tprod{t}{t} \spmid \tlist{t} \\[0.05in]
\end{array}
}
\]
\caption{Syntax of \lang}
\label{fig:syntax}
\end{figure}
\lstDeleteShortInline{|} % sigh...
\begin{figure}[t]
\[
\boxed{
\begin{array}{lcl}
  \V          & \defeq & \List{\R}\\
  \Runit      & \defeq & \{r \in \R\ |\ 0 \le r \le 1\} \\ % [0,1]\\
  \featuresym & : & \List{e \to \R} \\
  \labelsym   & : & e \times e \to \List{e} \\
  \extractsym & : & \List{e \to \R} \to e \times e \to \List{\V \times \B} \\
  \trainsym   & : & \List{\V \times \B} \to \Model \\
  \evalsym    & : & \Model \to \V \to \Runit \\
  \midrule
  \blamesym   & : & \Model \to e \to \List{e \times \Runit}
\end{array}
}
\]
\caption{
  %
  A high-level API for converting program pairs to
  feature vectors and labels.
  % %
  % A feature function is a function from expressions $e$ to real numbers
  % $\R$.
  % %
  % A feature vector $\V$ is a list of real numbers,
  % %
}
\label{fig:api}
\end{figure}
\lstMakeShortInline{|}


\mysubsection{Features}
\label{sec:nate:features}

% \begin{figure}[ht]
% \begin{minipage}{0.6\linewidth}
% \begin{code}
%   let rec sumList xs =
%     match xs with
%     | []     -> []
%     | hd::tl -> hd + sumList tl
% \end{code}
% \end{minipage}
% \begin{minipage}{0.3\linewidth}
% \begin{code}[numbers=left]
% []
% hd + sumList tl
% sumList tl
% tl
% \end{code}
% \end{minipage}
% \caption{An ill-typed program (left) and a selection of its
%   sub-expressions (right).}
% \label{fig:sumList}
% \end{figure}


The first issue we must tackle is formulating our learning task in
machine learning terms.
%
We are given programs over \lang, but learning algorithms expect to work
with \emph{feature vectors} $\V$ --- vectors of real numbers, where each
column describes a particular aspect of the input.
%
Thus, our first task is to convert programs to feature vectors.

We choose to model a program as a \emph{set} of feature vectors, where
each element corresponds an expression in the program.
%
Thus, given the |sumList| program in \autoref{fig:sumList} we
would first split it into its constituent sub-expressions and then
transform each sub-expression into a single feature vector.
%
We group the features into five categories, using \autoref{tab:sumList}
as a running example of the feature extraction process.

\mypara{Local syntactic features}
These features describe the syntactic category of each expression $e$.
%
In other words, for each production of $e$ in \autoref{fig:syntax} we
introduce a feature that is enabled (set to $1$) if the expression was
built with that production, and disabled (set to $0$) otherwise.
%
For example, the \IsNil feature in \autoref{tab:sumList} describes
whether an expression is the empty list $\enil$.

We distinguish between matching on a list vs.\ on a pair, as this
affects the typing derivation.
%
We also assume that all pattern matches are well-formed --- \ie all
patterns must match on the same type.
%
Ill-formed match expressions would lead to a type error; however, they
are already effectively localized to the match expression itself.
%
We note that this is not a \emph{fundamental} limitation, and one could
easily add features that specify whether a match \emph{contains} a
particular pattern, and thus have a match expression that enables multiple
features.

\mypara{Contextual syntactic features}
These are similar to local syntactic features, but lifted to describe the
parent and children of an expression.
%
For example, the \IsCaseListP feature in \autoref{tab:sumList} describes
whether an expression's \emph{parent} matches on a list.
%
If a particular $e$ does not have children (\eg a variable $x$) or a
parent (\ie the root expression), we leave the corresponding features
disabled.
%
This gives us a notion of the \emph{context} in which an expression
occurs, similar to the \emph{n-grams} used in linguistic
models \citep{Hindle2012-hf,Gabel2010-el}.

% Instead of just describing the immediate context, we could describe
% whether a particular syntax element occurs in the neighboring
% expressions (or even a count of how many times it occurs).
% %
% For example, the \CountVarP feature in \autoref{tab:sumList} describes
% how many variables are contained in the expression \emph{rooted} at the
% current expression's parent.
% %
% Such fuzzier notions of context may enable increased precision in the
% model, but they also introduce opportunities for \emph{overfitting} ---
% where the model memorizes particular inputs rather than learning general
% patterns.
% %
% We will investigate (\ES{maybe..}) the impact of these alternatives
% in \autoref{sec:nate:evaluation}.

\mypara{Expression size}
We also propose a feature representing the \emph{size} of each expression,
\ie how many sub-expressions does it contain?
%
For example, the \ExprSize feature in \autoref{tab:sumList} is set to three
for the expression |sumList tl| as it contains three expressions:
the two variables and the application itself.
%
This allows the model to learn that, \eg, expressions closer to the
leaves are more likely to be blamed than expressions closer to the root.

\mypara{Typing features}
A natural way of summarizing the context in which an expression occurs
is with \emph{types}.
%
Of course, the programs we are given are \emph{untypeable}, but we can
still extract a \emph{partial} typing derivation from the type checker
and use it to provide more information to the model.

A difficulty that arises here is that, due to the parametric type
constructors $\tarr{\cdot}{\cdot}$, $\tprod{\cdot}{\cdot}$, and
$\tlist{\cdot}$, there is an \emph{infinite} set of possible types ---
but we must have a \emph{finite} set of features.
%
Thus, we abstract the type of an expression to the set of type
constructors it \emph{mentions}, and add features for each type
constructor that describe whether a given type mentions the type
constructor.
%
For example, the type $\tint$ would only enable the $\tint$ feature,
while the type $\tarr{\tint}{\tbool}$ would enable the
$\tarr{\cdot}{\cdot}$, $\tint$, and $\tbool$ features.

We add these features for parent and child expressions to summarize the
context, but also for the current expression, as the type of an
expression is not always clear \emph{syntactically}.
%
For example, the expressions |tl| and |sumList tl|
in \autoref{tab:sumList} both enable \HasTypeList, as they
are both inferred to have a type that mentions $\tlist{\cdot}$.
%constructor.

Note that our use of typing features in an ill-typed program subjects us
to \emph{traversal bias} \citep{McAdam1998-ub}. For example, the
|sumList tl| expression might alternatively be assigned the type
$\tint$.
%
Our models will have to learn good localizations in spite this bias (see
\autoref{sec:nate:evaluation}).

\mypara{Type error slice}
Finally, we wish to distinguish between changes that could fix the
error, and changes that \emph{cannot possibly} fix the error.
%
Thus, we compute a minimal type error \emph{slice} for the program
(\ie the set of expressions that contribute to the error), and add a
feature that is enabled for expressions that are part of the slice.
%
The \InSlice feature in \autoref{tab:sumList} indicates whether an
expression is part of such a minimal slice, and is enabled for all of
the sampled expressions except for |tl|, which does not affect
the type error.
%
If the program contains multiple type errors, we compute
a minimal slice for each error.

In practice, we have found that \InSlice is a particularly important
feature, and thus include a post-processing step that discards all
expressions where it is disabled.
%
As a result, the |tl| expression would never actually be shown to the
classifier.
%
 We will demonstrate the importance of \InSlice empirically in
\autoref{sec:nate:feature-utility}.

\mysubsection{Labels}
\label{sec:nate:labels}
Recall that we make predictions in two stages.
%
First, we use $\evalsym$ to predict for each subexpression whether it
should be blamed, and extract a confidence score $\Runit$ from the
$\Model$.
%
Thus, we define the output of the $\Model$ to be a boolean label, where
``false'' means the expression \emph{should not} change and ``true''
means the expression \emph{should} change.
%
This allows us to predict whether any individual expression should
change, but we would actually like to predict the \emph{most likely}
expressions to change.
%
Second, we \emph{rank} each subexpression by the confidence $\Runit$
that it should be blamed, and return to the user the top $k$
most likely blame assignments (in practice $k=3$).

% Many learning algorithms produce not only a prediction, but also a
% metric that can be interpreted as a \emph{confidence} in the prediction.
% %
% Thus, we \emph{rank} the expressions by the model's confidence that they
% will change, and select only the top $k$ (in practice $k=3$) to present
% to the user.

We identify the fixes for each ill-typed program with an
expression-level diff~\citep{Lempsink2009-xf}.
%
We consider two sources of changes.
%
%\begin{enumerate}
%\item 
First, if an expression has been removed wholesale, \eg if $\eapp{f}{x}$
  is rewritten to $\eapp{g}{x}$, we will mark the expression $f$ as
  changed, as it has been replaced by $g$.
Second, if a new expression has been inserted \emph{around} an existing
  expression, \eg if $\eapp{f}{x}$ is rewritten to
  $\eplus{\eapp{f}{x}}{1}$, we will mark the application expression
  $\eapp{f}{x}$ (but not $f$ or $x$) as changed, as the $+$ operator now
  occupies the original location of the application.
%\end{enumerate}

\mysubsection{Learning Algorithms}
\label{sec:nate:models}
\lstDeleteShortInline{|} % sigh...

Recall that we formulate type error detection at a single expression as
a supervised classification problem.
%
This means that we are given a training data set
%$S = \{ \langle v_i, b_i \rangle \}_{i=1}^n$
$S : \List{\V \times \B}$
of labeled examples, and
our goal is to use it to build a \emph{classifier}, \ie a rule
that can predict a label $b$ for an input $v$.
%
Since we apply the classifier on each expression in the program to
determine those that are the most likely to be type errors, we also
require the classifier to output a \emph{confidence score} that measures
how sure the classifier is about its prediction.


%Recall that we formulate type error localization as a supervised
%classification problem.
%
%This means that we assume the existence of a training set
%$S = \{\langle \mathbf{v}_i, b_i \rangle\}_{i=1}^{n}$
%of sample inputs $\mathbf{v}$ and their corresponding labels $b$.
%
%The learning task is to model the probability $Pr(b\ |\ \mathbf{v})$
%of a label being assigned to a given input.

There are many learning algorithms to choose from, existing on a
spectrum that balances expressiveness with ease of training (and of
interpreting the learned model).
%
In this section we consider four standard learning algorithms: (1)
logistic regression, (2) decision trees, (3) random forests, and (4)
neural networks.
%
A thorough introduction to these techniques can be found in introductory
machine learning textbooks \citep[\eg][]{Hastie2009-bn}.
% \ES{TODO: huma/kamalika?}).
%

Below we briefly introduce each technique by describing the rules it
learns, and summarize its advantages and disadvantages.
%
For our application, we are particularly interested in three properties
-- expressiveness, interpretability and ease of generalization.
%
Expressiveness measures how complex prediction rules are allowed to be,
and interpretability measures how easy it is to explain the cause of
prediction to a human.
%
Finally ease of generalization measures how easily the rule generalizes
to examples that are not in the training set; a rule that is not
easily generalizable might perform poorly on an unseen test set even
when its training performance is high.

%We will briefly introduce each technique by describing the model it
%learns, the training procedure, any hyper-parameters that must be tuned
%by a human, and a summary of its advantages and disadvantages.

% \ES{FIXME: Wes thinks the entire rest of this section, up to ``4. Evaluation'',
% contains much more detail than an OOPSLA audience wants on
% introductory/undergraduate machine learning. (Especially since nothing
% after this point is a novel contribution we make.) Perhaps we can cut some
% space by focusing on the pros and cons of each technique rather than
% describing the details? This is a PL paper. } \ES{KC: I completely agree. Perhaps we can add a table here. Judging from what we use, the features we need are: (a) expressivity (b) proneness to overfitting and (c) interpretability. We can rate the four classifiers in these three areas.
% }

\mypara{Logistic Regression}
The simplest classifier we investigate is logistic regression:
a linear model where the goal is to learn a set of weights $W$
that describe the following model for predicting a label
$b$ from a feature vector $v$:
%
\[ \Pr(b = 1 | v) = \frac{1}{1 + e^{-W^{\top} v}} \]
%
The weights $W$ are learnt from training data, and the value of
$\Pr(b | v)$ naturally leads to a confidence score $\Runit$.
%
Logistic regression is a widely used classification algorithm, preferred
for its simplicity, ease of generalization, and interpretability.
%
Its main limitation is that the prediction rule is constrained to be a
linear combination of the features, and hence relatively simple.
%
While this can be somewhat mitigated by adding higher order (quadratic
or cubic) features, this often requires substantial domain knowledge.

% The simplest model we investigate is logistic regression, which learns a
% linear function of the features.
% %
% The goal is to learn a set of weights $W$ and biases $b$ such that the
% function
% %
% $$
% Pr(y\ |\ \mathbf{x}) = \frac{1}{1 + \mathsf{exp}(-W\mathbf{x} - b)}
% $$
% %
% effectively maps feature vectors $\mathbf{x}$ to labels $y$.

% \ES{TODO: talk about logistic regression and the loss function?}

%The simplest model we investigate is logistic regression, which can be
%understood in two steps.
%
%First, we compute the \emph{evidence} for a label $b$ as a linear
%function of the feature vector $\mathbf{v}$, using a weight vector
%$\mathbf{w}$ and a bias $w_0$ that will be learned from the training set.
%$$
%\mathsf{ev} = \sum_j w_j v_j + w_0
%$$
%Then we use the logistic function (also called a sigmoid function)
%$\sigma(x) = 1 / (1 + e^{-x})$ to compress the evidence term into the
%$[0,1]$ interval, which can then be interpreted as the probability that
%the label $b$ should be applied.
%
%This gives us the final model:
%$$
%Pr(b\ |\ \mathbf{v}) = \frac{1}{1 + \mathsf{exp}(-\sum_j w_j v_j - w_0)}
%$$
% $$
% Pr(y\ |\ \mathbf{x}) = \frac{1}{1 + e^{-\mathsf{ev}}}
% $$

%Training a logistic regression entails finding optimal values for the
%weights and bias, such that a \emph{cost} function is minimized.
%
%In practice, the cost function used for logistic regression is the
%\emph{cross entropy}, a measure of the similarity between two
%probability distributions, between the predicted labels and the ground
%truth.
%
%The search for optimal values is typically done via \emph{stochastic
%  gradient descent}, an iterative method for optimizing a cost function.
%
%Starting with an initial estimate of the weights (\eg a normal
%distribution), one repeatedly makes predictions for the training data,
%computes the cost function for the predictions, and updates the
%parameters according to the gradient of the cost function.
%
%This process is repeated until the parameters converge, or until a time
%limit is exhausted.
%
%Stochastic gradient descent requires a hyper-parameter $\eta$, often
%called the \emph{learning rate}, which controls how much the weights and
%bias are updated at each iteration.

%It has been observed that large weights often coincide with
%\emph{overfitting} of the model --- where the model performs well on the
%training samples but not on the testing samples.
%
%Thus, it is common to add a \emph{regularization} term to the cost
%function, a popular choice is the $L_2$ norm of the weights
%$\sum_j w_j^2$, which has the effect of penalizing the model for
%learning large weights \citep{Park2008-no}.
%
%The contribution of the regularization term to the overall cost function
%is controlled by another hyper-parameter $\lambda$, often called the
%regularization rate.

%As a generalized linear model, logistic regression is popular for its
%ease of training and of interpreting the resulting model; however, it
%can be limited in its applicability.
%
%In particular, it may perform poorly when asked to learn an inherently
%\emph{nonlinear} function.
%
%This limitation can be mitigated by adding transformed or compound
%features to the model --- \eg if one knows that the label has a
%quadratic relationship with a feature $v$, the transformed feature $v^2$
%can be added --- but this typically requires knowledge of the underlying
%model.

% This limitation can be mitigated to some extent by adding more features
% to the model (\eg combinations of the existing features \ES{CITE?}), but
% the fundamental issue remains.

\mypara{Decision Trees}
Decision tree algorithms learn a tree of binary predicates over the
features, recursively partitioning the input space until a final
classification can be made.
%
Each node in the tree contains a single predicate of the form
$v_j \leq t$ for some feature $v_j$ and threshold $t$, which determines
whether a given input should proceed down the left or right subtree.
%
Each leaf is labeled with a prediction and the fraction of training
samples that would reach it; the latter quantity can be interpreted as
the decision tree's confidence in its prediction.
%
This leads to a prediction rule that can be quite expressive depending
on the data used to build it.

Training a decision tree entails finding both a set of good partitioning
predicates and a good ordering of the predicates based on data.
%
This is usually done in a top-down greedy manner, and there are several
standard training algorithms such as C4.5 \citep{Quinlan1993-de} and
CART \citep{Breiman1984-qy}.

Another advantage of decision trees is their ease of interpretation ---
the decision rule is a white-box model that can be readily described to
a human, especially when the tree is small.
%
However, the main limitation is that these trees often do not generalize
well, though this can be somewhat mitigated by \emph{pruning} the tree.
% this is still a well-known problem.


%predicates and a good ordering of the predicates.
%
%This is usually done in a top-down greedy manner by selecting the
%predicate the best partitions the entire training set, and recursively
%partitioning the subsets.
%
%Standard algorithms for training decision trees include C4.5
%\citep{Quinlan1993-de} and CART \citep{Breiman1984-qy}.
%
%Overfitting can also be a problem for decision trees; the countermeasure
%is usually either halting the learning process before the tree grows too
%deep, or pruning the resulting tree.

%The advantages of decision trees are that they are relatively easy to
%train and that they produce a \emph{white-box} model, making it trivial
%to analyze the entire model and explain individual predictions.
%
%The disadvantages are that they are prone to overfitting and that the
%model can be unstable, \ie small changes to the training data may
%produce large changes in the model.
%
%Both disadvantages are addressed by random forests.

\mypara{Random Forests}
%
Random forests improve generalization by training an
\emph{ensemble} of distinct decision trees and using a majority
vote to make a prediction.
%
The agreement among the trees forms a natural
confidence score.
%
Since each classifier in the ensemble is a decision tree, this still
allows for complex and expressive classifiers.
%\ES{Huma: is this the confidence score that we are using?}

%One technique for increasing the accuracy of a classifier is to train an
%\emph{ensemble} of distinct classifiers and use a majority vote
%to make the final prediction.
%
%Random forests are an instance of this method, with decision
%trees as the underlying classifiers.
%
The training process involves taking $N$ random subsets of the training
data and training a separate decision tree on each subset --- the
training process for the decision trees is often modified slightly to
reduce correlation between trees, by forcing each tree to pick features
from a random subset of all features at each node.
%
%In order to make a prediction for an unseen sample, the random forest
%asks each decision tree to make a prediction, and then predicts the most
%common label.

The diversity of the underlying models tends to make random forests less
susceptible to the overfitting, but it
also makes the learned model more difficult to interpret.
%
%Furthermore, the number of trees in the forest, $N$, becomes an important
%hyper-parameter that must be set manually.


\mypara{Neural Networks}
%
The last (and most complex) model we use is a type of neural network
called a \emph{multi-layer perceptron} (see \citealt{Nielsen2015-pu} for
an introduction to neural networks).
%
A multi-layer perceptron can be represented as a directed acyclic
graph whose nodes are arranged in layers that are fully connected by
weighted edges.
%
The first layer corresponds to the input features, and the final to the
output.
%
%The output of a node $v$ in an internal layer is given by:
The output of an internal node $v$ is
%
\[ h_v = g(\sum_{j \in N(v)} W_{jv} h_j ) \]
%
where $N(v)$ is the set of nodes in the previous layer that are adjacent
to $v$, $W_{jv}$ is the weight of the $(j, v)$ edge and $h_j$ is the
output of node $j$ in the previous layer.
%
Finally $g$ is a non-linear function, called the activation function,
which in recent work is commonly chosen to be the \emph{rectified linear
  unit} (ReLU), defined as $g(x) = \mathsf{max}(0,x)$
\citep{Nair2010-xg}.
%
The number of layers, the number of neurons per layer, and the
connections between layers constitute the \emph{architecture} of a
neural network.
%
In this work, we use relatively simple neural networks which have an
input layer, a single hidden layer and an output layer.

A major advantage of neural networks is their ability to discover
interesting combinations of features through non-linearity, which
significantly reduces the need for manual feature engineering, and
allows high expressivity.
%
On the other hand, this makes the networks particularly difficult to
interpret and also difficult to generalize unless vast amounts of
training data are available.

%HEREHERE
%
%Neural networks can be understood as a collection of linear models, just
%like the one used in logistic regression, arranged in a directed graph.
%
%Each node in the graph corresponds to a single linear model, called a
%\emph{neuron}, and the edges propagate signals from the input features,
%through the neurons, to the output labels.
%
%The linear model in each neuron is wrapped in an \emph{activation
%  function}, which controls whether (or how strongly) the signal will be
%propagated out.
%
%These activation functions are what allow neural networks to accurately
%model nonlinear functions.
%
%A common choice of activation function is the same sigmoid function used
%by logistic regression, alternatives include the hyperbolic $\tanh$
%function, and more recently the \emph{rectified linear unit} (ReLU),
%defined as $f(x) = \mathsf{max}(0,x)$ \citep{Nair2010-xg}.

%The neurons are arranged in \emph{layers}; each layer contains a set of
%neurons that take in signals from the previous layer and propagate
%signals to the next layer.
%
% and are all important hyper-parameters.
%
%We consider a particular class of neural networks called
%\emph{multi-layer perceptrons} (MLP), where the layers a fully connected to
%each other, \ie each neuron propagates its signal to every neuron in the
%subsequent layer.
%
%We will only consider MLPs with a single layer of neurons, but we will
%vary the number of neurons in our experiments.

%Neural networks are trained similarly to logistic regression using
%stochastic gradient descent, and thus the learning rate $\eta$ and
%regularization rate $\lambda$ are also important hyper-parameters.
%
%They are, however, much more difficult to train as they require larger
%training sets.

%Neural networks are popular for their ability to uncover ``hidden''
%features in the data.
%
%Each neuron learns a different linear combination of the signals from
%the previous layer; combined with the activation function this can be
%thought of as the neuron learning to respond to a new, \emph{compound}
%feature of its inputs.
%
%This ability to discover interesting combinations of features makes
%neural networks easier to use in some respect; they can reduce the
%importance of manual \emph{feature engineering} (selecting interesting
%features to provide as inputs).
%
%However, the sheer size and complexity of neural networks makes the
%model particularly difficult to interpret --- they are generally regarded
%as ``black boxes.''
%
%Neural networks are also very susceptible to overfitting, so
%regularization methods like the $L_2$ norm used for logistic regression
%are important.

%\ES{should mention that logistic regression can be thought of as a degenerate MLP with 0 layers, and perhaps add a diagram?}

%\ES{need to say something about neurons corresponding to ``hidden'' features, give intuition..}

% The fundamental building block of an ANN is the linear model, the
% nonlinearity arises from the \emph{activation functions} that govern
% whether an individual linear model propagates its signal to the next layer.




% ES: keep this at the bottom... sigh
\lstMakeShortInline{|}

%%% Local Variables:
%%% mode: latex
%%% TeX-master: "main"
%%% End:
