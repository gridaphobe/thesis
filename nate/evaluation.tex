\mysection{Evaluation}
\label{sec:nate:evaluation}
\pgfplotstableset{col sep=comma}

% \pgfplotstableread{nate/data/sp14/op+type+size/linear/results.csv}{\FeatureLinearBench}
% \pgfplotstablevertcat{\FeatureLinearBench}{nate/data/sp14/op+context+type+size/linear/results.csv}
% \pgfplotstablevertcat{\FeatureLinearBench}{nate/data/sp14/op+context-has+type+size/linear/results.csv}
% \pgfplotstablevertcat{\FeatureLinearBench}{nate/data/sp14/op+context-count+type+size/linear/results.csv}
% \pgfplotstableread{nate/data/sp14/op+type+size/hidden-10/results.csv}{\FeatureHiddenTBench}
% \pgfplotstablevertcat{\FeatureHiddenTBench}{nate/data/sp14/op+context+type+size/hidden-10/results.csv}
% \pgfplotstablevertcat{\FeatureHiddenTBench}{nate/data/sp14/op+context-has+type+size/hidden-10/results.csv}
% \pgfplotstablevertcat{\FeatureHiddenTBench}{nate/data/sp14/op+context-count+type+size/hidden-10/results.csv}
% \pgfplotstableread{nate/data/sp14/op+type+size/hidden-500/results.csv}{\FeatureHiddenFHBench}
% \pgfplotstablevertcat{\FeatureHiddenFHBench}{nate/data/sp14/op+context+type+size/hidden-500/results.csv}
% \pgfplotstablevertcat{\FeatureHiddenFHBench}{nate/data/sp14/op+context-has+type+size/hidden-500/results.csv}
% \pgfplotstablevertcat{\FeatureHiddenFHBench}{nate/data/sp14/op+context-count+type+size/hidden-500/results.csv}

% \pgfplotstableread{nate/data/sp14/op+type+size/hidden-10/results.csv}{\HiddenBench}
% \pgfplotstablevertcat{\HiddenBench}{nate/data/sp14/op+type+size/hidden-25/results.csv}
% \pgfplotstablevertcat{\HiddenBench}{nate/data/sp14/op+type+size/hidden-50/results.csv}
% \pgfplotstablevertcat{\HiddenBench}{nate/data/sp14/op+type+size/hidden-100/results.csv}
% \pgfplotstablevertcat{\HiddenBench}{nate/data/sp14/op+type+size/hidden-250/results.csv}
% \pgfplotstablevertcat{\HiddenBench}{nate/data/sp14/op+type+size/hidden-500/results.csv}
% % \pgfplotstablevertcat{\HiddenBench}{nate/data/sp14/op+context-count+type+size/hidden-10/results.csv}
% % \pgfplotstablevertcat{\HiddenBench}{nate/data/sp14/op+context-count+type+size/hidden-25/results.csv}
% % \pgfplotstablevertcat{\HiddenBench}{nate/data/sp14/op+context-count+type+size/hidden-50/results.csv}
% % \pgfplotstablevertcat{\HiddenBench}{nate/data/sp14/op+context-count+type+size/hidden-100/results.csv}
% % \pgfplotstablevertcat{\HiddenBench}{nate/data/sp14/op+context-count+type+size/hidden-250/results.csv}
% % \pgfplotstablevertcat{\HiddenBench}{nate/data/sp14/op+context-count+type+size/hidden-500/results.csv}

\pgfplotstableread{nate/data/sp14/baseline.csv}{\SpringBench}
\pgfplotstablevertcat{\SpringBench}{nate/data/sp14/ocaml/results.csv}
\pgfplotstablevertcat{\SpringBench}{nate/data/sp14/mycroft/results.csv}
\pgfplotstablevertcat{\SpringBench}{nate/data/sp14/sherrloc/results.csv}
\pgfplotstablevertcat{\SpringBench}{nate/data/sp14/op+context+type+size/linear/results.csv}
\pgfplotstablevertcat{\SpringBench}{nate/data/sp14/op+context+type+size/decision-tree/results.csv}
\pgfplotstablevertcat{\SpringBench}{nate/data/sp14/op+context+type+size/random-forest/results.csv}
\pgfplotstablevertcat{\SpringBench}{nate/data/sp14/op+context+type+size/hidden-10/results.csv}
\pgfplotstablevertcat{\SpringBench}{nate/data/sp14/op+context+type+size/hidden-500/results.csv}

\pgfplotstableread{nate/data/fa15/baseline.csv}{\FallBench}
\pgfplotstablevertcat{\FallBench}{nate/data/fa15/ocaml/results.csv}
\pgfplotstablevertcat{\FallBench}{nate/data/fa15/mycroft/results.csv}
\pgfplotstablevertcat{\FallBench}{nate/data/fa15/sherrloc/results.csv}
\pgfplotstablevertcat{\FallBench}{nate/data/fa15/op+context+type+size/linear/results.csv}
\pgfplotstablevertcat{\FallBench}{nate/data/fa15/op+context+type+size/decision-tree/results.csv}
\pgfplotstablevertcat{\FallBench}{nate/data/fa15/op+context+type+size/random-forest/results.csv}
\pgfplotstablevertcat{\FallBench}{nate/data/fa15/op+context+type+size/hidden-10/results.csv}
\pgfplotstablevertcat{\FallBench}{nate/data/fa15/op+context+type+size/hidden-500/results.csv}

\pgfplotstableread{nate/data/models/linear-op+slice-no-slice.cross.csv}{\SliceLinearBench}
\pgfplotstablevertcat{\SliceLinearBench}{nate/data/models/linear-op+slice.cross.csv}
\pgfplotstablevertcat{\SliceLinearBench}{nate/data/models/linear-op+slice-only-slice.cross.csv}
\pgfplotstableread{nate/data/models/hidden-500-op+slice-no-slice.cross.csv}{\SliceHiddenBench}
\pgfplotstablevertcat{\SliceHiddenBench}{nate/data/models/hidden-500-op+slice.cross.csv}
\pgfplotstablevertcat{\SliceHiddenBench}{nate/data/models/hidden-500-op+slice-only-slice.cross.csv}
% \pgfplotstablecreatecol[create col/assign/.code={%
%     \edef\entry{\thisrow{features}/\thisrow{model}}
%     \pgfkeyslet{/pgfplots/table/create col/next content}\entry
%   }]{tool}{\SliceBench}

\pgfplotstableread{nate/data/models/linear-op.cross.csv}{\FeatureLinearBench}
\pgfplotstablevertcat{\FeatureLinearBench}{nate/data/models/linear-op+size.cross.csv}
\pgfplotstablevertcat{\FeatureLinearBench}{nate/data/models/linear-op+context.cross.csv}
\pgfplotstablevertcat{\FeatureLinearBench}{nate/data/models/linear-op+type.cross.csv}
\pgfplotstablevertcat{\FeatureLinearBench}{nate/data/models/linear-op+context+size.cross.csv}
\pgfplotstablevertcat{\FeatureLinearBench}{nate/data/models/linear-op+type+size.cross.csv}
\pgfplotstablevertcat{\FeatureLinearBench}{nate/data/models/linear-op+context+type.cross.csv}
\pgfplotstablevertcat{\FeatureLinearBench}{nate/data/models/linear-op+context+type+size.cross.csv}
\pgfplotstableread{nate/data/models/linear-op.cross.csv}{\FeatureHiddenBench}
\pgfplotstablevertcat{\FeatureHiddenBench}{nate/data/models/hidden-500-op+size.cross.csv}
\pgfplotstablevertcat{\FeatureHiddenBench}{nate/data/models/hidden-500-op+context.cross.csv}
\pgfplotstablevertcat{\FeatureHiddenBench}{nate/data/models/hidden-500-op+type.cross.csv}
\pgfplotstablevertcat{\FeatureHiddenBench}{nate/data/models/hidden-500-op+context+size.cross.csv}
\pgfplotstablevertcat{\FeatureHiddenBench}{nate/data/models/hidden-500-op+type+size.cross.csv}
\pgfplotstablevertcat{\FeatureHiddenBench}{nate/data/models/hidden-500-op+context+type.cross.csv}
\pgfplotstablevertcat{\FeatureHiddenBench}{nate/data/models/hidden-500-op+context+type+size.cross.csv}
% \pgfplotstablecreatecol[create col/assign/.code={%
%     \edef\entry{\thisrow{features}/\thisrow{model}}
%     \pgfkeyslet{/pgfplots/table/create col/next content}\entry
%   }]{tool}{\FeatureBench}


We have implemented our technique for localizing type errors for a
purely functional subset of \ocaml with polymorphic types and functions.
%
We seek to answer three questions in our evaluation:
%
\begin{itemize}
\item \textbf{Blame Accuracy}
  %
  How often does \toolname
  blame a \emph{correct}
  location for the type error?
  (\autoref{sec:nate:quantitative})
  %
  % We compare our technique with a variety of off-the-shelf classifiers
  % and find that our top-ranked blame assignments have an accuracy of
  % 72\%, compared to a state-of-the-art 56\%.
  % For how many ill-typed programs can we accurately predict the source
  % of the error?
\item \textbf{Feature Utility}
  %
  Which program \emph{features are required}
  to localize errors?
   (\autoref{sec:nate:feature-utility})
  % How much do the features described in \autoref{sec:nate:features}
  % contribute to our predictions?
\item \textbf{Interpretability}
  %
  %% Do the models learned by \toolname
  %% correspond to our intuition about
  %% the real causes of errors?
  Are the models \emph{interpretable} using
  our intuition about the causes of type errors?
  (\autoref{sec:nate:qualitative})
\end{itemize}
%
%\mypara{Summary of Results}
%
In the sequel we present our experimental
methodology \autoref{sec:nate:methodology} and
then drill into how we evaluated each of
the questions above.
%
However, for the impatient reader, we begin
with a quick summary of our main results:
%
%
%\begin{itemize}
%
%\item \textbf

\mypara{1. Data Beats Algorithms}
Our main result is that for type error
localization, data is indeed unreasonably
effective \citep{Halevy2009-so}.
%
When trained on student errors from one
instance of an undergraduate course and
tested on another instance,
\toolname's most sophisticated
\emph{neural network}-based
classifier's top-ranked
prediction blames the correct
sub-term \HiddenFhTopOne\% of the time
--- a good \ToolnameWinSherrloc points
higher than the state-of-the-art
\sherrloc's \SherrlocTopOne\%.
%
However, even \toolname's simple
\emph{logistic regression} based
classifier is correct \LinearTopOne\% of the time,
\ie \LinearWinSherrloc points better than \sherrloc.
%
When the top three predictions are considered,
\toolname is correct \HiddenFhTopThree\% of the time.

% \item \textbf
\mypara{2. Slicing Is Critical}
%
However, data is effective \emph{only}
when irrelevant sub-terms have been
sliced out of consideration.
%
In fact, perhaps our most surprising
result is that type error slicing and
local syntax alone yields
a classifier that is \SlicingWinOcaml points
better than \ocaml and on par with
\sherrloc.
%
That is, once we focus our classifiers on
slices, purely local syntactic features
perform as well as the
state-of-the-art.

%\item \textbf
\mypara{3. Size Doesn't Matter, Types Do}
%
We find that (after slices)
typing features
provide the biggest
improvement in accuracy.
%
Furthermore, we find contextual syntactic
features to be mostly (but not entirely)
redundant with typing features,
which supports the hypothesis that
the context's \emph{type} nicely
summarizes the properties of the
surrounding expressions.
%
Finally, we found that the \emph{size}
of the sub-expression was not very useful.
This was unexpected, as we thought
smaller expressions would be simpler, and
hence, more likely causes.

% \item \textbf
\mypara{4. Models Learn Typing Rules}
%
Finally, by investigating a few of the
predictions made by the \emph{decision tree}-based
models, we found that the models
appear to capture some simple and intuitive
rules for predicting well-typedness.
%
For example, if the left child of an application
is a function, then the application is likely
correct.

% in an application, if the
% left argument is a function then the
% error is likely on the right sub-term;
% in a function definition
% \RJ{fixme: orig
% defining a function is a fine thing
% to do?}
%
%\end{itemize}
% \RJ{eric please check the numbers -- we say 71, 72, 74, 91, 92, 94?}


\mysubsection{Methodology}
\label{sec:nate:methodology}

We answer our questions on two sets of data gathered from the
undergraduate Programming Languages course at
% \begin{anonsuppress}
UC San Diego (IRB \#140608).
% \end{anonsuppress}
% \begin{noanonsuppress}
% AUTHOR's INSTITUTION.
% \end{noanonsuppress}
%
We recorded each interaction with the \ocaml top-level system while the
students worked on 23 programs from the first three homework
assignments, capturing ill-typed programs and, crucially, their
subsequent fixes.
%
The first dataset comes from the Spring 2014 class (\SPRING), with a
cohort of 46 students. The second comes from the Fall 2015 class
(\FALL), with a cohort of 56 students.
%
The extracted programs are relatively small, but they demonstrate a
range of functional programming idioms, \eg higher-order functions and
(polymorphic) algebraic data types.

\mypara{Feature Selection}
We extract 282 features from each sub-expression in a
program, including:
%
\begin{enumerate}
\item 45 local syntactic features. In addition to the syntax of \lang,
  we support the full range of arithmetic operators (integer and
  floating point), equality and comparison operators, character and
  string literals, and a user-defined % |expr| type of simple
  arithmetic
  expressions. We discuss the challenge of supporting other
  % user-defined
  types in \autoref{sec:nate:discussion}.
\item 180 contextual syntactic features. For each sub-expression we
  additionally extract the local syntactic features of its parent and
  first, second, and third (left-to-right) children. If an expression
  does not have a parent or children, these features will simply be
  disabled. If an expression has more than three children, the
  classifiers will receive no information about the additional
  children.
\item 55 typing features. In addition to the types of \lang, we support
  |int|s, |float|s, |char|s, |string|s, and the user-defined |expr|
  mentioned above. These features are extracted for each sub-expression
  and its context. % for the contextual sub-expressions.
\item One feature denoting the size of each sub-expression.
\item One feature denoting whether each sub-expression is part of the
  minimal type error slice. We use this feature as a ``hard''
  constraint, sub-expressions that are not part of the minimal slice
  will be preemptively discarded. We justify this decision in
  \autoref{sec:nate:feature-utility}.
\end{enumerate}

\mypara{Blame Oracle}
Recall from \autoref{sec:nate:labels} that we automatically extract a blame
oracle for each ill-typed program from the (AST) diff between it and the
student's eventual fix.
%
A disadvantage of using diffs in this manner is that students may have
made many, potentially unrelated, changes between compilations; at some
point the ``fix'' becomes a ``rewrite''.
%
We do not wish to consider the ``rewrites'' in our evaluation, so we
discard outliers where the fraction of expressions that have changed is
more than one standard deviation above the mean, establishing a diff
threshold of 40\%.
%
This accounts for roughly 14\% of each dataset, leaving us with
2,712 program pairs for \SPRING and 2,365 pairs for \FALL.

% we discard any program pairs where more than 40\%
% of the sub-expressions have changed.
% %
% We picked 40\% as an estimate of the inflection point where we could
% still retain the large majority of program pairs.
% % FIXME: Can you say that this dataset curation is similar to any other
% % datasets (e.g., the washington one)? Anything you could cite and discuss
% % here would take some of the pressure off.


\mypara{Accuracy Metric}
All of the tools we compare (with the exception of the standard \ocaml
compiler) can produce a list of potential error locations.
%
However, in a study of fault localization techniques,
\citet{Kochhar2016-oc} show that most developers will not consider more
than around five potential error locations before falling back to manual
debugging.
%
Type errors are relatively simple in comparison to general fault
localization, thus we limit our evaluation to the top three predictions
of each tool.
%
We evaluate each tool on whether a changed expression occurred in its
top one, top two, or top three predictions.

\mysubsection{Blame Accuracy}
\label{sec:nate:quantitative}

First, we compare the accuracy of our predictions to the
state of the art in type error localization.

\mypara{Baseline}
We provide two baselines for the comparison: a random choice of location
from the minimized type error slice, and the standard \ocaml compiler.

\mypara{State of the Art}
\mycroft~\citep{Loncaric2016-uk} localizes type errors by searching for
a minimal subset of typing constraints that can be removed, such that
the resulting system is satisfiable.
%
When multiple such subsets exist it can enumerate them, though it has no
notion of which subsets are \emph{more likely} to be correct, and thus
the order is arbitrary.
%
\sherrloc~\citep{Zhang2014-lv} localizes errors by searching the typing
constraint graph for constraints that participate in many unsatisfiable
paths and comparatively few satisfiable paths.
%
It can also enumerate multiple predictions, in descending order of
likelihood.

Comparing source locations from multiple tools with their own parsers is
not trivial.
%
Our experimental design gives the state of the art tools the ``benefit
of the doubt'' in two ways.
% To ensure a fair comparison when evaluating \mycroft and
% \sherrloc,
First, when evaluating \mycroft and \sherrloc, we did not consider
programs where they predicted locations that our oracle could not match
with a program expression: around 6\% of programs for \mycroft and 4\%
for \sherrloc.
%
Second, we similarly ignored programs where \mycroft or \sherrloc timed
out (after one minute) or where they encountered an unsupported language
feature: another 5\% for \mycroft and 12\% for \sherrloc.
%

\mypara{Our Classifiers}
We evaluate five classifiers, each trained on the full feature set.
% features: 44 local syntactic features, 176 contextual syntactic
% features, 55 typing features, and a single expression size feature.
% %
% \ES{should explain the make-up of these groups}
%
% We preemptively discard expressions that are not part of the minimal
% type error slice --- we will explain the rationale for this in
% \autoref{sec:nate:feature-utility} --- and thus the final feature count is
% 276.
%
These include:
%Our classifiers are:
%
\begin{description}
\item[\linear] A logistic regression trained with a learning rate
  $\eta = 0.001$, an $L_2$ regularization rate $\lambda = 0.001$, and a
  mini-batch size of 200.
\item[\dectree] A decision tree trained with the CART algorithm
  \citep{Breiman1984-qy} and an impurity threshold of $10^{-7}$ (used to
  avoid overfitting via early stopping).
\item[\forest] A random forest \citep{Breiman2001-wo} of 30
  estimators, with an impurity threshold of $10^{-7}$.
\item[\hiddenT and \hiddenFH] Two multi-layer perceptron neural
  networks, both trained with $\eta = 0.001$, $\lambda = 0.001$, and a
  mini-batch size of 200.
  %
  The first MLP contains a single hidden layer of 10 neurons, and the
  second contains a hidden layer of 500 neurons.
  %
  This gives us a measure of the complexity of the MLP's model, \ie
  if the model requires many compound features, one would expect \hiddenFH
  to outperform \hiddenT.
  % This allows us to investigate how well the MLP can \emph{compress} its
  % model (cf.~\cite{FIXME}).
  %
  The neurons use rectified linear units (ReLU) as their activation
  function, a common practice in modern neural networks.
\end{description}
%
All classifiers were trained for 20 epochs on one dataset
--- \ie they were shown each program 20 times ---
before being evaluated on the other.
%
The logistic regression and MLPs were trained with the \textsc{Adam}
optimizer \citep{Kingma2014-ng}, a variant of stochastic gradient
descent that has been found to converge faster.


% colors from http://colorbrewer2.org/?type=sequential&scheme=Blues&n=3
\definecolor{blue1}{HTML}{DEEBF7}
\definecolor{blue2}{HTML}{9ECAE1}
\definecolor{blue3}{HTML}{3182BD}
\definecolor{green1}{HTML}{E5F5E0}
\definecolor{green2}{HTML}{A1D99B}
\definecolor{green3}{HTML}{31A354}

% \begin{figure}[ht]
% \centering
% \begin{tikzpicture}
% \begin{axis}[
%   % ybar stacked,
%   width=12cm,
%   height=8cm,
%   title={Impact of Feature Set on Accuracy},
%   ylabel={Accuracy},
%   %ymin=0.2,
%   ymax=1,
%   yticklabel={\pgfmathparse{\tick*100}\pgfmathprintnumber{\pgfmathresult}\,\%},
%   ytick style={draw=none},
%   ymajorgrids = true,
%   symbolic x coords={op+type+size, op+context+type+size, op+context-has+type+size, op+context-count+type+size},
%   % enlarge x limits=0.25,
%   xtick=data,
%   xtick style={draw=none},
%   xticklabels={Type, Context-Is, Context-Has, Context-Count},
%   x tick label style={rotate=45},
%   reverse legend,
%   transpose legend,
%   legend style={legend pos = outer north east, legend columns=4},
% ]
% % \addplot[draw=black, fill=blue1] table[x=tool, y=top-1] {\HiddenBench};
% % \addplot[draw=black, fill=blue2] table[x=tool, y expr=\thisrow{top-2} - \thisrow{top-1}] {\HiddenBench};
% % \addplot[draw=black, fill=blue3] table[x=tool, y expr=\thisrow{top-3} - \thisrow{top-2}] {\HiddenBench};

% \addplot[mark options={fill=blue1, scale=1.5}, mark=square*]
%   table[x=features, y=top-1] {\FeatureHiddenFHBench};
% \addplot[mark options={fill=blue2, scale=1.5}, mark=square*]
%   table[x=features, y=top-2] {\FeatureHiddenFHBench};
% \addplot[mark options={fill=blue3, scale=1.5}, mark=square*]
%   table[x=features, y=top-3] {\FeatureHiddenFHBench};
% \addlegendentry{Top-1}
% \addlegendentry{Top-2}
% \addlegendentry{Top-3}
% \addlegendimage{empty legend}
% \addlegendentry{\hiddenFH}

% \addplot[mark options={fill=blue1, scale=1.5}, mark=*]
%   table[x=features, y=top-1] {\FeatureLinearBench};
% \addplot[mark options={fill=blue2, scale=1.5}, mark=*]
%   table[x=features, y=top-2] {\FeatureLinearBench};
% \addplot[mark options={fill=blue3, scale=1.5}, mark=*]
%   table[x=features, y=top-3] {\FeatureLinearBench};
% \addlegendentry{Top-1}
% \addlegendentry{Top-2}
% \addlegendentry{Top-3}
% \addlegendimage{empty legend}
% \addlegendentry{\linear}

% \end{axis}
% \end{tikzpicture}
% \caption{reuslts!}
% \label{fig:results}
% \end{figure}

% \begin{figure}[ht]
% \centering
% \begin{tikzpicture}
% \begin{axis}[
%   ybar stacked,
%   width=12cm,
%   height=8cm,
%   title={Impact of Hidden Layer Size on Accuracy},
%   ylabel={Accuracy},
%   bar width=20pt,
%   %ymin=0.2,
%   ymax=1,
%   yticklabel={\pgfmathparse{\tick*100}\pgfmathprintnumber{\pgfmathresult}\,\%},
%   ytick style={draw=none},
%   ymajorgrids = true,
%   symbolic x coords={op+type+size/hidden-10, op+type+size/hidden-25, op+type+size/hidden-50,
%                      op+type+size/hidden-100, op+type+size/hidden-250, op+type+size/hidden-500},
%   % enlarge x limits=0.25,
%   xtick=data,
%   xtick style={draw=none},
%   xticklabels={\hiddenT, \hiddenTF, \hiddenF, \hiddenH, \hiddenTHF, \hiddenFH},
%   x tick label style={rotate=45},
%   reverse legend,
%   legend style={legend pos = north west},
% ]
% \addplot[draw=black, fill=blue1] table[x=tool, y=top-1] {\HiddenBench};
% \addplot[draw=black, fill=blue2] table[x=tool, y expr=\thisrow{top-2} - \thisrow{top-1}] {\HiddenBench};
% \addplot[draw=black, fill=blue3] table[x=tool, y expr=\thisrow{top-3} - \thisrow{top-2}] {\HiddenBench};
% % \addplot[draw=black, fill=blue1] table[x=tool, y=top-1] {\HiddenBench};
% % \addplot[draw=black, fill=blue2] table[x=tool, y=top-2] {\HiddenBench};
% % \addplot[draw=black, fill=blue3] table[x=tool, y=top-3] {\HiddenBench};
% \legend{Top-1, Top-2, Top-3}
% \end{axis}
% \end{tikzpicture}
% \caption{reuslts!}
% \label{fig:results}
% \end{figure}

\begin{figure}[t]
\centering
\begin{tikzpicture}
\begin{axis}[
  ybar stacked,
  width=14cm,
  height=6cm,
  title={Accuracy of Type Error Localization Techniques},
  ylabel={Accuracy},
  bar width=0.5cm,
  ymin=0.2,
  ymax=1,
  ytick={0.2, 0.3, 0.4, 0.5, 0.6, 0.7, 0.8, 0.9, 1.0},
  yticklabel={\pgfmathparse{\tick*100}\pgfmathprintnumber{\pgfmathresult}\,\%},
  ytick style={draw=none},
  ymajorgrids = true,
  symbolic x coords={baseline, ocaml, mycroft, sherrloc,
                     op+context+type+size/linear,
                     op+context+type+size/decision-tree,
                     op+context+type+size/random-forest,
                     op+context+type+size/hidden-10,
                     op+context+type+size/hidden-500},
  %enlarge x limits=0.07,
  xtick=data,
  xtick style={draw=none},
  xticklabels={\random, \ocaml, \mycroft, \sherrloc,
               \linear, \dectree, \forest, \hiddenT, \hiddenFH},
  x tick label style={rotate=45, anchor=north east},
  %x tick label style={font=\small},
  y tick label style={font=\small},
  reverse legend,
  transpose legend,
  legend style={legend pos = north west, legend columns=4, font=\footnotesize},
]

% ES: NOTE: ORDER OF PLOTS/LEGEND ENTRIES MATTERS

\addplot[draw=black, fill=green1, bar shift=.25cm] table[x=tool, y=top-1] {\FallBench};
\addlegendentry{Top-1}
\addplot[draw=black, fill=green2, bar shift=.25cm] table[x=tool, y expr=\thisrow{top-2} - \thisrow{top-1}] {\FallBench};
\addlegendentry{Top-2}
\addplot[draw=black, fill=green3, bar shift=.25cm] table[x=tool, y expr=\thisrow{top-3} - \thisrow{top-2}] {\FallBench};
\addlegendentry{Top-3}
\addlegendimage{empty legend}
\addlegendentry{\FALL}

\resetstackedplots

\addplot[draw=black, fill=blue1, bar shift=-.25cm] table[x=tool, y=top-1] {\SpringBench};
\addlegendentry{Top-1}
\addplot[draw=black, fill=blue2, bar shift=-.25cm] table[x=tool, y expr=\thisrow{top-2} - \thisrow{top-1}] {\SpringBench};
\addlegendentry{Top-2}
\addplot[draw=black, fill=blue3, bar shift=-.25cm] table[x=tool, y expr=\thisrow{top-3} - \thisrow{top-2}] {\SpringBench};
\addlegendentry{Top-3}
\addlegendimage{empty legend}
\addlegendentry{\SPRING}


%\legend{Top-1, Top-2, Top-3}
\end{axis}
\end{tikzpicture}
\caption[Results of our comparison of type error localization
  techniques.]{
  %
  Results of our comparison of type error localization
  techniques.
  %
  We evaluate all techniques separately on two cohorts of
  students from different instances of an undergraduate
  Programming Languages course.
  %
  Our classifiers were trained on one cohort and evaluated on the other.
  %
  All of our classifiers outperform the state-of-the-art techniques
  \mycroft and \sherrloc.%  by a 10--15\% margin in Top-1 accuracy (with
%   the exception of \linear which is only slightly better than \sherrloc).
%
}
\label{fig:accuracy-results}
\end{figure}


\mypara{Results}
\autoref{fig:accuracy-results} shows the results of our experiment.
%
Localizing the type errors in our benchmarks amounted, on average, to
selecting one of 3 correct locations out of a slice of 10.
%
Our classifiers consistently outperform the competition, ranging from
\LinearTopOne\% Top-1 accuracy (\LinearTopThree\% Top-3)
for the \linear classifier to
\HiddenFhTopOne\% Top-1 accuracy (\HiddenFhTopThree\% Top-3)
for the \hiddenFH.\@
%
Our baseline of selecting at random achieves \BaselineTopOne\% Top-1
accuracy (\BaselineTopThree\% Top-3),
while \ocaml achieves a Top-1 accuracy of \OcamlTopOne\%.
%
Interestingly, one only needs two \emph{random} guesses to outperform
\ocaml, with \BaselineTopTwo\% accuracy.
%
\sherrloc outperforms both baselines, and comes close to our \linear classifier,
with \SherrlocTopOne\% Top-1 accuracy (\SherrlocTopThree\% Top 3),
while \mycroft underperforms \ocaml at \MycroftTopOne\% Top-1 accuracy.
%
% Finally, we find that \emph{all} of our classifiers outperform \sherrloc,
% ranging from 58--62\% Top-1 accuracy (86--88\% Top-3) for the \linear
% classifier to 71--74\% Top-1 accuracy (91\% Top-3) for the \hiddenFH.

Surprisingly, there is little variation in accuracy between our
classifiers.
%
With the exception of the \linear model, they all achieve around 70\%
Top-1 accuracy and around 90\% Top-3 accuracy.
%
This suggests that the model they learn is relatively simple.
%
In particular, notice that although the \hiddenT has $50\times$ \emph{fewer}
hidden neurons than the \hiddenFH, it only loses around 4\% accuracy.
% In particular, notice that the \hiddenT only loses around 2\% accuracy
% compared to the \hiddenFH,
%
We also note that our classifiers consistently perform better when
trained on the \FALL programs and tested on the \SPRING programs than
vice versa.
% , they appear to generalize better from the \FALL data.
% FIXME: Why? What is your explanation for this? Is it just sizes of those
% datasets or something qualitative about the program pairs in them?

\mysubsection{Feature Utility}
\label{sec:nate:feature-utility}
We have shown that we can train a classifier to effectively localize
type errors, but which of the feature classes from
\autoref{sec:nate:features} are contributing the most to our accuracy?
%
We focus specifically on feature \emph{classes} rather than individual
features as our 282 features are conceptually grouped into a much
smaller number of \emph{categorical} features.
%
For example, the syntactic class of an expression is conceptually a
feature but there are 45 possible values it could take; to encode this
feature for learning we split it into 45 distinct binary features.
%
Analyses that focus on individual features, \eg \textsc{ANOVA},
are difficult to interpret in our setting, as they will tell us the
importance of the binary features but not the higher-level categorical
features.
%
Thus, to answer our question we investigate the performance of
classifiers trained on various subsets of the feature classes.

\mysubsubsection{Type Error Slice}
\label{sec:nate:type-error-slice}
First we must justify our decision to automatically exclude expressions
outside the minimal type error slice from consideration.
%
% The \InSlice feature should be highly predictive --- a fix must change
% at least one expression in the type error slice.
% %
% Thus, our first experiment seeks to quantify the impact of \InSlice by
% comparing the accuracy of our classifiers on three sets of features:
%
Thus, we compare our classifiers on three sets of features:
%
\begin{enumerate}
\item A baseline with only local syntactic features and no
  preemptive filtering by \InSlice.
\item The features of (1) extended with \InSlice.
\item The same features as (1), but we preemptively discard samples
  where \InSlice is disabled.
\end{enumerate}
%
The key difference between (2) and (3) is that a classifier for (2) must
\emph{learn} that \InSlice is a strong predictor.
%
In contrast, a classifier for (3) must only learn about the syntactic
features, the decision to discard samples where \InSlice is disabled has
already been made by a human.
%
This has a few additional advantages: it reduces the set of candidate
locations by a factor of 7 on average, and it guarantees that any
prediction made by the classifier can fix the type error.
%
We expect that (2) will perform better than (1) as it contains more
information, and that (3) will perform better than (2) as the classifier
does not have to learn the importance of \InSlice.

% FIXME: Wes feels that there should be a sentence in the next mypara
% explaining to the reader why we didn't just use an ANOVA or the ReliefF
% method or whatever to figure out feature importance. Feature overlap?

We tested our hypothesis with the \linear and
%
\hiddenFH\footnote{A layer of 500 neurons is excessive when we have so few
  input features --- we use \hiddenFH for continuity with the
  surrounding sections.}
%
classifiers, cross-validated ($k=10$) over the combined SP14/FA15
dataset.
% We used a learning rate $\eta=0.001$, $L_2$ regularization rate
%$\lambda=0.001$, and mini-batch size of 200.
%
We trained for a single epoch on feature sets (1) and (2), and for 8
epochs on (3), so that the total number of training samples would be
roughly equal for each feature set.
%
\lstDeleteShortInline{|} % sigh...
In addition to accuracy, we report each
classifier's \emph{recall} --- \ie ``How many true changes can we
remember?'' --- defined as
$$
\frac{|\mathsf{predicted} \cap \mathsf{oracle}|}
     {|\mathsf{oracle}|}
$$
where $\mathsf{predicted}$ is limited to the top 3 predictions, and
$\mathsf{oracle}$ is the student's fix, limited to changes that are in
the type error slice.
%
We make the latter distinction as:
%
(1) changes that are not part of the type error slice are noise in the
data set; and
%
(2) it makes the comparison easier to interpret since $\mathsf{oracle}$
never changes.
% NOTE: keep this at the end of the para or it screws up spacing...
\lstMakeShortInline{|}
%
\begin{figure}[t]
\centering
\begin{subfigure}[t]{\linewidth}
\centering
\begin{tikzpicture}
\begin{axis}[
  name=slice,
  % scale only axis,
  %at=(feature.above north), anchor=below south east,
  ybar stacked,
  width=0.5\linewidth,
  height=4cm,
  %title={Impact of Type Error Slice},
  ylabel={Accuracy},
  bar width=0.5cm,
  ymin=0.2,
  ymax=1,
  ytick={0.2, 0.3, 0.4, 0.5, 0.6, 0.7, 0.8, 0.9, 1.0},
  yticklabel={\pgfmathparse{\tick*100}\pgfmathprintnumber{\pgfmathresult}\,\%},
  ytick style={draw=none},
  ymajorgrids = true,
  symbolic x coords={op+slice-no-slice, op+slice, op+slice-only-slice},
  enlarge x limits=0.25,
  xtick=data,
  xtick style={draw=none},
  xticklabels={\textsc{Local Syntax}, +\InSlice, \textsc{Filter \InSlice}},
  x tick label style={font=\small},
  y tick label style={font=\small},
  reverse legend,
  transpose legend,
  legend style={
    %legend pos = outer north east,
    at={(1.75,0.5)},
    anchor=center,
    legend columns=5
  },
]

% ES: NOTE: ORDER OF PLOTS/LEGEND ENTRIES MATTERS

\addplot+[stack plots=false, draw=black, fill=none, thick, bar shift=.25cm] table[x=features, y=recall] {\SliceHiddenBench};
\addlegendentry{Recall}
\addplot[draw=black, fill=green1, bar shift=.25cm] table[x=features, y=top-1] {\SliceHiddenBench};
\addlegendentry{Top-1}
\addplot[draw=black, fill=green2, bar shift=.25cm] table[x=features, y expr=\thisrow{top-2} - \thisrow{top-1}] {\SliceHiddenBench};
\addlegendentry{Top-2}
\addplot[draw=black, fill=green3, bar shift=.25cm] table[x=features, y expr=\thisrow{top-3} - \thisrow{top-2}] {\SliceHiddenBench};
\addlegendentry{Top-3}
\addlegendimage{empty legend}
\addlegendentry{\hiddenFH}

\resetstackedplots

\addplot+[stack plots=false, draw=black, fill=none, thick, bar shift=-.25cm] table[x=features, y=recall] {\SliceLinearBench};
\addlegendentry{Recall}
\addplot[draw=black, fill=blue1, bar shift=-.25cm] table[x=features, y=top-1] {\SliceLinearBench};
\addlegendentry{Top-1}
\addplot[draw=black, fill=blue2, bar shift=-.25cm] table[x=features, y expr=\thisrow{top-2} - \thisrow{top-1}] {\SliceLinearBench};
\addlegendentry{Top-2}
\addplot[draw=black, fill=blue3, bar shift=-.25cm] table[x=features, y expr=\thisrow{top-3} - \thisrow{top-2}] {\SliceLinearBench};
\addlegendentry{Top-3}
\addlegendimage{empty legend}
\addlegendentry{\linear}
\end{axis}
\begin{axis}[
  ybar stacked,
  width=0.5\linewidth,
  height=4cm,
  ylabel={Recall},
  axis y line*=right,
  ymin=0.2,
  ymax=1,
  ytick={0.2, 0.3, 0.4, 0.5, 0.6, 0.7, 0.8, 0.9, 1.0},
  yticklabel={\pgfmathparse{\tick*100}\pgfmathprintnumber{\pgfmathresult}\,\%},
  ytick style={draw=none},
  ymajorgrids = false,
  xmin=0, xmax=1,
  hide x axis,
]
\end{axis}
\end{tikzpicture}
\caption{Impact of type error slice on blame accuracy.}\label{fig:slice-utility}
\end{subfigure}

%\hfill\mbox{}

\vspace{1\baselineskip}


\begin{subfigure}[t]{\linewidth}
\begin{tikzpicture}
\begin{axis}[
  % name=feature,
  % scale only axis,
  ybar stacked,
  width=0.9\linewidth,
  height=5cm,
  %title={Impact of Contextual Feature Classes},
  ylabel={Accuracy},
  bar width=0.5cm,
  ymin=0.5,
  ymax=1,
  ytick={0.5, 0.6, 0.7, 0.8, 0.9, 1.0},
  yticklabel={\pgfmathparse{\tick*100}\pgfmathprintnumber{\pgfmathresult}\,\%},
  ytick style={draw=none},
  ymajorgrids = true,
  symbolic x coords={op, op+size, op+context, op+type, op+context+size, op+type+size, op+context+type, op+context+type+size},
  % enlarge x limits=0.5,
  xtick=data,
  xtick style={draw=none},
  xticklabel style={align=center},
  xticklabels={
    \textsc{Local Syn}\\(45),
    \textsc{+Size}\\(46), \textsc{+Context}\\(225), \textsc{+Type}\\(100),
    +C+S\\(226), +T+S\\(101), +C+T\\(281),
    +C+T+S\\(282)
    % + Size, + Context, + Type,
    % + Context + Size, + Type + Size, + Context + Type,
    % + Context + Type + Size
  },
  x tick label style={font=\small},
  y tick label style={font=\small},
  %x tick label style={rotate=45, anchor=north east},
  % reverse legend,
  % transpose legend,
  % legend style={
  %   anchor = south east,
  %   at = {(1,1)},
  %   %legend pos = outer north east,
  %   legend columns=4},
]

% ES: NOTE: ORDER OF PLOTS/LEGEND ENTRIES MATTERS

\addplot+[stack plots=false, draw=black, fill=none, thick, bar shift=.25cm] table[x=features, y=recall] {\FeatureHiddenBench};
%\addlegendentry{Recall}
\addplot[draw=black, fill=green1, bar shift=.25cm] table[x=features, y=top-1] {\FeatureHiddenBench};
% \addlegendentry{Top-1}
\addplot[draw=black, fill=green2, bar shift=.25cm] table[x=features, y expr=\thisrow{top-2} - \thisrow{top-1}] {\FeatureHiddenBench};
% \addlegendentry{Top-2}
\addplot[draw=black, fill=green3, bar shift=.25cm] table[x=features, y expr=\thisrow{top-3} - \thisrow{top-2}] {\FeatureHiddenBench};
% \addlegendentry{Top-3}
% \addlegendimage{empty legend}
% \addlegendentry{\hiddenFH}

\resetstackedplots

\addplot+[stack plots=false, draw=black, fill=none, thick, bar shift=-.25cm] table[x=features, y=recall] {\FeatureLinearBench};
\addplot[draw=black, fill=blue1, bar shift=-.25cm] table[x=features, y=top-1] {\FeatureLinearBench};
% \addlegendentry{Top-1}
\addplot[draw=black, fill=blue2, bar shift=-.25cm] table[x=features, y expr=\thisrow{top-2} - \thisrow{top-1}] {\FeatureLinearBench};
% \addlegendentry{Top-2}
\addplot[draw=black, fill=blue3, bar shift=-.25cm] table[x=features, y expr=\thisrow{top-3} - \thisrow{top-2}] {\FeatureLinearBench};
% \addlegendentry{Top-3}
% \addlegendimage{empty legend}
% \addlegendentry{\linear}


%\legend{Top-1, Top-2, Top-3}
\end{axis}
\begin{axis}[
  ybar stacked,
  width=0.9\linewidth,
  height=5cm,
  ylabel={Recall},
  axis y line*=right,
  ymin=0.5,
  ymax=1,
  ytick={0.5, 0.6, 0.7, 0.8, 0.9, 1.0},
  yticklabel={\pgfmathparse{\tick*100}\pgfmathprintnumber{\pgfmathresult}\,\%},
  ytick style={draw=none},
  ymajorgrids = false,
  xmin=0, xmax=1,
  hide x axis,
]
\end{axis}

\end{tikzpicture}
\caption{
  %
  Impact of contextual features on blame accuracy.
  %
  % Starting from a baseline of local syntactic features,
  % we add each combination of
  % expression size, contextual syntactic, and typing features.
  %
  The total number of features is given in parentheses.
}
\label{fig:context-utility}
\end{subfigure}

\caption{
  %
  Results of our experiments on feature utility.
%
}
\label{fig:slice-utility-results}
\end{figure}

% \begin{table}[ht]
%   \caption{
%     Impact of Type Error Slice on Accuracy.
%     \ES{TODO: load these numbers from CSV}
%   }\label{tab:type-error-slice}
%   \centering
%   \begin{tabular}{lrcrrrrcrrrr}
%     \toprule
%                        &             & & \multicolumn{4}{c} \linear        & & \multicolumn{4}{c} \hiddenFH      \\
%                                          \cmidrule{4-7}                        \cmidrule{9-12}
%     Feature Set        & \# Features & & Top-1  & Top-2  & Top-3  & Recall & & Top-1  & Top-2  & Top-3  & Recall \\
%     \midrule
%     Local Syntax       & 44          & & 27.7\% & 46.7\% & 58.6\% & 38.2\% & & 32.6\% & 49.5\% & 60.2\% & 39.4\% \\
%     + \InSlice         & 45          & & 46.4\% & 65.1\% & 76.1\% & 48.9\% & & 55.4\% & 71.0\% & 82.1\% & 57.1\% \\
%     Filter by \InSlice & 44          & & 55.9\% & 71.9\% & 82.9\% & 57.5\% & & 57.8\% & 72.7\% & 82.9\% & 57.6\% \\
%     \bottomrule
%   \end{tabular}
% \end{table}

\mypara{Results}
\autoref{fig:slice-utility} shows the results of our experiment.
%
As expected, the baseline performs the worst, with a mere 25\% \linear
Top-1 accuracy.
%
Adding \InSlice improves the results substantially with a 45\% \linear Top-1
accuracy, demonstrating the importance of a minimal error slice.
%
However, filtering out expressions that are not part of the slice
\emph{further} improves the results to 54\% \linear Top-1 accuracy.
%
Interestingly, while the \hiddenFH performs similarly poor with no error
slice features, it recovers nearly all of its accuracy after being given
the error slice features.
%
Top-1 accuracy jumps from 29\% to 53\% when we add \InSlice, and only
improves by 1\% when we filter out expressions that are not part of the
error slice.
%
Still, the accuracy gain comes at zero cost, and given the other benefits
of filtering by \InSlice % out expressions that do not belong to the type error slice
--- shrinking the search space and guaranteeing our predictions are actionable ---
we choose to filter all programs by \InSlice.

\mysubsubsection{Contextual Features}
\label{sec:nate:contextual-features}

We investigate the relative impact of the other
three classes of features discussed in \autoref{sec:nate:features}, assuming
we have discarded expressions not in the type error slice.
%
For this experiment we consider again a baseline of only local syntactic
features, extended by each combination of
%
(1) expression size;
(2) contextual syntactic features; and
(3) typing features.
%
As before, we perform a 10-fold cross-validation,
% with $\eta = 0.001$,
% $\lambda = 0.001$, and a mini-batch size of 200
but we train for a full 20 epochs to make the differences more apparent.
%
% \begin{table}[ht]
%   \caption{
%     Impact of Contextual Features on Accuracy.
%     \ES{TODO: load these numbers from CSV}
%   }\label{tab:contextual-features}
%   \centering
%   \begin{tabular}{lrcrrrrcrrrr}
%     \toprule
%                              &             & & \multicolumn{4}{c} \linear        & & \multicolumn{4}{c} \hiddenFH      \\
%                                                \cmidrule{4-7}                        \cmidrule{9-12}
%     Feature Set              & \# Features & & Top-1  & Top-2  & Top-3  & Recall & & Top-1  & Top-2  & Top-3  & Recall \\
%     \midrule
%     Local Syntax             &  44         & & 56.6\% & 72.6\% & 83.2\% & 57.7\% & & 57.6\% & 72.6\% & 83.1\% & 57.7\% \\
%     \midrule
%     + Size                   &  45         & & 57.2\% & 73.7\% & 83.0\% & 57.4\% & & 60.9\% & 75.5\% & 84.1\% & 57.9\% \\
%     + Context                & 220         & & 61.1\% & 78.7\% & 86.7\% & 63.0\% & & 71.5\% & 84.8\% & 90.8\% & 69.0\% \\
%     + Types                  & 102         & & 63.1\% & 77.8\% & 85.2\% & 61.6\% & & 73.4\% & 85.4\% & 90.7\% & 68.9\% \\
%     \midrule
%     + Context + Size         & 221         & & 61.1\% & 78.8\% & 86.3\% & 62.4\% & & 71.5\% & 84.4\% & 90.8\% & 69.0\% \\
%     + Types + Size           & 103         & & 61.9\% & 78.7\% & 85.6\% & 62.0\% & & 73.1\% & 85.2\% & 91.1\% & 69.5\% \\
%     + Context + Types        & 275         & & 63.1\% & 80.9\% & 88.0\% & 65.0\% & & 77.2\% & 88.3\% & 92.5\% & 72.6\% \\
%     \midrule
%     + Context + Types + Size & 276         & & 62.8\% & 80.5\% & 88.1\% & 65.1\% & & 77.3\% & 88.1\% & 92.7\% & 72.7\% \\
%     \bottomrule
%   \end{tabular}
%   % \begin{minipage}{0.49\linewidth}
%   % \centering
%   % \hiddenF
%   % \begin{tabular}{lrrrr}
%   %   \toprule
%   %   Feature Set                 & Top-1  & Top-2  & Top-3  & Recall \\
%   %   \midrule
%   %   Local Syntax                & 56.9\% & 72.2\% & 82.8\% & 57.9\% \\
%   %   \midrule
%   %   + Size                      & 59.7\% & 74.6\% & 83.0\% & 57.4\% \\
%   %   + Context                   & 70.9\% & 83.7\% & 90.4\% & 69.2\% \\
%   %   + Types                     & 72.1\% & 84.1\% & 90.3\% & 69.3\% \\
%   %   \midrule
%   %   + Size + Context            & 69.8\% & 83.5\% & 90.2\% & 68.6\% \\
%   %   + Size + Types              & 72.3\% & 84.6\% & 90.3\% & 69.5\% \\
%   %   + Context + Types           & 75.5\% & 86.4\% & 91.5\% & 71.7\% \\
%   %   \midrule
%   %   + All                       & 75.0\% & 86.8\% & 91.9\% & 72.0\% \\
%   %   \bottomrule
%   % \end{tabular}
%   % \end{minipage}
% \end{table}
%
\mypara{Results}
\autoref{fig:context-utility} summarizes the results of this experiment.
%
The \linear classifier and the \hiddenFH start off
competitive when given only local syntactic features, but the \hiddenFH
quickly outperforms as we add features.

\ExprSize is the weakest feature, improving \linear Top-1
accuracy by less than 1\% and \hiddenFH by only 4\%.
%
In contrast, the contextual syntactic features improve \linear Top-1
accuracy by 5\% (\resp 16\%), and the typing features improve
Top-1 accuracy by 6\% (\resp 18\%).
%
Furthermore, while \ExprSize does provide some benefit when it is the
only additional feature, it does not appear to provide any real increase
in accuracy when added alongside the contextual or typing features.
%
This is likely explained by \emph{feature overlap}, \ie the contextual
features of ``child'' expressions additionally provide some information
about the size of the subtree.

As one might expect, the typing features are more beneficial than the
contextual syntactic features.
%
They improve Top-1 accuracy by an additional 1\% (\resp 3\%), and are much more
compact --- requiring only 55 typing features compared to 180
contextual syntactic features.
%
This aligns with our intuition that types should be a good summary of
the context of an expression.
%
However, typing features do not appear to \emph{subsume} contextual
syntactic features, the \hiddenFH gains an additional 4\% Top-1 accuracy
when both are added.



%%% Local Variables:
%%% mode: latex
%%% TeX-master: "main"
%%% End:
