In this chapter we describe a collection of novice interactions with the
\ocaml compiler --- including, importantly, type errors --- that we
gathered at UC San Diego over two quarters of the undergraduate CSE 130
course (IRB \#140608).
%
We have made the anonymized data publicly
available~\citep{Seidel2017-ko}, and hope that other researchers will
find it as valuable as we have.

The CSE 130 course is an upper-level (\ie generally consisting of third-
and fourth-year students) course that introduces students to typed
functional languages, specifically \ocaml.
%
For most students, this course will be their first exposure to both
functional programming and type systems with global inference.
%
We generally spend the first five weeks covering basic functional
programming in \ocaml, and then spend the last five weeks in
\textsc{Scala} covering more advanced concepts like traits and
monads (in the guise of |for-yield| comprehensions).
%
In the \ocaml portion of the course we cover standard functional idioms
like (tail-) recursion, higher-order functions, and user-defined
algebraic datatypes.

We recruited students from two instances of the course, Spring 2014
(\SPRING) and Fall 2015 (\FALL), to use an instrumented version of the
\textsc{Ocaml-Top} editor~\ES{CITE}, which logged each of their
interactions with the \ocaml top-level system.
%
46 students from the \SPRING quarter and 56 students from the \FALL
quarter participated, for a total of 102 participants.
%
The participants used our instrumented editor to complete the first
three programming assignments, which involved writing 23 \ocaml
programs.
%
