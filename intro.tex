Static type systems are a marvelous invention.
%
They rule out, at compile-time, an entire class of run-time failures,
\eg dereferencing a |null|-pointer or passing an |int| where a |bool|
was expected.
%
Languages like \ocaml and \haskell make
the value-proposition for types even more
appealing by using constraints to automatically
synthesize the types for all program terms,
without troubling the programmer for any
annotations.
%
Unfortunately, this automation comes at a price.
%
Type annotations signify the programmer's intent, and help to correctly
blame the erroneous sub-term when the code is ill-typed.
%
In the absence of such signifiers, automatic type inference algorithms
are prone to report type errors far from their source
\citep{Wand1986-nw}.
%
While this can seem like a minor annoyance to veteran programmers,
\citet{Joosten1993-yx} have found that novices often focus their
attention on the \emph{location} reported and disregard the
\emph{message}.

In this dissertation we present two new techniques designed to help
\emph{localize} and \emph{explain} type errors.
%
First, in this chapter, we begin with a review of the existing
literature on type errors.
%
Second, we present a new dataset of novice interactions with the \ocaml
type-checker, which will form the backbone of our evaluation.
%
Finally, we present our techniques and evaluate them in the context of
localizing and explaining novice type errors.

\section{A Running Example}
\label{sec:intro:sumList}

\begin{figure}[t]
\small
\begin{minipage}{0.45\linewidth}
\begin{ecode}
let rec sumList xs =
  match xs with
  | []   -> (*@\hlSherrloc{[]}@*)
  | h::t -> h + (*@\hlTree{sumList t}@*)
\end{ecode}
\end{minipage}
\begin{minipage}{0.49\linewidth}
\begin{verbatim}
File "sumList.ml", line 4, characters 16-25:
  This expression has type 'a list
  but an expression was expected of type int

\end{verbatim}
\end{minipage}
\caption{(left) An ill-typed \ocaml program that should sum the elements of a
  list, with highlights indicating two possible blame assignments based on:
  %
  (1) the \hlTree{\ocaml} compiler; and
  %
  (2) the student's \hlSherrloc{fix}.
  %
  % The \underline{underlined} expressions are equally valid
  % locations to blame. The expression blamed by the \ocaml compiler
  % is in \textbf{bold}.
  %
  % FIXME: This bolding is ambiguous, because ``let rec'', ``match'' and
  % ``with'' are also bolded (by \\ecode)! You need to find another way to
  % highlight what ocaml is yelling about.
  %
  (right) The error reported by \ocaml.}
\label{fig:intro:sumList}
\end{figure}

Consider the \ocaml program in \autoref{fig:intro:sumList}, which is
supposed to sum the integers in a list.
%
This program was written by an undergraduate student at UC San Diego,
and works as follows.
%
In functional languages like \ocaml, lists are recursively defined as
either the empty list (written |[]| and pronounced ``nil''), or a single
element |h| followed by the rest of the list |t| (written |h::t| and
pronounced ``h cons t'')\footnote{These variable names are conventional,
  \texttt{h} stands for ``head'' and \texttt{t} for ``tail''.}.
%
Given an input |xs|, the student |match|es it against the
two forms a list can take.
%
In the |[]| case she returns another empty list |[]|, and in the
|h::t| case she adds |h| to the recursive sum of |t|.

The observant reader will notice that this program is incorrect, given
\emph{any} non-empty list of integers, the addition on line 4 will
attempt to add an integer to |[]|, which is an invalid operation.
%
In fact, the program is \emph{ill-typed} and the \ocaml compiler rejects
it with the error message in \autoref{fig:intro:sumList}.
%
Unfortunately, \ocaml's error \emph{blames} the recursive call to
|sumList|, explaining that |sumList| returns a |list|, while the |+|
operator requires an |int|.
%
The real error is on line 3, where the student returns |[]| rather than
|0| as the sum of an empty list.

As we will see throughout the rest of this chapter, this rather simple
program is sufficient to illustrate many of the difficulties of
automatically locating the source of a type error, and explaining it to
the programmer.

\section{A Simple Type System}
\label{sec:intro:simple-type}

\begin{figure}[t]
\centering
\[
\boxed{
\begin{array}{rrcl}
\mathbf{Expressions}
& e & ::=    & x \spmid \efun{x}{e} \spmid \eapp{e}{e} \spmid \elet{x}{e}{e} \\
&   & \spmid & n \spmid \eplus{e}{e}\\
    % & \spmid & b \spmid \eif{e}{e}{e} \\
    % & \spmid & \epair{e}{e} \spmid \epcase{e}{x}{x}{e} \\
&   & \spmid & \enil \spmid \econs{e}{e} \spmid \elcase{e}{e}{x}{x}{e} \\[0.05in]

%\mathbf{Numbers}
& n & ::= &  0, 1, -1, \ldots \\[0.2in]

% b & ::= &  \etrue \spmid \efalse \\[0.05in]
\mathbf{Types}
& t & ::= & \alpha \spmid % \tbool \spmid
            \tint \spmid \tarr{t}{t} \spmid % \tprod{t}{t} \spmid
            \tlist{t} \\[0.05in]
\end{array}
}
\]
\begin{framed}
\judgementHead{Typing}{\jtype{\Gamma}{e}{t}}
\begin{gather*}
\inference[\textsc{Var}]
  {x:t \in \Gamma}
  {\jtype{\Gamma}{x}{t}}
\quad
\inference[\textsc{Lam}]
  {\jtype{\Gamma, x:t_1}{e}{t_2}}
  {\jtype{\Gamma}{\efun{x}{e}}{\tarr{t_1}{t_2}}}
\quad
\inference[\textsc{App}]
  {\jtype{\Gamma}{e_1}{\tarr{t_1}{t_2}}
   \quad
   \jtype{\Gamma}{e_2}{t_1}}
  {\jtype{\Gamma}{\eapp{e_1}{e_2}}{t_2}}
\\[0.1in]
\inference[\textsc{Lit}]
  {}
  {\jtype{\Gamma}{n}{\tint}}
\quad
\inference[\textsc{Plus}]
  {\jtype{\Gamma}{e_1}{\tint} \quad \jtype{\Gamma}{e_2}{\tint}}
  {\jtype{\Gamma}{\eplus{e_1}{e_2}}{\tint}}
\\[0.1in]
\inference[\textsc{Nil}]
  {\alpha \text{ is fresh}}
  {\jtype{\Gamma}{\enil}{\tlist{\alpha}}}
\quad
\inference[\textsc{Cons}]
  {\jtype{\Gamma}{e_1}{t} \quad \jtype{\Gamma}{e_2}{\tlist{t}}}
  {\jtype{\Gamma}{\econs{e_1}{e_2}}{\tlist{t}}}
\\[0.1in]
\inference[\textsc{Match}]
  {\jtype{\Gamma}{e_1}{t_1}
    \quad
   \jtype{\Gamma}{e_2}{t_2}
    \quad
   \jtype{\Gamma,\ x_1:t_1,\ x_2:\tlist{t_1}}{e_3}{t_2}
  }
  {\jtype{\Gamma}{\elcase{e_1}{e_2}{x_1}{x_2}{e_3}}{t_2}}
\end{gather*}
\end{framed}
\caption{A simple $\lambda$-calculus with integers and lists.}
\label{fig:intro:simple-lambda}
\end{figure}

To illustrate the difficulty of pinpointing the source of type errors,
let us consider the simple $\lambda$-calculus with integers and lists in
\autoref{fig:intro:simple-lambda}.
