Static type systems are a marvelous invention.
%
They rule out, at compile-time, an entire class of run-time failures,
\eg dereferencing a |null|-pointer or passing an |int| where a |bool|
was expected.
%
Languages like \ocaml and \haskell make
the value-proposition for types even more
appealing by using constraints to automatically
synthesize the types for all program terms,
without troubling the programmer for any
annotations.
%
Unfortunately, this automation comes at a price.
%
Type annotations signify the programmer's intent, and help to correctly
blame the erroneous sub-term when the code is ill-typed.
%
In the absence of such signifiers, automatic type inference algorithms
are prone to report type errors far from their source
\citep{Wand1986-nw}.
%
While this can seem like a minor annoyance to veteran programmers,
\citet{Joosten1993-yx} have found that novices often focus their
attention on the \emph{location} reported and disregard the
\emph{message}.

In this dissertation we present two new, complementary techniques
designed to help \emph{localize} and \emph{explain} type errors, drawing
inspiration from the fields of automated program testing and machine
learning.
%
In the rest of this chapter we will set the stage for our contributions.
%
First, we will motivate the program of type error diagnosis with a
simple example, and will review the state of the art in type error
diagnosis.
%
Second, we will provide a brief introduction to automated program
testing and machine learning.
%
Finally, we will outline our novel contributions to the field.


\section{A Running Example}
\label{sec:intro:sumList}

\begin{figure}[t]
\small
\begin{minipage}{0.45\linewidth}
\begin{ecode}
let rec sumList xs =
  match xs with
  | []   -> (*@\hlSherrloc{[]}@*)
  | h::t -> h + (*@\hlTree{sumList t}@*)
\end{ecode}
\end{minipage}
\begin{minipage}{0.49\linewidth}
\begin{verbatim}
File "sumList.ml", line 4, characters 16-25:
  This expression has type 'a list
  but an expression was expected of type int

\end{verbatim}
\end{minipage}
\caption{(left) An ill-typed \ocaml program that should sum the elements of a
  list, with highlights indicating two possible blame assignments based on:
  %
  (1) the \hlTree{\ocaml} compiler; and
  %
  (2) the student's \hlSherrloc{fix}.
  %
  % The \underline{underlined} expressions are equally valid
  % locations to blame. The expression blamed by the \ocaml compiler
  % is in \textbf{bold}.
  %
  % FIXME: This bolding is ambiguous, because ``let rec'', ``match'' and
  % ``with'' are also bolded (by \\ecode)! You need to find another way to
  % highlight what ocaml is yelling about.
  %
  (right) The error reported by \ocaml.}
\label{fig:intro:sumList}
\end{figure}

Consider the \ocaml program in \autoref{fig:intro:sumList}, which is
supposed to sum the integers in a list.
%
This program was written by an undergraduate student at UC San Diego,
and works as follows.
%
In functional languages like \ocaml, lists are recursively defined as
either the empty list (written |[]| and pronounced ``nil''), or a single
element |h| followed by the rest of the list |t| (written |h::t| and
pronounced ``h cons t'')\footnote{These variable names are conventional,
  \texttt{h} stands for ``head'' and \texttt{t} for ``tail''.}.
%
Given an input |xs|, the student |match|es it against the
two forms a list can take.
%
In the |[]| case she returns another empty list |[]|, and in the
|h::t| case she adds |h| to the recursive sum of |t|.

The observant reader will notice that this program is incorrect, given
\emph{any} non-empty list of integers, the addition on line 4 will
attempt to add an integer to |[]|, which is an invalid operation.
%
In fact, the program is \emph{ill-typed} and the \ocaml compiler rejects
it with the error message in \autoref{fig:intro:sumList}.
%
Unfortunately, \ocaml's error \emph{blames} the recursive call to
|sumList|, explaining that |sumList| returns a |list|, while the |+|
operator requires an |int|.
%
The real error is on line 3, where the student returns |[]| rather than
|0| as the sum of an empty list.

As we will see throughout the rest of this chapter, this rather simple
program is sufficient to illustrate many of the difficulties of
automatically locating the source of a type error, and explaining it to
the programmer.

\section{A Simple Type System}
\label{sec:intro:simple-type}

\begin{figure}[p]
\centering
\[
\boxed{
\begin{array}{rrcl}
\mathbf{Expressions}
& e & ::=    & x \spmid \efun{x}{e} \spmid \eapp{e}{e} \spmid \elet{x}{e}{e} \\
&   & \spmid & n \spmid \eplus{e}{e}\\
    % & \spmid & b \spmid \eif{e}{e}{e} \\
    % & \spmid & \epair{e}{e} \spmid \epcase{e}{x}{x}{e} \\
&   & \spmid & \enil \spmid \econs{e}{e} \spmid \elcase{e}{e}{x}{x}{e} \\[0.05in]

%\mathbf{Numbers}
& n & ::= &  0, 1, -1, \ldots \\[0.2in]

% b & ::= &  \etrue \spmid \efalse \\[0.05in]
\mathbf{Types}
& t & ::= & \alpha \spmid % \tbool \spmid
            \tint \spmid \tarr{t}{t} \spmid % \tprod{t}{t} \spmid
            \tlist{t} \\[0.2in]

\mathbf{Environments}
& \Gamma & ::= & \cdot \spmid \Gamma;x:t \\[0.2in]

\mathbf{Substitutions}
& \sigma & ::= & \cdot \spmid \sigma \{\alpha \mapsto t\}\\
\end{array}
}
\]
\begin{framed}
\judgementHead{Typing}{\jtype{\Gamma}{e}{t}}
\begin{gather*}
\inference[\textsc{Var}]
  {x:t \in \Gamma}
  {\jtype{\Gamma}{x}{t}}
\quad
\inference[\textsc{Lam}]
  {\jtype{\Gamma;x:t_1}{e}{t_2}}
  {\jtype{\Gamma}{\efun{x}{e}}{\tarr{t_1}{t_2}}}
\quad
\inference[\textsc{App}]
  {\jtype{\Gamma}{e_1}{\tarr{t_1}{t_2}}
   \quad
   \jtype{\Gamma}{e_2}{t_1}}
  {\jtype{\Gamma}{\eapp{e_1}{e_2}}{t_2}}
\\[0.1in]
\inference[\textsc{Let}]
  {\jtype{\Gamma}{e_1}{t_1}
   \quad
   \jtype{\Gamma;x:t_1}{e_2}{t_2}}
  {\jtype{\Gamma}{\elet{x}{e_1}{e_2}}{t_2}}
\\[0.1in]
\inference[\textsc{Lit}]
  {}
  {\jtype{\Gamma}{n}{\tint}}
\quad
\inference[\textsc{Plus}]
  {\jtype{\Gamma}{e_1}{\tint} \quad \jtype{\Gamma}{e_2}{\tint}}
  {\jtype{\Gamma}{\eplus{e_1}{e_2}}{\tint}}
\\[0.1in]
\inference[\textsc{Nil}]
  {\alpha \text{ is fresh}}
  {\jtype{\Gamma}{\enil}{\tlist{\alpha}}}
\quad
\inference[\textsc{Cons}]
  {\jtype{\Gamma}{e_1}{t} \quad \jtype{\Gamma}{e_2}{\tlist{t}}}
  {\jtype{\Gamma}{\econs{e_1}{e_2}}{\tlist{t}}}
\\[0.1in]
\inference[\textsc{Match}]
  {\jtype{\Gamma}{e_1}{t_1}
    \quad
   \jtype{\Gamma}{e_2}{t_2}
    \quad
   \jtype{\Gamma;\ x_1:t_1;\ x_2:\tlist{t_1}}{e_3}{t_2}
  }
  {\jtype{\Gamma}{\elcase{e_1}{e_2}{x_1}{x_2}{e_3}}{t_2}}
\end{gather*}
\end{framed}
\caption{A simple $\lambda$-calculus with integers and lists.
  %
  % This
  % language lacks common features like recursion and polymorphic
  % functions since they do not
}
\label{fig:intro:simple-lambda}
\end{figure}

To illustrate the difficulty of pinpointing the source of type errors,
let us consider the simple $\lambda$-calculus with integers and lists in
\autoref{fig:intro:simple-lambda}.

\section{Diagnosing Type Errors}
\label{sec:diagnosing-type-errors}
In this section we will review the state of the art in type error
diagnosis, which we will categorize as follows:
\begin{enumerate}
\item \emph{localizing} errors to a specific (set of) term(s);
\item \emph{explaining} the error to the programmer; and
\item automatically \emph{fixing} the error for the programmer.
\end{enumerate}

\subsection{Localizing Type Errors}
\label{sec:localizing-type-errors}

Note that the while the typing rules in
\autoref{fig:intro:simple-lambda} specify the requirements for an
expression $e$ to have type $t$, they do not dictate a procedure for
computing $t$ given $e$, \ie, they are not \emph{algorithmic}.
%
Thus, a type system cannot be used by a computer without an algorithm
that specifies how to \emph{check} the type of an expression.
%
In our case, as the source language does not specify any types, we are
actually interested in algorithms to \emph{infer} the type of an
expression.

\begin{figure}
\centering
\[
\boxed{
\begin{array}{lcl}
\W & : & \Gamma \times e \rightarrow \sigma \times t \\
\W(\Gamma, x)
   & = & (\cdot, t) \quad \text{ where } \Gamma(x) = t \\
\W(\Gamma, \efun{x}{e})
   & = & \text{let } (\sigma, t) = \W(\Gamma;x:\alpha, e),\ \alpha \text{ is fresh} \\
   &   & \text{in  } (\sigma, \sigma(\tarr{\alpha}{t})) \\
\W(\Gamma, \eapp{e_1}{e_2})
   & = & \text{let } (\sigma_1, t_1) = \W(\Gamma, e_1) \\
   &   & \quad\ (\sigma_2, t_2) = \W(\sigma_1(\Gamma), e_2) \\
   &   & \quad\ \sigma_3 = \unify{\sigma_2(t_1)}{\tarr{t_2}{\alpha}},\ \alpha \text{ is fresh} \\
   &   & \text{in } (\sigma_3(\sigma_2(\sigma_1)), \sigma_3(\alpha)) \\
\W(\Gamma, \elet{x}{e_1}{e_2})
   & = & \text{let } (\sigma_1, t_1) = \W(\Gamma, e_1) \\
   &   & \quad\ \Gamma' = \Gamma;\ x_1:t_1 \\
   &   & \quad\ (\sigma_2, t_2) = \W(\sigma_1(\Gamma'), e_2) \\
   &   & \text{in } (\sigma_2(\sigma_1), t_2) \\
\W(\Gamma, n)
   & = & (\cdot, \tint) \\
\W(\Gamma, \eplus{e_1}{e_2})
   & = & \text{let } (\sigma_1, t_1) = \W(\Gamma, e_1) \\
   &   & \quad\ (\sigma_2, t_2) = \W(\sigma_1(\Gamma), e_2) \\
   &   & \quad\ \sigma_3 = \unify{t_1}{\tint} \\
   &   & \quad\ \sigma_4 = \unify{t_2}{\tint} \\
   &   & \text{in } (\sigma_4(\sigma_3(\sigma_2(\sigma_1))), \tint) \\
\W(\Gamma, \enil)
   & = & (\cdot, \tlist{\alpha}), \quad \alpha \text{ is fresh} \\
\W(\Gamma, \econs{e_1}{e_2})
   & = & \text{let } (\sigma_1, t_1) = \W(\Gamma, e_1) \\
   &   & \quad\ (\sigma_2, t_2) = \W(\sigma_1(\Gamma), e_2) \\
   &   & \quad\ \sigma_3 = \unify{t_2}{\sigma_2(\tlist{t_1})} \\
   &   & \text{in } (\sigma_3(\sigma_2(\sigma_1)), \sigma_3(\tlist{t_1})) \\
\W(\Gamma, \elcase{e_1}{e_2}{x_1}{x_2}{e_3}\hspace{-2ex})
   & = & \text{let } (\sigma_1, t_1) = \W(\Gamma, e_1) \\
   &   & \quad\ \sigma_2 = \unify{t_1}{\tlist{\alpha}},\ \alpha \text{ is fresh} \\
   &   & \quad\ (\sigma_3, t_2) = \W(\sigma_2(\Gamma), e_2) \\
   &   & \quad\ \Gamma' = \Gamma;\ x_1:\alpha;\ x_2:\tlist{\alpha} \\
   &   & \quad\ (\sigma_4, t_3) = \W(\sigma_3(\sigma_2(\Gamma')), e_3) \\
   &   & \quad\ \sigma_5 = \unify{t_2}{\sigma_4(t_1)} \\
   &   & \text{in } (\sigma_5(\sigma_4(\sigma_3(\sigma_2(\sigma_1)))), \sigma_5(t_1)) \\
\end{array}
}
\]
\caption{Algorithm $\mathcal{W}$~\citep{Damas1982-uw}, adapted to our language. \ES{TODO: CHECK THE NEW RULES}}
\label{fig:intro:algo-w}
\end{figure}

The traditional Damas-Milner Algorithm
$\mathcal{W}$~\citep{Damas1982-uw}, extended to our language in
\autoref{fig:intro:algo-w}, takes as input a typing environment $\Gamma$
and an expression $e$, and returns a substitution $\sigma$, mapping free
type variables $\alpha$ to concrete types, and an inferred type $t$.
%
Mechanically, it traverses the program from the bottom up, collecting
typing constraints at each expression, and halting with an error when it
detects an inconsistent constraint during a call to
$\unifysym$. \ES{TODO: CITE UNIFICATION}
%
Thus, the placement of $\unifysym$ calls determines which expression will be blamed when a program is ill-typed.

Given our |sumList| program \W will infer that the |[]| case
returns a list while the |h::t| case returns an integer, which violates
the \textsc{Match} rule's constraint that both branches must have the
same type $t_2$.
%
This violation will manifest as an error in the second $\unifysym$
call in \W's |match| case, and \W will thus blame
% $\mathcal{W}$ will then \emph{blame} the expression that induced the
% inconsistent constraint, in this case
the entire |match| expression
even though the error was in the base case.

In this case $\mathcal{W}$'s error report is actually not so
bad, a well-written error message could convey that the error is due to
the two branches having different types, which may be sufficient to
isolate the source in the base case.
%
But in general, this behavior of bubbling up of typing constraints from
the leaves of the program can produce very poor errors~\citep[see
Fig. 1 for a particularly pathological example]{Lee1998-ys}.
%
% In particular, when the same variable is used in different contexts in
% different branches \ES{TODO...}

\paragraph{Alternate Traversal Strategies}

As a result, several authors have proposed alternate traversal
strategies to improve the error reports.
%
\citet{Lee1998-ys} describe a ``folklore'' algorithm $\mathcal{M}$ that
traverses the program top down, rather than from the bottom up, pushing
constraints inward from outer expressions.
%
Thus, while \W returned both a substitution and a type, \M takes an
\emph{expected} type as input, and returns only a substitution.
%
\citeauthor{Lee1998-ys} also prove that \M always terminates sooner that
$\mathcal{W}$ would, \ie at an expression deeper in the tree.
%
\begin{figure}
\centering
\[
\boxed{
\begin{array}{lcl}
\M & : & \Gamma \times e \times t \rightarrow \sigma \\
% \W(\Gamma, x)
%    & = & (\cdot, t) \quad \text{ where } \Gamma(x) = t \\
% \W(\Gamma, \efun{x}{e})
%    & = & \text{let } (\sigma, t) = \W(\Gamma;x:\alpha, e),\ \alpha \text{ is fresh} \\
%    &   & \text{in  } (\sigma, \sigma(\tarr{\alpha}{t})) \\
% \W(\Gamma, \eapp{e_1}{e_2})
%    & = & \text{let } (\sigma_1, t_1) = \W(\Gamma, e_1) \\
%    &   & \quad\ (\sigma_2, t_2) = \W(\sigma_1(\Gamma), e_2) \\
%    &   & \quad\ \sigma_3 = \unify{\sigma_2(t_1)}{\tarr{t_2}{\alpha}},\ \alpha \text{ is fresh} \\
%    &   & \text{in } (\sigma_3(\sigma_2(\sigma_1)), \sigma_3(\alpha)) \\
% \W(\Gamma, \elet{x}{e_1}{e_2})
%    & = & \text{let } (\sigma_1, t_1) = \W(\Gamma, e_1) \\
%    &   & \quad\ \Gamma' = \Gamma;\ x_1:t_1 \\
%    &   & \quad\ (\sigma_2, t_2) = \W(\sigma_1(\Gamma'), e_2) \\
%    &   & \text{in } (\sigma_2(\sigma_1), t_2) \\
% \W(\Gamma, n)
%    & = & (\cdot, \tint) \\
\M(\Gamma, \eplus{e_1}{e_2}, t)
   & = & \text{let } \sigma_1 = \unify{t}{\tint} \\
   &   & \quad\ \sigma_2 = \M(\sigma_1(\Gamma), e_1, \tint) \\
   &   & \quad\ \sigma_3 = \M(\sigma_2(\sigma_1(\Gamma)), e_2, \tint) \\
   &   & \text{in } \sigma_3(\sigma_2(\sigma_1)) \\
% \W(\Gamma, \enil)
%    & = & (\cdot, \tlist{\alpha}), \quad \alpha \text{ is fresh} \\
% \W(\Gamma, \econs{e_1}{e_2})
%    & = & \text{let } (\sigma_1, t_1) = \W(\Gamma, e_1) \\
%    &   & \quad\ (\sigma_2, t_2) = \W(\sigma_1(\Gamma), e_2) \\
%    &   & \quad\ \sigma_3 = \unify{t_2}{\sigma_2(\tlist{t_1})} \\
%    &   & \text{in } (\sigma_3(\sigma_2(\sigma_1)), \sigma_3(\tlist{t_1})) \\
\M(\Gamma, \elcase{e_1}{e_2}{x_1}{x_2}{e_3}\hspace{-2ex}, t)
   & = & \text{let } \sigma_1 = \M(\Gamma, e_1, \tlist{\alpha_1}), \alpha_1 \text{ is fresh} \\
   &   & \quad\ \sigma_2 = \M(\sigma_1(\Gamma), e_2, \alpha_2), \alpha_2 \text{ is fresh} \\
   &   & \quad\ \Gamma' = \Gamma;\ x_1:\alpha_1;\ x_2:\tlist{\alpha_1} \\
   &   & \quad\ \sigma_3 = \M(\sigma_2(\sigma_1(\Gamma')), e_3, \sigma_2(\alpha_2)) \\
   &   & \text{in } \sigma_3(\sigma_2(\sigma_1)) \\
\end{array}
}
\]
\caption{A selection of rules from algorithm \M~\citep{Lee1998-ys},
  extended to our language.}
\label{fig:intro:algo-m}
\end{figure}

\autoref{fig:intro:algo-m} contains a selection of \M's rules that are relevant to the error in our |sumList| program.
%
Note how the |match| rule no longer does any unification, it just makes
a series of recursive calls.
%
Rather, the |+| rule is now responsible for checking that its
surrounding context expects it to return an $\tint$.
%
Crucially, \M \emph{infers} a type for the |[]| case\footnote{Note how
  we pass a fresh type variable $\alpha_2$ as the ``expected'' type.}
and then checks that the |::| case returns the same type.
%
This is a subtle change from \W's behavior, but as a result \M will
blame the |+| expression for producing an $\tint$ rather than the
|match| expression.
%
This is better, though still not ideal as the actual error is in the
|[]| case.

% It pushes typing constraints inward from the outer expressions, \eg when
% checking the |match| expression in |sumList|, it would infer that the
% |[]| branch is a list and would then check that the |h::t| branch is
% also a list.
%
% This is a subtle but crucial change from $\mathcal{W}$, as $\mathcal{M}$
% will abort when it checks that the |+| expression is a list.

Another issue, present in both algorithm $\mathcal{M}$ and
$\mathcal{W}$, is that constraints are propagated from one branch to
others, known as the ``left-to-right'' bias~\citep{McAdam1998-ub}.
%
This bias is the reason that $\mathcal{M}$ (and most type-checkers in
practice) blames the |h::t| case rather than the |[]| case; however,
it is not limited to expressions that create a branch in the program's
control-flow.
%
Rather, the constraint propagation happens between branches of the
program's abstract syntax tree.
%
See, \eg the $\eapp{e_1}{e_2}$ case of \W\ --- we apply the substitution
$\sigma_1$ from $e_1$ to the environment when checking $e_2$.
%
Thus, any expression with multiple children will be subject to the
left-to-right bias.

Instead of propagating the constraints collected in one branch to
others, \citet{McAdam1998-ub} and \citet{Yang1999-yr} suggest a
\emph{symmetric} traversal that checks each branch independently of the
others and then reports an error when merging two inconsistent sets of
constraints from the branches.
%
Mechanically, this means that we no longer apply the substitution from
one branch to the other, but rather unify the substitutions from both
branches after they have been individually checked.
%
One might think we would then be right back at the issue we observed
with algorithm $\mathcal{W}$, with an error reported at the |match|
expression, but \citeauthor{Yang1999-yr} annotates each constraint with
its source location, so that they can report the conflict between the
|[]| on line 3 and the |+| on line 4.

\paragraph{Type Error Slicing}
\citet{Tip2001-qp} and \citet{Haack2003-vc} extend the idea of
\citeauthor{McAdam1998-ub} and \citeauthor{Yang1999-yr}, and compute a
full type error \emph{slice}, \ie all of the sub-expressions that are
required for the error to manifest and no more.
%
For example, one type error slice for |sumList| would be the |match|
expression and its two children, the |[]| and the |+| expression,
which \cite{Haack2003-vc} would report as follows:
%
\begin{verbatim}
  Type error: type constructor clash, endpoints list vs. int.
  match .. with | [] -> [] | h::t -> .. + ..
\end{verbatim}

There is at least one other error slice for |sumList|, which includes
the |[]|, the |+|, the recursive |sumList| call, and the |let rec|
binder, which captures the issue that a list can be passed back through
the recursive call to the |+| operator.
%
In general there may be many distinct error slices for the same error,
and while computing one slice can be done efficiently,
computing all slices is exponential.

A drawback of type error slicers is that they typically involve
rewriting the type checker to use a specialized constraint language and
solver.
%
Production compilers for languages like \ocaml and \haskell generally
feature more advanced type languages than Hindley-Milner with heavily
optimized type checkers, so the prospect of a full rewrite to support
slicing is probably quite daunting.
%
The only type error slicer we are aware of that claims to support a full
language is the \textsc{Skalpel} slicer for
\textsc{SML}~\citep{Rahli2010-ps,Rahli2015-tt}, though
\citet{Schilling2011-yf} has shown how to how to turn any type checker
into a slicer by treating it as a black-box.

Further, while type error slicers can guarantee enough information to
diagnose an error, they can fall into the opposite trap of providing
\emph{too much} information, producing a slice that is not much smaller
than the original program.
%
In other words, a type error slicer will produce \emph{every possible}
expression that could be blamed for the error, but some expressions are
more likely to be at fault than others.

\paragraph{Finding Likely Sources of Errors}
Thus, recent work has focused on finding the \emph{most likely} source
of a type error.
%
\citeauthor{Zhang2015-yu}~\citep{Zhang2014-lv,Zhang2015-yu} use Bayesian
reasoning to search the constraint graph for constraints that
participate in many unsatisfiable paths and relatively few satisfiable
paths, based on the intuition that the program should be mostly correct.
%
\citet{Pavlinovic2014-mr,Pavlinovic2015-kh} translate the localization
problem into a MaxSMT problem, asking an off-the-shelf solver to find
the smallest set of constraints that can be removed such that the
resulting system is satisfiable.
%
\citet{Loncaric2016-uk} improve the scalability of
\citeauthor{Pavlinovic2014-mr} by reusing the existing type checker as
a theory solver in the Nelson-Oppen~\citep{Nelson1979-td}
style, and thus require only a MaxSAT solver.
%
All three of these techniques support \emph{weighted} constraints to
incorporate knowledge about the frequency of different errors,
but only \citeauthor{Pavlinovic2014-mr} use the weights, setting them to
the size of the term that induced the constraint.


\subsection{Explaining Type Errors}
\label{sec:explaining-type-errors}
The techniques we have discussed so far have focused primarily on the
task of \emph{localizing} a type error, but a good error report should
also \emph{explain} the error.
%
\citet{Wand1986-nw}, \citet{Beaven1993-hb}, and \citet{Duggan1996-by}
attempt to explain type errors by collecting the chain of inferences
made by the type checker --- essentially the typing derivation ---
and presenting them to the user.
%
Such explanations can be lengthy, as an attempt to compress the
explanation \citet{Yang2000-kz} present a visualization of the
inference process.
%
\citet{Gast2004-zd} produces a slice enhanced by arrows
showing the dataflow from sources with different types to a
shared sink, borrowing the insight of dataflows-as-explanations from
\textsc{MrSpidey}~\citep{Flanagan1996-bu}.
%

\paragraph{Interactive Explanations}
Such \emph{static} explanations of type errors run the risk of
overwhelming the user with too much information, it may be preferable to
treat type error diagnosis as an \emph{interactive} debugging session.
%
\citet{Bernstein1995-yj} extend the type inference procedure to handle
\emph{open} expressions (\ie with unbound variables), allowing users to
interactively query the type checker for the types of sub-expressions.
%
\citet{Chitil2001-td} proposes \emph{algorithmic debugging} of type
errors, presenting the user with a sequence of yes-or-no questions about
the inferred types of sub-expressions that guide the user to a specific
explanation.

\paragraph{Programmatic Explanations}
%
The best explanation of a type error, however, might be given by an
expert, \eg the compiler or library author.
%
\citet{Hage2006-hc} catalog a set of heuristics for
improving the quality of error messages by examining errors made by
novices.
%
\citet{Marceau2011-ok,Marceau2011-cy} study the effectiveness of error
messages in novice environments and present suggestions for improving
their quality and consistency.
%
\citet{Heeren2003-db}, \citet{Christiansen2014-qc}, and
\citet{Serrano2016-oo} extend the ability to customize error messages to
library authors, enabling \emph{domain-specific} errors.
%
The 8.0 release of the
Glasgow Haskell Compiler\footnote{\url{https://ghc.haskell.org/trac/ghc/wiki/Proposal/CustomTypeErrors}}
incorporates these ideas, allowing library authors to supply
custom errors when type-class resolution or type-family reduction fail,
but not for ordinary unification failures.

\subsection{Fixing Type Errors}
\label{sec:fixing-type-errors}
Finally, some techniques go beyond explaining or locating a type error,
and actually attempt to \emph{fix} the error automatically.
%
\citet{Lerner2007-dt} searches for fixes by enumerating a
set of local mutations to the program and querying the type checker to
see if the error remains.
%
\citet{Chen2014-gd} use a notion of \emph{variation-based} typing to
track choices made by the type checker and enumerate potential
% type (and expression)
changes that would fix the error.
%
They also extend the algorithmic debugging technique of
\citeauthor{Chitil2001-td} by allowing the user to enter the expected
type of specific sub-expressions and suggesting fixes based on these
desired types \citep{Chen2014-vm}.

\section{Automated Program Testing}
\label{sec:automated-program-testing}

Software testing is a time-consuming and expensive task. The National
Institute of Standards \& Technology (NIST) reports that an average of
50-70\% of development time for a software product is spent on testing
\citep{tassey_economic_2002}.
%
Furthermore, the cost of inadequate testing can be severe, NIST
estimates the annual economic cost of inadequate software testing
infrastructure at \$22.2 - \$59.5 billion.
%
Thus there is a strong incentive to develop improved testing methods and
tools.

At a high level, software testing can be divided into three levels,
based on the object being tested~\citep{bourque_swebok:_2014}.
%
Unit testing involves testing individual components of a system --
often at the level of individual functions or modules -- in isolation,
and is almost always performed by the programmers
themselves.
%
Integration testing refers to testing the interactions between units of
a system.
%
Finally, system testing exercises the entire assembled software
product, and may be performed by developers or target users.
%
In this section we will focus on automating unit testing, though the
techniques we present could also be applied to the other levels.


Two important questions that an automated test-case generation technique
must answer are:

\begin{enumerate}
\item How shall we determine the success of a test run?
\item How shall we generate appropriate input vectors?
\end{enumerate}

\paragraph{Characterizing correctness}
At the unit testing level, a common practice is to collect a set of
input/output pairs that characterize the expected behavior of the
system, and test that given an input the system produces the expected
output.
%
However, on its own this does not scale well to automatic
test-generation (how do we determine \emph{a priori} the expected
output?).
%
Thus, the systems we investigate will all use some notion of
\emph{specification-based testing}~\cite{laycock_theory_1993}.
%
A specification is a formal statement of the expected behavior of the
system under test, and may range from the simple ``this program should
not crash'' to the relatively complex ``this program takes as input a
sorted list $xs$ and an element $x$, and returns a new sorted list
containing all elements of $xs$ as well as $x$''.
%
In fact, a complex safety specification can be reduced to crash-freedom
by calling the function under scrutiny from another function that
dynamically checks the safety condition and crashes if it is violated.
%
Thus, in this report we will focus more on the method of generating
input vectors than the specification mechanism.

\paragraph{Input generation}
There are two broad approaches to constructing test vectors for a
system.
%
\emph{Functional}, or \emph{black-box}, testing treats the
system as opaque and selects input candidates based solely on the
external specification.
%
On the other hand, \emph{structural}, or \emph{white-box}, testing uses
knowledge of the system's internals to select test vectors (e.g. to
optimize statement coverage or attempt to steer control-flow to a
dangerous location).
%
Both categories are widely used, and in this section we will investigate
techniques that fall into both.

\subsection{Enumerating Inputs}
\label{sec:enumerating-inputs}
Perhaps the simplest method of automatically testing a program is to
enumerate valid inputs based on its external interface -- an instance of
\emph{black-box testing} \cite{adrion_validation_1982} -- and check
whether the program behaves correctly on these inputs.
%
Of course, enumerating \emph{all} inputs is generally infeasible, so we
must find some way of narrowing the search space.
%
The two common solutions are enumerating \emph{small} inputs and random
sampling of the entire space.

\paragraph{Enumerating ``small'' inputs}
The \emph{small-scope hypothesis} \cite{jackson_software_2006} argues
that if a property is invalid, there is likely a small counterexample,
i.e. if a program contains a bug there is likely a small input that will
trigger it. Thus, we can restrict our enumeration to ``small'' inputs
and still gain a large degree of confidence in our program.

SmallCheck \cite{runciman_smallcheck_2008} is a testing library for
\haskell programs
% that does the simplest thing possible.
% that uses \haskell's type-class system \cite{wadler_how_1989} to
that can automatically
enumerate all possible inputs up to some depth, guaranteeing a
\emph{minimal} counterexample if one exists within the depth-bound.
% %
% Programmers specify how to enumerate input values using \haskell's
% type-class system \cite{wadler_how_1989} and provide a boolean-valued
% function that describes the property they wish to check, and SmallCheck
% enumerates all possible inputs up to some depth, validating the property
% on each input vector.
% %
% The input generators are expected to produce values in increasing size
% (indeed, such a generator can be automatically derived from a datatype
% definition), thus SmallCheck is guaranteed to find the \emph{minimal}
% counterexample if it lies within the depth-bound.
%
Naturally, such a brute-force enumeration of input vectors is bound to
be wasteful (\ie it will produce many inputs that trigger the same code
path), thus the authors also introduce a \emph{lazy} variant that takes
advantage of \haskell's inherent laziness to prune the search space.
%
The key distinction of Lazy SmallCheck is that it produces
\emph{partially-defined} values instead of fully-defined values,
i.e. Lazy SmallCheck will initially leave each field of a value
uninitialized,
%
only computing an actual value once the field is demanded by
evaluation.
% so that the Haskell runtime will thrown an exception if
% the field is read.
% %
% Only when a field is demanded will Lazy SmallCheck fill in the hole with
% an actual value.
Thus, Lazy SmallCheck reduces the search space by
dynamically discovering which parts of a value are relevant to the
property being tested.
%
% Initially, Lazy SmallCheck had several disadvantages compared to
% SmallCheck.
% %
% It could not generate functional values (commonly used in Haskell), and
% it could not print the partially-defined counterexample, instead it
% would have to concretize the input vector, thus \emph{losing}
% information.
% %
% These were both added later as enhancements \cite{reich_advances_2013}.
%
% A remaining concern is that programmers must be careful about the order
% in which they conjoin predicates, giving precedence to the lazier
% predicate.
% %
% For example, a binary-search tree implementation will want to check the
% ordering invariant before balancedness; the former will often fail
% within the first few nodes whereas the latter will force the entire
% spine.

%\paragraph{TestEra / Korat}
TestEra \cite{khurshid_testera:_2004,marinov_testera:_2001} and Korat
\cite{boyapati_korat:_2002} provide (bounded) exhaustive testing of Java
methods.
% against an Alloy
% \cite{jackson_automating_2000} specification
%
%Unlike SmallCheck, which enumerates input vectors by unfolding each
% possible data constructor and enumerating its arguments,
TestEra begins
with a pre-defined universe of atomic values (e.g. primitive types and
uninitialized objects), and uses the Alloy Analyzer
\cite{jackson_alcoa:_2000} to enumerate all valid \emph{relations}
between the atoms, \ie setting the pointers appropriately such that the
method's preconditions are satisfied.
%
% Thus, an important optimization that TestEra implements is filtering
% \emph{isomorphic} inputs.
% %
% For each symbolic input vector produced by Alloy, TestEra
% \emph{concretizes} the input to a set of Java objects, executes the
% method, and then \emph{abstracts} the result back to an Alloy
% value.
% %
% Finally, it queries the Alloy Analyzer again to determine whether the
% output satisfies the method's postcondition, then proceeding to the next
% input vector.
%
Unlike SmallCheck, TestEra is not guaranteed to find a minimal
counterexample as the testing order is left up to Alloy.
%
Korat optimizes TestEra by using a custom enumeration algorithm instead
of calling out to Alloy.
%
It enumerates all input candidates, and runs an instrumented version of
a programmer-supplied \texttt{repOk} method (to check class invariants
and preconditions) that tracks field accesses.
%
If \texttt{repOk} returns \emph{true} Korat immediately tests all
(non-isomorphic) input vectors that share the current valuation of the
\emph{accessed} fields, as the valuation of the unread fields cannot
have affected the outcome of \texttt{repOk}.
%
Otherwise it backtracks, skipping all variant candidates that share
the current valuation of the accessed fields, as \emph{none} of them can
be valid.
%
Thus, Korat takes advantage of the inherent laziness of
invariant-checking predicates to quickly prune the search space of
invalid inputs, in a very similar manner to Lazy SmallCheck.

\paragraph{Random sampling of inputs}
The obvious drawback to the approaches described above is that they will
not detect bugs that only present on ``large'' inputs.
%
Thus, a common alternative to enumeration of small inputs is random
selection of input vectors from the entire search space.

% \paragraph{QuickCheck}
QuickCheck \cite{claessen_quickcheck:_2000,claessen_testing_2002}
enables randomized testing of \haskell programs by providing an embedded
domain-specific language for generating \emph{arbitrary} values of a
given datatype.
%
% As with SmallCheck, QuickCheck properties are boolean-valued \haskell
% functions whose inputs can be (randomly) generated, and the
% input-generators (usually) operate by unfolding a specific data
% constructor and generating sub-values for the constructor's
% fields.
%
Unlike SmallCheck, it is impractical to automatically derive QuickCheck
generators for datatypes, as one must take care to ensure the generator
covers a uniform distribution of values.
%
% For example, a generator for a simple ``list'' type
% %
% \begin{verbatim}
%   type 'a list = Nil | Cons of 'a * 'a list
% \end{verbatim}
% %
% that chooses between \texttt{Nil} and \texttt{Cons} with equal
% probability is highly unlikely to generate lists with more than a
% handful of elements.
%
A further concern arising from random testing is that the returned
counterexample may be quite large, as demonstrated by Pike
\cite{pike_smartcheck:_2014}.
%
Thus, subsequent iterations of QuickCheck introduced support for
\emph{shrinking} counterexamples \cite{hughes_quickcheck_2006}.
%
Once QuickCheck has found a counterexample it will invoke a user-defined
\texttt{shrink} function on the input vector, which will return a list
of smaller inputs.
%
QuickCheck will test each small candidate in turn and repeat the
shrinking process on any new counterexamples, finally returning the
smallest counterexample it could find.
%
Notably, the shrinking process is not guaranteed to find a
\emph{minimal} counterexample.
%
Pike \cite{pike_smartcheck:_2014} builds on top of QuickCheck with
SmartCheck, which provides automatically-derivable shrinking definitions
that are shown to perform favorably compared to handwritten
\texttt{shrink} functions, and produce smaller counterexamples.
%
More interestingly, SmartCheck also introduces \emph{counterexample
  generalization}, which attempts to produce a universal property
describing a class of counterexamples.
%
% For example, Pike shows that SmartCheck can reduce a large
% counterexample like
% %
% \begin{verbatim}
%   StackSet
%    (Screen (Workspace 0 (-1)
%      (Just (Stack 'S' "" ""))) 1 1)
%    [Screen (Workspace 2 (-1) Nothing) 2 (-1),
%     Screen (Workspace 3 (-1) Nothing) 0 (-1)]
%    [Workspace 1 (-1) (Just (Stack 'NUL' "" "")),
%     Workspace 4 (-1) (Just (Stack 'I' "" ""))]
%    (fromList [])
% \end{verbatim}
% %
% to a comparatively simple formula
% %
% \begin{verbatim}
%   forall values x0 x1 x2 x3:
%     StackSet
%       (Screen (Workspace x0 (-1) (Just x1)) 1 1)
%       x2 x3 (fromList [])
% \end{verbatim}
% %
% thus abstracting away the irrelevant portions of the counterexample.
% %
% The key insight is that if one can replace a sub-value by another
% arbitrary value without affecting the test outcome, then the sub-value
% must not affect the outcome.
% %
% Thus, SmartCheck systematically replaces all sub-values of the
% counterexample with other random values and generalizes the
% counterexample accordingly.

% QuickCheck does not currently have good support for testing properties
% with preconditions, due to the low probability of randomly generating a
% value that satisfies the precondition.
% %
% \citep{claessen_generating_2014} describes an algorithm for random
% generation of constrained inputs based on \cite{duregard_feat:_2012}, by
% defining a function to index into a uniform distribution of constrained
% values, and then generating random indices, but the work has not yet
% been incorporated into QuickCheck.

% \paragraph{JCrasher}
Instead of constructing input vectors directly, JCrasher
\cite{csallner_jcrasher:_2004} constructs them indirectly via
\emph{method chaining}.
%
Given a set of Java classes, JCrasher constructs a parameter graph,
where the nodes are public methods, constructors, and primitive values,
and the edges run from method parameters to nodes producing values of
the needed type.
%
By randomly choosing paths through the graph starting from the method
under test, JCrasher creates sequences of method and constructor calls
that should produce valid input vectors.
%
JCrasher executes these method chains followed by the target method,
under the assumption that public methods and constructors should not
produce inputs that will crash a program.
%
There is some subtlety in the use of ``crash'', as Java methods will
frequently throw exceptions when given invalid input parameters.
%
We should not consider precondition violations as ``crashes'' as the
responsibility of providing valid inputs rests with the \emph{caller},
not the \emph{callee} \cite{meyer_applying_1992}.
%
Thus, JCrasher includes a number of heuristics to determine whether a
thrown exception should be considered a bug.
%
% For example, an \texttt{IllegalArgumentException} can be considered a
% bug if it was thrown by a transitively-called method, but not if it was
% thrown directly by the method under test, as that would indicate that
% our test vector was at fault.
% %
% On the other hand, an \texttt{ArithmeticException} (e.g. divide-by-zero)
% can always be classified as a bug, as the method under test should have
% caught and handled it.

%\paragraph{Randoop}
Randoop \cite{pacheco_feedback-directed_2007} extends the
method-chaining approach of JCrasher by incorporating feedback from
previously generated test vectors.
%
The main insight is that if a sequence of methods $s$ results in a
crash, there is no point in checking any sequences that include $s$ as a
prefix.
%
Thus, Randoop iteratively constructs longer chains of method calls, only
extending existing chains if they do not cause the program to
crash.
%
Furthermore, Randoop applies some filters to the generated sequences
before testing them to further reduce the search space, e.g. by
discarding sequences that immediately throw an exception or produce a
value that equal to the result of an existing sequence.

\subsection{Enumerating Code Paths}
The drawback to explicit enumeration of input vectors is that many
inputs will trigger similar behavior in the program under test.
%
Indeed unit testing texts often advise programmers to first partition
program inputs into \emph{equivalence classes}, and then test a single
input vector from each equivalence class, thereby minimizing the number
of handwritten tests required \cite{burnstein_practical_2003}.
%
So instead of enumerating inputs, perhaps we should enumerate program
behaviors, i.e. paths through the program.
%
This necessarily requires knowledge of the internal structure of the
program under test, thus tools that take this approach will fall in the
category of \emph{white-box testing} \cite{adrion_validation_1982}.

Tools that take this approach typically use \emph{dynamic-symbolic
  execution}, which combines traditional symbolic execution with
concrete execution, to quickly explore different paths through the
program.
%
The two main categories of dynamic-symbolic execution-based testing
tools are concolic testing and execution-generated testing, both
introduced independently in 2005
\cite{Godefroid2005-am,cadar_execution_2005}.

\paragraph{Symbolic Execution}
Symbolic execution as a method of testing programs is not a new idea, it
was introduced in 1976 by King \cite{king_symbolic_1976}.
%
The key difference in between symbolic and concrete execution is that
instead of mapping program variables to \emph{values}, a symbolic
executor maps them to \emph{symbolic expressions}.
%
For example, given the simple program
%
\begin{verbatim}
  int f (int x, int y) {
    return 2 * (x + y);
  }
\end{verbatim}
%
a concrete execution may begin with input vector
$\{x \mapsto 1, y \mapsto 2\}$ and return $6$.
%
A symbolic execution, however, will begin with an input vector
$\{x \mapsto \alpha_1, y \mapsto \alpha_2\}$ -- where $\alpha_i$ are
symbolic variables -- and return $2 * (\alpha_1 + \alpha_2)$, thereby
precisely describing \emph{all} possible executions of \verb~f~.
%
Another key difference of symbolic execution is its handling of
conditionals. Consider the first conditional in the following program.
%
\begin{verbatim}
  int f (int x) {
    if (x > 0) {
      if (x == 0) {
        abort();
      }
    }
    return 0;
  }
\end{verbatim}
%
With the input vector $\{x \mapsto \alpha_1\}$, the symbolic executor
does not know which direction of the branch it should take, as it knows
nothing about the symbolic variable $\alpha_1$.
%
Therefore it must follow both directions!
%
When following a branch, the symbolic executor records the symbolic
expression associated with the chosen direction in its \emph{path
  constraint}, which we will write as a sequence of expressions
$\langle e_1, e_2, \ldots \rangle$.
%
For example, in the outer conditional above, the ``true'' case would
record $\langle \alpha_1 > 0 \rangle$ and the false case would record
$\langle \lnot (\alpha_1 > 0) \rangle$.
%
Thus, it remembers what properties of the program inputs will trigger
specific paths through the code.
%
When the symbolic executor reaches a branch point, it consults the
current path constraint to determine with directions are feasible.
%
For example, upon reaching the inner conditional above, the symbolic
executor will check whether $\alpha_1 = 0$ is consistent with the path
condition $\langle \alpha_1 > 0 \rangle$, i.e. is the formula
$\alpha_1 = 0\ \land\ \alpha_1 > 0$ satisfiable?
%
As the formula is clearly unsatisfiable, the symbolic executor will
decide that the ``true'' branch is \emph{unreachable}, and continue by
only pursuing the ``false'' branch.
%
Thus, a symbolic executor can statically determine that the
\verb~abort()~ call above can \emph{never} be executed.

While a powerful idea in theory, symbolic execution crucially relies on
a theorem prover to solve the symbolic expressions it creates, and as
such it went relatively unused until recent advances in constraint
solving technology.

\paragraph{Concolic Testing}
Godefroid et al. introduced \emph{concolic testing} in 2005
\cite{Godefroid2005-am}.
%
Concolic testing performs symbolic and concrete execution of a program
in tandem.
%
Thus, when confronted with a program expression that the symbolic
executor cannot reason about, a concolic tester can fall back to the
concrete value and continue execution with more precision than a purely
symbolic approach.

DART \cite{Godefroid2005-am} instruments a C program to execute
each instruction both concretely and symbolically, then performs a
depth-first search of all paths through the program, starting
with a random input vector.
%
At each branch point, DART records the branch condition and the
direction taken, thereby building a \emph{path constraint}.
%
% For example, suppose DART is testing the following C program with
% initial inputs $\{x \mapsto 5, y \mapsto 6\}$.
% %
% \begin{verbatim}
% int f (int x, int y) {
%   if (x == 5) {
%     if (2 * y == x) {
%       abort();
%     }
%   }
%   return 0;
% }
% \end{verbatim}
% %
% This execution will satisfy $x = 5$ but not $2y = x$, thus the path
% constraint will be $\langle x = 5,\ 2y \neq x \rangle$.
%
% Next,
After completing a program run, DART will negate the last predicate in
the path constraint and query a constraint solver for a solution
% to $x = 5 \land 2y = x$,
in order to produce a new input vector.
%
% There is only one solution to this constraint,
% $\{x \mapsto 5, y \mapsto 10\}$, which will force execution through the
% \emph{true} branch of both conditionals, right into the erroneous
% \texttt{abort()} call.
%
% Since the concrete execution reached the \texttt{abort()} call, we know
% it is a real bug as opposed to a false positive that could come from a
% purely symbolic approach, i.e. DART \emph{soundly} reports bugs.
Since the program is always being executed concretely, we know
that any crashes discovered by DART are true bugs, \ie DART is
\emph{sound}.

When confronted with an expression that it cannot reason about
symbolically, e.g. multiplication of two variables or a dereference of a
pointer that depends on program input, DART will fall back to recording
the result of the concrete evaluation.
% For example, given
% %
% \begin{verbatim}
% int f (int x, int y) {
%   if (x == y*y) {
%     abort();
%   }
%   return 0;
% }
% \end{verbatim}
% %
% and starting inputs $\{x \mapsto 5, y \mapsto 2\}$, DART will produce a
% path constraint $\langle x \neq 4 \rangle$ for the first
% execution.
% %
% Refuting this path constraint will \emph{not} produce an input vector
% that is guaranteed to take the \emph{true} branch -- indeed the solver
% may return the original input vector -- thus DART suffers a severe loss
% of precision when the path-constraint veers outside the language of the
% constraint solver.
%
In effect, this means DART degenerates to brute-force enumeration of
inputs, as in \autoref{sec:enumerating-inputs}.
%
Furthermore, DART's depth-first enumeration of paths means that it may
fail to discover all paths when presented with recursive programs,
e.g. a program that checks the ordering invariant of a binary-search
tree.
%
% In this case DART will loop forever, generating increasingly deep trees
% whose right sub-trees are always \texttt{NULL} (assuming the program
% checks the left sub-tree first).

Sen et al. introduced CUTE \cite{Sen2005-ju} later that year, an
extension of DART that adds support for testing complex datatypes.
%
CUTE enhances DART's technique by adding support for (dis)equality
constraints on pointers, and by switching to a \emph{bounded}
depth-first search.
%
% \paragraph{Pointer (dis)equality}
% \label{sec-3-2-2-1}
% Whereas DART maintained a single map of memory locations to symbolic arithmetic
% expressions, CUTE maintains two maps of memory locations: (1) $\mathcal{A}$ to
% arithmetic expressions and (2) $\mathcal{P}$ to pointer expressions. $\mathcal{A}$
% contains the usual linear arithmetic expressions as in DART; however,
% $\mathcal{P}$ contains expressions of the form $x_p \cong y_p$ where
% $x_p$ is either a symbolic variable or the constant symbol \texttt{NULL} and
% $\cong\ \in \{=, \neq\}$. When solving a pointer constraint, CUTE partitions the
% variables in $\mathcal{P}$ into equivalence classes and applies the arithmetic
% constraints to all members of the equivalence class. For example, given
%
% \begin{verbatim}
% int f (int *x, int *y) {
%   if (x == y) {
%     if (*x == 5) {
%       return 0;
%     }
%   }
%   return 0;
% }
% \end{verbatim}
%
% and the path constraint $\langle x = y,\ *x \neq 5 \rangle$, when CUTE refutes
% the $*x \neq 5$ conjunct, the value of $*y$ will \emph{also} be forced to $5$ as $x$
% and $y$ are in the same equivalence class.
%
% \paragraph{Bounded Depth-First Search}
% \label{sec-3-2-2-2}
% In order to avoid an infinite loop from the repeated inlining of a loop body or
% recursive call, CUTE places a configurable bound $k$ on the number of predicates
% in the path constraint. Once the path constraint is full, CUTE stops recording
% any further nested branch conditions, thereby forcing the refutation process to
% negate an earlier constraint. For example, given
%
% \begin{verbatim}
% int f (int n) {
%   for (int i = 0; i < n; i++ ) {
%     ...
%   }
%   return 0;
% }
% \end{verbatim}
%
% and $k = 4$, CUTE will never force more than four iterations of the
% loop body, as the path constraint will be cut off at
% $\langle i_0 < n,\ i_1 < n,\ i_2 < n,\ i_3 < n \rangle$. Negating the last conjunct
% will force $n \leq 3$, and CUTE will begin to backtrack through the path
% constraint until it terminates. While this tactic forces broad rather than deep
% coverage, it also means that CUTE may miss bugs deep in the execution graph of
% the program, e.g. if the loop body above were \verb~if (i == 5) abort();~.
%
% Another tactic CUTE employs to quickly achieve high coverage is branch
% prediction. Since CUTE only refutes the final conjunct of the path constraint,
% the outcomes of the previous branches should remain the same. Deviation from the
% previous path at an earlier branch indicates an imprecision in the symbolic
% executor; in this case CUTE will decide to restart execution with random inputs
% instead of allowing the loss of precision.
%
% \subsubsection{PEX}
% \label{sec-3-2-3}
Tillman and Halleaux further extended concolic testing with Pex
\cite{Tillmann2008-qc} in 2008, adding heuristics to improve
path-selection, modeling of interactions with the environment, and a
richer constraint language.
%
% \paragraph{Richer constraints}
% \label{sec-3-2-3-1}
% Whereas previous systems had limited constraint languages -- linear
% arithmetic for DART, with the addition of pointer equality for CUTE --
% Pex takes advantage of the rich constraint language offered by Z3
% \cite{de_moura_z3:_2008}.
% %
% Pex supports linear arithmetic, bit-vectors, arrays directly via Z3.
% %
% Pex further supports floating-point numbers with an approximation to
% rational numbers.
% %
% % \paragraph{Improving path-selection}
% % \label{sec-3-2-3-2}
% Instead of performing a depth-first search of all program paths, Pex
% maintains a tree of all branch conditions it has encountered.
% %
% After exploring a path, Pex will choose a new unexplored path from the
% unexplored leaves of the execution tree, using several heuristics to
% partition branches into equivalence classes and then choosing a new
% branch from the least-often chosen class.
% %
% Thus, Pex favors a more breadth-oriented search than DART or CUTE, while
% avoiding randomness in its path-selection.
% %
% % \paragraph{Dealing with the environment}
% % \label{sec-3-2-3-3}
% Pex builds a model of the environment by recording the inputs and
% outputs of function calls where the source code is unavailable.
% %
% This allows Pex to increase its precision when determining the
% feasibility of a path, but it also makes Pex unsound as the model is
% necessarily an under-approximation.


\subsection{Execution-Generated Testing}
\label{sec-3-3}
Instead of performing symbolic and concrete execution in tandem,
\emph{execution-generated testing} \cite{cadar_execution_2005} begins
with pure symbolic execution and lazily generates concrete inputs on
demand.
%
When a dangerous operation (e.g. division or memory read/write) is about
to be executed, the system will insert an implicit branch denoting the
possibility of an error (e.g. divide-by-zero or out-of-bounds write).
%
If the error branch is deemed feasible, the system will then solve the
path constraint for an input vector designed to trigger the error
condition.
%
Similarly, function calls into uninstrumented code, e.g. library
functions or system calls, will induce a call to the constraint solver
for a concrete set of inputs designed to trigger the call.
%
When the external call returns, the system will continue execution with
the concrete result, thus improving precision over pure-symbolic
approaches that would have to somehow model the interaction with the
external world (often simply assuming nothing about the result).

% \subsubsection{EXE}
% \label{sec-3-3-1}
Cadar et al. introduced execution-generated testing with EXE
\cite{cadar_exe:_2006}.
%
EXE models program memory as arrays of bitvectors, enabling bit-precise
reasoning about the C programs it tests via the co-developed constraint
solver STP \cite{ganesh_decision_2007}.
%
This crucial distinction from DART and CUTE allows EXE and STP to view
program values in the same way as the systems software they test, as
untyped bytes.
%
% At each branch EXE forks execution for each direction of the branch that is
% deemed feasible. The child processes add their direction to the path contraint
% and go to sleep. A master process then decides which child (path) should
% continue executing, using a combination of depth-first and best-first
% search. The master process chooses the child blocked on the instruction with the
% lowest execution count and runs it and its children in DFS for some period of
% time. Then it picks another best candidate and repeats the process.
%
% An important optimization of EXE is \emph{aggressive concretization}. If the operands
% are all concrete (i.e. constant values), EXE will simply perform the operation
% and record the resulting concrete value. This helps simplify the queries sent
% to STP, such that the only symbolic variables in a query will have a data
% dependence on one of the initial symbolic variables.
%
% \subsubsection{KLEE}
% \label{sec-3-3-2}
In 2008, Cadar et al. rewrote EXE as KLEE \cite{Cadar2008-kg}, which
symbolically executes LLVM IR \cite{lattner_llvm:_2004} and provides
several enhancements over EXE.
%
% \paragraph{Compact process representation}
% \label{sec-3-3-2-1}
EXE cleverly used the host OS to fork processes for each branch,
thus avoiding the need for complex backtracking logic to enumerate
program paths, and to share memory between the processes.
%
Unfortunately, this did not scale well to larger programs with many
program paths, so KLEE implemented its own backtracking and sharing
logic in order to scale up to testing all of GNU Coreutils.
%
%  processes relied on the host OS to share memory and was
% thus limited to page-level granularity, KLEE implements sharing with a
% granularity of individual objects, thus tracking many more processes
% than EXE could with the same memory limit.
% %
% This optimization enabled KLEE to scale up to testing all of GNU
% Coreutils.
%
% \paragraph{Random path selection and Coverage-optimized search}
% \label{sec-3-3-2-2}
KLEE also employs two path selection strategies in round robin to
prevent either one from getting stuck.
%
\emph{Random path selection} maintains a tree of all branches KLEE has
encountered.
%
It starts at the root and randomly picks a child node until it hits a
leaf, and schedules the corresponding process for execution.
%
This favors broad and shallow coverage, while still allowing for deep
paths to be chosen.
%
\emph{Coverage-optimized search} weights each process according to some
heuristics, e.g. distance to an unexecuted instruction, and biases the
choice accordingly.
%
% \paragraph{Environment modeling}
% \label{sec-3-3-2-3}
% KLEE models the environment at the level of system calls, by replacing
% the actual system call with a simplified C implementation. Thus there is
% no ``foreign'' code and the developers can model interactions with the
% external world with as much precision as they desire. The drawback is
% that KLEE must now additionally reason about the mock system calls (as
% well as any library code leading up to them).


\section{Machine Learning}
\label{sec:machine-learning}

\ES{TODO}

\section{Our Contributions}
\label{sec:our-contributions}

The thesis of this dissertation is that we can adapt the wealth of work
in automated program testing and machine learning to the task of
providing better diagnostic information for type errors.
%
To that end we present three concrete contributions:
%
\begin{enumerate}
\item In Chapter~\ref{chp:data-collection} we present a dataset of
  novice interactions with the \ocaml type checker, which will form the
  backbone of our evaluation.
  %
  This dataset contains thousands of ill-typed programs from over one
  hundred different students, the largest set of novice type errors we
  are aware of.
  % We gathered these interactions from students in an undergraduate
  % course at UC San Diego over the course of two quarters of instruction.
\item In Chapter~\ref{chp:nanomaly} we use techniques from automated
  program testing to search for \emph{witnesses} to type errors, \ie
  input vectors that would cause the program to crash.
  %
  Once we have found a witness, we compute an execution trace that
  \emph{demonstrates} how the program goes wrong, and provide an
  interactive debugger with which students can explore the erroneous
  computation.

\item In Chapter~\ref{chp:nate} we use machine learning techniques to
  \emph{learn} a model of where type errors are most likely to occur in
  our students' programs.
  %
  This model allows us to make significantly more accurate predictions
  of where the error is most likely to be found.
\end{enumerate}

Our contributions are presented and evaluated in the context of \ocaml
programs written by novice programmers, but they are not restricted to
this domain.
%
Rather, we chose \ocaml as the target language as it has a powerful type
system with global inference, showcasing both the advantages of static
typing and the limitations of type inference.
%
We use novice programs for our benchmarks as we feel novices are the
most in need of assistance in diagnosing type errors.
%
In our experience, working with the type system can be quite pleasant
once you have gotten used to it; however, getting that point is
difficult.
%
% We will note specific limitations of our techniques and describe
% potential remedies in the relevant chapters.

\section*{Endnotes}
\paragraph{Acknowledgments}
This chapter contains material adapted from the following publications:
\begin{itemize}
\item \ES{CITE NANOMALY (ICFP \& JFP)}
\item \ES{CITE NATE}
\end{itemize}



%%% Local Variables:
%%% mode: latex
%%% TeX-master: "main"
%%% End:
